% \setchapterimage[6.5cm]{seaside}
\setchapterpreamble[u]{\margintoc}
\chapter{Conclusions and further remarks}
\labch{conclusions}

% Conclusions

\section{Discussion of results} \labsec{end_discussion}

\subsection{Feasibility of these models}

testo

\subsection{Indefinite object drop cross-linguistically}

testo

\subsection{Relevance of the two new constraints}

testo (voleva AL)

\subsection{Comments on PISA models of semantic selectivity}

testo (voleva AL)


\section{Future directions} \labsec{end_future}


\subsection{Expanding the model}


\paragraph{Additional constraints}
testo

% ampliare il set dei constraints (riferimento a hopper thompson 1980! v. capitolo 2)

\paragraph{Corpus frequencies}

As shown in \textcite{Boersma2004, BoersmaHayes2001empirical}, Stochastic Optimality Theory can be used to model corpus frequencies just as well as grammaticality judgments, namely, via the evaluation of constraints that get re-ranked along a continuous numerical scale (refer to \refch{modeltheory}). Rather than trivially duplicating the results I obtained and discussed in this dissertation, new models of corpus frequencies are sure to shed a different light on the indefinite implicit object construction. As I anticipated in \refsec{frequencyfail}, neither \textcite{Resnik1993, Resnik1996} nor \textcite{Medina2007} found a precise correlation between corpus frequencies and gradient grammaticality judgments provided in behavioral experiments about indefinite implicit objects.\\
Indeed, linguistic research has long since shown that there is no clear-cut correspondence between ratings elicited from native speakers and corpus frequencies \parencite{manning2003probabilistic}. In particular, it is often the case that low-frequency utterances (or other linguistic items) receive mid-to-high acceptability judgments in behavioral experiments \parencite{KempenHarbusch2005, BermelKnittl2012, BaderHaussler2010, Boersma2004, KellerAsudeh2002}. There is also no strict relation between the relative grammaticality of a linguistic structure with respect to another and their relative corpus frequencies, since, for instance, \textcite[315-316]{BaderHaussler2010} report that they found no pairs of syntactic structures in their study where a member of the pair was judged as more grammatical than the other but occurred with a smaller frequency in the corpus, while \textcite{Boersma2004} argues in favor of the opposite. Moreover, \textcite{BaderHaussler2010} experimental results show both a "ceiling mismatch"\sidenote{The authors also observe that this is not a measurement artifact due to the use of a capped scale, such as binary or 7-point Likert ratings, because it is also found with Magnitude Estimation ratings, which are open-ended both at the top and at the bottom.} (meaning that two syntactic structures may be judged as maximally grammatical, but occur with different frequencies in the corpus) and a "floor mismatch" (meaning that two syntactic structures may never or almost-never occur in the corpus, but receive different acceptability judgments).\\
A common worry about linguistic research based on corpus material is that frequencies are less reliable than human judgments because there is no way to control language production as one controls an experimental design. This line of reasoning would surely curb easy enthusiasm about the replication of the current study to model corpus frequencies of indefinite object drop, if \textcite{Steube2008, Schutze2016} did not observe that acceptability ratings are too "contaminated by performance factors", that is to say, biased by other tasks the raters perform in addition to the one they are explicitly asked to carry out (e.g., they judge the similarity between the target sentence and the one they consider its "ideal delivery" paraphrase). Thus, if linguistics gladly relies on acceptability judgments (and, oftentimes, the results of one's own introspection), provided they are based on a rigorous experimental design, there should be no qualms about modeling corpus frequencies, provided they are interpreted in the light of the factors possibly influencing them.\\
Modeling corpus frequencies of indefinite null objects using the very same model(s) I defined in this dissertation may present additional challenges if compared to modeling acceptability ratings, since it is impossible to manipulate aspectual and discourse factors in a corpus study as in a behavioral experiment. However, the possible absence (or very low frequency) of a given object-less verb in a given aspect may well be considered an interesting, modelable datum in itself, provided one adjusts the model to account for such findings. Alternatively, it would be possible to design a production experiment to design an \textit{ad-hoc} corpus to model the frequency of indefinite null objects in controlled speech or writing. It is also important to note that it would be possible, if not even easy, to include discourse and world-knowledge context (somewhat ancillary to semantics and aspectual factors in this dissertation) in a model of object drop based on frequencies extracted from a large corpus, given that these null objects appear in sentences which are part of larger documents with explicit context information. Moreover, a corpus study of object drop may provide an answer to a question foreshadowed by \textcite{KempenHarbusch2005, Medina2007} (refer to \refsec{frequencyfail}), namely, whether a "production threshold" exists blocking mid-to-low grammaticality structures from ever being uttered and, if so, which numerical value has to be assigned to this threshold.


\paragraph{Other indefinite implicit complements}
testo

% parlare degli strumenti (consultare notebook_ideas.txt e appunti mPad) + Boland (2005) e gli strumenti come caso ibrido tra argomenti e aggiunti! e forse Resnik (1993)
% verbi di moto?

\paragraph{Indefinite object drop diachronically}
testo

% *Goldberg (2001: 518)*  
% MOLTO IMPORTANTE!!! PROVARE A FARE QUESTO STUDIO DIACRONICO?  
% > Interestingly, the same set of verbs frequently occurs in generic contexts with a
% habitual interpretation: Pat drinks; Pat smokes; Chris sings; Sam bakes. It seems
% likely that the frequent appearance of this usage, which is licensed by the Omission
% under Low Discourse Prominence principle, led to the grammaticalization of a lex-
% ical option for these verbs, whereby they could appear intransitively in less con-
% strained contexts. Corpus and historical work, to determine the frequencies of usage
% and the historical evolution, would be required to determine whether this hypothesis
% is correct.

% *Glass (2020: 5)*
% > if a verb appears frequently in generic or habitual contexts, it is likely to omit its objectin those contexts, a pattern which would diachronically generalize to episodic contexts as well

\paragraph{Typologically different languages}

testo

% lingue tipologicamente diverse da inglese e italiano che marcano l'aspetto (perfectivity) in modo diverso, come il russo

% Dire che in inglese e in italiano telicity/perf/tense interagiscono (rif Cap 3) e la loro interpretazione dipende in parte l'uno dall'altro, il che spiega il ruolo importante di tel e perf (soprattutto tel) nel modello e anche l'assenza di assolutezza. Ipotizzo invece (v. riferimenti biblio in Cap 3) che siccome nelle lingue slave l'aspetto è lessicalizzato e evidence mostra che non c'è object drop coi perfettivi, un modello come il mio in una lingua slava darebbe risultati molto più netti.

% OTHER LANGUAGES
% > Jackendoff (2002, p. 134, fn. 65)
% Many languages, such as Korean and Japanese, are much freer than English in omitting arguments. In such languages there may be no justification for distinguishing
% between obligatorily and optionally expressed semantic arguments.
% [mi interessa soprattutto il punto qui sopra]
% There is often an impulse to conjecture that the obligatoriness of arguments is predictable from the
% semantics. For instance, devour is more specific in its semantics than eat, so perhaps more specific verbs tend to make their arguments obligatory. This conjecture is
% immediately counterexemplified by contrasts like serve/give the food to Sally vs. serve/*give the food, where serving is a more specific form of giving, and insert/put the letter in the slot
% vs. insert/*put the letter, where inserting is a more specific form of putting. I also note the verbs juggle and flirt, which have got to be among the more highly specific verbs in the
% language, yet take an optional syntactic argument: juggle (six balls), flirt (with Kim). These arguments are not optional in semantics: one certainly can't juggle without juggling
% something, nor flirt without flirting with someone. I conclude that the obligatoriness of syntactic arguments must be encoded as an idiosyncratic lexical property.An
% important and little-mentioned distinction among implicit syntactic arguments is whether they are construed as “indefinite” or “definite.” For instance, the implicit argument
% in John is eating is indefinite: it can be paraphrased by John is eating something. By contrast, the implicit argument in I'll show you is definite: there has to be an understood discourse
% argument, and the explicit paraphrase is I'll show it to you. The choice between indefinite and definite implicit arguments seems specific to the verb (or possibly the semantic
% verb class). See e.g. Grimshaw (1979), among others, for discussion of this distinction.

\subsection{Different underlying math}

testo

% non-linear function nel modello! v. Medina p. 110 del pdf, nota 18

% rendere StOT simile a lmem()! io mi limito a minimizzare l'effetto-partecipanti normalizzando i giudizi, ma LMEM si occupa di controllare sia i verbi che i partecipanti

\subsection{A possible follow-up on recoverability and prototypicality}

testo

% un follow-up a riempimento di spazi per scoprire cosa è che viene omesso quando viene omesso un oggetto! e.g. "John was drinking _", per poi testare la recoverability con PISA + rispondere alle ipotesi del capitolo 2 sulla prototypicality + vedere se ha ragione Fillmore (citato in Cap 2 e Cap Model) ed effettivamente è un senso specifico del verbo (e uno specifico tipo di oggetto) a determinare object-droppability
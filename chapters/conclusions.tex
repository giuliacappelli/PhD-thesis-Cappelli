% \setchapterimage[6.5cm]{seaside}
\setchapterpreamble[u]{\margintoc}
\chapter{Conclusions and further remarks}
\labch{conclusions}

Conclusions

\section{Discussion of results} \labsec{end_discussion}

testo

% Direi però anche qualcosa del ruolo dei nuovi constraint che hai introdotto rispetto a Medina. Così come farei anche qualche commento rispetto al modo di computare la semantic specifity,


\section{Conclusions} \labsec{end_conclusions}

testo


\section{Future directions} \labsec{end_future}

% parlare degli strumenti (consultare notebook_ideas.txt e appunti mPad) + Boland (2005) e gli strumenti come caso ibrido tra argomenti e aggiunti! e forse Resnik (1993)

% lingue tipologicamente diverse da inglese e italiano che marcano l'aspetto (perfectivity) in modo diverso, come il russo

% modellare corpus frequencies invece di grammaticality judgments (cite i.a. Manning 2002: 16), fare riferimento al capitolo "2.2.5.2 Expected Frequency and Relative Grammaticality" di Medina a p. 110 del pdf

% ampliare il set dei constraints (riferimento a hopper thompson 1980! v. capitolo 2)

% non-linear function nel modello! v. Medina p. 110 del pdf, nota 18

% rendere StOT simile a lmem()! io mi limito a minimizzare l'effetto-partecipanti normalizzando i giudizi, ma LMEM si occupa di controllare sia i verbi che i partecipanti

% un follow-up a riempimento di spazi per scoprire cosa è che viene omesso quando viene omesso un oggetto! e.g. "John was drinking _", per poi testare la recoverability con PISA + rispondere alle ipotesi del capitolo 2 sulla prototypicality + vedere se ha ragione Fillmore (citato in Cap 2 e Cap Model) ed effettivamente è un senso specifico del verbo (e uno specifico tipo di oggetto) a determinare object-droppability

% OTHER LANGUAGES
% > Jackendoff (2002, p. 134, fn. 65)
% Many languages, such as Korean and Japanese, are much freer than English in omitting arguments. In such languages there may be no justification for distinguishing
% between obligatorily and optionally expressed semantic arguments.
% [mi interessa soprattutto il punto qui sopra]
% There is often an impulse to conjecture that the obligatoriness of arguments is predictable from the
% semantics. For instance, devour is more specific in its semantics than eat, so perhaps more specific verbs tend to make their arguments obligatory. This conjecture is
% immediately counterexemplified by contrasts like serve/give the food to Sally vs. serve/*give the food, where serving is a more specific form of giving, and insert/put the letter in the slot
% vs. insert/*put the letter, where inserting is a more specific form of putting. I also note the verbs juggle and flirt, which have got to be among the more highly specific verbs in the
% language, yet take an optional syntactic argument: juggle (six balls), flirt (with Kim). These arguments are not optional in semantics: one certainly can't juggle without juggling
% something, nor flirt without flirting with someone. I conclude that the obligatoriness of syntactic arguments must be encoded as an idiosyncratic lexical property.An
% important and little-mentioned distinction among implicit syntactic arguments is whether they are construed as “indefinite” or “definite.” For instance, the implicit argument
% in John is eating is indefinite: it can be paraphrased by John is eating something. By contrast, the implicit argument in I'll show you is definite: there has to be an understood discourse
% argument, and the explicit paraphrase is I'll show it to you. The choice between indefinite and definite implicit arguments seems specific to the verb (or possibly the semantic
% verb class). See e.g. Grimshaw (1979), among others, for discussion of this distinction.

% ------------------------------------------------------
% *Goldberg (2001: 518)*  
% MOLTO IMPORTANTE!!! PROVARE A FARE QUESTO STUDIO DIACRONICO?  
% > Interestingly, the same set of verbs frequently occurs in generic contexts with a
% habitual interpretation: Pat drinks; Pat smokes; Chris sings; Sam bakes. It seems
% likely that the frequent appearance of this usage, which is licensed by the Omission
% under Low Discourse Prominence principle, led to the grammaticalization of a lex-
% ical option for these verbs, whereby they could appear intransitively in less con-
% strained contexts. Corpus and historical work, to determine the frequencies of usage
% and the historical evolution, would be required to determine whether this hypothesis
% is correct.

% *Glass (2020: 5)*
% > if a verb appears frequently in generic or habitual contexts, it is likely to omit its objectin those contexts, a pattern which would diachronically generalize to episodic contexts as well
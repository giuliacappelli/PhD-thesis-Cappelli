% \setchapterstyle{kao}
\setchapterpreamble[u]{\margintoc}
\chapter{Acceptability judgments}
\labch{judgments}

\section{Materials and methods} \labsec{matandmethods}


\subsection{Operative choices} \labsec{participants}
The development of both experiments is organized in three steps, i.e. building, running, and recruiting, with a different platform employed for each step. The merits of the PsychoPy-Pavlovia-Prolific pipeline, described below in full detail, make it a growingly popular choice among behavioral experimenters. Let us examine each step separately.

\paragraph{Building the experiment} I built the experiment using the graphical interface (called "Builder") of PsychoPy v2020.2.10 \parencite{peirce2019psychopy2}. PsychoPy is an open-source, cross-platform software package allowing experimenters to build any kind of experiment in psychology, neuroscience, psychophysics, linguistics, and other behavioral sciences. It makes it possible to code an experiment from scratch using the Coder interface (or any Python-friendly Integrated Development Environment), or to build one using the graphical interface provided by the Builder. To a skilled Python programmer, the main appeal of the Builder lies in that it has a feature to translate the built-in Python functions into JavaScript code that can run online, while Coder-created experiments can only be run locally on the experimenter's computer.

\paragraph{Running the experiment online} In order to be run online on the participants' devices, a PsychoPy Builder experiment has first to be uploaded on Pavlovia, a hosting platform for behavioral experiments coded using PsychoPy, lab.js, or jsPsych. Pavlovia can be fine-tuned to launch experiments online with a variety of options (e.g. launch pilot or full-fledged experiment, save or discard incomplete submissions), and it integrates seamlessly with popular recruiting platforms. Moreover, it doubles up as a source code repository thanks to integration with GitLab.\\
The source code for both my experiments, as well as the stimuli used in them, is available here for \href{https://gitlab.pavlovia.org/gcappelli/grammaticality_eng}{English} and \href{https://gitlab.pavlovia.org/gcappelli/grammaticality_ita}{Italian}. The files can be read directly online, but the code requires PsychoPy to be edited.

\paragraph{Recruiting participants} Both surveys were run on Prolific (formerly known as "Prolific Academic"), a large crowdsourcing platform that was specifically developed to cater to the needs of researchers.\\
The experimental tasks were carried out by 30 people for the study on Italian and 30 people for the study on English. All 60 people are native speakers of Italian and English, untrained in Linguistics, holding at least a Bachelor's degree (in order to minimize the effect of education on their judgments), and lacking any knowledge of the goals of this dissertation.\\
The estimated duration for the experiments was about 30 minutes. Participants who completed the experiment were given €-.-- each as compensation for their effort, in compliance with Prolific's policy calling for ethical rewards.


\subsection{Target verbs} \labsec{verbs}
Both for English and for Italian, the verb dataset used to build the stimuli comprises 30 target transitive verbs and 10 filler intransitive verbs. The reader will find both sets, together with the relevant frequency information, in \refapp{app_verbs}.\\
The optimal verb set used in the creation of the stimuli must be balanced in every relevant aspect. This means that it has to include the same number of English and Italian verbs (leading to the same number of stimuli in both Likert experiments), that the verbs span over different frequency ranges within the same language (i.e. they are not all high-frequency or low-frequency verbs), and that each English verb has a corresponding Italian verb with roughly the same meaning and comparable (relative) frequency in the corpus. This last requirement on verb frequency is crucial, since word frequency is typically confounded with the variables under consideration in (psycho)linguistic experiments\sidenote{See, for instance, \textcite{arunachalam2013experimental, brysbaert2018word} on the word frequency effect and other confounding variables.}.\\
The creation of a verb set for the two Likert experiments was accomplished in two steps, as detailed below.

\paragraph{Creation of verb lists} First of all, a list was made of several transitive English verbs (5 from \textcite{Resnik1993}, 8 from \textcite{Levin1993}, 4 from both papers, and 5 commonly used verbs of my choosing). I further added to the list 12 verbs specified with respect to the manner component (discussed in \nrefsec{predictor_mannspec}), based on my personal opinion as a linguist and person in charge of this project. This resulted in a list containing 30 transitive verbs, reported in \reftab{verbs_engita}.\\
The English verbs in the list have unambiguous meanings (e.g. \textit{to slice, to wash}). I did not include all of Resnik's original set, which featured highly polysemous verbs such as \textit{to have} and \textit{to do}, precisely to comply with this requirement. In order to verify that the 30 verbs are indeed monosemous, quantifying my intuition in a more rigorous way, I used WordNet \parencite{Miller1995} to check that each verb belongs to just one synset (i.e. has just one sense) or, if it belongs to more than one as it often happens, that these synsets are very close together semantically. In order to achieve this result, I computed the Wu-Palmer similarity\sidenote{
	The Wu-Palmer similarity (WP) is a similarity measure based on the depth of the two synsets (s1 and s2) in the taxonomy and the depth of their closest ancestor, i.e. the Least Common Subsumer (LCS).
	\begin{equation*}
  WP = 2 * \frac{depth(LCS(s1,s2))}{depth(s1) + depth(s2)}
	\end{equation*}
	It can vary between 0 and 1 (0 $<$ WP $\leq$ 1), with higher scores corresponding to more similar senses.
} between all the WordNet synsets each verb belongs to. I defined strict criteria for monosemy, namely, a verb is only considered monosemous if no more than 20\% of its senses have a Wu-Palmer similarity score lower than a set threshold. For lack of a standardized threshold in the literature, I set it at 0.15, since it's close to the scores of very different word senses of the notoriously non-monosemous English noun \textit{bank}, a now classic textbook example (0.1428 for 'bank\textsubscript{1}: sloping land' versus 'bank\textsubscript{2}: financial institution', 0.1538 for 'bank\textsubscript{1}: sloping land' versus 'bank\textsubscript{6}: gambling house funds'). All the 30 verbs in the set are acceptable, based on these requirements. It's possible to reproduce the results (or apply the test to novel data) using my Python script, available \href{https://github.com/giuliacappelli/checkPolysemy}{here on GitHub}.\\
Keeping the semantic requirement for monosemy in mind, I created the set of 30 Italian verbs (also listed in \reftab{verbs_engita}) by translating each English verb from the list into Italian, and checked that they are not polysemous using the same criteria applied to the English verb set. For each verb in the English set, the corresponding Italian verb is the first translation found in the WordReference English-Italian Dictionary \copyright 2020. I had to choose the second translation by WordReference just for \textit{to chop} (because the first translation, \textit{tagliare}, suits best the verb \textit{to cut}) and for \textit{to swig} (because the first translation, \textit{tracannare}, features in the itWaC corpus only 32 times, while \textit{trangugiare} is found 647 times).\\
Based on the information discussed in \nrefch{predictors}, I computed the PISA scores for each verb in both lists, and I annotated each verb pair with their telicity and manner specification features (the full details are collected in \refapp{app_predictors}).\\

\begin{table}[htb] % the "htb" makes table env unfloaty
\caption{The sets of English and Italian verbs of interest.}
\labtab{verbs_engita}
\begin{tabular}{ll}
English verbs & Italian verbs \\
\hline
behead	&	decapitare	\\
break	&	rompere	\\
build	&	costruire\\	
chop	&	spaccare	\\
clean	&	pulire	\\
cook	&	cucinare	\\
cut	&	tagliare	\\
devour	&	divorare	\\
doodle	&	scarabocchiare\\	
drink	&	bere	\\
eat	&	mangiare	\\
embroider	&	ricamare	\\
hum	&	canticchiare	\\
kill	&	uccidere	\\
knife	&	accoltellare	\\
poison	&	avvelenare	\\
polish	&	lucidare	\\
pour	&	versare	\\
sew	&	cucire	\\
sign	&	firmare	\\
sing	&	cantare	\\
sip	&	sorseggiare	\\
slice	&	affettare\\	
smoke	&	fumare	\\
steal	&	rubare	\\
swig	&	trangugiare	\\
teach	&	insegnare	\\
wash	&	lavare	\\
watch	&	guardare	\\
write	&	scrivere	       
\end{tabular}
\end{table}

\paragraph{Frequency check} The second step in the creation of the verb set is a "sanity check" of the verb frequencies, both within-language and between-language, as detailed in this paragraph. The (absolute) frequency of each verb is extracted from ukWaC for English and itWaC for Italian \parencite{baroni2009wacky}.\\
Absolute frequencies have to be transformed in order to be compared, since they are corpus-dependent, and also because words occur in a corpus according to the power law known as "Zipf's law".
\sidenote{
	According to Zipf's law, The $r$th most frequent word has a frequency $f(r)$ that scales according to:
	\begin{equation*}
  f(r) \propto \frac{1}{r^\alpha}
	\end{equation*}
for $\alpha \approx 1$. Based on this law, the distribution of word frequencies w.r.t. the word rank $r$ is a logarithmic distribution (crucially, not a linear one).
}
Computing the relative frequency, or the frequency per million words, would solve the first problem but not the second. Log-transforming either of these would solve both problems, but low-frequency words would yield negative logarithms, so that the scale would not be easily human-readable and it would be quite difficult to use for the purposes of this experiment.\\
In order to overcome these problems, I used the "Zipf scale" by \textcite{van2014subtlex}, i.e. a logarithmic scale going from 1 (very-low-frequency words) to 7 (very-high-frequency words), much like a Likert scale. The human (or automatic) interpretation of values on the Zipf scale is straightforward, and it does not vary across corpora. The middle of the scale, i.e. 4, is the tipping point between low- and high-frequency words, and words with a Zipf score higher than 6 are very likely to be function words (i.e. non-content words, such as determiners and auxiliaries, often called "stop words" in computational literature).\\
Knowing the absolute frequency of a verb in a corpus ($f$) and the corpus size in tokens ($c$), Zipf scores ($Z$) are easy to compute as in \refeq{zipf_orig} or, equivalently, as in \refeq{zipf_short} (both are the base 10 logarithm of the frequency-per-billion-words).

\begin{equation} \labeq{zipf_orig}
Z = \log _{10} \frac{f*1,000,000,000}{c}
\end{equation}

\begin{equation} \labeq{zipf_short}
Z = \log _{10} \frac{f}{c} + 9
\end{equation}

Having chosen a suitable scale to compare my verb frequencies, I performed the within- and between-language tests. As for the within-language check, I verified that the Zipf scores of the verbs for both languages were compatible with a distribution spanning from low- to high-frequency verbs, avoiding extremes (English verbs 2.078 $\div$ 5.520, Italian verbs 1.681 $\div$ 5.763). Lastly, I made sure that the same verb (modulo the translation into Italian) belonged in the same Zipf score tier in both languages by computing the difference between the English and the Italian Zipf scores. Since the Zipf scale is designed so that words in each of the 7 tiers have significantly different corpus frequencies, I take an English-Italian Zipf difference to be acceptable if smaller than 1, while I would reject any English-Italian verb pairs with a the difference equal to or greater than 1. The within- and between-language frequency tests showed that the two verb sets comply with the above requirements. The reader will find the frequencies and Zipf scores in \refapp{app_verbs}.

\subsection{Design} \labsec{design}

Both experiments follow a within-subject fully crossed design, where every participant sees all the stimuli in random order and provides judgments for each on a 7-point Likert scale.\\
The stimuli are organized in a 2x2x2 factorial design, summarized in \reftab{mydesign}, with 3 independent variables (presence of an overt direct object, perfectivity, iterativity) having 2 levels each (presence or absence of the feature). This experimental design is more complex than the 2x2 design by \textcite{Medina2007}, since I am adding iterativity as an independent variable.

\begin{table}[htb] % the "htb" makes table env unfloaty
\caption{The 2x2x2 factorial design used in both Likert experiments.}
\labtab{mydesign}
\begin{tabular}{ccc}
overt dObj & perfectivity & iterativity \\
\hline
+          & +            & +           \\
+          & +            & -           \\
+          & -            & +           \\
+          & -            & -           \\
-          & +            & +           \\
-          & +            & -           \\
-          & -            & +           \\
-          & -            & -          
\end{tabular}
\end{table}

Each verb of interest (fully listed in \refapp{app_verbs}) participates in each of the 8 experimental conditions. Since telicity, semantic recoverability and manner specification are inherent properties of each verb, they are not part of the experimental design.

\subsection{Stimuli} \labsec{stimuli}

All the verbs of interest plus 10 intransitive filler verbs participate in all the experimental conditions, leading to a total of 320 sentence stimuli (twice as in \textcite{Medina2007}) for each language in the study.\\
The list of English and Italian intransitive filler verbs is provided in \reftab{fillers_engita}. Since these verbs are part of the experimental design but irrelevant for the subsequent analysis, they were not controlled for frequency, nor were they annotated with semantic or aspectual information.

\begin{table}[htb] % the "htb" makes table env unfloaty
\caption{The sets of English and Italian filler verbs.}
\labtab{fillers_engita}
\begin{tabular}{ll}
English fillers & Italian fillers \\
\hline
clap	&	applaudire	\\
fast	&	digiunare	\\
knock	&	bussare	\\
laugh	&	ridere	\\
limp	&	zoppicare	\\
rest	&	riposarsi\\
scream	&	urlare	\\
sleep	&	dormire	\\
smile	&	sorridere	\\
stagger	&	barcollare       
\end{tabular}
\end{table}

As mentioned in \nrefch{predictors}, semantic recoverability, telicity, and manner specification vary \textit{across verbs}, while the presence of an overt direct object, perfectivity, and iterativity vary \textit{across sentences}. This means that the two groups of object drop predictors are treated in different ways. Recoverability, telicity, and manner specification are intrinsic characteristics of each verb, respectively continuous the first and binary the last two (refer to \refapp{app_predictors} for the complete set of verbs with their features). On the contrary, the presence of a direct object, perfectivity, and iterativity are binary features that need to be encoded by creating a pair of minimally different sentences for each.\\
Let us consider the eight\sidenote{based on the design in \refsec{design}} example stimuli for the verb \textit{to eat} in \ref{eat_simulation}:
\ex. \label{eat_simulation} \a. John had eaten pizza again. \hfill {\small [dObj+, perf+, iter+]}
\b. John had eaten pizza. \hfill {\small [dObj+, perf+, iter-]}
\c. John was eating pizza again. \hfill {\small [dObj+, perf-, iter+]}
\d. John was eating pizza. \hfill {\small [dObj+, perf-, iter-]}
\e. John had eaten again. \hfill {\small [dObj-, perf+, iter+]}
\e. John had eaten. \hfill {\small [dObj-, perf+, iter-]}
\e. John was eating again. \hfill {\small [dObj-, perf-, iter+]}
\e. John was eating. \hfill {\small [dObj-, perf-, iter-]}

All the 30 verbs of interest and the 10 filler verbs, both for English and Italian, will feature in stimuli like the ones listed in \ref{eat_simulation}. In transitive sentences, regardless of the verb being transitive or intransitive in nature, the direct object is semantically compatible with the meaning of the verb so that the violation of selectional preferences do not act as a confound in the experiment. The reader will find the full list of stimuli and verb-object pairings in \refapp{app_stimuli}, and detailed information on the manipulation of perfectivity and iterativity in \nrefsec{predictor_perfectivity}.


\subsection{Setting} \labsec{setting}
 
 \paragraph{Informed consent} First of all, the participants had to accept the privacy policy in order to proceed to the survey, or else decline the terms and exit the experiment. Below you will find the English and the Italian versions of the privacy policy.

\begin{kaobox}[frametitle=Privacy policy for the English survey]
During this survey, you will not be asked personal information. In no way will it be possible for anyone to find out your identity by having access to your answers to the survey. The data you will provide will be used for the purposes of this linguistic experiment and they may be shared with third parties anonymously. By completing the survey, you accept these terms. If you leave the survey early, your data will not be used in the study and you will not be compensated.
\end{kaobox}

\begin{kaobox}[frametitle=Privacy policy for the Italian survey]
Nel questionario non ti saranno chieste informazioni personali. Non sarà possibile per nessuno risalire alla tua identità a partire dalle risposte che darai nel questionario. I dati che fornirai saranno usati ai fini di questo esperimento linguistico e potranno essere condivisi con terzi in forma anonima. Completando il questionario, dichiari di accettare questi termini. Se abbandoni il questionario in anticipo, i tuoi dati non saranno usati nello studio e non riceverai alcun compenso.
\end{kaobox}

\paragraph{Instructions} Then, participants were given instructions in their native language (i.e. the target language of the survey). The full text is available right below for both English and Italian.

\begin{kaobox}[frametitle=Instructions for the English survey]
This survey takes about 30 minutes to complete, and you will be rewarded € -.-- as compensation if you complete the survey.\\
You will see a series of sentences, one by one. For each, you are asked to judge how acceptable it is to you on a graded scale. You should rate a sentence 1 if it sounds utterly bad, 7 if it sounds perfectly fine, or choose any in-between score if you think it applies. Let us consider some examples:
\begin{itemize}
    \item John laughs stories.
\end{itemize}
This sentence should score 1, because you can't "laugh something".
\begin{itemize}
    \item Mario walked on the path.
\end{itemize}
This sentence should score 7, because it's perfectly acceptable.\\
This is not an exam! By virtue of being a native speaker of English, you will provide the right answers. Beware: to avoid cheating and random clicking, the survey is interspersed with hidden control questions. If you fail them, you will be kicked out of the survey and receive no compensation.
\end{kaobox}

\begin{kaobox}[frametitle=Instructions for the Italian survey]
Il questionario richiede - minuti in media per essere completato e riceverai -.--€ come compenso se lo completerai tutto.\\
Vedrai una serie di frasi, una alla volta. Per ciascuna, devi giudicare quanto ti sembra accettabile in una scala di valori. Dovresti dare a una frase il punteggio 1 se ti sembra del tutto sbagliata, 7 se ti suona del tutto corretta, o scegliere punteggi intermedi se ti sembra il caso. Guardiamo alcuni esempi:
\begin{itemize}
    \item Gianni ride storie.
\end{itemize}
A questa frase dovresti dare 1, perché non si può "ridere qualcosa".
\begin{itemize}
    \item Mario camminava sul sentiero.
\end{itemize}
A questa frase dovresti dare 7, perché è perfettamente accettabile.\\
Questo non è un esame! Le tue risposte sono tutte giuste, perché sei un parlante nativo di italiano. Attenzione, però: per impedire che vengano date risposte a caso, il questionario contiene domande di controllo nascoste. Se le sbaglierai, ti sarà impedito di continuare a rispondere e non riceverai alcun compenso.
\end{kaobox}

\paragraph{Screening survey} Finally, the participants were asked to complete a short screening survey before entering the actual linguistic judgment survey. The screening questions are presented in \ref{screening_eng} for English and in \ref{screening_ita} for Italian:
\ex. \label{screening_eng} \a. Are you a native speaker of English?
\b. Have you got a Bachelor's (or higher) degree?
\c. Have you understood the instructions above?

\ex. \label{screening_ita} \a. Sei un parlante nativo di italiano?
\b. Hai una laurea triennale (o titolo superiore)?
\c. Hai capito le istruzioni presentate sopra?

Participants could click on either "Yes" or "No" buttons to answer. Answering "No" to any screening questions meant being automatically kicked out of the survey.

\paragraph{Training session} Before entering the actual experimental session, participants had the chance to accustom themselves to the task in a short training session. They were asked to judge the acceptability of each sentence on a 7-point Likert scale, as in the full experimental session. Likert scales are a reliable method to test grammaticality \parencite{WeskottFanselow2011}, and the 7-point variant (unlike the 5-point scale used by \textcite{Medina2007}) is the most common in experimental linguistics \parencite{Juzek2016}.\\ In order to keep the training session as short as possible while maximizing usefulness, subjects only judged a fully-grammatical sentence and a fully-ungrammatical sentence (in \ref{training_eng} for the English survey and \ref{training_ita} for the Italian survey).

\ex. \label{training_eng} \a. \label{training_eng_yes} Jack had opened a bar.
\b. \label{training_eng_no} *Ann went a sandwich.

\ex. \label{training_ita} \a. \label{training_ita_yes} Sergio ha aperto un bar.
\b. \label{training_ita_no} *Anna è andata un panino.

Unlike the real experimental session, the training session allowed participants to keep rating a sentence as many times as they wanted, instead of kicking them out in the case of a mistake. After each judgment on the Likert scale, participants received immediate feedback in the same interface window. They were either prompted to provide a different judgment on the same sentence, if the score they chose was off-scale, or they were asked to continue to the next task, if they rated the sentence correctly. Expected scores in the training session were quite strict, i.e. no less than 6 for \ref{training_eng_yes} and \ref{training_ita_yes}, and no more than 2 for \ref{training_eng_no} and \ref{training_ita_no}.

\paragraph{Experimental session} The 320 stimuli were presented in randomized order, since order is well-known to have an effect on acceptability judgments \parencite{Myers2009, Juzek2016}. In general, randomizing stimuli or counterbalancing conditions are good experimental practice to counteract the effects of carryover, fatigue, and practice. Moreover, the stimuli were presented one by one, in order to prevent participants to compare the stimuli one to another instead of judging them individually. The concern for similar task-specific strategies and the need to avoid eliciting them is raised, among others, by \textcite{Myers2009}.\\
Participants were instructed to click on their chosen score and then press their spacebar to proceed to the next stimulus, so they could change their mind before submitting their judgment for good. Naturally, the experiment was programmed as to prevent participants from skipping stimuli.

\paragraph{Reliability of judgments} The reliability of the collected judgments was ensured in two different ways. On a general note, paid compensation is standard practice in behavioral studies where attentiveness is key, as well as being a popular way to make up for the time participants invested in the experiment\sidenote{Please refer to \textcite{PERMUTHWEY2009280} for an extensive debate on the ethical and practical implications of financial remuneration in behavioral research.}.\\
A more task-specific method to elicit reliable judgments involves using the clearly ungrammatical and the clearly grammatical control sentences as monitoring stimuli. This means that ungrammatical control sentences like \ref{filler_sentence}, where an intransitive filler verb appears with a direct object, should get a very low score on the 7-point Likert scale.
\ex. \label{filler_sentence} * John had slept pillows.

Likewise, grammatical control sentences like \ref{control_sentence_trans}, where a transitive verb appears with a semantically compatible direct object, or one like \ref{control_sentence_intrans}, where an intransitive verb is used intransitively, should get a very high score on the 7-point Likert scale.
\ex. \a. \label{control_sentence_trans} John was eating pizza.
\b. \label{control_sentence_intrans} John was limping.

Summing up, based on the experimental design previously depicted in \reftab{mydesign}, the stimuli are divided into target stimuli and (un)grammatical control stimuli as in \reftab{kinds}.

\begin{table}[htb] % the "htb" makes table env unfloaty
\caption{Summary of which stimuli are targets and which ones are (un)grammatical controls in both experiments.}
\labtab{kinds}
\begin{tabular}{ccc|cc}
overt dObj & perfectivity & iterativity & trans. verbs & intrans. verbs \\
\hline
+          & +            & + & / & *control           \\
+          & +            & - & control & *control         \\
+          & -            & + & / & *control           \\
+          & -            & - & control & *control        \\
\hline
-          & +            & + & target & /          \\
-          & +            & - & target & control          \\
-          & -            & + & target & /          \\
-          & -            & - & target & control          
\end{tabular}
\end{table}

Since getting a single judgment wrong out of 320 would cost a participant his reward (and, incidentally, cost this study valuable data), the requirements for a judgment to be deemed correct were softened with respect to those used in the training session. Thus, participants who provided a score higher than 3 (i.e. not in the lower half of the scale) to filler sentences or lower than 5 (i.e. not in the higher half of the scale) to control sentences were kicked out of the experiment and did not receive any compensation, since out-of-range scores would mean that they were not paying enough attention to the task or was downright clicking at random.\\ Notably, iterative sentences with transitive verbs used transitively and intransitive verbs used intransitively were not included among the control stimuli, because they are not prototypical examples of perfectly grammatical sentences. It would have been frustrating for participants to be excluded from the experiment (and the reward) for a single mistake on such a sentence, especially considering that half the stimuli already were control sentences.\\
The results of the English and Italian experiments are presented in full detail in \nrefch{results}.

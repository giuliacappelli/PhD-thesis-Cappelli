\setchapterpreamble[u]{\margintoc}
\chapter{Introduction}
\labch{intro}

\section{Overview} \labsec{intro_intro}

 valutare se fare una breve intro e poi il primo capitolo teorico sull'object drop, oppure incorporare quel capitolo direttamente nell'intro

\subsection{Theoretical relevance}
what is indefinite object drop and why is it relevant?, titolo di Keller

\subsection{State of the art}
letteratura teorica, letteratura sperimentale, riferimenti interni a dove ne parlo nel dettaglio

\subsection{Central claims}
cambiare titolo, l'ho preso da Keller! qui devo dire cosa voglio dimostrare nella tesi e fare riferimento a come l'ho dimostrato nel testo


\section{Contents, materials and related work} \labsec{intro_contents}

\subsection{Chapters}
descrizione dettagliata di capitoli e sezioni

\subsection{Supporting materials}
script e dati miei su GitHub!

\subsection{LAVORI MIEI}
Keller ha una sezione "Collaborations and published work", io qui metterei Computational PISA (e le tre conferenze), la conferenza di Venezia, altro dello stesso tipo



% \section{Citations} % \textcite = \citet, \parencite = \citep

% \index{citations}
% To cite someone \cite{Visscher2008,James2013}, \textcite{Visscher2008,James2013}, \parencite{Visscher2008,James2013} is very simple: just 
% use the \Command{sidecite}\index{\Command{sidecite}} command. It does 
% not have an offset argument yet, but it probably will in the future. 
% This command supports multiple entries, as you can see, and by default 
% it prints the reference on the margin as well as adding it to the 
% bibliography at the end of the document. Note that the citations have 
% nothing to do with the text, \sidecite{James2013} but they are completely 
% random as they only serve the purpose to illustrate the feature.
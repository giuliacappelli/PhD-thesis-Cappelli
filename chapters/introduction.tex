\setchapterpreamble[u]{\margintoc}
\chapter{Introduction}
\labch{intro}

\section{Overview} \labsec{intro_intro}

\subsection{Relevance of this thesis}

This thesis is about the omission of direct objects from predicates headed by verbs with two semantic participants, i.e., an Agent (in the syntactic subject position) and a Patient (in the syntactic object position). These "optionally transitive" verbs, deviating from the transitive prototype defined by \textcite{HopperThompson1980}, appear in a wide variety of contexts cross-linguistically, and are licensed by different semantic, aspectual, pragmatic, and discourse factors. Within this broad area of interest, I will focus on \textit{indefinite} null objects, corresponding to what \textcite{Fillmore1986} called "indefinite null complements". These omitted objects, as shown in \ref{introintro1}, refer to something that is "unknown or a matter of indifference" \parencite[96]{Fillmore1986}. Indeed, what matters in the example is that Giulia is writing something, and in particular, something that is usually written. The actual product of the writing event, be it a novel, a shopping list, or a doctoral dissertation, is irrelevant. On the contrary, the referent of \textit{definite} null objects (which I am not studying in this thesis) "must be retrieved from something given in the context", as in \ref{introintro2}. In this case, the context is provided in the first sentence in the example, where a reference is made to grad school. Thus, the omitted object in the second sentence can be understood to refer to a doctoral thesis (the thing one wants to defend soon when in grad school) rather than, say, the title of Olympic champion or the national borders.

\ex. \label{introintro} \a. \label{introintro1} Giulia is writing $\varnothing$\textsubscript{dObj}.
\b. \label{introintro2} Grad school is hard. Giulia hopes to defend $\varnothing$\textsubscript{dObj} soon.

The available literature suggests indefinite object drop to be possible with different types of verbs to varying extents (for instance, change-of-state verbs such as \textit{to kill} are much more resistant to object drop than incremental-theme verbs such as \textit{to eat}), and for any given verb, to be more likely under some specific semantic, aspectual, and pragmatic circumstances. For instance, a direct object can only participate in the indefinite implicit object construction if it is recoverable from the meaning of the verb itself, and transitive verbs are much more likely to be used without a direct object when they are used in imperfective or iterative contexts than in perfective, single-occurrence contexts.\\
While many pages have been written about the role of several linguistic factors in facilitating or blocking indefinite object drop, as I will detail in the first section of this thesis, way fewer attempts have been made to understand the nature of this phenomenon via experimental means, modeling the joint effect of several predictors of object drop. \textcite{Medina2007} made a substantial step in this direction in her (linear) Stochastic Optimality Theoretic model of indefinite object drop in English, taking into consideration the joint effect of object recoverability, telicity, and perfectivity on the grammaticality of indefinite null objects (as gauged via gradient acceptability judgments elicited from native speakers) occurring with 30 transitive verbs. This model shows that:
\begin{itemize}
    \item indefinite object drop is a gradient, non-categorical phenomenon;
    \item it is possible with virtually any transitive verb, but in different degrees depending on the verb semantics;
    \item for any given verb, different aspectual features may favor or hinder object drop.
\end{itemize}

In the experiments I will perform to study the indefinite implicit object construction, I inherit Medina's methodology and employ the same variant of Stochastic Optimality Theory she devised, with several updates I will illustrate in \refsec{goalsandnovelty} and, in more detail, in the experimental section of this thesis.


\subsection{Main goals and elements of novelty} \labsec{goalsandnovelty}

This thesis is meant as an expansion upon the original model of indefinite object drop designed by \textcite{Medina2007}. I will collect acceptability judgments following her same experimental design and model the data thus collected within the bounds of the same framework (her variant of Stochastic Optimality Theory). In doing so, I add several elements of novelty to the study:
\begin{itemize}
    \item I will model indefinite implicit objects both in English (like Medina did) and in Italian (which is included in such a probabilistic model for the first time), analyzing language-specific differences in the way several factors facilitate object drop;
    \item in addition to using \posscite{Resnik1993} Selectional Preference Strength measure to quantify semantic selectivity (as a proxy to object recoverability), following \textcite{Medina2007}, I will also define a novel computational measure based on distributional semantics (Computational PISA, presented in \textcite{CappelliLenciPISA}) and a behavioral measure (Behavioral PISA) meant to improve on Medina's Object Similarity;
    \item in addition to the three predictors included by Medina in her model (semantic selectivity, telicity, and perfectivity), I will also add iterativity and manner specification as predictors to find out how they affect indefinite object drop and whether a more complex, five-predictor model actually provides a more accurate view on this phenomenon than the original three-predictor model;
    \item in order to make it easier for future research to build on my results (possibly applying my method to other languages, or to the same languages with different predictors) or to replicate them, I intend to share my materials and document my methods (as well as my Python scripts), as detailed in \refsec{supportingmaterials}.
\end{itemize}

Assuming that my probabilistic models of the gradient grammaticality of indefinite object drop are solid, this thesis will be an additional cobblestone on the well-trodden road of studies about indefinite implicit objects, omitted arguments, and transitivity-related phenomena. More in general, it will add to the understanding of the ways several linguistic factors give rise to phenomena that, despite appearing to happen arbitrarily on a lexically-determined basis, are quite systematic in their behavior. Thus, looking at this thesis from a much broader perspective, it can also be argued to be a contribution to research about the systematicity (i.e., rule-abiding behavior) of human language and cognition, and about the interaction of semantic, aspectual, world-knowledge, pragmatic, and discourse factors in determining the way we translate our communicative intentions into syntactically well-formed utterances.

%  small, hopefully not insignificant, contribution

\section{Contents within and without} \labsec{intro_contents}

\subsection{Chapters of the thesis and their structure}
This thesis is divided into two main parts, one devoted to the review of the literature on object drop, Optimality Theory, and gradient models of indefinite null objects (from \refch{objectdrop} to \refch{medina}), and another devoted to my own experiments and the results thereof (from \refch{predictors} to \refch{model}). Let us consider each Chapter in more detail.

\paragraph{Theory and literature review}
In \nrefch{objectdrop} I will define the \textit{indefinite} implicit object construction as a deviation from the transitive prototype (see \refsec{theory_transitivity}) and in contrast with \textit{definite} object drop (see \refsec{theory_def_vs_indef}), based on the literature. In \refsec{theory_defindefinite} I will argue that there is virtually no reason why a transitive verb should not be able to participate in the indefinite implicit object construction (provided favorable aspectual, semantic, and discourse conditions), and that the implicit object is understood to be the prototypical Patient for a given sense of a given verb. In \refsec{theory_entries} I will argue that optionally transitive verbs should only be taken to have a single entry in the lexicon, realized syntactically either with an overt or with an implicit object, rather than having two separate lexical entries. In \refsec{theory_workingdef} I will provide the perspective on indefinite object drop I adopt in my experiments and throughout this thesis.\\
In \nrefch{factors} I will discuss the role played by semantic factors (recoverability, Agent affectedness, and manner specification, in \refsec{semanticfactors}), aspectual factors (telicity and perfectivity, in \refsec{aspectualfactors}), and pragmatic factors (routine\sidenote{In the sense intended by \textcite{Glass2013, Glass2020, glass2022english}.}, iterativity, habituality, and discourse factors, in \refsec{pragmaticfactors}) in facilitating or blocking object drop with optionally transitive verbs. After some considerations in \refsec{frequencyfail} on the reasons why corpus frequency is not included among the relevant factors, I conclude the Chapter in \refsec{factorsofchoice} with the reasoning behind my choice of predictors of object drop to be used in the experimental section of this thesis.\\
\nrefch{modeltheory} will explain the main tenets of Optimality Theory relative to syntax (in \refsec{classicot}), the limits of standard Optimality Theory as a model of the indefinite implicit object construction, and, therefore, why it is advisable to resort to probabilistic models of grammar that are able to account for the gradient grammaticality shown by indefinite object drop, such as Stochastic Optimality Theory (as argued in \refsec{weightedot}).\\
In particular, in this thesis I will adopt the variant of Stochastic Optimality Theory specifically designed by \textcite{Medina2007} to model indefinite object drop, which I describe and discuss in \nrefch{medina}. The contents of the input and the output will be presented in \refsec{inputmedina}. In \refsec{predictorsmedina} I will discuss the implementation of the three predictors the author used in her model (semantic selectivity, telicity, and perfectivity). The probabilistic ranking of the constraints, which I will introduce in \refsec{constraintsmedina}, will be defined in a top-down perspective (from constraint ranking as a function of semantic selectivity to object-drop probability as gradient grammaticality) in \refsec{rankingmedina}, and finally implemented in a bottom-up perspective (from the acceptability judgments to the estimation of parameters of the linear functions) in \refsec{medinacomputation}.

% the reasons why standard Optimality Theory would be a bad fit for a model of the indefinite implicit object construction

\paragraph{Experiments and results}
\nrefch{predictors} opens the experimental part of this thesis. I will present five facilitating factors (a continuous factor and four binary factors) of object drop I will use as predictors in my Stochastic Optimality Theoretic model, picked among the ones introduced in \refch{factors}. The continuous factor is object recoverability, which I will model via three different measures of semantic selectivity described in \refsec{predictor_sps}, namely \posscite{Resnik1993} Selectional Preference Strength (following \textcite{Medina2007}), Computational PISA (a novel measure based on distributional semantics I contributed to define in \textcite{CappelliLenciPISA}), and Behavioral PISA (a similarity-based measure inspired by Computational PISA and Medina's Object Similarity measure). The four binary factors are telicity in \refsec{predictor_telicity}, perfectivity in \refsec{predictor_perfectivity}, iterativity in \refsec{predictor_iterativity}, and manner specification in \refsec{predictor_mannspec}.\\
In \nrefch{judgments} I will present the materials and methods employed in the behavioral experiments to collect acceptability judgments from native speakers of English and Italian relative to the indefinite implicit object construction. In particular, in \refsec{participants} I will describe how I built the experiment with PsychoPy, how I ran it on Pavlovia, and how I recruited the participants via Prolific. Finally, I will present my 30-verb target dataset in \refsec{verbs}, the experimental design in \refsec{design}, the stimuli in \refsec{stimuli}, and the experimental setting in \refsec{setting}.\\
A first analysis of the data collected with these behavioral experiments will be provided in \nrefch{results}, where I will describe the structure of the Python script I devised to perform the analysis and to compute the models, as well as the procedures of data preprocessing employed (see \refsec{likert_scripts}), before discussing the separate and joint effects of semantic selectivity and the four binary predictors on the acceptability judgments in English and in Italian (see \refsec{eng_judgresult} and \refsec{ita_judgresult}, respectively). In \refsec{sumup_judgresult}, I will argue that the five factors facilitating indefinite object drop are able to predict, to a non-negligible extent, the likelihood a transitive verb will appear without an overt object in a statistical (linear mixed-effects) model, and I will also explain why Medina's variant of Stochastic Optimality Theoretic is a more linguistically-motivated way of modeling these results than the linguistically-naive statistical model.\\
I will define and discuss my Stochastic Optimality Theoretic models of indefinite object drop in English and Italian in \nrefch{model}. In \refsec{introfitting}, I will describe and evaluate my 18 models, stemming from the union of three measures of semantic selectivity, three increasingly more complex constraint sets (Medina's basic set, another with the addition of iterativity, and a full set with manner specification too), and two target languages. In \refsec{stot_full}, I will discuss the theoretical aspects and computational implementation of the two full models of object drop in English and Italian where semantic selectivity is modeled via Behavioral PISA. I will then compare my models with Medina's model and with regression models in \refsec{stot_conclusions}.\\
Finally, I will provide my conclusions and propose some possibile future directions for research about modeling the indefinite implicit object construction in \nrefch{conclusions}.


\subsection{Supporting materials} \labsec{supportingmaterials}
With an eye to the Open Science environment, I used open source software and programming languages to collect and analyse data for this thesis whenever possible, and I am sharing my data, scripts and results on GitHub. %Should anyone in the future read these pages and find themselves interested in replicating my results, or testing new models on the data I collected, or contributing to enrich my repositories, they will be able to do so \href{https://github.com/giuliacappelli}{on my GitHub profile}\sidenote{https://github.com/giuliacappelli}.\\ %  effortlessly (and without spending money on proprietary software)
The interested reader will find my data, i.e., the stimuli for each experiment and the raw results I got from participants, \href{https://github.com/giuliacappelli/dissertationData}{in a dedicated GitHub repository}\sidenote{https://github.com/giuliacappelli/ dissertationData}. In more detail, this repository contains:
\begin{itemize}
    \item 30 target verbs and 10 filler verbs both for English and for Italian, used in all the computational (see \refsec{predictor_sps}) and behavioral (see \refsec{behavPisa} and \refch{judgments}) experiments, as in \refapp{app_verbs};
    \item full lists of the direct objects of each target verb as extracted from the ukWaC corpus for English and from itWaC for Italian, both raw and manually cleaned (as detailed in \refsec{compuPisa});
    \item stimuli, full judgments elicited from 25 participants per language on a 7-point Likert scale, and final scores obtained in the Behavioral PISA experiment (see \refsec{behavPisa}), also provided in \refapp{app_behavPisa};
    \item each verb tagged with its features relative to the verb-specific predictors of object drop, i.e., telicity, manner specification, and semantic selectivity, as in \refapp{app_predictors};
    \item stimuli and full judgments elicited from 30 participants per language on a 7-point Likert scale in the main behavioral experiment of this thesis (see \refch{judgments}), aimed towards creating a Stochastic Optimality Theoretic model of object drop in English and Italian (see \refch{model}), as in \refapp{app_stimuli}.
\end{itemize}
As for the data processing, analysis of results, computational implementation of experimental designs, and creation of stimuli, I coded several Python scripts and documented their usage on GitHub. In detail, they are as follows:
\begin{itemize}    
    \item \href{https://github.com/giuliacappelli/checkPolysemy}{Quantify the polysemy of words in a list}\sidenote{https://github.com/giuliacappelli/ checkPolysemy} using WordNet (Wu-Palmer Similarity), as in \refsec{verbs};
    \item \href{https://github.com/giuliacappelli/behavioralPISA}{Behavioral PISA}\sidenote{https://github.com/giuliacappelli/ behavioralPISA}, a (behavioral) measure of Preference In Selection of Arguments to model verb argument recoverability, as in \refsec{behavPisa}. The script takes care both of creating the stimuli for the experiment and of generating Behavioral PISA scores based on the Likert-scale acceptability judgments provided by human participants;
    \item \href{https://github.com/giuliacappelli/psychopy_exps}{PsychoPy Builder source code}\sidenote{https://github.com/giuliacappelli/ psychopy\_exps} of my behavioral experiments to collect acceptability judgments relative to the indefinite implicit object construction from native speakers of English and Italian, described in \refch{judgments};
    \item \href{https://github.com/giuliacappelli/PsychopyToMedina}{Psychopy-to-Medina converter}\sidenote{https://github.com/giuliacappelli/ PsychopyToMedina}, to convert the output of my PsychoPy behavioral experiment (see \refch{judgments}) into a suitable input for my scripts to analyse the results (see \refch{results}) and create Stochastic OT models of the implicit object construction following \textcite{Medina2007} (see \refch{model});
    \item \href{https://github.com/giuliacappelli/MedinaStochasticOptimalityTheory}{Modeling the grammaticality of implicit objects}\sidenote{https://github.com/giuliacappelli/ MedinaStochasticOptimalityTheory} based on Medina (2007)'s variant of Stochastic Optimality Theory, as in \refch{results} and \refch{model};
    \item \href{https://github.com/giuliacappelli/generateMockLikertGrammaticalityJudgments}{Generate mock Likert-scale acceptability judgments}\sidenote{https://github.com/giuliacappelli/ generateMockLikertGrammaticalityJudgments} based on factor levels specified in the input, to test the above Stochastic Optimality Theoretic model on ideal data before running the experiment proper.
\end{itemize}

\subsection{Published work and outreach}
Relevant parts of the experimental section of this thesis have been shared with the scientific community, both in written form and during conferences. The original distributional measure of Preference In Selection of Arguments (Computational PISA) presented in \textcite{CappelliLenciPISA} and discussed here in \refsec{compuPisa}, tested on large sets of transitive verbs and Instrument verbs in English, was presented at:
\begin{itemize}
    \item *SEM 2020, 9th Joint Conference on Lexical and Computational Semantics, December 12-13th 2020, online due to the Covid-19 pandemic (originally in Barcelona, Spain);
    \item LSA 2021, 95th Annual Meeting of the Linguistic Society of America, January 7-10th 2021, online due to the Covid-19 pandemic (originally in San Francisco, California);
    \item CLiC-it 2020, 7th Italian Conference on Computational Linguistics, March 1-3rd 2021, online due to the Covid-19 pandemic (originally in Bologna, Italy).
\end{itemize}
The results of the main behavioral experiment of this thesis (detailed in \refch{judgments} and \refch{results}), especially the ones pertaining to Italian, were presented at:
\begin{itemize}
    \item SyntOp 2022, Syntactic Optionality in Italian, July 4-5th 2022, Venice (Italy).
\end{itemize}
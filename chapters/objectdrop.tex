\setchapterpreamble[u]{\margintoc}
\chapter{Object-dropping verbs}
\labch{objectdrop}

Summary

% *Cummins & Roberge (2005: 47)*  
% > Based on work by Roberge
% (2002, to appear), we hold that all null objects are syntactically represented.2
% Roberge proposes a Transitivity Requirement (TR), parallel to the Extended
% Projection Principle (EPP) for subjects, whereby the direct-object position,
% complement of the verb, is given by Universal Grammar.

\section{Definite \textit{vs} indefinite object drop} \labsec{theory_mythesis}

\subsection{Definite or indefinite: discrete accounts} \labsec{theory_discrete}

Definite or indefinite: discrete accounts -- Fillmore (1969)’s seminal work

% *Liu (2008: 296)*  
% > As Rutherford (1998:191) correctly points out, “Verbs
% exhibiting surface characteristics of intransitivity, however, are a mixed bag and can
% be distinguished syntactically and/or semantically.”

%  *Liu (2008: 289)*
%  tutto il paper riguarda questa distinzione in gruppi per transitività  
%  > verbs used without an object into four categories: 1) pure intransitive verbs, such as
% arrive, rise, and sleep; 2) ergative intransitive verbs, such as break, increase, and open; 3)
% transitive-converted intransitive verbs of activity, such as eat, hunt, and read; 4) object deleting
% verbs, warranted by discourse or situational context, such as know, notice, and
% promise.

% *Medina (2007: 13)*
% > The indefinite implicit object construction is to be distinguished here, as much as possible, from 
% definite implicit objects which are defined here to be objects in which the speaker does need to have 
% the specific identity of the object in mind.This distinction serves to distinguish implicit objects whose 
% particular meaning can be recovered from the preceding discourse or disambiguating physical context 
% (definite implicit objects) from implicit objects whose meaning is recoverable only from the verb in the
%  sentence (indefinite implicit objects)

% *Petho & Kardos (2006: 30)*  
% > so-called indefinite null complements (INC) and definite null complements (DNC). This
% distinction was introduced by Fillmore (1986), and aims to capture a semantic difference between
% two types of verbs. Indefinite null complements (i.e. implicit objects) of verbs receive an
% “existentially quantified” interpretation, e.g. I am eating approximately means ‘I am eating
% something’, but not ‘I am eating it.’. On the other hand, a definite null complement is interpreted
% anaphorically and must therefore have an appropriate antecedent in context to make sense

% *Garcia-Velasco & Munoz (2002: 7-8)*  
% IMPORTANTE APPUNTO SULLA RECOVERABILITY! e sulla world knowledge!  
% > The fact the phenomenon is influenced by two types of factors (lexical and
% discoursive) has led some scholars to suggest the existence of two corresponding types of
% argument omission: contextual and lexical omission. This distinction is first introduced in
% Allerton (1975), to be taken over and developed by Fillmore (1986) and Groefsema (1995).
% Fillmore establishes an interesting distinction between definite null complements (DNC),
% which basically correspond to Allerton’s contextual omission and indefinite null complements
% (INC). Fillmore (1986: 96) employs the following test to distinguish the two categories:  
% << One test for the INC/DNC distinction has to do with determining whether it would sound odd for a speaker to
% admit ignorance of the identity of the referent of the missing phrase. It’s not odd to say things like, “He was
% eating; I wonder what he was eating”; but it is odd to say things like “They found out; I wonder what they found
% out.” The missing object of the surface-intransitive verb EAT is indefinite; the missing object of the surface-
% intransitive verb FIND OUT is definite. The point is that one does not wonder about what one already knows.>>  
% The distinction is then based on the possibility of recovering the missing element. INC verbs
% do not allow recoverability from the context. [...]  
% It seems that those verbs which allow the transition from accomplishment to activity might
% correspond to IO-verbs. That is, the IO type of omission can be considered to be lexical in
% nature, and therefore influenced by both the type and nature of the verbal object and the
% semantic class of the verb itself. [...]  
% The factors of relevance here include the semantic structure of the verb, which
% itself may give prominence to one semantic component (as in the manner-result opposition),
% the speaker’s communicative intentions, which may lead him to focus on the activity itself,
% thus downgrading the referential status of the object, and world knowledge, which allows him
% to construe an action as an autonomous activity

% *Tonelli & Delmonte (2011: 55 sgg.)*  
% IMPORTANTE: TERMINI-OMBRELLO PER DEFINITE+INDEFINITE OBJ DROP  
% > In this work, we focus on null complements, also
% called pragmatically controlled zero anaphora (Fill-
% more, 1986), understood arguments or linguistically
% unrealized arguments [...]  
%  definite null complements or instantia-
% tions (DNI) and are lexically specific in that they ap-
% ply only to some predicates.  
%  indefinite null complements or instantiations (INI) and
% are constructionally licensed in that they apply to
% any predicate in a particular grammatical construc-
% tion.

% *Eu (2018: 523)*
% >  Allerton (1975) and Fillmore (1986) understand omitted objects as being indefinite, by which they mean
% that omitted objects are unknown and insulated from their potential referents available
% in the context. However, this squib presents data that challenge this understanding of
% indefiniteness, and proposes that the indefiniteness of omitted objects may be more
% precisely understood as their indeterminacy over their potential referents. [...]  
% Omitted objects
% are ‘semantically obligatory’ (Somers 1984: 510), and yet object omission is generally
% distinguished from the phenomenon where an object is missing on the surface but
% ‘recoverable’ from the context. The difference between the two types of missing
% objects, however, has not been clearly explained, and this squib investigates this
% difference to gain a more precise view of object omission.
% Allerton (1975) and Fillmore (1986) explain the difference in terms of definiteness:
% omitted objects are indefinite in reference, while recoverable objects are definite.

% *Eu (2018: 524)*
% > For Fillmore and Allerton, however, the unknownness of INC-objects does not
% mean that they are known only as the abstract semantic role assigned to the object.
% Rather, they note that INC-objects are often understood in ‘semantic specialization’,
% for instance, as ‘a meal’ as in (4a), ‘alcoholic beverages’ as in (4b), and ‘breads or
% pastries’ as in (4c):  
% (4) (a) We’ve already eaten.  
% (b) I’ve tried to stop drinking.  
% (c) I spent the afternoon baking.  
% (Fillmore 1986: 96–7)

% *Eu (2018: 525)*
% > The question, then, is how semantic specialization differs from the knownness of
% DNC-objects. Allerton (1975: 218) answers this by saying: ‘the sentence: John’s been
% drinking again may imply a particular KIND of object, but it does not refer to one
% established as DEFINITE contextually’. In other words, semantically specialized INC-
% objects always refer to a category, or ‘a particular KIND of object’, and never a specific
% instance of the category, while DNC-objects refer to specific individuals in the context.

% *Eu (2018: 525)*
% > Indefiniteness as unknownness, however, is challenged by cases where INC-objects
% fairly clearly refer to a specific individual in the context. Groefsema (1995: 142, 44)
% introduces (5a, b) against Fillmore’s comment on (3a) and says that here eat and drink
% do take the contextual referents (sandwiches, glass of beer) as their objects, although
% the amount eaten or drunk is ‘unspecified’:
% (5) (a) John brought the sandwiches and Ann ate.
% (b) John picked up the glass of beer and drank.
% (Groefsema 1995: 142, 144)

% *Eu (2018: 527)*
% >  The point here is that INC-objects themselves do not fix the actual referent
% as the obvious one; hence, whenever the context is flexible enough as in (5a–d), it is
% possible to dissociate the actual referent from the obvious one without sounding odd.
% In contrast, DNC-objects require a clear and fixed referent. So in all contexts if the
% actual referent is different from what the hearer expects on the basis of the context, the
% object cannot be deleted; therefore, if an object is deleted as a DNC-object, the actual
% referent cannot differ from the obvious referent no matter how flexible the context may
% be. The referents of DNC-objects are determinate.

% *Ruppenhofer & Michaelis (2010: 159)*  
% > Fillmore (1986) distinguishes two major types of null comple-
% ments, definite and indefinite, based on the potential for a discourse antecedent,
% and Goldberg (2006: Chapter 9) uses the discourse prominence of participants to
% explain why constructions like the English experiential perfect license argument
% omissions that are not allowable in episodic contexts (e.g., She has never failed to
% impress Ø ).

% *Stark & Meier (2018: 11)*  
% > Cummins/Roberge (2004). In their qualitative analysis of object drop in French,
% they distinguish two types of null objects (NOs): “internally-licensed Null Objects” and
% “referential” Null Objects.

% *Stark & Meier (2018: 13)*  
% > This kind of null topic [referential null objects] is controlled pragmatically according to Raposo (1986, 375), i. e.
% it has either to be given linguistically, in the preceding context, or extralinguistically, in
% the situational context.

% *Ahringberg (2015: 9)*  
% > Fillmore (1986, p. 103) further recognizes several complement constructions that may be
% omitted as definite null complements, including lexical noun phrase direct objects, indicative
% that-clause direct objects, subjunctive that-clause direct objects, and prepositional phrase
% complements of transitive verbs, to name a few.

% *Dvorak (2017: 117)*  
% >  The terms were coined by Fillmore (1969, 1986), but the distinction itself goes
% back even further, to Katz and Postal (1964) who distinguished between the deletion of
% it versus something at the level of D-structure. Fillmore (1969) proposed that it has to
% be specified for each predicate whether it can have a null complement with an indfefinite
% interpretation, or with a definite interpretation. Fraser and Ross (1970) assume that the null
% object of Max read / Max is reading undergoes ‘unspecified object deletion’, which makes
% it different from the null object of the verbs like I approved / I began / I insisted, which
% has an anaphoric interpretation. 

% *Dvorak (2017: 118)*  
% > Fillmore (1986) reinforces the lexicon-based view of ‘indefinite null complements’ by
% saying that they are “limited to particular lexically defined environments”, such as the
% object slot of verbs like eat, read, sing, cook, sew, bake (Fillmore 1986:95). He suggests that
% these verbs, when used intransitively, have an understood object that could be paraphrased
% as stuff. The referential identity of such an object is unknown, or a matter of indifference,
% as shown by the follow-up clause in (200-a). On the other hand, ‘definite null complements’
% correspond to something that is already known from the context, so they do not allow the
% same continuation (see (200-b)).  
% (200) a. He was eating; I wonder what he was eating.  
% b. #They found out; I wonder what they found out.  
%  Fillmore 1986:96  
% He is also aware that in some highly restricted mini-genres, the possibility of object omission
% is much higher: Store in a cool place, Shake before using, Keep out of reach of children. For
% a recent take on these special registers, see Ruda 2014.

% *Melchin (2019: 53-54)*  
% > The understood objects of verbs that have undergone UOA are interpreted as indefi-
% nite masses or plurals. For their indefinite status, I give evidence from AnderBois (2012).
% AnderBois notes that the understood objects may not be coreferential with any previous
% discourse referents. [...] The unavailability of coreferential interpretations is a property of indefinite DPs; definite
% DPs (including personal pronouns) typically corefer with other referents in the discourse.

%  *Cote (1996: 110)*
%  > an examination of English null objects leads one into the murky waters of the distinction between arguments of verbs and information available from world
% knowledge about events.

%  *Cote (1996: 113)*
% > null objects found in (3) and (4) which occur with a different set of verbs and which may not co-refer with a discourse antecedent, but which do seem to create a discoure
% entity available for subsequent reference. Following previous work, I refer to the verbs that take this type of null object as Indefinite Object Alternation (IOA) verbs.


% *Liu (2008: 290)*  
% in risposta a quelli che dicono che non c'è differenza tra INDEFINITE object deletion e DEFINITE obj del  
% (dovrei avere una cosa del genere negli appunti, ma dove??)
% > For example, without any contextual
% information preceding or following the sentence Mary read for a while before she
% went to bed, it is impossible to decide whether a specific object, like a book or a magazine,
% has been deleted after the verb read. Making this distinction is important
% because whether a specific object has been deleted determines whether the verb is
% used as an intransitive verb focusing on activity or as an instance of object deletion,
% a distinction that, as I will argue later in this paper, should and can be made.

% *Liu (2008: 293)*  
% corrobora il punto precedente  
% > As some linguists (e.g., Garcia-Velasco & Munoz 2002; Quirk et al. 1985) have
% correctly pointed out, in the case of the second type, the shift of the verbs’ function
% from transitive to intransitive often involves a change in the focus of meaning, i.e.,
% the focus turns from the object in the transitive use to the activity (the verb) itself in
% the intransitive use. There is no such semantic shift in the object-deleting verbs’ shift
% from transitive to intransitive.

% *Liu (2008: 300)*  
% > two groups. For instance, Quirk et al. (1985:1565) list both the ergative and the
% transitive-converted intransitive verbs as instances of “transitive → intransitive.”
% There are three significant differences between the two groups of verbs. First, the
% subject or the sole argument of an ergative verb plays the theta role of theme, but the
% sole argument of a transitive-converted intransitive verb plays the theta role of agent
% (e.g., The window broke vs. Mary ate). Second, while there is a subject change in the
% use of an ergative intransitive verb compared to its transitive counterpart (e.g., from
% They opened the door to The door opened), no such a change is involved in the use
% of a transitive-turned intransitive verb [...]  
% The third difference between the two groups is that there is no object deletion
% involved at all in ergative intransitive verbs when they shift from transitive to intransitive
% because the shift entails only a movement of the object into the subject position,
% i.e., the object is not deleted, just moved to a different position in the sentence. [...]  
% Of course, some scholars argue that there are a few instances
% in which a transitive-converted intransitive verb of activity may assume “a more specific
% meaning, so a particular kind of object is ‘understood’” (Quirk et al. 1985:1169).
% An example given by Quirk et al. is John drinks heavily. They suggest that the verb
% drink in the utterance means ‘drink alcohol,’ i.e., alcohol is the omitted object. While
% it is true that drink here means ‘drink alcohol,’ the focus of the utterance, in the final
% analysis, is not on the object because it does not really refer to any specific kind or
% amount of alcoholic drink, something that is usually mentioned if the focus is on the
% object. In short, transitive-turned-intransitive verbs of activity focus on the activity,
% not the object.4

% *Liu (2008: 302)*  
% > The third compelling reason to consider these verbs intransitive verbs of activity
% is that they can function meaningfully without a clear discourse or situational context
% as shown in the following BNC examples:  
% (26) She likes to read.  
% (27) The python has already eaten.  
% (28) She might have been drinking.  
% In contrast, true object-deleting verbs, such as know, promise, and understand,
% cannot be used this way

% *AnderBois (2012: 44)*
% >  two properties which distinguish overt indefinites
% and definites in English: the possibility of apparently anaphoric readings (§2.1) and the ability to serve
% as the inner antecedent for Sluicing (§2.2) [...]  
% Sluicing is the name given by Ross (1969) to the phenomenon in (7) in which the crossed out
% material can be elided in the presence of a suitable antecedent clause in prior discourse:  
% (7) [ Someone left]A , but I don’t know [who left]E .


\subsection{Neither definite nor indefinite: continuous accounts} \labsec{theory_continuous}

Neither definite nor indefinite: continuous accounts

% *Olsen & Resnik (1997: 1)*
% > Early accounts eliminate the object from the syntax via an "object
% deletion" transformation (Katz and Postal 1964, Browne, 1971). More recent
% work has observed that implicit object constructions have aspectual constraints
% (Mittwoch 1982, Brisson 1994) and that omitted objects must be "typical,"
% inferable, or partially specified by the semantics of the verb (Brisson 1994,
% Fellbaum and Kegl 1989, Lehrer 1970, Mittwoch 1982, Rice 1988). Recently
% Resnik (1996) has substantiated and formalized the inferability claim using an
% information-theoretic account of selectional constraints. In this paper we show
% that the aspectual and selectional criteria for implicit objects are accounted for
% within the framework described by Hopper and Thompson's (1980; H&T)
% transitivity hypothesis. We locate English implicit object constructions on a
% continuum of transitivity, with indefinite implicit object constructions (1a) closer
% to intransitives, and definite implicit object constructions (1b) closer to transitives.

% *Olsen & Resnik (1997: 2)*
% > According to H&T, clauses that
% show telic aspect and individuated objects are more transitive than those with
% atelic aspect and nonindividuated objects. Furthermore, if aspect and
% individuation covary, telicity and high individuation should occur together in
% transitive clauses, and atelicity and low individuation together in intransitive
% clauses. 

% *Glass (2013: 1)*  
% UTILI CONSIDERAZIONI TEORICHE E BIBLIOGRAFICHE  
% > I'll sketch the scope of these data before I turn to the analysis. Although the literature (e.g.,
% Fillmore1986) distinguishes between “indefinite” and “definite” IOs, I have some qualms about
% this distinction, which I elaborate later on (see also Anderbois 2012, Scott 2006). Therefore, I
% consider data from both sides of this distinction. However, I limit myself to IOs that seem to
% stand in for DP’s (rather than CP’s). I also don't take a stand on whether IOs are represented in the
% semantics, as in e.g., Anderbois (2012), or whether the verbs are simply intransitive and a patient
% argument (the thing eaten) is pragmatically inferred (as in Recanati 2007); the term “IO” is de-
% scriptive only.


% *AnderBois (2012: 43)*
% > Dating back to Fillmore (1969)’s seminal work, the literature on implicit arguments (IAs) has
% consistently distinguished two types of IAs : those which can be paraphrased with an overt indefinite, as
% in (1a)1 , and those which are better paraphrased with a pronoun or definite description, such as (1b).  
% (1) a. John ate [THEME].  ⇐ Indefinite Implicit Argument  
% b. Maribel noticed [S TATE O F A FFAIRS ]. ⇐ Definite Implicit Argument [...]  
% a third category, to which we give the descriptive moniker flexible IAs. As we will show below, flexible
% IAs appear to pattern with definite IAs on some occasions and indefinite IAs on others.  
% (2) The Giants won [CONTEST]. ⇐ Flexible Implicit Argument

% *Cummins & Roberge (2005: 46)*  
% CFR. ANDERBOIS 2012 SUBITO SOPRA IN QUESTO NOTEBOOK!  
% > Larjavaara’s generic/latent distinction is based on
% the ability of the hearer to identify a possible referent. Lambrecht and
% Lemoine’s categories of indefinite (those that cannot refer to an entity in the
% discourse) and definite (those that must be interpreted as referring to an entity
% in the discourse) capture a similar distinction, and they add a third
% category—ÔÔlibreÕÕ (ÔfreeÕ)—to handle the indeterminate cases.

% *Naess (2007: 130-134)*
% CONTINUOUS ACCOUNT SU BASE PRAGMATICA
% CRITICA DI QUESTA COSA --> TORNA BENE CON IL MIO "CONTINUUM GRADATUM"
% In a departure from the traditional approach based on verb semantics, Goldberg
% (2001) analyses IOD as essentially a discourse-pragmatic phenomenon. Her start-
% ing-point is the data presented in (6.1) above, which demonstrate that in the right
% context, just about any transitive verb in English can be used without an object.
% In order to account for these data, Goldberg introduces the notion of discourse
% prominence, which subsumes both topic and focus. An argument that is either
% topical or focal is discourse-prominent, and English generally requires discourse-
% prominent arguments to be expressed. In the typical situation the patient argument
% of a causative verb is a highly prominent one (cf. 3.2.2); “one typically does not as- --  
% sert that a participant changes state unless one wishes to discuss or draw attention
% to that participant” (Goldberg 2001 :510). Nevertheless, a patient of a causative
% verb which is neither topical nor focal, and therefore has low discourse promi-
% nence, may be omitted under the further condition that the action is emphasised.
% Emphasis on the action can take a number of forms; iteration, discussed earlier, is
% one possibility, but there are others. Goldberg among others mentions that the ac-
% tion can be the discourse topic, as in the example He was always opposed to the idea
% of murder, but in the middle of the battlefield, he had no trouble killing; or an action
% can be emphasised by the speaker’s strong affective stance toward the action, as in
% Why would they give this creep a light prison term? He murdered!
% Goldberg concludes that patient arguments can be omitted when they are
% deemphasised vis à vis the action. This leaves her with the cases of apparently
% lexically conditioned omission – the familiar “core set” of IOD verbs like drink,
% smoke, sing, bake, read, eat, which frequently omit their objects even when none of
% the above-mentioned trappings of “action emphasis” are present. She states that
% intuitively there does appear to be a stronger emphasis on the action in Pat read in
% the car than in Pat read a book in the car, but does not attempt to explain or sub-
% stantiate this intuition. Instead, she notes that this set of verbs frequently appears
% in generic contexts with a habitual interpretation: Pat drinks; Pat smokes; Chris
% sings; Sam bakes. As this usage is licensed by her discourse-prominence analysis,
% she assumes that the frequent usage of these verbs in such contexts has led to a
% grammaticalisation of object omission as a lexical option for these verbs.
% There are two main difficulties with this analysis. Firstly, it is unclear exactly
% what the condition of “emphasis on the action” should be taken to include. There
% are instances where object omission is possible but where it is not clear that there
% is any intended emphasis on the action as such. In a sentence like John murdered
% for the money, no iteration is necessarily implied; John might perfectly well have
% committed just one murder for the purposes of financial gain. But neither does the
% action as such seem to be emphasised in any other way; if anything, the emphasis
% here seems to be on John’s motivation for performing the act in question rather
% than on the action itself, and in fact the NP specifying this motivation may take
% contrastive stress: John murdered for MONEY, not for love. This goes against Gold-
% berg’s claim that emphasis on the action as such is required, rather than just the
% presence of some focal element other than the patient (p. 513).
% Secondly, the attempt to locate the properties licensing IOD exclusively at the
% level of discourse leads to a rather vague and unsatisfactory account of the in-
% stances where IOD does seem to be related to verbal semantics. There is no expla-
% nation for why just the sort of verbs cited by Goldberg (note that all of these are
% either affected-agent or effected-object verbs) should be so frequent in generic/ha-
% bitual contexts as to grammaticalise the possibility of IOD. Furthermore, it is sim- --  
% ply not true that all of these verbs are frequent in “generic contexts with a habitual
% interpretation”. The most frequently-cited IOD verb of them all, ‘eat’, is highly un-
% natural in a generic/habitual sentence: ??John eats. Similarly, the generic/habitual
% construction with drink only applies when drink is understood to mean ‘drink al-
% cohol’; if taken to mean something like ‘John habitually imbibes liquids’, ??John
% drinks is just as peculiar as ??John eats. It is not obvious, therefore, how the pro-
% posed grammaticalisation of the possibility of IOD would have come about with
% these verbs; and the question also arises of exactly where the ‘drink alcohol’ mean-
% ing of objectless ‘drink’ comes from (see 6.3.3 below).

% *Naess (2007: 134)*
% > Structurally, the most obvious characteristic of the indefinite object deletion construction is
% that it is a formally intransitive clause, as opposed to the transitive construction
% which appears when the verb in question is used with ano overt object NP. The
% logical explanation to such an alternation between higher vs. lower formal transi-
% tivity would be that it reflects a corresponding difference in semantic transitivity.
% From this perspective, IOD is most felicitously analysed not as a lexical quirk
% of certain specific verbs or classes of verbs, but as a syntactic detransitivisation
% mechanism, a means of expressing in a formally intransitive clause events which
% are construed as deviating from the transitive prototype.

% *Naess (2007: 135)*  
% IMPORTANTISSIMO!!! SI RICOLLEGA AL DISCORSO DI DRINK=ALCOHOL ETC  
% > Analysing IOD as a construction which applies only to verbs or clauses which
% are not fully transitive semantically explains a striking observation made by Fill-
% more (1986): certain verbs undergo what he calls a semantic “specialisation” when
% used in an IOD context. He cites the verb bake as an example: in the sentence I
% spent the afternoon baking, “the missing object is taken to include breads or pas-
% tries, but not potatoes or hams” (Fillmore 1986 :96).
% If we take into consideration the difference in semantic transitivity between
% affected-object and effected-object verbs, an explanation for this “semantic spe-
% cialisation” is readily forthcoming. The verb bake in English is ambiguous between
% an affected-object (bake potatoes, where the potatoes already existed prior to the
% action and are only affected, not effected by the act of baking) and an effected-ob-
% ject reading (bake pastries, where the act of baking brings the pastries into exist-
% ence). Affected-object bake is high in transitivity and so does not easily undergo
% IOD. Effected-object bake, on the other hand, unproblematically omits its object,
% because this verb has a nonreferential, nonaffected object and therefore is not ful-
% ly transitive semantically. Consequently, under IOD, only the effected-object read-
% ing is possible – the intransitive construction can only be read as containing a verb
% relatively low in transitivity.
% Similar behaviour can be found e.g. with the ambiguous verb paint in generic/
% habitual statements. If we say of someone that He paints, we mean that he paints
% pictures – either for a living or as a hobby – not that he is a housepainter.

% *Naess (2007: 136)*  
% > The IOD construction with a purpose clause – John murdered for the money –
% does not necessarily have an iterative reading, as pointed out above. Rather, such
% clauses are construable as a kind of affected-agent construction where the affected-
% agent reading is not imposed by the semantics of the verb, but rather by the pur-
% pose clause. A statement of the agent’s motivation or purpose in performing an act
% is essentially a statement of the benefits that the agent hopes to achieve in acting;
% in other words the intended effect of the act on the agent. Affectedness of the
% agent, then, is not necessarily inherent to the semantics of a specific verb, but may
% be introduced by other elements of a clause.


\subsection{A working definition of "indefinite object drop"}

citare Medina e tutta la teoria possibile che la conforti
è una definizione pre-teoretica che prescinde dalle proiezioni sintattiche?
QUI METTERE IL MIO DIAGRAMMA DI VENN CONCENTRICO

% *Naess (2007: 124-125)*
% > In this chapter, the term “indefinite object deletion” (IOD) will be used to describe
% the situation where a verb which allows an argument which syntactically must be
% characterised as a direct object is employed without any overt direct-object argu-
% ment. We will restrict the discussion to cases of so-called context-independent
% object deletion.
% Many languages allow omission of objects whose reference is expected to be
% retrievable from context. Such context-dependent object deletion may take place
% when the object in question has already been mentioned in the previous discourse,
% or where the general context provides sufficient clues as to the identity [...]  
% By contrast, context-independent object deletion does not depend on the ref-
% erent of the missing object being retrievable or identifiable by the hearer. In order
% to correctly understand a sentence such as He is eating, it is not necessary to be
% able to infer what is being eaten; in fact, there is no expectation whatsoever that
% the referent of the object should be identifiable. This is not to say that the use of the
% verb eat without an object does not imply that an object is present; the act of eating
% unquestionably presupposes that something is eaten. However, it is only this pres-
% ence of an object in general that must be inferred by the hearer in such cases: the
% hearer knows that something is being eaten, but is not expected to be able to iden-
% tify which specific edible object is being referred to. [...]  
% Some approaches assume a further division between contexts where objects
% are omitted because they are obvious and contexts where they are omitted because
% they are indefinite. For example, Heath (1976a) claims that in the sentence He
% drinks, “the T[ransitive] O[bject] is deleted because it is obvious in the context,
% rather than because of indefiniteness. On the other hand, in the type Speed kills the
% deleted TO is indefinite rather than obvious”.
% It is not clear, however, how the object of He drinks can be “obvious in the
% context” when the sentence is in fact presented without any context whatsoever.
% The only way to interpret this would seem to be that the object is “obvious in the
% context of deletion”, that is, it is obvious because it is deleted; the very fact that no
% object is present allows us to assign it the “obvious” interpretation. But this is a
% circular argument: the object is deleted because it is obvious, and it is obvious be-
% cause it is deleted. No such difference between “obvious” and “indefinite” deleted
% objects will be assumed here. 

% *Haegeman (1987: 236)*  
% > The (b)-examples illustrate what has come to be known as indefinite
% object drop: a transitive verb lacks an object, but an indefinite object
% is understood.
%  Rizzi ( 1986) argues convincingly that these cases of
% indefinite object drop in English are lexically governed. While eat, for
% instance, allows object drop, the semantically closely related devour
% does not:

% *Yasutake (1987: 46)*  
% calls them "detransitive verbs"

% *Glass (2013: 1)*  
% > “Implicit object” (IO) is a name for what happens when a verb we normally consider transitive
% appears without an object

% *Petho & Kardos (2006: 29)*  
% >  implicit object arguments, also referred
% to in the literature as null complements or understood arguments

% *Glass (2020: 1)*
% > Other words for this phenomenon include object drop,  unspecified object,  null object,  null argument,  null complement, null instantiation, diathesis alternation, implicit object  

% *Gillon (2012: 314)*
% > Fodor and Fodor (1980) seem to have been the first authors to have brought to the
% attention of linguists that implicit indefinite objects pose a problem for grammatical
% theory. Though Fodor and Fodor did not put it this way, we can succinctly formulate
% the problem as a dilemma. Either there is only one verb to read or there are two.

% *Glass (2020: 1)*
% > It is a longstanding question in lexical semantics which normally-transitive verbs in Englishcan omit their objects to describe an event with an unexpressed theme, which cannot, and why1.Even near-synonyms differ in how natural they sound with their objects omitted (1) (Fillmore1986, Rice 1988, Mittwoch 2005, Gillon 2012), making it difficult to explain omission in termsof meaning, and leading some researchers to characterize this phenomenon as partly or fullyarbitrary (Fillmore 1986, Jackendoff 2002: 134, Ruppenhofer 2004, Gillon 2012, Ruppenhofer& Michaelis 2014).

% *Dvorak (2017: 1)*  
% > Non-overt direct objects with an existential interpretation, as in (1), have been the topic of
% linguistic discussions since Chomsky introduced the transformational rule of ‘Unspecified
% object deletion’ in 1962. In the English-oriented literature, they are usually assumed to be
% represented lexically, either in the form of de-transitivizing rules that operate on individual
% predicates, see (2a), or as two separate predicates, transitive and intransitive, linked by
% a predicate-specific meaning postulate, see (2b). Alternatively, Cote (1996)  enriches the
% lexical conceptual structure (LCS) of a given verb by marking the relevant argument as
% unsubscripted zero [...]  
% More recently, this view has been challenged by Alexiadou et al. (2014), who argue for a
% syntactic re-interpretation of Bresnan’s rule of Existential Closure (EC) in (2a), such that it
% applies to any predicate with an unsaturated internal argument: JECK = λ fhe,stiλ es ∃x[f(x)(e)]
% (see Babko-Malaya 1999 and Martı́ 2011 for related proposals).

% *Lorenzetti (2008: 62)*  
% > Early accounts of the phenomenon proposed the elimination of the object from the syntax
% via “object deletion transformation” [Katz and Postal 1964, Browne 1971, Allerton 1975],
% while later works introduced the view that implicit object constructions have aspectual
% constraints [Mittwoch 1982] and that omitted objects are frequently typical, inferable or
% partially specified by the semantics of the verb [Lehrer 1970, Rice 1988].


\section{Defining the indefinite} \labsec{theory_defindefinite}


\subsection{Two meanings, two verbs: the naive account} \labsec{theory_twoentries}

Two entries in the lexicon: the naive account

% *Petho & Kardos (2006: 29)*  
% DECISAMENTE NO! MOLTO NAIVE, DIRE CHE IO NON PENSO QUESTO
% >   Some of these can be characterised as properties of
% the verbal predicate itself, i.e. the verb has both a transitive and an intransitive version in the
% lexicon, like eat, drink, cook, read.

% Pragmatic *Cote (1996: 120)* 
% > The optional argument (pragmatic) explanation supposes that the verb may optionally have no
% linguistic representation for an object, even in the lexicon. In other words, this explanation asserts that
% there is a true transitive/intransitive alternation at work. 

% *Bourmayan & Recanati (2013: 122)*  
% con notazioni logiche lambda! utili per la tesi   
% > on the ‘intransitive’ uses of the verb the object
% remains covert—it is not articulated in surface syntax, though it shows up at
% LF. The logical form of ‘John eats’ is therefore something like ‘John eats [some
% thing]’, where the materials within square brackets correspond to a covert syntactic
% element with indefinite value. We will refer to this view as the covert indefinite view
% (CIV). The other view, which we call the genuine-intransitive view (GIV), assigns
% two distinct (though related) lexical entries to transitive and intransitive ‘eat’.



% *Lorenzetti (2008: 60)*  
% ## CONTRO LA DISTINZIONE IN DUE DIVERSE ENTRATE LESSICALI
% > we argue that positing different lexical entries in the case of null-object verbs
% is often counterintuitive and inappropriate,

% *Naess (2007: 130-134)*  
% ## CONTRO LA DISTINZIONE IN DUE DIVERSE ENTRATE LESSICALI
% Rather than focus on the relations between individual verbs and the types of
% objects they typically take, a number of approaches to IOD appeal to event-struc-
% ture analyses along the lines of Vendlerian semantics. Mittwoch (1982) character-
% ises intransitive ‘eat’ as an activity and transitive ‘eat’ as an accomplishment, thus
% supporting on event-semantic grounds the view that the two uses of ‘eat’ must on
% some level be analysed as involving two separate lexical items. Van Valin and La-
% Polla (1997) similarly assume a distinction between an “activity” and an “active
% accomplishment” reading of verbs such as ‘eat’, where the activity version has a
% nonreferential second argument which is not instantiated as a semantic macrorole
% in the verb’s logical structure, whereas the active accomplishment version has two
% macrorole arguments (Van Valin and LaPolla 1997 :148–150). Van Valin and La-
% Polla’s analysis of ‘eat’ was discussed in 4.5.
% In Brisson (1994), a case is made for two distinct classes of verbs permitting
% IOD (in Brisson’s terms, verbs allowing unspecified objects): “write verbs”, exam-
% ples of which are write, knit, bake, draw, paint, sew, drink, type, dig, and eat; and
% “sweep verbs” such as sweep, plow, pack, dust, vacuum, clean, mow and rake – study
% and read are given in parentheses as atypical examples of this class because they
% behave like these verbs under some, but not all, conditions.
% In brief and somewhat simplified terms, Brisson’s argument is that the sweep
% verbs are not true accomplishment verbs (among other things, she argues that they
% do not necessarily entail a result state, and they can take adverbs of duration with-
% out acquiring an iterative reading) and that this accounts for their ability to occur
% without an overt object, a claim essentially parallel to that made for ‘eat’ by Van
% Valin and LaPolla to the effect that the second argument of activity predicates is
% qualitatively different from that of accomplishment predicates, and therefore omis-
% sible. The write verbs, on the other hand, are accomplishment verbs proper and
% therefore require a different account.
% However, Brisson is not able to provide such an account, beyond proposing
% that write verbs do not really “delete” their objects but rather have two distinct
% variants represented in the lexicon, one intransitive activity verb and one transitive
% accomplishment verb. With respect to the intransitive activity verbs, she discusses ---  
% whether there is not in fact an “implicit argument” present; does not John wrote
% seem to entail that he wrote something? However, she claims that this may not be
% relevant and that “the assumption that ‘something’ was written might be related to
% real world knowledge” (Brisson 1994 :99).
% The appeal to real-world knowledge appears rather ad hoc in this case, as it is
% not clear why this would apply only to the write verbs. If the assumption that
% something is written is attributable to real-world knowledge, why would the as-
% sumption that something is swept require a different analysis? If the interpretation
% of objectless clauses depends on real-world knowledge in some cases, could this
% not be generalised to all cases, making all further analysis superfluous?
% A further problem with this analysis is that it produces a complicated and
% rather vague account of the verb eat, which has traditionally been considered one
% of the most central instances of the IOD phenomenon. Eat in this account is clas-
% sified with the write verbs, but behaves rather differently from the rest of this class
% when used intransitively: “monadic eat seems to strongly suggest that a meal has
% been eaten in the unmarked case: John ate means that he ate dinner, or lunch, or
% whatever may constitute a meal in John’s diet. This completive sense is in strong
% contrast to the monadic write verbs: John wrote does not have any such completive
% sense. However, it is still possible to obtain a “pure” activity reading for monadic
% eat, as in John ate for ten minutes. It is likely that monadic eat would have to be
% lexically marked to account for these properties” (Brisson 1994 :100).
% An analysis in terms of event structure, then, cannot account for the fact that
% a number of presumed accomplishment verbs are frequent in the IOD construc-
% tion, beyond claiming that such verbs occur in two distinct lexical variants. It also
% has difficulties in accounting for the behaviour of ‘eat’ under IOD, which will be
% further discussed in 6.3.3.

\subsection{One verb, two meanings: the state-of-the-art account} \labsec{theory_incorporation}

qui parlare dell'"oggetto logico" e del focus sull'activity!!!\\
è questo il paragrafo in cui distinguere object drop e verbi intransitivi\\
cambiare il titolo in "focus sull'activity" e fare sottoparagrafi, uno per noun incorporation\\
manner/result complementarity?

\paragraph{Inherent objects}
testo

% *Naess (2007: 130-134)* 
% Rice appeals to the notion of prototypical complement of a verb: there is, she
% claims, a basic-level NP which is evoked as the understood object when a verb is
% used without an overt object NP: “Clearly, an omitted object should not be read as
% zero. Rather, on a neutral reading, an omission activates a prototype or a particular
% semantic frame in which the action is prototypical” (Rice 1988 :204). This assump-
% tion is meant to account for the fact that the presence of an object is implied with
% IOD constructions, even though no specific reference is assigned to it.
% The notion of a prototypical complement, however, begs the question of how
% one determines, for a given verb, which possible complement is “prototypical”.
% One of Rice’s examples of such a verb is drink, which is assumed to have the pro-
% totypical complement ‘alcohol’. On this analysis, the fact that the English verb
% drink may be used without an object is a consequence of this verb having a proto-
% typical object associated with it, and this prototypical object is recoverable from
% the objectless sentence, giving the reading of John drinks as John drinks alcohol.
% The question is, apart from the fact that objectless drink in English (and many ---  
% other languages) has this reading, what is the evidence for alcohol being a proto-
% typical object of drink? In the daily life of most people, a typical act of drinking
% does not involve alcohol, and it is not clear why such a specialised notion as that of
% ingesting intoxicating fluids should be “prototypical” with respect to the everyday
% concept of drinking things in general. Without any independent justification, such
% an assumption is circular: Object NPs may be omitted if they represent “proto-
% typical complements” of the verb in question, and since in English objectless drink
% is generally interpreted as meaning ‘drink alcohol’, ‘alcohol’ must be the prototypi-
% cal object of ‘drink’.

% *Garcia-Velasco & Munoz (2002: 4)*  
% >  it is not only the presence/absence of a verbal object that allows
% the transition from an activity to an accomplishment reading with some verbs. When the
% verbal object is non-specific, indefinite or generic, it is possible to obtain the same effect:  
% (3) a. He ate a plate of spaghetti in ten minutes (accomplishment)  
% b. He ate spaghetti for ten minutes (activity)  
% Van Valin & LaPolla (1997) note that this situation is frequent with verbs of creation or
% consumption. According to the authors (1997: 122), the second argument in activity
% predicates does not show referential properties, which usually imparts a generic or habitual
% interpretation to the predication. [...]  
% On the basis of these observations, the authors make the following claim (1997: 122-123):  
% << Thus, the second argument with an activity verb like eat will be called an INHERENT ARGUMENT, an
% argument which expresses an intrinsic facet of the meaning of the verb and does not refer specifically to any
% participants in an event denoted by the verb; it serves to characterize the nature of the action rather than to refer
% to any of its participants. >>

% *Cote (1996: 120)* 
% Lexical (the winner, page 130) -- la spiegazione del perché gli altri due fattori (in particolare il sintattico) non sono vincenti è GENIALE!!!  
% e credo che mi consenta anche di spiegare per quale motivo lo stesso modello funzioni (speriamo) sia per i dObj, che sono argomenti prototipici, sia per gli Strumenti, molti dei quali sono aggiunti prototipici
% > The lexical analysis approach requires that an argument be present in the lexicon but not projected to syntactic structure. 

% *Cote (1996: 156)*  
% CRUCIALE per distinguere gli indefinite dObj verbs dai veri intransitivi (così PM è tranquillo)
% > Also, unlike strict intransitives, when IOA verbs like eat are used, there is a new discourse entity
% (an “eatee”) which is made salient enough to be referred to with a pronoun in a subsequent utterance.

% *Dvorak (2017: 2)*
% > The literature on null arguments
% generally advocates two major theoretical approaches: (A) a null argument is syntactically
% represented and it corresponds to a null pronoun/DP; (B) a null argument is not syntacti-
% cally represented at all and it is a part of a lexical entry for a given predicate.

% *Dvorak (2017: 112)*
% >  Examples like John ate were discussed within the transformati-
% onal generative grammar already in Chomsky 1964 where they were analyzed as ‘deleted
% unspecified object’ (see also Chomsky 1962, Katz and Postal 1964). [...]  
% Bresnan (1978) proposed an alternative, lexical solution to the same issue. She argues that
% the verb eat has a logical object even if it lacks a grammatical object, which is what makes
% it different from the verbs like sleep which have no object at all. 

% *Yasutake (1987: 48)*  
% > they are different from pure intransitives in that the action will not be complete without some lexically implied (but unspecified) object

% *Melchin (2019: 52)*
% > some properties of the omitted objects. I show that they are
% interpreted as indefinite masses, corresponding to the “typical” object of the verb; however,
% there is no evidence that they are present in the syntax. This suggests that it is really an
% instance of omission of an argument,4 rather than the presence of a null pronoun or variable
% present in the syntax

% *Naess (2007: 130-134)*  
% > “In English... condi-
% tions for the omission of non-subject complements are limited to particular lexi-
% cally defined environments. The most commonly discussed of these is the object
% slot for such verbs as EAT, READ, SING, COOK, SEW and BAKE... which are un-
% derstood as having, when used intransitively, an understood object roughly repre-
% sented as the word STUFF” (Fillmore 1986 :95). Fillmore assumes that the ability to
% occur with what he calls Indefinite Null Complements (that is, “deleted” objects) is
% specified in the lexicon for each verb: “From the reality that omissibility phenom-
% ena of the sort discussed in this paper are tighly connected with specific senses of
% specific words, it seems unavoidable that (at least in these cases) closely related
% word senses must be listed separately in lexical entries” (Fillmore 1986 :106).
% Rice (1988) suggests that IOD is not a possibility specified in the lexical entry
% of a particular lexeme, but rather a function of the different ways in which the
% events denoted by certain verbs can be construed: “[C]ertain construals of transi-
% tive events are such that they focus on the active participant and leave the acted-
% upon participant unspecified and, most importantly, to be filled in by a default
% value” (Rice 1988 :203). She argues that objects can be omitted when there is a
% default interpretation for a missing object, exemplifying this with ‘John ate a meal’
% as the default interpretation for John ate. The verbs allowing such omission must
% be “semantically neutral”, not conflating action and manner (contrast John ate with
% *John nibbled), and the omitted object shows a “low degree of semantic independ-
% ence from the verb” (Rice 1988 :204).

\paragraph{Prototypicality}

Prototypicality and other aspects such as noun incorporation

% *Mittwoch (2005: 20)*  
% DUE CONSIDERAZIONI (IN NOTA) FONDAMENTALI!!!
% > It is sometimes claimed that in such restrictions the omitted object is
% prototypical. As an attempt at explanation, this risks circularity. What makes us
% regard the interior of a house as a ‘prototypical’ object of the verb clean is
% precisely the fact that it is typically ‘understood’ in the absence of an overt
% object. [...]  
% In Mittwoch (1971) and (1982) I pointed out that the deleted element could not
% be the indefinite pronoun something, since this would be incompatible with the
% atelic nature of the resulting sentence.

% *Yasutake (1987: 48-50)*  
% 3 tipi di implicit object:
% * read/telephone = "activities where there is some standard or typical kind of object"
% * drink/shave (+ expect/propose/drive) = "the objectless use of this subtype has, conventionally or socially, acquired 
% a slightly different sense from the habitual one" (CFR. GLASS 2020! ROUTINE, WORLD KNOWLEDGE con riferimenti in Yasutake stesso)
% * steal/see/annihilate = "a highly specialized kind of activity, and in many cases imposes practically no limitation on possible objects"

% *Dvorak (2017: 118)*  
% TUTTE COSE FONDAMENTALI DA SPOSTARE NEI PARAGRAFI GIUSTI!  
% >  For Rice (1988), INO represent variation
% which is “not strictly a function of the verb’s inherent meaning”, therefore, it “does not
% warrant additional lexical entries”. In her view, this forces researchers like Bresnan (1982)
% or Hale and Keyser (1986) to make unwarranted assumptions about the lexicon’s power.
% Rice sees object omission as a result of a collection of paradigmatic rather than idiosyncratic
% semantic factors, such as the verb type (verbs that conflate action and manner tend to resist
% object omission: *Celia nibbled/chewed/bit versus Celia ate), or the object type (objects
% denoting wholes are more likely to be left out than objects denoting parts: Travis let Billy
% drive (the car) versus Travis let Billy gun *(the motor)). In general, Rice claims, the se-
% mantically ‘neutral’ verbs with objects that are neither too specific nor too general are the
% most prone to object omission. The omitted object then represents the verb’s ‘prototypical
% complement’, giving rise to the default interpretation, cf. When he goes to Boston, John
% drives (a car / *a Toyota / *a motorcycle / *a vehicle).

% *Ahringberg (2015: 7)*  
% > 2.3.1 Lexical licensing
% According to Fillmore (1986, p. 98) the lexical properties of the predicate are essential in the
% licensing of null instantiation, which implies, in in broad terms, that some transitive predicates
% can allow their objects to be left out whereas other cannot. This is generally conceived as the
% fundamental factor also in other studies on the subject (e.g. Goldberg, 1995; Lambrecht &
% Lemoine, 2005; Prytz, 2009; Bäckström, 2013). Fillmore (1986, p. 98) further claims that this
% kind of lexical description is more consistent than a semantic one, as some verbs allow null
% instantiation whereas predicates with synonymous meanings may require their complements to
% be expressed. For example, the verbs protest and find out can be used without objects, whereas
% oppose and discover cannot, even though the verbs are “semantically related” (Fillmore, 1986,
% p. 98).

% *Ahringberg (2015: 8)*  
% > A single predicate may denote several senses and
% Fillmore (1986, p. 99) emphasises that with polysemous transitive verbs, in other words verbs
% with several different senses, it is rather certain types of the senses and not the predicates per
% se that permit leaving out the object. Likewise, it appears that only particular kinds of
% complements allow being omitted in some cases. For example, a left out object of the verb lose
% (Fillmore, 1986, p. 100) can only refer to a certain kind of competition or election, but not to
% an item which one has forgotten or mislaid. It therefore appears that null instantiation cannot
% be fully accounted for from lexical factors only. 

% *Newman & Rice (2006: 4)*  
% > Huddleston and Pullum (2002: 303-305) refine Huddleston [1988]’s notion of
% intransitivity by offering a sub-categorization of types of ‘unexpressed
% objects’ of intransitive verbs. EAT and DRINK participate in two such
% patterns of omissibility: ‘specific category indefinites’ and ‘normal
% category indefinites’. The former refers to the possibility of understanding
% the intransitive uses of EAT and DRINK specifically as ‘eat a meal’ and
% ‘drink alcoholic drink’ respectively; the latter refers to the use of
% intransitive EAT and DRINK when the unexpressed object is interpreted as
% the ‘indefinite, typical, unexceptional’ exemplar (‘food’ in the case of EAT,
% ‘water’ or ‘beverage’ presumably, in the case of DRINK).  
% The traditional view of an intransitive vs. (mono)transitive distinction,
% as enunciated in Huddleston (1988) and Huddleston and Pullum (2002), is
% by no means compelling. One could just as well argue that the intransitive
% use in (3a) really involves one participant (the agent phrase) and describes
% an activity of that participant, similar to the way in which the intransitive
% verb run in English describes an activity of a runner. Other associated
% entities can be a necessary part of a larger semantic frame of intransitive
% verbs (legs in the case of run, food in the case of eat), but this does not
% require us to say that they are second participants which are simply
% unexpressed. [...]  
% A more provocative view of transitivity can be found in Van Valin and
% LaPolla (1997: 115). They speak of the English predicate as having either
% one or two arguments in its logical structure, similar to Huddleston’s
% distinction between intransitive and monotransitive uses of EAT. Their
% representation of the logical form of EAT expresses the alternatives through
% the parenthesized (y) embedded in the argument structure.  
% (4) do' (x, [eat' (x, (y))]  
% x=CONSUMER, y=CONSUMED

% *Dvorak (2017: 116)*  
% > It is somewhat disturbing that the most indepth discussions of INO in 1970s and 1980s
% were revolving around a single verb to eat. The notable breakthrough in this tradition is
% Levin (1993:33), who listed over forty verbs in English as examples of the ‘unspecified object
% alternation’. They include the verbs bake, carve, clean, cook, drink, eat, hunt, paint, play,
% sing, study, wash, write, etc.22 Levin also notes that the intransitive variants of these verbs
% are ‘understood to have as object something that qualifies as a typical object of the verb’.  
% (199) Mike ate. (→ Mike ate a meal or something one typically eats.)

% *Lorenzetti (2008: 59)*  
% QUESTA COSA DELLO SHIFT SEMANTICO LA DICO ALTROVE! FORSE IN SPECIAL REGISTERS?  
% > This paper argues that the phenomenon of the null instantiation of objects, i.e. the property of
% some transitive verbs to omit their direct complements, can be viewed as a polysemy-trigger.
% Our study, adopting a lexical complexity perspective, suggests that in the majority of cases
% verbs retain traits of their prototypical meaning, which becomes the starting point for possible
% inferences, contributing to the overall interpretative process, and leading to the dynamic
% emergence of different semantic interpretations and nuances through complex mechanisms of
% figure and ground.

\paragraph{INDEFINITE = "SOMETHING"}

testo testo

% *Melchin (2019: 55)*  
% > Thus, the understood objects of UOA verbs pattern with bare masses and plurals, which are
% weak indefinites.
% Further evidence for a weak indefinite reading of understood objects comes from the
% analysis of Steedman (2015). Steedman starts with the observation, attributed to Fodor and
% Fodor (1980), that these understood objects, unlike indefinite pronouns like something, always
% have low scope with respect to other quantifiers in the sentence.

% *Lorenzetti (2008: 63)*  
% QUESTA PAGINA DICE IN MODO SINTETICO COSE MOLTO IMPORTANTI!!!  
% > 3.1.1. Indefinite Null Objects
% This category, also known as Indefinite Null Complements (henceforth INCs) [Fillmore
% 1986] or Unspecified Object Alternation [Browne 1971], is typical of a variety of activity
% verbs of the eat type, such as drink, sing, bake, cook and paint among the others, which have
% a pronounced manner component in their meaning and fairly circumscribed selectional
% restrictions. Hence, the content of the null object is more or less predictable: it will
% correspond to the literal rather than to the metaphorical meaning of the verb , and is
% sometimes argued to be restricted in usage, i.e. an expression such as I’m cleaning IS most
% likely to refer to the interior of a house, rather than to one’s teeth. [...]  
% However, postulating a reading of these verbs in terms of stereotypic entities associated to
% them does not always seem appropriate, since while in example (3) the phantom object is a
% meal, i.e. the apparently stereotypic entity associated to the verb eat, in an example like the
% most likely context would not lead to the interpretation that the person is accustomed to
% having a meal or as many meals as she can during the whole day.
% (5) I started working out, but I would eat all day after that.
% On the contrary, the most typical interpretation in this case is likely to be achieved through
% the underspecified word “food”, a representative of the entire class of edible things. However,
% the fact that the object is unexpressed in this case suggests that what the person eats is
% irrelevant for the current purpose of the interaction.
% We can suppose, in this respect, that a better explanation of the restrictions on these null
% objects is that they have to be consistent with the underlying context, the intentional structure
% of discourse and the shared relevance [Sperber and Wilson 1986] at the time of utterance.


\paragraph{Focus on the activity}
testo

% *Mittwoch (2005: 2)*  
% PARAGRAFO IMPORTANTISSIMO!!!  tra activity e prototypicality
% > There is a well-known transitivity alternation involving a class of common
% process verbs which can be both transitive and intransitive with the same
% subject argument, so that the intransitive variety is unergative:  
% (1) John is reading / drinking.  
% John is reading a letter / drinking juice.  
% A representative list of English verbs participating in the alternation as it will be
% understood here is given in (2).  
% (2) a. read, study, revise (what has been learnt) rehearse, practise  
% b. sing, dance, play (music), act  
% c. write, compose (music) paint (a picture), draw, etch, sew, knit, crochet,
% weave, spin, cook, bake2  
% d. type, print, photocopy, dictate, record, film  
% e. eat, drink, chew, smoke  
% f. sow, plough, harvest, weed, hunt  
% g. wash, iron, mend, darn, clean, sweep, dust, hoover, paint (apply paint
% to), embroider, tidy up  
% The verbs all have a pronounced manner component in their meaning, and fairly
% circumscribed selection restrictions. Hence the content of the phantom object is
% more or less predictable. It will correspond to the literal rather than
% metaphorical meaning of the verb (e.g. read written or printed material rather
% than, say, the stars or coffee grounds) and may be further restricted in usage
% (e.g. the understood object of intransitive clean in he is cleaning is the interior of
% a house, rather than a car, shoes or teeth; that of mend is clothes rather than
% electric gadgets or roads).3
% Aspectually, intransitive predicates with these verbs pattern together, as atelics,
% with VPs consisting of transitive verbs + bare NP objects [−DELIMITED
% QUANTITY], whereas transitive verbs + quantized object yield telic VPs.

% *Ahringberg (2015: 6)*  
% > In cases where the omitted complement is indefinite, the focus is placed on the action
% which the predicate denotes (Fillmore, 1986, p. 96), and some other verbs that allow this type
% of null instantiation include clean, drink, embroider, hunt, iron, read, sing, study, teach and
% write (Levin, 1993, p. 33). It should be stressed that compared to typical intransitive verbs, such
% as those mentioned in the introduction section, there is always an implied object involved,
% which could be interpreted as representing either the word “stuff” or “something” as exemplified
% in (9) (Fillmore, 1986, p. 95), or a more specific concept which is generally conceived and
% linked with the verb, such as dinner in (10).
% (9) He’s too stressed out to be able to eat {stuff}.

% *Yasutake (1987: 50)*
% cita Munro (1982) --> nell'uso intransitivo, "the general action is of more interest than the specific unspecified object"

% *Yasutake (1987: 52)*  
% > verbs with decategorized objects thus share a common property with objectless transitive,
% viz. the communicative intent of the speaker in both is to provide information concerning the
% subject by way of emphasizing the action-type. Removal of an object noun phrase is hence
% regarded as an extreme form of de-categorization

% *Melchin (2019: 52-53)*  
% (questa cosa la diceva anche PM: preparare obiezioni!) + dire anche in altro par, ma quale?
% >  Chomsky (1986) cites an observation by Howard Lasnik
% that the meaning of the intransitive use of eat is somewhat different from the transitive use,
% and is specifically something more like dine; [...]  Similarly, Fillmore
% (1986) notes that while one can, and often does, bake a number of foods besides typical
% “baked goods” such as breads and pastries, including potatoes and hams, John is baking
% generally cannot be taken to mean John was baking, for example, potatoes.

% *Lorenzetti (2008: 66)*
% QUI È USATO PER GIUSTIFICARE ANCHE **DEFINITE** DOBJ DROP!
% > As to the factors more directly connected to the domain of discourse, it is worth
% mentioning the topic/focus distinction. The omitted arguments in (11) and the following are
% all highly predictable, and therefore they are not good candidate for focal status, since "the
% focus is that portion of a proposition which cannot be taken for granted at the time of speech,
% the unpredictable and pragmatically non-recoverable element in an utterance" [Lambrecht
% 1994: 207].  
% A sentence topic, by contrast, is usually defined as “a matter of already established current
% interest which a statement is about and with respect to which a given proposition is to be
% interpreted as relevant” [Lambrecht 1994: 119].  
% (11) a. I thought you said your dog doesn’t bite!  
% b. Religion integrates and unifies.  
% Every sentence requires at least one focus, namely an assertion containing new information
% (Chafe 1994). It would be tempting to claim that when objects are omitted, the focus is on the
% activity itself.

% *Kardos (2010: 5-6)*  
% IMPORTANTE CLASSIFICAZIONE DELLE TEORIE E SPIEGAZIONE!!!  
% > When it comes to matters of argument realization, researchers are split into several camps
% regarding the role of the lexicon and that of aspectual notions in this process. As I will discuss
% in section 4 in greater detail, some bolster the so-called Free Argument Projection Hypothesis
% which says that ′′arguments of verbs are projected freely onto syntax, with verbs being
% unspecified for those components of meaning that determine argument expression5′′
% (Rappaport Hovav and Levin 2005: 275). Others attribute argument realization patterns purely
% to aspectual properties such as incremental theme and measure. Unlike the former two
% ' camps' , Levin and Rappaport Hovav strongly reject the Free Argument Projection Hypothesis
% and also question the sole role of aspect in argument expression. Their theory is
% fundamentally lexically-based as is seen below in the exposition of the system. [...]  
% << THE ARGUMENT-PER-SUBEVENT CONDITION: There must be at least one argument XP in the
% syntax per subevent in the event structure.  
% (Rappaport Hovav and Levin 2001: 779) >>  
% The popularity of this condition is apparent from the fact that it has been accommodated in
% the work of numerous researchers, e.g Goldberg (2005)6, Grimshaw and Vikner (1993), van
% Hout (1996), inter alia. An important consequence of this is that verbs denoting simple events
% must project a single argument, whereas verbs expressing complex eventualities alway occur
% with at least two obligatory elements in the syntax. The former class is represented by activity
% type verbs. For instance, run lexicalizes a single event in its lexical representation, which in
% turn yields the obligatory appearance of a single argument (i.e. an agent) in subject position.
% A prototypical verb exemplifying the latter class is the causative verb, break, which is
% associated with two subevents, one being the causing event and the other one, the event
% caused by the causing event. The causing event is a simple activity, while in the second
% subevent an externally caused result state is brought about.

% *Sugayama (2007: 1)*  
% PRAGMATIC AND DISCOURSE FACTORS?
% > researchers have proposed that causative verbs obligatorily express the argument that
% undergoes the change of state in all contexts (Browne 1971; Brisson 1994; van Hout 2000; Ritter and
% Rosen 1996, 1998; Rappaport Hovav & Levin 1998). This generalisation is too strong to accommodate
% real fact [...]  
% counterexamples may be accounted for by the following Principle of Omission under Low Discourse
% Prominence in Goldberg’s (2005) Construction Grammar. [...]  
% (2) Principle of Omission under Low Discourse Prominence: Omission of the patient argument
% may be possible when the patient argument is construed to be de-emphasized/unprofiled in the
% discourse vis-а-vis the action [...]  
% That is, omission is possible when the patient argument is (or focal) in the discourse, and
% NOT TOPICAL the action is particularly emphasised [...]  
% the attention can be shifted away from the (definite) argument in favour of the action, if the action is sufficiently emphasised due to the
% TOPICAL patient argument being present and salient in the IMMEDIATE NON-LINGUISTIC CONTEXT

% *Mittwoch (2005: 3)* 
% TRA ACTIVITY E NOUN INCORPORATION
% > In the early days of generative grammar the intransitive version of these verbs
% was derived by a transformation deleting the object. Today it is generally
% thought that the objects of the verbs concerned in this alternation, though
% appearing in the lexicon, need not be projected in the syntax.4
% Thus an influential paper by Grimshaw (1993) draws a distinction between
% structural and content components of meaning. The objects of change-of-state
% verbs are structural, and must be projected; the objects of activity verbs are
% content arguments, and are in principle optional (subject to certain ill-
% understood restrictions). The reason is that change-of-state verbs have a
% complex event structure involving something like x cause y to change state,
% where y represents the object, whereas activity verbs have a simple structure: x
% act. Additional components of meaning that distinguish between different verbs
% in each structure are content components in this theory.

\paragraph{Object drop as noun incorporation}
queste teorie sostengono che l'oggetto non sia rappresentato sintatticamente! (vero?)

% *Dvorak (2017: 119)*  
% > An unorthodox approach to INO is presented by Martı́ (2011), who is primarily moti-
% vated by defeating the view that English INO are purely pragmatic in nature (Groefsema
% 1995, a.o.). Martı́ argues that the INO of verbs like eat, bake, smoke, drink, read, write,
% hunt, cook, sing, carve, knit, weed, file, write, etc. are grammatically represented, number-
% neutral nouns, not too different from nouns incorporating into verbs in noun-incorporating
% languages. Her argumentation is based on the fact that English verbs with implicit indefi-
% nite objects are generally atelic (except for John ate for/in and hour ), and they describe
% conventional, name-worthy, institutionalized, habitual activities – just like verbs that have
% undergone noun-incorporation (cf. Mithun 1984, Dayal 2011b:164). I get back to the ties
% between INO derivation and noun-incorporation in 5.2.3.

% *Martì (2015: 453)*  
% NOUN INCORPORATION! TORNA SPESSO, MERITA UN PARAGRAFO A PARTE?  
% > That there is a notional, even if unuttered, object (indicated in parentheses) in
% the interpretations of the object-less versions of these sentences is uncontrover-
% sial. The controversial question is whether there is an object at any point in the
% grammatical derivation of the object-less sentences, despite its lack of phonological
% realization.
% Carston (2004), Groefsema (1995), Hall (2009), Iten et al. (2004), Recanati
% (2002) and Wilson and Sperber (2000) have suggested that, in at least some of
% the uses of these seemingly object-less sentences, there is indeed no object at any
% level of linguistic representation. Instead, that object is provided for pragmatically.
% In these cases, given appropriate pragmatic pressures, language users enrich gram-
% matical interpretations in such a way as to provide the missing material. Here I
% argue that an object is, on the contrary, part of the linguistic representation of
% sentences such as (37)–(41) in all of their uses. This object, however, is not a
% run-of-the-mill syntactic object, but an object whose properties are those of nouns
% in noun incorporation constructions found in languages such as West Greenlandic,
% Frisian, and many others. The argument rests on the following logic: if these
% notional objects behave like linguistic category X (i.e. incorporated nouns), then
% the null hypothesis is that the notional objects are themselves instantiations of that
% linguistic category X. The only difference between the objects of interest here and
% those that appear in noun incorporation is that the former, but not the latter, are
% phonologically null.14

% *Martì (2015: 454)*  
% > Noun incorporation is a word formation process whereby a verb and a noun (usually,
% its object) are put together to form a verb. [...]  
% When the noun and the verb are put together, a number of associated effects follow.
% Importantly, the object noun loses its status as a regular syntactic object. This can
% be seen in the fact that the nouns in these constructions must always appear bare

% *Martì (2015: 455)*  
% ALTRE IMPORTANTI CONSIDERAZIONI SULLA NOUN INCORPORATION  
% > In West Greenlandic, an erga-
% tive language, objects of transitive verbs and subjects of intransitive (unergative) verbs
% are marked with ABSOLUTIVE Case, and subjects of transitive verbs are marked with
% ERGATIVE Case. In (42)b, where the noun has incorporated, the subject is marked
% with ABSOLUTIVE, not ERGATIVE, Case.
% Word order is also affected, so that incorporated nouns always precede the
% verb, independently of the canonical position of regular syntactic objects in the
% language in question. [...]  
% There are a number of semantic characteristics that are associated with
% noun incorporation. The incorporated noun is interpreted indefinitely and
% non-specifically: compare the a and b examples in (42)–(44). The indefinite,
% non-specific semantics of the nouns that participate in noun incorporation is a
% widely noted fact in the literature (see Carlson, 2006; van Geenhoven, 1998; Gerdts
% and Marlett, 2008; Mardirussian, 1975, p. 386; Mithun, 1984; de Reuse, 1994;
% Spencer, 1991; Sullivan, 1984, among others). A second cross-linguistically stable
% semantic property of incorporated nouns is that they typically take narrow scope
% with respect to other operators in the sentence 

% *Martì (2015: 456)*  
% > Verbs that are incorporated into typically designate name-worthy, typical activi-
% ties. For example, Axelrod (1990, p. 193) says that ‘ ... incorporation provides the
% lexicalized expression of a typical activity’. And Mithun (1984, p. 848) says that
% ‘some entity, quality or activity is recognized sufficiently often to be considered
% name-worthy in its own right’.

% *Martì (2015: 457)*  
% > It is well known that implicit indefinite objects are interpreted as non-specific
% indefinites with plain existential import (Bresnan, 1978; Dowty, 1981; Fillmore,
% 1969, 1986; Fodor and Fodor, 1980; Gillon, 2012; Mittwoch, 1980; Shopen, 1973;
% Thomas, 1979). In (49), the speaker does not convey the idea that something in
% particular has been eaten today, the same way that that would not have been conveyed
% had s/he said ‘John has already eaten food/a meal today’:
% (49) John has already eaten today.

% *Martì (2015: 458)*  
% > Another property that implicit indefinite objects share with incorporated nouns
% is that they take obligatory narrow scope with respect to other operators in the
% sentence, such as negation or intensional verbs (Fillmore, 1986; Fodor and Fodor,
% 1980; Mittwoch, 1982; Wilson and Sperber, 2000): [...]  
% Again like incorporated nouns, implicit indefinite objects are semantically
% number-neutral. As far as I know, this fact has not been noted before. Thus, if (64)
% is true, then it is immaterial whether John is smoking half, one or many cigarettes:

% *Martì (2015: 459)*  
% SIGNIFICATO CONVENZIONALIZZATO DEGLI USI INTRANSITIVI  
% > When verbs such as eat, drink, write, etc. take on implicit indefinite objects, they
% typically give rise to ‘conventionalized’ meanings, in a similar fashion to incorpo-
% rated nouns. Thus, if John is eating, then he can only be eating edible things, things
% that are conventionally eaten. Whereas it is perfectly possible to say that John is eat-
% ing his bed, strange as that may be, when one says that John is eating, one means
% that he is eating things that are normally eaten. 

% *Martì (2015: 461)*  
% QUESTO CONSENTE DI DISTINGUERE INDEFINITE DROP DA DEFINITE DROP!!!
% > Frisian allows us to test the predictions of the proposed analysis in an interest-
% ing way. That’s because this language has both implicit indefinite objects and noun
% incorporation, as we saw in Section 3.1. The fact that Frisian is such a close relative
% of English allows for a controlled comparison between the two languages.
% In fact, this is precisely the argument made by Dyk (1997) regarding the nature
% of what he calls ‘detransitivized’ verbs (in Dowty’s 1989 terminology), which are
% the verbs that take implicit indefinite indefinite objects, such as English eat. [...]  
% This means that unergatives, detransitivized and incorporated-into verbs form a
% natural class. Then, there a number of restrictions on the type of verb that can be
% incorporated into in Frisian and these very same restrictions are strikingly observed
% for detransitivized verbs. For example, only verbs that select for a Patient object
% allow incorporation. The verbs corresponding to English notice, hate and know don’t
% take Patients as objects and do not allow noun incorporation or detransitivized uses
% (Dyk, 1997, pp. 95, 108):

% *Martì (2015: 463)*  
% QUESTO SIGNIFICA CHE NON DIPENDE DALL'OBJ=PATIENT, MA DALL'AGENT AFFECTEDNESS?  
% (conferma: esiste bird-watching, soggetto Agent, Obj != Patient)  
% > A verb like know never has a volitional subject, and, accordingly, in Frisian this verb
% never allows incorporation or detransitivization.

% *Martì (2015: 466)*  
% > A common denominator in various proposals for the analysis of noun incorpo-
% ration is the semantic function that is assigned to the incorporating noun: that of
% restricting the internal argument of the verb. I build upon Chung and Ladusaw’s
% (2004) implementation of this idea.
% Chung and Ladusaw (2004) propose that there is a mode of composition called
% Restrict that applies in cases of formal incorporation. This rule applies in cases where
% a predicate is being combined with a property. The property argument is interpreted
% as a restrictive modifier of the predicate. The operation does not reduce the predi-
% cate’s degree of unsaturation; i.e. semantically, the predicate is still missing an internal
% argument. Restrict is illustrated in (87):

% *Martì (2015: 467)*  
% > Any pragmatic approach must consider the cluster of properties we discussed in
% the previous sections as accidental. Nothing in this type of approach predicts that
% implicit indefinite objects always take narrow scope, are number-neutral, or are
% interpreted indefinitely and non-specifically.

% *Martì (2010: 7)*  
% IN RISPOSTA A CHI DICE CHE NON C'È DIFFERENZA TRA DEFINITE E INDEFINITE?  
% > Notice the difference in meaning between incorporated and non-incorporated
% objects in these languages. For example, whereas in (15), the grammatical and
% notional object is interpreted specifically, in (12) the notional object is interpreted
% indefinitely. The indefinite, non-specific semantics of incorporated nouns is a widely
% noted fact in the incorporation literature (see Mithun 1984, Sullivan 1984, de Reuse
% 1994, Spencer 1995, etc.).

% *Martì (2010: 7)*  
% >  the semantic properties of implicit indefinite objects
% in languages like English are in fact the same as the “cross-linguistically stable
% properties of the semantics of incorporation” (in Farkas and de Swart’s 2003 and
% Carslon’s 2006 terminology). After presenting the case for semantic incorporation, I
% show that, formally, implicit indefinite nouns undergo compound noun incorporation,
% not classificatory noun incorporation.

% *Yasutake (1987: 51)*  
% OBJECT DE-CATEGORIZATION AND NON-INDIVIDUALITY  
% parla di object incorporation (John did some deer-hunting) and "the use of an explicit but non-individuated object" (John hunted deer)

% *Melchin (2019: 56)*  
% > 3.2.2  Omitted objects are not present in the syntax

% *Bourmayan & Recanati (2013: 125)*  
% drink alcohol --> alcohol-drink --> drink(2)  
% write with a pen --> pen-write --> write(2)  [questa è una mia ipotesi, devo ragionarci su]  
% > Through free enrichment, a lexical item (or, for that matter, a complex phrase) can be
% understood in a more specific sense than the sense it literally has. For example,
% intransitive ‘drink’ is often understood in the specific sense DRINK ALCOHOL [...]
% can be achieved by morphosyntactic means. In languages such as
% West Greenlandic, some verbs can undergo a process of ‘incorporation’ of their direct
% objects which yields noun-verb combinations behaving like single, verbal, morphological items [...]
% According to van Geenhoven (1998), the incorporated object of the verb does not
% correspond to a genuine argument: it does not denote an individual of type e, but
% rather a property of type <e,t>.

% *Naess (2007: 129)*
% ALTRI QUI NEL PARAGRAFO AVEVANO DETTO CHE EAT E DRINK SONO VERBI PROTOTIPICI
% > Two main proposals have been made as to
% how to integrate such verbs into a broader theory of valency in general: either the
% “intransitive” and the “transitive” variants of a verb are counted as two different
% lexical entries with different argument structures, or the lexemes in question have
% as part of their lexical entry that an indefinite direct object need not be overtly ex-
% pressed. The problem, then, becomes one of identifying the set of verbal lexemes
% which have the property of being able to participate in this “alternation”, and, if pos-
% sible, providing a semantic characterisation of the verbs belonging to this set.
% As Marantz (1984 :192) puts it, “it is an interesting and important problem to
% characterize the transitive verbs that permit indefinite object deletion”. Marantz
% assumes the ingestive verbs, of which he gives ‘eat’, ‘drink’, and ‘learn’ as examples,
% to be the core candidates for IOD, recurring in language after language. In other
% instances, he claims, “the alternation is created not by a productive lexical rule but
% by generalization by analogy with certain core verbs exhibiting the alternation”
% (Marantz 1984 :193).
% (NOTA MIA: eat, drink e learn sono affected agent verbs! v. naess 2011: 413)

\paragraph{EATING AND DRINKING} 
EATING AND DRINKING

% *Naess (2011: 413)*
% > Verbs referring to acts of eating and drinking show a crosslinguistic tendency to behave in ways
% which distinguish them from other verbs in a language. Specifically, they tend to pattern like
% intransitive verbs in certain respects, even though they appear to conform to the definition of
% ‘prototypical transitive verbs’. The explanations which have been suggested for this behaviour fall
% into two main categories: those referring to telicity or Aktionsart, and those referring to the fact
% that such verbs describe acts which have ‘affected agents’, i.e. they have an effect on their agent as
% well as on their patient participant. The latter observation has further led to reexaminations of the
% notion of transitivity in general. [...]
% the notion of transitivity as a prototype concept. This idea was first articulated by Hopper and Thompson (1980)

% *Petho & Kardos (2006: 30)*  
% INVECE È CHIARO! LOCK NON È UN AFFECTED-AGENT VERB
% > It does not become clear
% why e.g. eat can be used intransitively, as opposed to lock, which requires its object to appear on
% the syntactic surface, even though its relevant selection restrictions do not seem to be any less
% specific (eat requires food as its object, whereas lock requires an object that has a lock, e.g. a car
% or a door).

% *Newman & Rice (2006: 5-6)*  
% > Van Valin and LaPolla (1997:112) explicitly remark that
% “...eat is not inherently telic, unlike kill and break; hence it must be
% analyzed as an activity verb, with an active accomplishment use”. For
% them, the ‘activity verb’ use (He ate, He ate spaghetti for ten minutes) is
% the ‘basic’ meaning of EAT.

% *Newman & Rice (2006: 14)*  
% > There is proportionately more intransitive usage with DRINK than there is
% with EAT . The difference is arguably influenced by the existence of
% specialized meanings associated with the intransitive (the ‘specific category
% indefinite’ kind of interpretation à la Huddleston and Pullum 2002: 303-
% 305 or Rice 1988). In the case of EAT the specific interpretation is ‘meal’,
% whereas with DRINK it is ‘alcoholic beverage’ (especially when consumed
% in an habitual and/or excessive manner). This use of intransitive DRINK is a
% very familiar one in casual conversation (some examples from sBNC are
% All they do in that house is drink and smoke; Because her daddy drinks in
% there in the pub...; He bought a bottle of brandy at the first liquor store he
% found and he began to drink), reflecting the prominence of alchol
% consumption as a topic of discourse. Comparing EAT and DRINK in this way
% is instructive for demonstrating the kind of variation that can exist between
% lexical items, even those which define and exhaust a class (cf. Levin 1993:
% 213-214). 

% *Naess (2011: 414)*  
% IMPORTANTISSIMO ELENCO DI CARATTERISTICHE + ESEMPI ESOTICI  
% > Types of ‘intransitive behaviour’ exhibited by eating and drinking verbs include (but
% are not restricted to) the following

% *Kardos (2010: 1)*  
% IMPORTANTISSIMO!!! SI RICOLLEGA ALL'AFFECTED-AGENT ACCOUNT (riflettere, ma penso di sì!)  
% >  it is often argued that pseudo-transitives, some core examples of which are eat
% and drink, exhibit both transitive and intransitive properties as the roles that their arguments
% play in the denoted event show overlapping properties. 

\paragraph{DOBJ IS THERE IN SYNTAX}
questa teoria vede l'oggetto sintatticamente rappresentato! (recuperare altri pezzetti)\\
trovare il modo di separare questo paragrafo e quello sull'incorporation in una sezione sintassi vs semantica? o anche rappresentato non rappresentato? oppure syntactic transitivity, semantic transitivity?

% *Cummins & Roberge (2004: 2)*
% > These possibilities cannot be attributed solely to lexical properties of the verb; if this
% were the case, certain verbs would always be able to appear without their objects
% regardless of the construction or discourse context, and others would never be able to
% appear without an object. As we will show, this is not the case. Rather, following Roberge
% (2003), we propose that null or implicit objects can be attributed to a Transitivity
% Requirement (TR) just as null subjects are ultimately due to the EPP. Recoverability for
% the EPP is morphologically based, as is evident in null subject languages, while
% recoverability involving the TR may also be semantically and pragmatically based; as we
% will show below, such recovery may be based on information derived from the verb's
% lexical semantics and Generalised Conversational Implicatures (formalised as in Levinson
% 2000) involved in the interpretation of reduced nominal forms. The factors that contribute
% to licensing superficial intransitivity—the absence of an overt object—may include lexical
% semantics, functional elements, discourse factors, and trans-clausal structural elements.
% This view is supported by a comparative study of null object possibilities in French and
% English. [...]  
% The concept of transitivity has been interpreted as a continuum in certain works, and a
% distinction has been proposed between syntactic transitivity and semantic transitivity; see,
% among many others, Blinkenberg (1960), Desclés (1998), Hopper et Thompson (1980),
% Lazard (1994). Surprisingly little is ever said about the object position itself. The
% hypothesis in Roberge (2003) is that there exists a Transitivity Requirement (TR),
% whereby an object position is always included in VP, independently of the lexical choice
% of V. The empirical motivation of this hypothesis is the well documented evidence (see in
% particular Blinkenberg (1960), Larjavaara (2000)) that any “transitive” verb has the
% potential to appear without a direct object and any “unergative” verb has the potential to
% appear with a direct object. To account for these facts, there must be a mechanism to
% generate the direct-object position, either optionally or obligatorily. The TR represents the
% second, more restrictive, possibility and conveys the concept of transitivity as a property
% of the predicate (the VP), rather than as a property of the lexical content of V. The TR is
% the internal-argument counterpart to the EPP.

% *Cummins & Roberge (2004: 3)*
% > For the purpose of our
% discussion, we define unexpressed objects interpretatively: there is an x such that x is (1)
% phonologically null, (2) involved in the event denoted by the VP, and (3) not an external
% argument. [...]  
% Two recent studies—Larjavaara (2000) on French and García Velasco & Portero
% Muñoz (2002) on English. [...]  
% The two agree that indefinite or generic null objects do not have a
% contextually available referent. GP point out that generic null objects can give rise to an
% activity rather than an accomplishment reading of the verb; L notes that null objects can
% focus attention on the activity. Both point out that the lexical characteristics of the verb
% can help to identify the referent of the null object. [...]  
% In a third study, Goldberg (2001) investigates unexpressed objects of causative verbs
% (those that entail a change of state in the patient argument) in English. She concludes that
% the option of leaving these arguments unexpressed depends largely on factors relating to
% information structure: the unexpressed object is typically neither topical nor focal, and the
% verb is emphasized somehow, by being iterative or generic, by being contrasted with
% another verb, or by having a narrow focus.

% *Cummins & Roberge (2004: 4)*
% > All of these authors implicitly or explicitly adopt the position that the missing
% argument is not syntactically represented: syntactically the verb is intransitive. In a
% generative framework, this position finds a counterpart in Rizzi (1986:509-510), who
% proposes that both the arbitrary third-person human interpretation, meaning “people in
% general” or “some people”, and the prototypical-object interpretation, where the verb's
% lexical semantics identify the object, are available lexically to saturate the argument's
% theta role and block projection. Thus, the verbs are intransitive in syntax. The absence of a
% syntactic object explains why, in Rizzi’s account, the type of sentence exemplified in (13)
% is impossible in English: there is no object that can bind the anaphor or be modified by the
% adjective. However, such sentences are grammatical in Romance; hence several accounts
% (Rizzi 1986; Authier 1989; Roberge 1991) posit a syntactically present null object.  
% (13)a.b.Ce gouvernement rend __ malheureux.  
% *“This government makes __ unhappy.”  
% Une bonne bière reconcilie __ avec soi-même.  
% *“A good beer reconciles __ with oneself.”  
% Under the TR, the object position is projected and the verb remains transitive in syntax
% in both English and French. Although we do not find sentences like those in (13) in
% English (as shown by the ungrammaticality of the glosses), there is nonetheless evidence
% that a null object has an effect on syntax in both English and French.3 
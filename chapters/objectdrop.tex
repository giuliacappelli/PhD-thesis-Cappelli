\setchapterpreamble[u]{\margintoc}
\chapter{Indefinite object drop} % Object-dropping verbs
\labch{objectdrop}

As mentioned in \refch{intro}, this thesis is about indefinite implicit objects. What is "indefinite" about them? In what sense can they be considered "implicit"? And ultimately, what is objecthood itself? This Chapter will answer these questions in reverse order, from the most general to the most specific one. In \refsec{theory_transitivity} I will make reference to the transitivity continuum, as defined by \textcite{HopperThompson1980} and further explored by later literature. In \refsec{theory_def_vs_indef} a crucial distinction will be made between definite and indefinite object drop, following \textcite{Fillmore1986} and subsequent works. The nature of \textit{indefinite} object drop will be finally described in \refsec{theory_defindefinite} and \refsec{theory_entries}, and a working definition (for the purposes of this thesis) will be provided in \refsec{theory_workingdef}.

\section{Transitivity as a prototype} \labsec{theory_transitivity}

% \subsection{Transitivity in Hopper \& Thompson (1980)} \labsec{theory_ht1980}
School kids everywhere are used to call "transitive" the verbs which take an overt direct object, with reference to the syntax. In a traditional semantic definition, a clause is deemed "transitive" if it describes an event where the action performed by an Agent "passes over"\sidenote{Hence the name of the concept of transitivity, from Latin \textit{transire} 'to go over'.} to a Patient, which usually undergoes some kind of transformation.\\
Going beyond these naive definitions, but still capturing their spirit, \textcite{HopperThompson1980} first proposed an account where transitivity is interpreted as a scalar concept whose strength depends on several parameters, or, to use more modern terminology, as a prototype category \parencite{Naess2007}. In particular, they identified ten parameters \parencite[252]{HopperThompson1980}, reported almost \textit{verbatim} in \reftab{ht1980_parameters}.

\begin{table}[htb] % the "htb" makes table env unfloaty
\caption{\textcite[252]{HopperThompson1980} defined transitivity as a prototype concept determined by ten parameters.}
\labtab{ht1980_parameters}
\begin{tabular}{rl|ll}
 & & \textbf{high transitivity} & \textbf{low transitivity} \\
 \hline
A. & \textbf{Participants} & 2+ (Agent and Object)  & 1 participant  \\
B. & \textbf{Kinesis} & action  & non-action  \\
C. & \textbf{Aspect} & telic  & atelic  \\
D. & \textbf{Punctuality} & punctual  & non-punctual  \\
E. & \textbf{Volitionality} & volitional  & non-volitional  \\
F. & \textbf{Affirmation} & affirmative  & negative  \\
G. & \textbf{Mode} & realis  & irrealis  \\
H. & \textbf{Agency} & A high in potency  & A low in potency   \\
I. & \textbf{Affectedness of O} & O totally affected  & O not affected  \\
J. & \textbf{Individuation of O} & O highly individuated  & O non-individuated  
\end{tabular}
\end{table}

These parameters are potentially active in all languages, but languages may differ from one another with respect to the actual subset of parameters they select as necessary criteria for transitivity. This depends on the "recursivity" \parencite[29]{Naess2007} of prototypical concepts, which assign membership in a category (in this case, transitive clauses) on the basis of attributes which are prototype concepts themselves \parencite[61]{taylor1995linguistic}.\\
Parameters A, B, E, F, and G from \reftab{ht1980_parameters} are self-explanatory. Parameter C (telicity) will be discussed in more detail in \refsec{telicity}. Parameter D (punctuality) refers to the phase between inception and completion of an action, which is non-existent with verbs like \textit{to kick} and noticeable with verbs like \textit{to carry}. Parameter H (agency) separates animate and inanimate subjects. Parameter I (affectedness of the object) determines that sentences like \textit{I drank up the milk} are more transitive than sentences like \textit{I drank some of the milk}, since the milk is only partially affected by the drinking in the second sentence. Finally, parameter J (individuation of the object) refers to the distinctness of the object both from the Agent and from the background, as summarized in \reftab{ht1980_individuation} \parencite[253]{HopperThompson1980}.

\begin{table}[htb] % the "htb" makes table env unfloaty
\caption{\textcite[252]{HopperThompson1980} defined transitivity as a prototype concept determined by ten parameters.}
\labtab{ht1980_individuation}
\begin{tabular}{c|c}
 \textbf{individuated} & \textbf{non-individuated} \\
 \hline
proper & common \\
human, animate & inanimate \\
concrete & abstract \\
singular & plural \\
count & mass \\
referential, definite & non-referential
\end{tabular}
\end{table}

The individuation parameter is the most controversial among the ten proposed ones, as observed by \textcite[128]{comrie1989language} and later on by \textcite[18]{Naess2007}. According to what has come to be known as "Comrie's generalization", in prototypical transitive clauses both animacy and definiteness are high in the Agent and low in the Patient, \textit{contra} \textcite{HopperThompson1980}. The weak argumentation \textcite{HopperThompson1980} provide in favor of the individuation parameter is that speakers would be more likely to focus on the Patient in \textit{I bumped into Charles} than in \textit{I bumped into the table}, since bumping is more likely to affect human beings than tables. \textcite{comrie1989language} makes a much more compelling point with reference to cross-linguistic typology, basing the generalization on the animacy hierarchy and on referential case-marking (which I will not discuss here, since it would lead me too far from the intended argumentation). Later literature \parencite{Naess2007, kemmer1993middle, Kardos2010, Naess2009} reinforced this point by assuming that prototypical transitive events are described by verbs whose subject and object are maximally distinct from a semantic point of view.\\
To sum up, the terse summary by \textcite[15]{Naess2007} clearly shows the relation between the naive definitions of transitivity and the ten-parameter account by \textcite{HopperThompson1980}. A prototypical clause is understood to describe an event such that:
\begin{itemize}
    \item a volitional Agent (E, H)
    \item performs an action (B)
    \item with a tangible, lasting effect on a Patient (A, I, J),
    \item and it is presented as real and concluded (C, D, F, G).
\end{itemize}

\textcite[78]{Lorenzetti2008} provides a tighter cluster of parameters, arguing that only a subset of the ones proposed by \textcite{HopperThompson1980} are truly relevant. In particular, the author ditches the criteria relative to the transitive event being real and concluded (C, D, F, G), and only keeps agentivity, affectedness, and individuation of the object among the other groups of criteria.


\section{Definite \textit{vs} indefinite drop} \labsec{theory_def_vs_indef}

In \refsec{theory_transitivity} I introduced transitivity as a prototype concept depending on a cluster of parameters and, specifically, involving an Agent acting on a Patient. But what about the utterances where events of this kind are expressed without an overt syntactic object? In this Section I will provide an account of the literature on the matter.

\subsection{Either definite or indefinite: discrete accounts} \labsec{theory_discrete}

\paragraph{Introduction}
Verbs behaving intransitively are, using \textcite[191]{rutherford1998workbook}'s words, "a mixed bag". Consider, for instance, the examples in \ref{intro}. The sentence in \ref{intro1} features a typical intransitive verb, describing an event where the subject is \textit{not} performing an action with effects on some Patient. The sentence in \ref{intro2} also describes an event where the subject is not volitionally acting on a Patient, but it is nevertheless clear that there has to exist something that John knows (unlike in \ref{intro1}, where there cannot be something that John sleeps). The sentence in \ref{intro3} has an Agent acting volitionally on a Patient, which is however not instantiated syntactically as an overt direct object.

\ex. \label{intro} \a. \label{intro1} John slept.
\b. \label{intro2} John knew.
\c. \label{intro3} John ate.

There is a clear similarity between \ref{intro2} and \ref{intro3}, as opposed to \ref{intro1}. They both require a Patient/Theme semantically \parencite[510]{Somers1984}, and they both surface as object-less syntactically. Quoting \textcite[48]{Yasutake1987}, "they are different from pure intransitives in that the action will not be complete without some lexically implied (but unspecified) object". Oddly, some literature (\textcite{BourmayanRecanati2013} a.o.) does not interpret such verbs as transitive-become-intransitive verbs via the omission of the direct object, but as intransitive-made-transitive verbs. Such an interpretation is totally counter-intuitive and it goes against the generally-accepted tenet that a core feature of so-called "intransitive verbs" is that they have no object slot available in the syntax.\\
There is, however, a crucial difference between these \ref{intro2} and \ref{intro3}. Native speakers of English understand that they have to be provided some context in order to understand \ref{intro2} \textemdash what is it exactly that John knew? On the contrary, \ref{intro3} can be interpreted to mean that John had a meal at a certain moment in time, without resorting to any additional context. This distinction was captured and defined by \textcite{Fillmore1986} (building upon \textcite{fillmore1969types, Allerton1975}), the established seminal work on the distinction between so-called "definite" and "indefinite" object drop (here represented by \ref{intro2} and \ref{intro3}, respectively).

\paragraph{Fillmore's account}
\textcite[96]{Fillmore1986} distinguishes between Indefinite Null Complements (hence, INC) and Definite Null Complements (hence, DNC) by testing "whether it would sound odd for a speaker to admit ignorance of the identity of the reference of the missing phrase". So, making reference to \ref{intro} again, there would be no issue with saying "John ate. I wonder what he ate.", but it would be quite odd to say "John knew. I wonder what he knew.". Thus, the missing object in \ref{intro2} is a DNC, while the missing object in \ref{intro3} is an INC. Fillmore than splits INCs into two sub-groups based on whether the omitted object is "of considerable generality" or "requiring the specification of various degrees of semantic specialization". The examples in \ref{fillmore} \parencite[96-97]{Fillmore1986} show increasing degrees of, using his words, "semantic specialization". In \ref{fillmore1}, the subject cannot perform the very act of eating or drinking, regardless of the actual ingested item. In \ref{fillmore2} something specific was eaten, but its specific nature is irrelevant inasmuch the speaker is referring to eating as the act of having a meal. In \ref{fillmore3} the omitted object is referring not just to any drinkable liquid, but to alcohol specifically. Finally, in \ref{fillmore4} something very specific was baked by the subject, but this information is backgrounded to focus on the activity itself (I will come back to this in \refsec{theory_incorporation}).

\ex. \label{fillmore} \a. \label{fillmore1} When my tongue was paralyzed I couldn't eat or drink.
\b. \label{fillmore2} We've already eaten.
\c. \label{fillmore3} I've tried to stop drinking.
\d. \label{fillmore4} I spent the afternoon baking.

However, as it will be shown throughout this Chapter, this secondary division of INCs into subgroups does not have to be a binary theoretical distinction. On the contrary, it follows from several finer-grained considerations on the nature of INCs and the factors allowing them. Moreover, this binary division actually opens the door to discussions about a DNC-INC continuum (more details on this in \refsec{theory_continuous}). Specifically, how is "semantic specialization" different from the "knownness" of the object in DNC constructions \parencite[525]{Eu2018}? An answer can be found in \textcite[218]{Allerton1975}, where the case is made that semantically specialized INCs (just like any INC) refer to a category of individuals, while DNCs refer to specific instances of a given category.\\
Going back to the main distinction between INCs and DNCs, finally, \textcite{Fillmore1986} formally defines the former as objects whose "referent's identity is unknown or a matter of indifference" and the latter as objects whose referent "must be retrieved from something \textit{given} in the context". This context "has either to be given linguistically, in the preceding context, or extralinguistically, in the situational context" \parencite[13]{StarkMeier2017}.

\paragraph{Other accounts}
Several other researchers made use of the distinction between definite and indefinite implicit objects brought to the fore by \textcite{Fillmore1986}, providing slightly different definitions which capture different aspects of the phenomenon. \textcite{Allerton1975}, then further developed by Fillmore, distinguishes between "contextual omission" and "lexical omission", which respectively refer to Fillmore's DNCs and INCs. \textcite{CumminsRoberge2004} distinguish between "internally-licensed null objects" (INCs) and "referential null objects" (DNCs). According to \textcite[30]{PethoKardos2006}, INCs "receive an \textit{existentially quantified} interpretation", while DNCs are "interpreted anaphorically and must therefore have an appropriate antecedent in context to make sense" (\textcite{Fillmore1986} himself referred to this in the title itself, "Pragmatically Controlled Zero Anaphora"). \textcite[13]{Medina2007}, the foundational work upon which I am basing my own model of the indefinite object construction, sees DNCs as "implicit objects whose particular meaning can be recovered from the preceding discourse or disambiguating physical context" and INCs as "implicit objects whose meaning is recoverable only from the verb in the sentence". Here the focus is all on recoverability, and the author goes on to show that semantic recoverability can be a reliable predictor of object drop in INC sentences. \textcite[293]{Liu2008}, following \textcite{Garcia-VelascoMunoz2002}, takes the shift away from lexical semantics and onto aspectual territory. In particular, the point is made that INCs involve a change of focus "from the object in the transitive use to the activity (the verb) itself in the intransitive use" (an idea that I will discuss in full detail in \refsec{theory_incorporation}), while DNCs do not determine such a shift.\\
The accounts provided so far are not in conflict with Fillmore's formulation of the problem at hand, nor are they in conflict with the view I am adopting in this thesis in order to provide a probabilistic model of the implicit object construction. Other accounts, on the other hand, are more challenging and deserve a more thorough clarification. Let us consider the most relevant ones for my argumentation.\\
\textcite[55]{TonelliDelmonte2011} argue that, while INCs are "constructionally licensed, in that they apply to any predicate in a particular grammatical construction", DNCs are "lexically specific, in that they apply only to some predicates". Later in this Section (on \refpage{recipes}) and in \refsec{theory_continuous} I will bring evidence in support of the opposite point of view, which is in favor of seeing DNCs as (extra- and intra-linguistically) contextually, not lexically, determined. Moreover, \textcite{TonelliDelmonte2011}'s account is in direct conflict with \textcite[95]{Fillmore1986}, who argues that INCs are "limited to particular lexically defined environments" (such as the object slot of \textit{to eat} and \textit{to read}). In this regard, I side with \textcite{TonelliDelmonte2011}. My probabilistic model of INCs (the results thereof are discussed in \refch{results} and modeled in \refch{model}) will provide strong evidence in support of the idea that any transitive verb can participate in INC constructions, provided the right aspectual, semantic, and pragmatic features. Indeed, as noted by \textcite[216]{HuddlestonEtAl2002}, transitivity is better thought of as a property of verb \textit{use}, rather than a feature of verbs themselves.\\
In a pragmatic (in particular, not lexical) perspective, \textcite[44]{AnderBois} and \textcite[53-54]{Melchin2019} both stay true to Fillmore's original interpretation of DNCs as "pragmatic anaphoras", arguing that DNCs corefer with other referents in the discourse. Moreover, they maintain that INCs lack the possibility of having coreferential interpretations. Fillmore himself \parencite[97]{Fillmore1986} made this point with example \ref{coref}, where \ref{coref2} cannot be considered a proper answer to the question in \ref{coref1}. This is taken to mean that there is no co-reference between the sandwich and the implicit object of \textit{to eat} in \ref{coref2}.

\ex. \label{coref} \a. \label{coref1} What happened to my sandwich?
\b. \label{coref2} *Fido ate.

However, examples can be provided in support of the opposite. \textcite[142-144]{groefsema1995understood} makes use of sentences such as the ones in \ref{groefsema} to argue that INCs can indeed refer to specific individuals, provided sufficient linguistic context.

\ex. \label{groefsema} \a. \label{groefsema1} John brought the sandwiches and Ann ate.
\b. \label{groefsema2} John picked up the glass of beer and drank.

Nevertheless, this account does not disrupt Fillmore's firm foundation. As explained by \textcite[527]{Eu2018}, not even in sentences like \ref{groefsema} do INCs force the identification of a specific referent. What happens, instead, is that native speakers processing INCs in a flexible context of this kind can be induced to understand the missing object as if co-referring to the previously mentioned one. Thus, INCs can grammatically dissociate the mentioned referent from the one implied by the missing object, while this possibility is not active for DNCs (which are always co-referential, regardless of the context). Considering eventualities like this, it really is no wonder that \textcite[110]{Cote1996}, with respect to implicit objects in English, spoke of "murky water" in relation to the distinction between lexically-provided information and context available via world knowledge.



\paragraph{Genre-based implicit objects: a special case of definite object drop} \labpage{recipes}

I will now discuss genre-based implicit objects, a type of DNCs whose very existence goes in favor of DNCs being virtually possible with any verb, provided it appears in a discourse context that is conducive to object omission (\textit{contra} \textcite[55]{TonelliDelmonte2011}, \textit{pro} \textcite{Goldberg2001}). To quote \textcite[175]{RuppenhoferMichaelis2010}, "argument omission can but need not be licensed by a lexeme". This possibility was first observed by \textcite[95]{Fillmore1986}, who acknowledged that in "certain kinds of highly restricted mini-genres" (e.g. instructional imperatives in recipes) the omission of objects and other non-subject complements is not lexically determined (see also \textcite[237]{Haegeman1987}). Crucially, it is not the case that there is a special grammar of recipe contexts that supersedes the actual grammar of the language the recipe uses \parencite{Culy1996, Cote1996}. On the contrary, recipes and other specialized genres just serve to provide an encompassing discourse and world-knowledge context to the listener/reader. Not only that, but genres are so intertwined with argument omission that it is sometimes possible to evoke a genre just by performing the right kind of DNC, as \textcite[159]{RuppenhoferMichaelis2010} exemplify by making reference to the title of a novel by Cynthia P. Lawrence, "Chill $\varnothing$ before Serving $\varnothing$: A Mystery Novel for Food Lovers" (a clear reference to instructional imperatives found in recipes).\\
Many linguistic analyses of DNCs licensed in "mini-genres" focus on recipe contexts \parencite{Ahringberg2015, Garcia-VelascoMunoz2002, Megitt2019, Ruda2014, MassamRoberge1989, Bender, Culy1996}. In particular, \textcite{Culy1996} performed a multiple regression analysis on diachronic sets of contemporary and historical recipes with several predictors, finding that the style of a recipe book and discourse factors are the most important predictors of recipe DNCs.\\
Other authors provided accounts pertaining to a broader spectrum of genres. For instance, in addition to recipes, \textcite{Cote1996} also considers "telegraphese", i.e. the telegraphic register used in telegraphs, memos, and signs. \textcite{Weir2017} focuses on what he calls "reduced written register" in English, i.e. the absence of objects in recipes, instructional/directive imperatives\sidenote{As noted by \textcite[162]{RuppenhoferMichaelis2010}, non-instructional imperatives cannot participate in DNC constructions, as shown in their example \textit{Take *(the money) and run}.}, diaries, text messages, internet-based communication, and similar contexts. The presence of DNCs in text messages and internet-based communication is further explored by \textcite{StarkMeier2017}, focusing on Whatsapp messages. \textcite[304]{Liu2008} mentions instructional languages, such as that found on manuals, warning signs, and product labels. \textcite{Paesani2006} provides a thorough account of object (and subject) drop in special registers (such as recipes, diaries, and headlines), noting clear similarities between DNCs in recipes and the broader phenomenon of Topic drop \parencite[165]{Paesani2006}. In an unconventional account of football language, \textcite{BerghOhlander2016} argue that verbs licensing DNCs are "monotransitives" \parencite[54]{quirk1985grammar} in this sublanguage, since they can only take one argument. Let us consider the examples in \ref{berghohlander}.

\ex. \label{berghohlander} \a. \label{berghohlander1} Iniesta passed (the ball) and Messi finished (the attack) clinically.
\b. \label{berghohlander2} John passed *(the salt) and finished *(his steak).

In \ref{berghohlander1}, the direct objects can be omitted because the two footballers are performing acts that need no further explanation in the football community. In this game, you can only pass balls and finish attacks. Moreover, given the presence of a single ball against many players, match reports are much more likely to DNC the ball rather than the footballers \parencite[167]{RuppenhoferMichaelis2010}. On the contrary, in the probable context of a dinner in \ref{berghohlander2}, it is not possible to say that John just "passed" or "finished", let alone clinically. \textcite[22]{BerghOhlander2016} explain the existence of DNCs in football reports (and, more generally, object omission) as a manifestation of the "principle of least effort" \parencite{zipf1949leasteffort} and also of the Gricean pragmatic maxim of quantity, which compels speakers to avoid being more informative than necessary. However, as \textcite[166]{RuppenhoferMichaelis2010} observed before, "genre-based omissions are never obligatory", since the maxim of quantity (favoring implicit objects) is counterbalanced by the need for informativeness (favoring overt objects).


\subsection{Neither definite nor indefinite: continuous accounts} \labsec{theory_continuous}

The account of genre-determined DNCs offered in \refsec{theory_discrete} opens the door to a broader discussion of \textcite{Fillmore1986}'s distinction between definite and indefinite omitted objects. As argued in \textcite[165]{RuppenhoferMichaelis2010} and \textcite[24]{BerghOhlander2016}, the main factor allowing for an object to be omitted is its recoverability (refer to \refsec{recoverability} for a full discussion), which depends on linguistic aspects as well as on contextual and discourse factors, and on world knowledge too. Focusing on recoverability makes it possible to go beyond the binary distinction between INCs and DNCs provided by \textcite{Fillmore1986} and many others, and also beyond the need to postulate verb-specific object-dropping capabilities. In particular, it paves the way for a non-binary account of object drop, where no clear-cut distinction between two types of omission have to be postulated (something that, in essence, was already thinkable under \textcite{HopperThompson1980}'s assumptions).\\
If recoverability is the cornerstone of object-droppability, and if it is a scalar, or even graded, feature of objects, then it stands to reason that object-droppability itself is a graded phenomenon. Taking a small step forward in this direction, \textcite{AnderBois} posits the existence of "flexible implicit arguments" to explain sentences like \textit{The Giants won $\varnothing$}, whose implicit object has a referent known to the reader (as with DNCs), which is, crucially, known because of world knowledge\sidenote{In this case, world knowledge about American football.} and not because of the presence of a linguistic antecedent (as with INCs). \textcite{CumminsRoberge2005} provide what they call a "modular account" of null objects in French, stemming from the intersection of several syntactic, semantic, pragmatic and discourse factors (a similar account of object drop, still abiding to Fillmore's binary distinction, is provided by \textcite{Cennamo2017}).\\
A more cogent, continuous account is offered by \textcite{Glass2013}, who acknowledges that there is "plenty of middle ground" between minimum recoverability (an object has to exist but it is unknown) and maximum recoverability (the specific identity of the object is known). In particular, she argues that, in order to be omitted, an object just has to be sufficiently recoverable for speakers to communicate felicitously, and that community- or genre-specific sublanguages are more prone to certain kinds of object omission simply because those smaller contexts favor object recoverability. Moreover, she explicitly argues against a INC-DNC distinction \parencite[1]{Glass2013}. A pioneering attempt to bring evidence in favor of the intuition that recoverability is the key in object omission is found in \textcite{Resnik1993, Resnik1996}, an information-theoretic account of selectional constraints testing, among other things, the relationship between transitivity and discourse context (more on Resnik's method in \refsec{resnik_sps}).\\
As I will argue later in \refsec{theory_workingdef}, I stay agnostic with respect to the binary, scalar, or continuous nature of object-droppability in my account of indefinite null objects. Following binary or scalar accounts, such a study would simply be a matter of considering those factors which are known to favor the emergence of INCs. On the other hand, under continuous-droppability assumptions, it would be a more complex matter of modeling both contexts and linguistic factors determining any kind of object drop, trying to position implicit null objects in a specific portion of the object-droppability spectrum.


\section{Defining the indefinite} \labsec{theory_defindefinite}

In this Section I will delve into a detailed discussion on indefinite implicit objects, the focus of this dissertation. In \refsec{theory_def_vs_indef} I reported a series of both now-classic and more recent accounts of the differences between so-called definite and indefinite null objects. Let us now comment on the nature of the latter, which were given several labels in the literature (objects of "detransitive verbs" in \textcite[46]{Yasutake1987}, "implicit objects" in \textcite{Glass2013} and \textcite[29]{PethoKardos2006}, or "pseudo-intransitive", "labile", "ambitransitive", "null complements", "understood arguments", "unspecified objects", "null instantiations" in other authors).

% \subsection{The involved parties} 

\subsection{Which verbs?}

Traditionally (\textit{contra} this thesis and \textcite[55]{TonelliDelmonte2011}, among others), indefinite null objects are taken to only be possible with a restricted set of activity verbs. For instance, \textcite[510]{Rizzi1986} provides an argument in favor of indefinite object drop being "lexically governed" in English on the basis that in some pairs of semantically related verbs (e.g. \textit{to eat} and \textit{to devour}) one member of the pair allows for object drop, while the other does not. \textcite[236]{Haegeman1987} does not hesitate to define this account "convincing", and similar notes are also found in \textcite{Fillmore1986, Rice1988, Mittwoch2005, Gillon2012}. I will come back on the theory referring to specific case of these "semantic minimal pairs" in \refsec{mannerspec}. The important aspect, here, is that traditional or traditionally-leaning literature has trouble motivating indefinite implicit objects on the basis of meaning alone, but it also needs it to be lexically determined. With that said, which verbs does then the literature identify as allowing indefinite object drop?\\
First of all, the verbs under consideration drop \textit{syntactic} arguments, crucially, not \textit{semantic} ones. In other words, indefinite object drop are obligatory semantic arguments of such verbs \parencite[120]{Cote1996}, while they do not surface syntactically (more on the syntax of indefinite implicit objects in \refsec{theory_incorporation}). \textcite[134]{Jackendoff2003} specifically observes that it is quite inaccurate to say that such verbs "licence an optional argument", since this definition "conflates semantic and syntactic argument structure". He illustrates this point by comparing \textit{to eat} and \textit{to swallow} in \ref{eatswallow}. Both show identical syntactic behavior, but while it is possible to swallow without swallowing anything, it is not possible to eat without eating something.

\ex. \label{eatswallow} \a. \label{eatswallow1} Bill ate (the food).
\b. \label{eatswallow2} Bill swallowed (the food).

Another key point in traditional literature on indefinite implicit objects concerns the difference between change-of-state verbs (such as \textit{to break, to harden, to open}) and pseudo-transitive verbs (such as \textit{to eat, to write, to sweep}). Verbs belonging to the former class are prototypically transitive \parencite{HopperThompson1980, Kardos2010}, since they feature two maximally distinct arguments (a volitional Agent subject and a non-volitional Patient object), while verbs belonging to the latter class exhibit both transitive and intransitive features. Thus, only pseudo-transitives can licence indefinite null objects in this dichotomy. They also appear to be a semantically rich class, comprising verbs of creation (e.g. \textit{to cook, to write, to knit}), verbs of ingestion or consumption (e.g. \textit{to eat, to drink}), and verbs of surface contact (e.g. \textit{to sweep}). While syntactically they have their ambivalent behavior in common, semantically they share the fact that their objects all are "incremental themes"\sidenote{Please refer to \textcite[279]{RappaportHovavLevin2005} and \textcite[4]{Kardos2010} for an extensive account of incremental themes.}, a term originally proposed by \textcite{dowty1991thematic} to refer to verbs showing homomorphism between the physical extent of their object and the temporal progress of the event. Thus, \textit{to eat} is an incremental-theme verb because the Patient gets progressively smaller while ingested by the Agent, \textit{to write} because the Theme gets progressively more wordy while the Agent types or pens it, and so on. On the contrary, \textit{pace} Dowty's attempt to apply this analysis to change-of-state verbs \parencite[568]{dowty1991thematic}, \textit{to close} is not an incremental-theme verb because sentences like \textit{Matt closed the door half-way} do not entail that half the door was closed \parencite[279]{RappaportHovavLevin2005}. Since incremental-theme verbs can behave both transitively and intransitively, in \textcite[33]{Levin1993} they are said to participate in the "unspecified object alternation". The author also provides a list of more than 40 verbs allowing for indefinite object drop, an event which \textcite[116]{Dvorak2017} praises as a major breakthrough after previous literature only focusing, "somewhat disturbingly", on the sole verb \textit{to eat}\sidenote{An unfortunate choice for a single example of the whole category of object-dropping verbs, as we will see later in \refsec{theory_incorporation}.}.\\
As I will show with the probabilistic model of indefinite null objects I define in this dissertation (final results presented in \refch{model}), object drop is possible both with change-of-state verbs and with incremental-theme verbs, the difference being a matter of degrees (determined by several linguistic factors, see \refch{factors} and \refch{predictors}), not a binary feature as in traditional accounts.


\subsection{Which objects?}

While syntactically unexpressed and semantically unspecified (at least with respect to a specific entity), indefinite implicit objects of the verbs allowing them still have to refer to \textit{something}. To what, though? Is it possible to generalize the answer?

\paragraph{Omitting \textit{something}}
This \textit{something} that object-dropping verbs refer to has been interpreted quite literally in traditional literature on the issue. \textcite{KatzPostal1967integrated} and \textcite{FraserRoss1970idioms} distinguish between the deletion of \textit{it} (expressed by the constructions which \textcite{fillmore1969types, Fillmore1986} made known as "definite null complements") and the deletion of \textit{something} (expressed by indefinite null objects). A similar consideration is also found, in passing, in \textcite[30]{PethoKardos2006}.\\
However, several objections can be made to this idea. Historically, the first came from \textcite{Mittwoch1982, Mittwoch2005}, who argued that the omitted object cannot be \textit{something} because "this would be incompatible with the atelic nature of the resulting sentence". I will come back to telicity, and the somewhat different approach I will embrace in the next Chapters, in \refsec{telicity}. Other authors \parencite{Marti2015, fodor1980functional, Melchin2019, Gillon2012, Dvorak2017} argue instead that indefinite implicit objects have to be interpreted as "weak indefinites" \parencite[55]{Melchin2019} as bare masses and plurals, instead of the indefinite pronoun \textit{something}, because only the former have obligatory narrow scope with respect to other quantifiers in the sentence (a behavior shown by indefinite null objects)\sidenote{Please refer to \textcite{carnie2021syntax}, a thorough handbook of generative syntax, for a gentle introduction to scope-taking constituents and the use of logic operators in syntax.}. Let us consider the examples in \ref{scope}, taken from \parencite[55]{Melchin2019}.

\ex. \label{scope} \a. \label{scope1} Everyone ate. \hfill $\forall$ > $\exists$ / #$\exists$ > $\forall$
\b. \label{scope2} Everyone ate something. \hfill $\forall$ > $\exists$ / $\exists$ > $\forall$

Let us unpack this notation. This means that

% (15)
% a.Everyone ate.
% b.Everyone ate something.
% ∀ > ∃/#∃ > ∀
% ∀ > ∃/∃ > ∀


    % Gillon (2012: 316)
    % THIS ALSO WORKS WITH RESPECT TO OTHER LOGIC OPERATORS THAT ARE NOT QUANTIFIERS
% verbs have a second property noted by Fodor and Fodor (1980, p. 760). The property,
% loosely formulated, is that the implicit object, expressible by an indefinite noun phrase,
% behaves as though it has narrow scope with respect to other scope taking constituents
% in its clause. [...]
% (4.0) Bill did not read.
% (4.1) ¬∃x Rbx
% (4.2) ∃x¬Rbx.

% $\neg$


% *Martì (2015: 458)*  
% > Another property that implicit indefinite objects share with incorporated nouns
% is that they take obligatory narrow scope with respect to other operators in the
% sentence, such as negation or intensional verbs (Fillmore, 1986; Fodor and Fodor,
% 1980; Mittwoch, 1982; Wilson and Sperber, 2000): [...] 

% *Melchin (2019: 55)*  
% > Thus, the understood objects of UOA verbs pattern with bare masses and plurals, which are
% weak indefinites.
% Further evidence for a weak indefinite reading of understood objects comes from the
% analysis of Steedman (2015). Steedman starts with the observation, attributed to Fodor and
% Fodor (1980), that these understood objects, unlike indefinite pronouns like something, always
% have low scope with respect to other quantifiers in the sentence.

weak indefinites (bare masses and plurals) sì, ma quali? v. paragrafo PROTOTYPICAL qui sotto!

\paragraph{Prototypical objects}

\textcite[122]{vanvalinlapolla1997syntax} coined the term "inherent arguments" to describe the indefinite implicit objects occurring with activity verbs, based on the idea that they denote a facet of the meaning of the verb, characterizing the action itself rather than a participant. I will devote some space to a detailed discussion of intransitivization as a means to focus on the activity itself on \refpage{activityfocus}. Resorting once again to the concept of linguistic prototype, which is indeed central in this discussion of transitivity (\refsec{theory_transitivity}) and object drop, much literature agrees on indefinite null objects being understood as prototypical\sidenote{Note that while all referenced works appeal to the notion of prototypical argument, not all of them phrase this idea exactly in these terms. Some refer to "implied arguments", "default interpretations", "standard objects", and other phrases in the same venue.} arguments of the verb \parencite{Rice1988, Naess2007, bresnan1978realistic, Melchin2019, Mittwoch2005, Dvorak2017thesis, Levin1993, Lorenzetti2008}. However, what is a prototypical object of a given transitive verb? \textcite[204]{Rice1988} provides the examples in \ref{rice}.

\ex. \label{rice} \a. \label{rice1} John smokes (cigarettes / *Marlboros / *a pipe / *SMOKING MATERIALS).
\b. \label{rice2} John drinks (alcohol / *gin / *water / *coffee / *LIQUIDS).
\c. \label{rice3} When he goes to Boston, John drives (a car / *a Toyota / *a motorcycle / *A VEHICLE).
\d. \label{rice4} Each afternoon, John reads (a book / *Ulysses / *the newspaper / *PRINTED MATTER).

Examples in \ref{rice1}, \ref{rice3}, and \ref{rice4} all appeal to our world knowledge, in particular, to our knowledge of what is the most probable choice of the average Joe (or John, in these examples). People usually smoke cigarettes which do not have to be necessarily Marlboros, they drive differently branded cars in their trips out of town, and they like to read generic books in the afternoon\sidenote{Apparently, for many people, especially the ones still liking their news to be printed on paper, reading the news is a leisurely activity to be specifically enjoyed in the mornings while having breakfast. This habit had to be even more common in the late '80s than today.}.\\
As noted by \textcite[125]{Naess2007}, however, example \ref{rice2} poses a challenge to the protypicality-enabled omission theory. The most typical liquid one can drink is usually water, not alcohol. And if one was to understand that the omitted substance is alcohol due to its very omission, would not this argumentation become circular?\sidenote{See also \textcite[20]{Mittwoch2005} on the issue of circularity.} The problem alcohol poses for linguists (or better, for linguistic theory) can be explained from different angles. I, for one, would appeal to the gricean maxim of relevance, in that water is indeed the typical liquid we drink, but it is so much typical (being necessary for good health and even life) that it would be actually weird to mention water-drinking in casual conversation. No one would bat an eye at John drinking water, so it would make little sense to utter \ref{rice2} implying water-drinking. Indeed, we only refer to the act of drinking water when it becomes relevant, for instance during hot summers (\textit{Remember to drink!}) or on a Sunday morning (\textit{I am so glad I drank water before going to bed.}). Thus, since the only socially relevant, statistically likely, choice of a drink for John in \ref{rice2} is alcohol, that is what we intend as a prototypical, omissible object for the verb \textit{to drink}. This perspective is also echoed by \textcite[14]{NewmanRice2006}, who ascribe the intransitive use of \textit{to drink} to "the prominence of alcohol consumption as a topic of discourse". Another possible account is the one by \textcite[21-28]{Goldberg2005}, where "taboo verbs" are argued to facilitate object drop due to our culturally-induced shame in mentioning that which is perceived as unmentionable in polite society (such as bodily fluids or, in this case, enjoying alcoholic drinks).Changing perspective, when presented with puzzling verb behavior such as the one expressed in \ref{rice2}, \textcite[303-305]{HuddlestonEtAl2002}\sidenote{Following \textcite[96-97]{Fillmore1986}, as discussed in \refsec{theory_def_vs_indef}.} needlessly assume that such verbs participate in two different patterns of object-droppability, i.e. "specific category indefinites" (where the omitted liquid would be interpreted as being of the alcoholic variety) and "normal category indefinites" (where the omitted liquid would be interpreted as being water). This account is flimsy at best, since it puts labels on a given state of affairs without actually providing an explanation for this epiphenomenal dichotomy.\\
Interestingly, \textcite[48-50]{Yasutake1987} suggests a three-way graded account of the different types of objects which can participate in indefinite implicit object constructions where the prototypicality of the omitted object is taken to be a rather flexible requirement for omission. In fact, the omissibility-as-prototypicality accounts I presented so far in this Section all made reference to the omitted object being somewhat "typical" of the verb, so that less typical Patients of the same verb are less likely to be omitted (or require quite the flight of fancy to be accounted for, as seen in the proposal by \textcite{HuddlestonEtAl2002} about the verb \textit{to drink}). Yasutake's perspective integrates the prototypicality intuition with other accounts based on world knowledge, envisioning these three types of implicit object:
\begin{itemize}
    \item typical objects (e.g. \textit{to read, to telephone});
    \item socially-understood objects (e.g. \textit{to drink, to shave, to drive});
    \item semantically unspecified objects of highly specialized activities (e.g. \textit{to steal, to see, to annihilate}).
\end{itemize}


\section{How many lexical entries?} \labsec{theory_entries}

In \refsec{theory_defindefinite} I discussed some traditional views on the characteristics a verb has to express in order to licence indefinite implicit objects and I also presented different views on the semantics of the omitted object. Now, another question arises about the nature of object-dropping verbs. When a verb participates in the so-called "implicit object alternation", is the transitive form of the verb actually distinct from the intransitive form in the lexicon, or are they different syntactic expressions of a single lexical entry? I will devote this Section to possible answers to this dilemma, to use \textcite{Gillon2012}'s word.


\subsection{Two meanings, two verbs: the naive account} \labsec{theory_twoentries}

The problem of having a single verb exhibiting two syntactically different behaviors (transitivity and intransitivity) was first identified by \textcite{fodor1980functional} and \textcite{dowty1981quantification}, an explicit reply to the former. Both treat verbs allowing for implicit objects as ambiguous between two different lexical entries, one transitive and one intransitive. This view is shared by other traditional literature on the matter \parencite{Cote1996, Mittwoch1982, vanvalinlapolla1997syntax, brisson1994licensing} and by more recent accounts \parencite{PethoKardos2006, BourmayanRecanati2013}.\\
Such an interpretation is clean on the surface, as clear-cut binarisms often are, but it does little to describe the complexity of reality \textemdash again, as binarisms often do. In a broader theory of semantics, the problem of the two uses of a single verb mirrors the well-known problem of deciding, for instance, whether \textit{bank} is a polysemous noun with two interpretations ("financial institution" and "river bank") or whether it has two homonymic interpretations. Say we go for the second, safer, account, since the only factor keeping the two senses together, i.e. etymology, is not transparent to native speakers of English nowadays. On the other hand, we would be much more keen to ascribe a polysemous interpretation to the different senses of the noun \textit{man}, which depending on the context can be used to mean "human being", "male human being", or "adult male human being". Crucially, the different senses of \textit{man} are all facets of the same entity, while the different meanings of \textit{bank} are not. Going back to the issue at hand, i.e. the distinction between transitive and intransitive senses of a given verb, it would indeed seem that these senses capture different facets of the same action performed by the Agent, instead of being two totally different meanings. This interpretation, fully consistent with the hypothesis that transitivity is a prototype (refer back to \refsec{theory_transitivity}), is further explored in \refsec{theory_incorporation} with reference to relevant literature.


\subsection{One verb, two meanings: the state-of-the-art account} \labsec{theory_incorporation}

testo

% *Lorenzetti (2008: 60)*  
% ## CONTRO LA DISTINZIONE IN DUE DIVERSE ENTRATE LESSICALI
% > we argue that positing different lexical entries in the case of null-object verbs
% is often counterintuitive and inappropriate,

% *Naess (2007: 134)*
% > Structurally, the most obvious characteristic of the indefinite object deletion construction is
% that it is a formally intransitive clause, as opposed to the transitive construction
% which appears when the verb in question is used with ano overt object NP. The
% logical explanation to such an alternation between higher vs. lower formal transi-
% tivity would be that it reflects a corresponding difference in semantic transitivity.
% From this perspective, IOD is most felicitously analysed not as a lexical quirk
% of certain specific verbs or classes of verbs, but as a syntactic detransitivisation
% mechanism, a means of expressing in a formally intransitive clause events which
% are construed as deviating from the transitive prototype.



\paragraph{Focus on the activity} \labpage{activityfocus}
testo

%  *Liu (2008: 289)*
% QUI, ADDIRITTURA, SI PARTE DALL'IDEA DELL'ACTIVITY INTRANSITIVA E LA SI FA DIVENTARE TRANSITIVA!
%  tutto il paper riguarda questa distinzione in gruppi per transitività  
%  > verbs used without an object into four categories: 1) pure intransitive verbs, such as
% arrive, rise, and sleep; 2) ergative intransitive verbs, such as break, increase, and open; 3)
% transitive-converted intransitive verbs of activity, such as eat, hunt, and read; 4) object deleting
% verbs, warranted by discourse or situational context, such as know, notice, and
% promise.

% *Liu (2008: 300)*  
% > two groups. For instance, Quirk et al. (1985:1565) list both the ergative and the
% transitive-converted intransitive verbs as instances of “transitive → intransitive.”
% There are three significant differences between the two groups of verbs. First, the
% subject or the sole argument of an ergative verb plays the theta role of theme, but the
% sole argument of a transitive-converted intransitive verb plays the theta role of agent
% (e.g., The window broke vs. Mary ate). Second, while there is a subject change in the
% use of an ergative intransitive verb compared to its transitive counterpart (e.g., from
% They opened the door to The door opened), no such a change is involved in the use
% of a transitive-turned intransitive verb [...]  
% The third difference between the two groups is that there is no object deletion
% involved at all in ergative intransitive verbs when they shift from transitive to intransitive
% because the shift entails only a movement of the object into the subject position,
% i.e., the object is not deleted, just moved to a different position in the sentence. [...]  
% Of course, some scholars argue that there are a few instances
% in which a transitive-converted intransitive verb of activity may assume “a more specific
% meaning, so a particular kind of object is ‘understood’” (Quirk et al. 1985:1169).
% An example given by Quirk et al. is John drinks heavily. They suggest that the verb
% drink in the utterance means ‘drink alcohol,’ i.e., alcohol is the omitted object. While
% it is true that drink here means ‘drink alcohol,’ the focus of the utterance, in the final
% analysis, is not on the object because it does not really refer to any specific kind or
% amount of alcoholic drink, something that is usually mentioned if the focus is on the
% object. In short, transitive-turned-intransitive verbs of activity focus on the activity,
% not the object.4

% *Garcia-Velasco & Munoz (2002: 7-8)*  
% The factors of relevance here include the semantic structure of the verb, which
% itself may give prominence to one semantic component (as in the manner-result opposition),
% the speaker’s communicative intentions, which may lead him to focus on the activity itself,
% thus downgrading the referential status of the object, and world knowledge, which allows him
% to construe an action as an autonomous activity

% *Mittwoch (2005: 2)*  
% PARAGRAFO IMPORTANTISSIMO!!!  tra activity e prototypicality
% > There is a well-known transitivity alternation involving a class of common
% process verbs which can be both transitive and intransitive with the same
% subject argument, so that the intransitive variety is unergative:  
% (1) John is reading / drinking.  
% John is reading a letter / drinking juice.  
% A representative list of English verbs participating in the alternation as it will be
% understood here is given in (2).  
% (2) a. read, study, revise (what has been learnt) rehearse, practise  
% b. sing, dance, play (music), act  
% c. write, compose (music) paint (a picture), draw, etch, sew, knit, crochet,
% weave, spin, cook, bake2  
% d. type, print, photocopy, dictate, record, film  
% e. eat, drink, chew, smoke  
% f. sow, plough, harvest, weed, hunt  
% g. wash, iron, mend, darn, clean, sweep, dust, hoover, paint (apply paint
% to), embroider, tidy up  
% The verbs all have a pronounced manner component in their meaning, and fairly
% circumscribed selection restrictions. Hence the content of the phantom object is
% more or less predictable. It will correspond to the literal rather than
% metaphorical meaning of the verb (e.g. read written or printed material rather
% than, say, the stars or coffee grounds) and may be further restricted in usage
% (e.g. the understood object of intransitive clean in he is cleaning is the interior of
% a house, rather than a car, shoes or teeth; that of mend is clothes rather than
% electric gadgets or roads).3
% Aspectually, intransitive predicates with these verbs pattern together, as atelics,
% with VPs consisting of transitive verbs + bare NP objects [−DELIMITED
% QUANTITY], whereas transitive verbs + quantized object yield telic VPs.

% *Ahringberg (2015: 6)*  
% > In cases where the omitted complement is indefinite, the focus is placed on the action
% which the predicate denotes (Fillmore, 1986, p. 96), and some other verbs that allow this type
% of null instantiation include clean, drink, embroider, hunt, iron, read, sing, study, teach and
% write (Levin, 1993, p. 33). It should be stressed that compared to typical intransitive verbs, such
% as those mentioned in the introduction section, there is always an implied object involved,
% which could be interpreted as representing either the word “stuff” or “something” as exemplified
% in (9) (Fillmore, 1986, p. 95), or a more specific concept which is generally conceived and
% linked with the verb, such as dinner in (10).
% (9) He’s too stressed out to be able to eat {stuff}.

% *Yasutake (1987: 50)*
% cita Munro (1982) --> nell'uso intransitivo, "the general action is of more interest than the specific unspecified object"

% *Yasutake (1987: 52)*  
% > verbs with decategorized objects thus share a common property with objectless transitive,
% viz. the communicative intent of the speaker in both is to provide information concerning the
% subject by way of emphasizing the action-type. Removal of an object noun phrase is hence
% regarded as an extreme form of de-categorization

% *Melchin (2019: 52-53)*  
% (questa cosa la diceva anche PM: preparare obiezioni!) + dire anche in altro par, ma quale?
% >  Chomsky (1986) cites an observation by Howard Lasnik
% that the meaning of the intransitive use of eat is somewhat different from the transitive use,
% and is specifically something more like dine; [...]  Similarly, Fillmore
% (1986) notes that while one can, and often does, bake a number of foods besides typical
% “baked goods” such as breads and pastries, including potatoes and hams, John is baking
% generally cannot be taken to mean John was baking, for example, potatoes.

% *Lorenzetti (2008: 66)*
% QUI È USATO PER GIUSTIFICARE ANCHE **DEFINITE** DOBJ DROP!
% > As to the factors more directly connected to the domain of discourse, it is worth
% mentioning the topic/focus distinction. The omitted arguments in (11) and the following are
% all highly predictable, and therefore they are not good candidate for focal status, since "the
% focus is that portion of a proposition which cannot be taken for granted at the time of speech,
% the unpredictable and pragmatically non-recoverable element in an utterance" [Lambrecht
% 1994: 207].  
% A sentence topic, by contrast, is usually defined as “a matter of already established current
% interest which a statement is about and with respect to which a given proposition is to be
% interpreted as relevant” [Lambrecht 1994: 119].  
% (11) a. I thought you said your dog doesn’t bite!  
% b. Religion integrates and unifies.  
% Every sentence requires at least one focus, namely an assertion containing new information
% (Chafe 1994). It would be tempting to claim that when objects are omitted, the focus is on the
% activity itself.

% *Kardos (2010: 5-6)*  
% IMPORTANTE CLASSIFICAZIONE DELLE TEORIE E SPIEGAZIONE!!!  
% > When it comes to matters of argument realization, researchers are split into several camps
% regarding the role of the lexicon and that of aspectual notions in this process. As I will discuss
% in section 4 in greater detail, some bolster the so-called Free Argument Projection Hypothesis
% which says that ′′arguments of verbs are projected freely onto syntax, with verbs being
% unspecified for those components of meaning that determine argument expression5′′
% (Rappaport Hovav and Levin 2005: 275). Others attribute argument realization patterns purely
% to aspectual properties such as incremental theme and measure. Unlike the former two
% ' camps' , Levin and Rappaport Hovav strongly reject the Free Argument Projection Hypothesis
% and also question the sole role of aspect in argument expression. Their theory is
% fundamentally lexically-based as is seen below in the exposition of the system. [...]  
% << THE ARGUMENT-PER-SUBEVENT CONDITION: There must be at least one argument XP in the
% syntax per subevent in the event structure.  
% (Rappaport Hovav and Levin 2001: 779) >>  
% The popularity of this condition is apparent from the fact that it has been accommodated in
% the work of numerous researchers, e.g Goldberg (2005)6, Grimshaw and Vikner (1993), van
% Hout (1996), inter alia. An important consequence of this is that verbs denoting simple events
% must project a single argument, whereas verbs expressing complex eventualities alway occur
% with at least two obligatory elements in the syntax. The former class is represented by activity
% type verbs. For instance, run lexicalizes a single event in its lexical representation, which in
% turn yields the obligatory appearance of a single argument (i.e. an agent) in subject position.
% A prototypical verb exemplifying the latter class is the causative verb, break, which is
% associated with two subevents, one being the causing event and the other one, the event
% caused by the causing event. The causing event is a simple activity, while in the second
% subevent an externally caused result state is brought about.

% *Sugayama (2007: 1)*  
% PRAGMATIC AND DISCOURSE FACTORS?
% > researchers have proposed that causative verbs obligatorily express the argument that
% undergoes the change of state in all contexts (Browne 1971; Brisson 1994; van Hout 2000; Ritter and
% Rosen 1996, 1998; Rappaport Hovav & Levin 1998). This generalisation is too strong to accommodate
% real fact [...]  
% counterexamples may be accounted for by the following Principle of Omission under Low Discourse
% Prominence in Goldberg’s (2005) Construction Grammar. [...]  
% (2) Principle of Omission under Low Discourse Prominence: Omission of the patient argument
% may be possible when the patient argument is construed to be de-emphasized/unprofiled in the
% discourse vis-а-vis the action [...]  
% That is, omission is possible when the patient argument is (or focal) in the discourse, and
% NOT TOPICAL the action is particularly emphasised [...]  
% the attention can be shifted away from the (definite) argument in favour of the action, if the action is sufficiently emphasised due to the
% TOPICAL patient argument being present and salient in the IMMEDIATE NON-LINGUISTIC CONTEXT

% *Mittwoch (2005: 3)* 
% TRA ACTIVITY E NOUN INCORPORATION
% > In the early days of generative grammar the intransitive version of these verbs
% was derived by a transformation deleting the object. Today it is generally
% thought that the objects of the verbs concerned in this alternation, though
% appearing in the lexicon, need not be projected in the syntax.4
% Thus an influential paper by Grimshaw (1993) draws a distinction between
% structural and content components of meaning. The objects of change-of-state
% verbs are structural, and must be projected; the objects of activity verbs are
% content arguments, and are in principle optional (subject to certain ill-
% understood restrictions). The reason is that change-of-state verbs have a
% complex event structure involving something like x cause y to change state,
% where y represents the object, whereas activity verbs have a simple structure: x
% act. Additional components of meaning that distinguish between different verbs
% in each structure are content components in this theory.

% *Wierzbicka (1982: 758)*  
% RIFLETTERE SOPRATTUTTO SUL SECONDO CAPOVERSO  
% > the action reported in a have a V frame cannot have an external
% goal: it must be either aimless, or aimed at some experience of the agent. [...]  
% Finally, the action (or process) must be seen as repeatable. Having a swim
% (or a read, or a try) is something that can be done again and again. There is
% something arbitrary about the length of a walk, a lie-down, or a read. Since
% these activities (when reported in a have a V frame) are aimless, devoid of any
% external goal, they can not only be extended or terminated at will, but can also
% be resumed at will. Thus actions which cannot be repeated cannot be described
% in a have a V frame. For example, the contrast of have a bite or a lick or a
% taste vs. ?have an eat may result at least partly from the contrast in repeatability
% of the actions in question. One could bite John's sandwich, or lick his ice
% cream, or taste his soup-not once but twice, or more-but one could eat his
% sandwich only on[ce]

% *Wierzbicka (1982: 759)*  
% > A translation into a more conventional metalanguage could read: 'The have
% a V construction is agentive, experiencer-oriented, antidurative, atelic, and
% reiterative.' 

% *Wierzbicka (1982: 771)*  
% > Examples are have a bite, a lick, a suck, a chew, a nibble.
% The verbs included in this subtype are transitive or semi-transit
% meaning of the verb ensures that the undergoer of the action is on
% affected by it. A lick, bite, or nibble is not enough to make muc
% to the object involved; and a chew or a suck would also not affect
% greatly, because of the slow nature of the process. This means tha
% verb is transitive, and the action requires an undergoer as well as
% undergoer can be ignored; the action can be viewed as really invo
% one participant (the agent). However, if someone eats an apple or
% the object in question is TOTALLY affected, and thus impossible to ignore

% *Wierzbicka (1982: 776)*  
% QUESTO VA INSERITO NEL DISCORSO SULLA TELICITY  
% > This means that the 'unlimited substance' type, like the 'objectless action' type
% (?4), demands strict atelicity. Drinking or smoking (like walking or swimming)
% is something that can be prolonged for as long as the agent wishes; but drinking
% a glass of water or smoking a cigarette cannot be indefinitely extended (just
% as walking to the post office or swimming to the shore cannot). Thus

% *Wierzbicka (1982: 788)*  
% > . This does not mean that the frame takes only intransitive verbs. It
% allows transitive verbs, too, but only if the sentence with a transitive verb
% permits an interpretation compatible with the requirement of one core partic-
% ipant: the agent/experiencer. Only two types require one-argument verbs: the
% type which refers to aimless and objectless individual action conducive to
% feeling good, and that which refers to potentially therapeutic, semi-voluntary,
% 'corrective' individual act

% *Naess (2011: 419)*  
% UTILE ANCHE PER I VERBI DI MOTO!!! fare una nuotata/caduta/corsa/scalata != *fare un'andata  
% (V. CONSIDERAZIONI IN FONDO ALLA STESSA PAGINA SUL VERBO "RUN")  
% > In other words, affectedness of the agent
% is associated not only with specific case-marking patterns, but also with the possibility of
% intransitive behaviour. [...]  
% Wierzbicka (1982) discusses constructions of the type have a drink, have a swim, and
% accounts for their distribution to a large extent in terms of affectedness of the agent participant. [...]  
% Wierzbicka links this affected-agent semantics explicitly to reduced formal transitivity:
% ‘have works as a detransitiviser: the object is de-emphasised, the predication concerning
% the object is backgrounded, and at the same time the emphasis on the agent increases’
% (Wierzbicka 1982:291). [...]  
% Nжss claims that it is a crucial property of transitive constructions that
% their arguments are maximally distinct – not just as physical entities, but in terms of the role
% they play in an event. Thus a prototypical transitive construction has one single controlling
% participant and one single affected participant. Verbs of eating and drinking deviate from this
% pattern because both the agent and the patient are affected

\paragraph{Object drop as noun incorporation}
queste teorie sostengono che l'oggetto non sia rappresentato sintatticamente! (vero?)

% *Dvorak (2017: 119)*  
% > An unorthodox approach to INO is presented by Martı́ (2011), who is primarily moti-
% vated by defeating the view that English INO are purely pragmatic in nature (Groefsema
% 1995, a.o.). Martı́ argues that the INO of verbs like eat, bake, smoke, drink, read, write,
% hunt, cook, sing, carve, knit, weed, file, write, etc. are grammatically represented, number-
% neutral nouns, not too different from nouns incorporating into verbs in noun-incorporating
% languages. Her argumentation is based on the fact that English verbs with implicit indefi-
% nite objects are generally atelic (except for John ate for/in and hour ), and they describe
% conventional, name-worthy, institutionalized, habitual activities – just like verbs that have
% undergone noun-incorporation (cf. Mithun 1984, Dayal 2011b:164). I get back to the ties
% between INO derivation and noun-incorporation in 5.2.3.

% *Martì (2015: 453)*  
% NOUN INCORPORATION! TORNA SPESSO, MERITA UN PARAGRAFO A PARTE?  
% > That there is a notional, even if unuttered, object (indicated in parentheses) in
% the interpretations of the object-less versions of these sentences is uncontrover-
% sial. The controversial question is whether there is an object at any point in the
% grammatical derivation of the object-less sentences, despite its lack of phonological
% realization.
% Carston (2004), Groefsema (1995), Hall (2009), Iten et al. (2004), Recanati
% (2002) and Wilson and Sperber (2000) have suggested that, in at least some of
% the uses of these seemingly object-less sentences, there is indeed no object at any
% level of linguistic representation. Instead, that object is provided for pragmatically.
% In these cases, given appropriate pragmatic pressures, language users enrich gram-
% matical interpretations in such a way as to provide the missing material. Here I
% argue that an object is, on the contrary, part of the linguistic representation of
% sentences such as (37)–(41) in all of their uses. This object, however, is not a
% run-of-the-mill syntactic object, but an object whose properties are those of nouns
% in noun incorporation constructions found in languages such as West Greenlandic,
% Frisian, and many others. The argument rests on the following logic: if these
% notional objects behave like linguistic category X (i.e. incorporated nouns), then
% the null hypothesis is that the notional objects are themselves instantiations of that
% linguistic category X. The only difference between the objects of interest here and
% those that appear in noun incorporation is that the former, but not the latter, are
% phonologically null.14

% *Martì (2015: 454)*  
% > Noun incorporation is a word formation process whereby a verb and a noun (usually,
% its object) are put together to form a verb. [...]  
% When the noun and the verb are put together, a number of associated effects follow.
% Importantly, the object noun loses its status as a regular syntactic object. This can
% be seen in the fact that the nouns in these constructions must always appear bare

% *Martì (2015: 455)*  
% ALTRE IMPORTANTI CONSIDERAZIONI SULLA NOUN INCORPORATION  
% > In West Greenlandic, an erga-
% tive language, objects of transitive verbs and subjects of intransitive (unergative) verbs
% are marked with ABSOLUTIVE Case, and subjects of transitive verbs are marked with
% ERGATIVE Case. In (42)b, where the noun has incorporated, the subject is marked
% with ABSOLUTIVE, not ERGATIVE, Case.
% Word order is also affected, so that incorporated nouns always precede the
% verb, independently of the canonical position of regular syntactic objects in the
% language in question. [...]  
% There are a number of semantic characteristics that are associated with
% noun incorporation. The incorporated noun is interpreted indefinitely and
% non-specifically: compare the a and b examples in (42)–(44). The indefinite,
% non-specific semantics of the nouns that participate in noun incorporation is a
% widely noted fact in the literature (see Carlson, 2006; van Geenhoven, 1998; Gerdts
% and Marlett, 2008; Mardirussian, 1975, p. 386; Mithun, 1984; de Reuse, 1994;
% Spencer, 1991; Sullivan, 1984, among others). A second cross-linguistically stable
% semantic property of incorporated nouns is that they typically take narrow scope
% with respect to other operators in the sentence 

% *Martì (2015: 456)*  
% > Verbs that are incorporated into typically designate name-worthy, typical activi-
% ties. For example, Axelrod (1990, p. 193) says that ‘ ... incorporation provides the
% lexicalized expression of a typical activity’. And Mithun (1984, p. 848) says that
% ‘some entity, quality or activity is recognized sufficiently often to be considered
% name-worthy in its own right’.

% *Martì (2015: 457)*  
% > It is well known that implicit indefinite objects are interpreted as non-specific
% indefinites with plain existential import (Bresnan, 1978; Dowty, 1981; Fillmore,
% 1969, 1986; Fodor and Fodor, 1980; Gillon, 2012; Mittwoch, 1980; Shopen, 1973;
% Thomas, 1979). In (49), the speaker does not convey the idea that something in
% particular has been eaten today, the same way that that would not have been conveyed
% had s/he said ‘John has already eaten food/a meal today’:
% (49) John has already eaten today.

% *Martì (2015: 458)*  
% > Another property that implicit indefinite objects share with incorporated nouns
% is that they take obligatory narrow scope with respect to other operators in the
% sentence, such as negation or intensional verbs (Fillmore, 1986; Fodor and Fodor,
% 1980; Mittwoch, 1982; Wilson and Sperber, 2000): [...]  
% Again like incorporated nouns, implicit indefinite objects are semantically
% number-neutral. As far as I know, this fact has not been noted before. Thus, if (64)
% is true, then it is immaterial whether John is smoking half, one or many cigarettes:

% *Martì (2015: 459)*  
% SIGNIFICATO CONVENZIONALIZZATO DEGLI USI INTRANSITIVI  
% > When verbs such as eat, drink, write, etc. take on implicit indefinite objects, they
% typically give rise to ‘conventionalized’ meanings, in a similar fashion to incorpo-
% rated nouns. Thus, if John is eating, then he can only be eating edible things, things
% that are conventionally eaten. Whereas it is perfectly possible to say that John is eat-
% ing his bed, strange as that may be, when one says that John is eating, one means
% that he is eating things that are normally eaten. 

% *Martì (2015: 461)*  
% QUESTO CONSENTE DI DISTINGUERE INDEFINITE DROP DA DEFINITE DROP!!!
% > Frisian allows us to test the predictions of the proposed analysis in an interest-
% ing way. That’s because this language has both implicit indefinite objects and noun
% incorporation, as we saw in Section 3.1. The fact that Frisian is such a close relative
% of English allows for a controlled comparison between the two languages.
% In fact, this is precisely the argument made by Dyk (1997) regarding the nature
% of what he calls ‘detransitivized’ verbs (in Dowty’s 1989 terminology), which are
% the verbs that take implicit indefinite indefinite objects, such as English eat. [...]  
% This means that unergatives, detransitivized and incorporated-into verbs form a
% natural class. Then, there a number of restrictions on the type of verb that can be
% incorporated into in Frisian and these very same restrictions are strikingly observed
% for detransitivized verbs. For example, only verbs that select for a Patient object
% allow incorporation. The verbs corresponding to English notice, hate and know don’t
% take Patients as objects and do not allow noun incorporation or detransitivized uses
% (Dyk, 1997, pp. 95, 108):

% *Martì (2015: 463)*  
% QUESTO SIGNIFICA CHE NON DIPENDE DALL'OBJ=PATIENT, MA DALL'AGENT AFFECTEDNESS?  
% (conferma: esiste bird-watching, soggetto Agent, Obj != Patient)  
% > A verb like know never has a volitional subject, and, accordingly, in Frisian this verb
% never allows incorporation or detransitivization.

% *Martì (2015: 466)*  
% > A common denominator in various proposals for the analysis of noun incorpo-
% ration is the semantic function that is assigned to the incorporating noun: that of
% restricting the internal argument of the verb. I build upon Chung and Ladusaw’s
% (2004) implementation of this idea.
% Chung and Ladusaw (2004) propose that there is a mode of composition called
% Restrict that applies in cases of formal incorporation. This rule applies in cases where
% a predicate is being combined with a property. The property argument is interpreted
% as a restrictive modifier of the predicate. The operation does not reduce the predi-
% cate’s degree of unsaturation; i.e. semantically, the predicate is still missing an internal
% argument. Restrict is illustrated in (87):

% *Martì (2015: 467)*  
% > Any pragmatic approach must consider the cluster of properties we discussed in
% the previous sections as accidental. Nothing in this type of approach predicts that
% implicit indefinite objects always take narrow scope, are number-neutral, or are
% interpreted indefinitely and non-specifically.

% *Martì (2010: 7)*  
% IN RISPOSTA A CHI DICE CHE NON C'È DIFFERENZA TRA DEFINITE E INDEFINITE?  
% > Notice the difference in meaning between incorporated and non-incorporated
% objects in these languages. For example, whereas in (15), the grammatical and
% notional object is interpreted specifically, in (12) the notional object is interpreted
% indefinitely. The indefinite, non-specific semantics of incorporated nouns is a widely
% noted fact in the incorporation literature (see Mithun 1984, Sullivan 1984, de Reuse
% 1994, Spencer 1995, etc.).

% *Martì (2010: 7)*  
% >  the semantic properties of implicit indefinite objects
% in languages like English are in fact the same as the “cross-linguistically stable
% properties of the semantics of incorporation” (in Farkas and de Swart’s 2003 and
% Carslon’s 2006 terminology). After presenting the case for semantic incorporation, I
% show that, formally, implicit indefinite nouns undergo compound noun incorporation,
% not classificatory noun incorporation.

% *Yasutake (1987: 51)*  
% OBJECT DE-CATEGORIZATION AND NON-INDIVIDUALITY  
% parla di object incorporation (John did some deer-hunting) and "the use of an explicit but non-individuated object" (John hunted deer)

% *Melchin (2019: 56)*  
% > 3.2.2  Omitted objects are not present in the syntax

% *Bourmayan & Recanati (2013: 125)*  
% drink alcohol --> alcohol-drink --> drink(2)  
% write with a pen --> pen-write --> write(2)  [questa è una mia ipotesi, devo ragionarci su]  
% > Through free enrichment, a lexical item (or, for that matter, a complex phrase) can be
% understood in a more specific sense than the sense it literally has. For example,
% intransitive ‘drink’ is often understood in the specific sense DRINK ALCOHOL [...]
% can be achieved by morphosyntactic means. In languages such as
% West Greenlandic, some verbs can undergo a process of ‘incorporation’ of their direct
% objects which yields noun-verb combinations behaving like single, verbal, morphological items [...]
% According to van Geenhoven (1998), the incorporated object of the verb does not
% correspond to a genuine argument: it does not denote an individual of type e, but
% rather a property of type <e,t>.

% *Naess (2007: 129)*
% ALTRI QUI NEL PARAGRAFO AVEVANO DETTO CHE EAT E DRINK SONO VERBI PROTOTIPICI
% > Two main proposals have been made as to
% how to integrate such verbs into a broader theory of valency in general: either the
% “intransitive” and the “transitive” variants of a verb are counted as two different
% lexical entries with different argument structures, or the lexemes in question have
% as part of their lexical entry that an indefinite direct object need not be overtly ex-
% pressed. The problem, then, becomes one of identifying the set of verbal lexemes
% which have the property of being able to participate in this “alternation”, and, if pos-
% sible, providing a semantic characterisation of the verbs belonging to this set.
% As Marantz (1984 :192) puts it, “it is an interesting and important problem to
% characterize the transitive verbs that permit indefinite object deletion”. Marantz
% assumes the ingestive verbs, of which he gives ‘eat’, ‘drink’, and ‘learn’ as examples,
% to be the core candidates for IOD, recurring in language after language. In other
% instances, he claims, “the alternation is created not by a productive lexical rule but
% by generalization by analogy with certain core verbs exhibiting the alternation”
% (Marantz 1984 :193).
% (NOTA MIA: eat, drink e learn sono affected agent verbs! v. naess 2011: 413 + haspelmath negli appunti al prossimo paragrafo)

\paragraph{Agent affectedness} 
Agent affectedness: why eating and drinking are poor choices for typical transitivity examples\\
in "transitivity as a prototype" cito NAESS e altri per dire che subject e object nelle frasi transitive devono essere massimamente distinti da un punto di vista semantico!

% *Naess (2007: 144)*  
% > IOD is considerably more common with some verb types than with others (affected-agent
% verbs, where the agent can be construed as the “endpoint” of the event, and ef-
% fected-object verbs, whose objects are inherently nonreferential and therefore eas-
% ily omissible),

% *Naess (2007: 135)*  
% IMPORTANTISSIMO!!! SI RICOLLEGA AL DISCORSO DI DRINK=ALCOHOL ETC  
% > Analysing IOD as a construction which applies only to verbs or clauses which
% are not fully transitive semantically explains a striking observation made by Fill-
% more (1986): certain verbs undergo what he calls a semantic “specialisation” when
% used in an IOD context. He cites the verb bake as an example: in the sentence I
% spent the afternoon baking, “the missing object is taken to include breads or pas-
% tries, but not potatoes or hams” (Fillmore 1986 :96).
% If we take into consideration the difference in semantic transitivity between
% affected-object and effected-object verbs, an explanation for this “semantic spe-
% cialisation” is readily forthcoming. The verb bake in English is ambiguous between
% an affected-object (bake potatoes, where the potatoes already existed prior to the
% action and are only affected, not effected by the act of baking) and an effected-ob-
% ject reading (bake pastries, where the act of baking brings the pastries into exist-
% ence). Affected-object bake is high in transitivity and so does not easily undergo
% IOD. Effected-object bake, on the other hand, unproblematically omits its object,
% because this verb has a nonreferential, nonaffected object and therefore is not ful-
% ly transitive semantically. Consequently, under IOD, only the effected-object read-
% ing is possible – the intransitive construction can only be read as containing a verb
% relatively low in transitivity.
% Similar behaviour can be found e.g. with the ambiguous verb paint in generic/
% habitual statements. If we say of someone that He paints, we mean that he paints
% pictures – either for a living or as a hobby – not that he is a housepainter.

% *Naess (2007: 150)*  
% AMBITRANSITIVE/LABILE VERBS  
% > In other words, what characterises the verb cook semantically is the same
% property that was demonstrated for bake in 6.3.1 above: the transitive cook is am-
% biguous between an affected-object and an effected-object reading. We saw above
% that for such verbs, indefinite object deletion is only permissible with the effected-
% object reading. That is, an intransitive clause where the S is interpreted as an agent
% can only be read as referring to the agent’s bringing something into existence.
% Dad is baking can only mean that he is producing things from loose ingredients
% through a process of mixing and heating them (baking bread or cookies); it cannot
% mean that he is putting already existent objects through a process which produces
% an alteration in them (baking potatoes). On the other hand, the patient-subject
% variety, for instance the potatoes are baking, clearly has the affected-object reading,
% as predicted by the restriction of inchoative-causative verb pairs to verbs referring
% to a change of state. 

% *Naess (2007: 141-144)*  
% ANCORA SU EAT/DRINK --> CI DEVO FARE UN PARAGRAFO A PARTE!  
% > A recurring question in discussions of IOD with ingestive verbs is not simply why
% such verbs so frequently undergo IOD, but also why the use of these verbs without
% an object tends to give rise to certain very specific readings. In the words of Fill-
% more, “EAT is used to mean something like ‘eat a meal’ – not merely ‘eat some-
% thing’, and DRINK is used to mean ‘drink alcoholic beverages’” (Fillmore 1986 :96).
% It is this use that motivates Rice’s analysis of ‘alcohol’ as being somehow a proto-
% typical object for drink, and it leads Fillmore to assume that IOD in fact comes in
% two distinct varieties, “one involving a semantic object of considerable generality,
% the other requiring the specification of various degrees of semantic specialization”
% (Fillmore 1986 :96). There are obvious weaknesses in this analysis, since it does
% not explain how a language user determines whether an absent object should be
% understood as being of “considerable generality”, or, on the contrary, representing
% a “semantic specialization”. [...]  
% The semantic property which gives rise to these
% specialised meanings is in fact the same property that makes these verbs eligible for
% indefinite object deletion in the first place: affectedness of the agent. I suggested in
% 4.3.1 that omitting the object with these verbs amounts to highlighting the effects
% that the action has on the agentive participant, while backgrounding those on the
% patient, by omitting the patient argument altogether. The so-called specialised
% readings arise precisely as a result of this emphasis on the effect on the agent.
% To understand how objectless eat and drink acquire their conventionally un-
% derstood meanings of ‘eat a meal’ and ‘drink alcohol’, respectively, we must con-
% sider the contexts in which these verbs are generally used without an object. As it
% turns out, these two verbs seem to be in almost complementary distribution with
% respect to which grammatical contexts favour their use without an overt object NP
% (see also Newman and Rice 2006).
% The English verb eat is mostly used without an object in the past tense or per-
% fect: John ate, I have eaten, or in the progressive, I am eating. The generic/habitual
% ??John eats, on the other hand, is highly unnatural, if not downright ungrammati-
% cal.  
% By contrast, objectless uses of the verb drink are most frequently found pre-
% cisely in a generic/habitual construction, as in John drinks, or Fillmore’s example
% I’ve tried to stop drinking. The past-tense John drank seems to be mainly interpret-
% able as a habitual statement in the past tense, while the perfect ??John has drunk is
% extremely peculiar without an object. The progressive John is drinking is possible,
% though perhaps a little strange out of context; it does not appear to favour the
% reading ‘drinking alcohol’ to quite the extent that the generic statement does, al-
% though this reading is facilitated by the inclusion of a context evoking the act of
% drinking alcohol: John is in the pub drinking.  
% In fact, if we attempt to interpret drink as referring simply to the ingestion of
% any kind of drinkable fluid, then the same thing happens with this verb as with eat:
% it becomes very unnatural in a generic/habitual clause without an object – ??John
% drinks. Most analyses seem to assume that this is because the conventional use of
% drink to mean ‘drink alcohol’ somehow blocks the reading ‘drink things in general’,
% or that the verb drink somehow selects ‘alcohol’ as a kind of privileged argument
% (Rice 1988). However, this does not explain how this “privileged” reading arose in
% the first place. Rather than explain the absence of a generic use of drink to mean
% ‘drink anything’ by invoking an “overriding” reading of such clauses as ‘drink alco-
% hol’, I suggest that the latter should in fact be explained in terms of the former. It is
% not the case that the reading ‘drink anything’ is unavailable for objectless clauses
% because the reading ‘drink alcohol’ overrides it; such a statement has little or no
% explanatory value, saying in effect that objectless drink means ‘drink alcohol’ be- --  
% cause it means ‘drink alcohol’. Rather, the reading ‘drink alcohol’ is available for
% generic statements precisely because the ‘drink anything’ reading is pragmatically
% unnatural and therefore unlikely to be expressed.  
% The naturalness of generic statements is constrained by properties of the world.
% When verbs like eat and drink do not easily occur in generic objectless statements
% of the type John smokes, it is for a very simple reason. Eating and drinking are the
% two most fundamental actions of human existence; their habitual performance is
% the most basic prerequisite for sustaining life. ??John eats, in the sense that eating
% is a habit of John’s, is highly peculiar for the obvious reason that humans and ani-
% mates who do not have this habit simply do not exist – they would be dead (cf. also
% Fillmore 1977 :135). Referring to inanimate beings, on the other hand, such sen-
% tences are much less peculiar; one can imagine, for example, an advertisement
% listing the attractions of a lifelike doll: She eats, she cries, she sleeps!
% While the sentence John drinks, taken to mean ‘John habitually imbibes liq-
% uids’ is unnatural because is is close to tautological – if John did not have this
% habit, he would not exist – there is nothing in the English grammar which pre-
% cludes the construction of such a sentence. Drink is an affected-agent verb highly
% suited to appear in low-transitivity contexts such as objectless generic construc-
% tions; it is the real-world interpretation of this construction which makes it odd.
% The sentence is syntactically well-formed but semantically strange – which leaves
% open the possibility of assigning it an alternative interpretation.  
% There are a couple of reasons why a suitable candidate for such an interpreta-
% tion should be ‘drink alcohol’. The consumption of alcoholic beverages has an im-
% portant social and cultural function in the Western cultural sphere, and increas-
% ingly in the rest of the world. Newman and Rice (2006) take the ‘alcohol’ reading
% of intransitively-used ‘drink’ to reflect “the prominence of alcohol consumption as
% a topic of discourse”.  
% In addition to these pragmatic considerations, however, the semantic proper-
% ties of objectless ‘drink’ – the emphasis on the affectedness of the agent – are in
% themselves conducive to this particular reading. The reading ‘drink alcohol’ very
% much preserves, if not augments, the affected-agent semantics of ‘drink’ and of the
% objectless construction. The act of drinking alcohol involves a highly specific in-
% tended effect on the agent – intoxication – which is not only clearly noticeable to
% the agent himself but also frequently observable to others. It is this intention of
% achieving a particular effect, and subsequently the habit of being in a particular
% state, namely intoxication, that we attribute to people when we use drink without
% an overt object: a statement like John drinks clearly implies that he drinks for the
% purpose of getting drunk, and that he does so frequently – not that he occasion-
% ally sips a glass of wine with his dinner because he enjoys the taste.  
% Given that the default reading ‘drink liquids’ is unavailable, or rather unneces-
% sary, for pragmatic reasons, then, the ‘drink alcohol’ reading is ideally suited to
% take over as standard interpretation for generic drink: affectedness of the agent is
% crucially relevant to its semantics, and it has a sociocultural significance which
% makes it highly eligible for being cast in some simplified, conventionalised expres-
% sion. In short, to say of someone that ‘he drinks things’ does not convey any inter-
% esting information about the person in question, while saying that ‘he drinks alco-
% hol (frequently/habitually/too much)’ does; so the construction which would
% normally be used to express the former is more usefully employed as a conven-
% tionalised expression of the latter.  
% Objectless eat, on the other hand, shows a different distribution. There does
% not appear to be any such socially significant variety of the act of eating which
% could usefully take over the unnecessary generic construction He eats in English;
% which means that eat is not normally found in this kind of construction – cer-
% tainly not with reference to humans or animates. On the other hand, objectless eat
% is found with reference to specific acts of eating that are either ongoing or con-
% cluded; and in this use it appears to have the sense of ‘eat a meal’.
% Recall again that the semantic effect of using an affected-agent verbs in an ob-
% jectless construction is to emphasise the affectedness of the agent, to the extent that
% the patient, being immaterial to the meaning conveyed, is suppressed. An object-
% less ‘eat’ sentence, then, is a statement about the effect on himself that the agent
% achieves through the act of eating. Typically, the effect one seeks to achieve by eat-
% ing is that of eliminating hunger; a typical act of eating consists of eating until one
% is full. Since the amount of food required to achieve this is what we usually refer to
% as ‘a meal’, this explains the interpretation of objectless ‘eat’ as ‘eat a meal’: if nothing
% else is specified, we interpret ‘eat’ as meaning ‘eat until full’ – i.e. eat a meal.
% Indefinite object deletion, then, functions much like antipassivisation in that
% it is essentially a syntactic mechanism used in contexts of low semantic transitivity.

% *Naess (2007: 55)*  
% > A phenomenon which in a sense is similar to that of indefinite object deletion
% is the occurrence of suppletive verb pairs for transitive and intransitive uses of ‘eat’.

% *Naess (2007: 55-56)*  
% SPOSTARE IN CAPITOLO SU ASPECT (TELICITY?)  
% > An affected argument is described by Tenny (1994) as “one that makes the
% event described by the verb delimited, by undergoing a change of state that marks
% the temporal end of the event”. As a result of this, affected arguments are “measur-
% ing arguments”, an argument type which “measures out and imposes delimited-
% ness on the event” (Tenny 1994 :158). In other words, affectedness is a property
% which defines the perceived endpoint of a verbal event; [...]  
% This view of affectedness has interesting implications for the notion of Af-
% fected Agent: If affected arguments “measure out” events, then the agent of a verb
% like ‘eat’, by virtue of being affected, might function as an argument serving to de-
% limit the verbal action. In other words, an Affected Agent is, or can be construed
% as, the endpoint of the act of eating. [...]  
%  It is essential to note that it is this effect, rather than that on the
% patient, which is the main goal of the agent’s act: we eat in order to achieve an ef-
% fect on ourselves, not primarily on the food.  
% In Tenny’s model, an event cannot show more than one instance of “measur-
% ing out” by an argument, and only internal arguments can measure out events.
% Facts such as those just discussed, however, suggest that it is in fact possible for an
% event to have two potentially measuring arguments, although only one of them
% may function as an actual measuring argument in any given clause. Furthermore,
% they suggest that the restriction on which arguments may be measuring arguments
% is not a formal restriction in terms of internal vs. external arguments, but rather a
% semantic restriction in terms of affectedness: any affected argument has the po-
% tential of serving to measure out an event.

% *Naess (2007: 57)*  
% > The choice of which argument should be taken to measure out an event such as
% that of eating is in reality a question of perspective. Either one may choose to focus
% on the objectively most affected participant in the event, the Patient, whose af-
% fectedness is immediately observable to any objective bystander. Or else one may
% focus on the most salient effect from the agent’s point of view, namely the effect
% that the agent registers directly on himself and which constitutes his motivation
% for engaging in the act of eating.

% *Naess (2007: 61-63)*  
% QUINDI EAT NON È TRANSITIVO PROTOTIPICO MANCO PE GNENTE  
% > A similar, though more complex, situation is found in the Mayan language
% Yukatek of Mexico. To begin with, it should be noted that Yukatek has no under-
% ived transitive verb meaning ‘eat’; han ‘eat’ is intransitive, and can only be used
% with a direct object if transitivised with an overt transitivising suffix (Jürgen
% Bohnemeyer p.c.). [...]  
% In general, the class of inherently imperfective verbs, which take the transitive
% subject cross-referencing marker in their unmarked aspect, consists of “activity
% verbs” – verbs such as ‘work’, ‘run’, ‘bathe’ and ‘dance’ belong to this class. The class
% of inherently perfective verbs, which in their unmarked aspect show the marker
% otherwise used to cross-reference transitive objects, consists mainly of “change-of-
% state” verbs such as ‘arrive’, ‘awaken’, ‘die’ and ‘fall’.
% Han ‘eat’, however, despite being considered an activity verb by Krämer and
% Wunderlich (1999 :447) belongs to the inherently perfective class. In other words,
% when it is unmarked for aspect, it takes the subject cross-referencing marker which
% with transitive verbs is used for objects. [...]  
% The classification of ‘eat’ as an “inherently perfective” verb in Yukatek can sim-
% ilarly be understood as a consequence of the verb’s having an Affected Agent argu-
% ment. 

% *Naess (2007: 63)*  
% > Yukatek is by no means the only language where the process of causativisation
% functions differently with respect to ‘eat’ than for other verbs. Amberber (2002)
% notes that in a number of languages, causative morphemes or constructions which
% in general are restricted to occurring on intransitive verbs exceptionally occur on
% just one small class of transitives: the ingestive verbs.

% *Naess (2007: 75)*  
% > 4.4.3 ‘Eat’ grammaticalised as marker of agent affectedness  
% The semantics of affectedness inherent to ‘eat’ is reflected in a number of languag-
% es through the use of this verb as an auxiliary, as a light verb in noun-verb con-
% structions, and other grammaticalised uses. In most of these languages such uses
% of ‘eat’ express a sense of undergoing, affectedness or adversativity:

% *Naess (2007: 77)*  
% SPOSTARE IN CAPITOLO SU TELICITY! (CONTINUA DOPO, CREDO)  
% > Of the properties included in Hopper and Thompson’s list of Transitivity pa-
% rameters, the most obvious candidate for an explanation for the intransitive be-
% haviour of ‘eat’ verbs would appear to be parameter C, “aspect” – telic vs. atelic. As
% Hopper and Thompson’s Transitivity notion is explicitly defined as a property of
% clauses, not all their parameters are directly applicable to individual verbs; but for
% those that are, the verb ‘eat’ must be ranked as “high” at least for parameters A, B,
% E, H, and I: it involves two participants, denotes an “action”, is necessarily voli-
% tional, and takes an A which is “high in potency” (human or animate) and a high-
% ly affected O. Telicity, on the other hand, has sometimes been appealed to as an
% explanation for the aberrant behaviour of ‘eat’ verbs. Thus Van Valin and LaPolla
% characterise ‘eat’ as “not inherently telic” (1997 :112), while Tenny (1994) main-
% tains that the telicity of ‘eat’ depends on the “delimitedness” or “non-delimited-
% ness” of its measuring argument, that is, its object: “If Chuck eats an apple, he fin-
% ishes eating when the apple is gone, but if he eats ice cream, he continues eating for
% an indefinite period of time, because there is an indefinite quantity of ice cream.
% (He may even continue eating ice cream forever, if he is in a world that never runs
% out of ice cream)” (Tenny 1994 :24).
% Quite apart from the fact that the real world does not actually work this way

% *Naess (2007: 78-79)*  
% COME SI CONCILIA QUESTO CON MEDINA? PER LEI EAT È TELICO O ATELICO?  
% > It is well known that telicity depends not only on the predicate of a clause but
% also on the nature of its object, if it has an object, so that bare-plural or mass noun
% objects lead to atelic readings (Dowty 1991): John built a chair in an hour/*for an
% hour vs. John built chairs for an hour/*in an hour. This variation is not specific to
% ‘eat’ but is apparently characteristic of verbs with incremental themes (Dowty
% 1991, Tenny 1994). The interesting question in the case of ‘eat’ is how this verb
% behaves when it is used intransitively, i.e. when the presence of different kinds of
% objects cannot influence the reading.
% Intransitive ‘eat’ is in fact perfectly compatible with an adverbial of comple-
% tion: I ate in five minutes, then rushed off to work. This clearly shows that intransi-
% tive ‘eat’ does in fact have a delimited – that is, a telic – reading; and since no object
% is present, this telic reading cannot derive from the “delimitedness” of the object.
% Rather, the telic reading arises from the affectedness of the agent. [...]  
% On the other hand, we also find ‘eat’ with adverbials of duration: We ate all
% evening. This is a rather striking alternation which to my knowledge is relatively
% rare with intransitive verbs, though it does occur with at least two other types of
% English intransitives. One is certain “reflexive” verbs of body care such as shower
% and bathe, which can also be construed either with adverbials of completion or of
% duration: I showered in five minutes or I showered for half an hour. The completive
% reading here implies a particular result state of the agent; I showered in five minutes
% means that it took me five minutes to attain the desired degree of cleanness, where-
% as I showered for half an hour only means that I spent half an hour standing under
% the shower. In the latter case, no result state is entailed; the implication of clean-
% ness can be cancelled by a sentence such as I showered for half an hour but I still
% couldn’t get all the dirt off my skin.
% Secondly, the alternation is shown by at least some verbs whose subject under-
% goes a process which is conceived of as typically leading to a result state, but which
% is in principle independent of this result state; that is, it may be halted before the
% state is achieved or extended past the point where the state is achieved, while still
% being essentially the same process. This is the case for the verbs cook and bake; [...]  
% What these different verbs have in common is that they all refer, when used with
% an adverb of completion, to a result state of the subject argument; that is, their sub-
% jects are affected by the action in question. Once again, then, eat here patterns with
% verbs which are characterised by affecting their subject argument; reflexive verbs
% such as shower and bathe and patient-subject verbs like (intransitive) cook and bake. [...]  
% Furthermore, the possibility of a telic reading for intransitive ‘eat’ is unusual for
% intransitive uses of verbs that may delete their objects in English; we may say Dad
% cooked dinner in half an hour or Dad cooked for half an hour but not *Dad cooked
% in half an hour; Mary sewed a dress in an hour or Mary sewed for an hour but not
% *Mary sewed in an hour.

% *Naess (2007: 79)*  
% TELICITY BOCCIATA COME FATTORE PER EAT VERBS  
% > What is relevant is that the verb has a
% telic reading even when it is used intransitively. In other words, the intransitive use
% of ‘eat’ verbs cannot be explained by appealing to the atelicity of such verbs: if ‘eat’
% can be used intransitively because it has a low value for the parameter of telicity,
% then we would expect an intransitive clause with ‘eat’ to necessarily have an atelic
% reading. But if a verb may be telic and still occur in intransitive contexts, then it can-
% not be the telic-atelic parameter which is the source of its reduced transitivity.

% *Naess (2007: 80)*  
% >  (Van Valin and LaPolla 1997 :149). It is this lack of a second macrorole which is taken to
% account for the frequently observed intransitive behaviour of ‘eat’ verbs.

% *Naess (2007: 81)*  
% >  The claim made by the analysis presented here is that it is precisely the
% affectedness of the agent which allows the object of ‘eat’ and similar verbs to be
% construed as “irrelevant”; since the effect of the act on the agent himself is the
% primary goal of the agent’s acting, the patient is of secondary interest and rela-
% tively low in distinctness, and may therefore be demoted or omitted altogether.

% *Naess (2007: 126)*
% > For most languages for which I have been able to find data, it is the case that if
% they allow indefinite object deletion with any verb, they allow it with ‘eat’ (and
% usually also ‘drink’). [...]  
% many of the languages which may omit the objects of ‘eat’ and ‘drink’ also
% include in their set of IOD verbs other verbs denoting events which affect their
% agents or their A participant. Another example given for Malayalam in Asher and
% Kumari (1997) is vaayikkuka ‘read’, another act having an effect on its agent. 

% *Naess (2007: 127-128)*  
% EFFECTED OBJECTS  
% > However, there is another type of verb which are crosslinguistically common
% in IOD constructions, which do not fall under this explanation as their agent par-
% ticipant is not typically affected, namely verbs with effected objects, discussed in
% 5.3.5. As noted there, an effected object is one that comes about as a result of the
% verbal action; it did not exist before the action began, nor does it come into exist-
% ence if the action is interrupted before it is completed. If someone is, for instance,
% writing a letter but is interrupted before they finish, we cannot say that a letter has
% been produced. [...]  
% The unique property of effected-object verbs is that
% nonreferentiality of the object is inherent to the semantics of the verb itself; that is,
% the use of the verb itself indicates that the object is nonreferential and therefore
% less prominent in the discourse context, less distinct from the general background,
% and not affected in the usual sense. Note that IOD with such verbs is most felici-
% tous in grammatical contexts presenting the act as ongoing, and consequently, the
% object as not yet completed: I am/was writing. By contrast, in perfective contexts,
% where the object must be understood as having been completed, and therefore
% referential, omission is all but impossible:??I have written,??I have baked. (Com-
% pare these to the very common perfect use of objectless eat: I have eaten; cf 6.3.3). [...]  
% The property shared by the omissible objects with affected-agent and effected-
% object verbs is a relatively low degree of distinctness. We saw in chapter 4 that Pa-
% tient participants of affected-agent verbs are not maximally distinct from the Af-
% fected Agent, and that they are omissible because the Affected Agent may itself be
% construed as the endpoint of the event, by virtue of being affected. Effected objects
% are nonaffected and nonreferential, and therefore show a low degree both of dis-
% tinguishability from the Agent and distinctness from the general background; this
% low degree of distinctness makes them highly susceptible to omission.

% *Naess (2007: 138)*  
% > In fact, a similar effect can be observed for English. Consider the verb kill, by
% any account a very highly transitive verb, with an active and, in the default inter-
% pretation at least, controlling agent, and a radical effect on its patient. A generic
% objectless clause with a human subject is not particularly good for kill: *John kills,
% meaning that he habitually kills people or animals, is unacceptable in English, al-
% though for the reasons outlined above it may be improved by the addition of a
% purpose clause: John kills for the sheer thrill of it. However, with an inanimate and
% therefore necessarily nonagentive subject a generic objectless clause is unproblem-
% atic: Smoking kills. Note that both the generic reading and the inanimate subject
% are required for the object to be omissible; *Smoking killed is unacceptable.

qui iniziava il vecchio paragrafo "eating and drinking"

% *Naess (2007: 13)*  
% KILL OR CUT, NOT EAT! discutere sul fatto che tutti i paper in genere parlano di "eat", che non è molto prototipico...  
% > Thus, for example,
% verbs like ‘kill’ or ‘cut’ are expected to be treated as transitive across languages – in-
% deed, verbs like these are frequently the basis for determining what should count
% as the basic transitive clause of a language. 

% *Naess (2011: 413)*
% > Verbs referring to acts of eating and drinking show a crosslinguistic tendency to behave in ways
% which distinguish them from other verbs in a language. Specifically, they tend to pattern like
% intransitive verbs in certain respects, even though they appear to conform to the definition of
% ‘prototypical transitive verbs’. The explanations which have been suggested for this behaviour fall
% into two main categories: those referring to telicity or Aktionsart, and those referring to the fact
% that such verbs describe acts which have ‘affected agents’, i.e. they have an effect on their agent as
% well as on their patient participant. The latter observation has further led to reexaminations of the
% notion of transitivity in general. [...]
% the notion of transitivity as a prototype concept. This idea was first articulated by Hopper and Thompson (1980)

% *Petho & Kardos (2006: 30)*  
% INVECE È CHIARO! LOCK NON È UN AFFECTED-AGENT VERB
% > It does not become clear
% why e.g. eat can be used intransitively, as opposed to lock, which requires its object to appear on
% the syntactic surface, even though its relevant selection restrictions do not seem to be any less
% specific (eat requires food as its object, whereas lock requires an object that has a lock, e.g. a car
% or a door).

% *Newman & Rice (2006: 5-6)*  
% > Van Valin and LaPolla (1997:112) explicitly remark that
% “...eat is not inherently telic, unlike kill and break; hence it must be
% analyzed as an activity verb, with an active accomplishment use”. For
% them, the ‘activity verb’ use (He ate, He ate spaghetti for ten minutes) is
% the ‘basic’ meaning of EAT.

% *Newman & Rice (2006: 14)*  
% > There is proportionately more intransitive usage with DRINK than there is
% with EAT . The difference is arguably influenced by the existence of
% specialized meanings associated with the intransitive (the ‘specific category
% indefinite’ kind of interpretation à la Huddleston and Pullum 2002: 303-
% 305 or Rice 1988). In the case of EAT the specific interpretation is ‘meal’,
% whereas with DRINK it is ‘alcoholic beverage’ (especially when consumed
% in an habitual and/or excessive manner). This use of intransitive DRINK is a
% very familiar one in casual conversation (some examples from sBNC are
% All they do in that house is drink and smoke; Because her daddy drinks in
% there in the pub...; He bought a bottle of brandy at the first liquor store he
% found and he began to drink), reflecting the prominence of alchol
% consumption as a topic of discourse. Comparing EAT and DRINK in this way
% is instructive for demonstrating the kind of variation that can exist between
% lexical items, even those which define and exhaust a class (cf. Levin 1993:
% 213-214). 

% *Naess (2011: 414)*  
% IMPORTANTISSIMO ELENCO DI CARATTERISTICHE + ESEMPI ESOTICI  
% > Types of ‘intransitive behaviour’ exhibited by eating and drinking verbs include (but
% are not restricted to) the following

% *Kardos (2010: 1)*  
% IMPORTANTISSIMO!!! SI RICOLLEGA ALL'AFFECTED-AGENT ACCOUNT (riflettere, ma penso di sì!)  
% >  it is often argued that pseudo-transitives, some core examples of which are eat
% and drink, exhibit both transitive and intransitive properties as the roles that their arguments
% play in the denoted event show overlapping properties. 


% *Naess (2011: 420)*  
% PER SPIEGARE "not clear why alcoholic drinks should be seen as ‘prototypical’; most acts of drinking by
% most people involve objects other than alcohol"  
% > An alternative explanation which has been proposed starts from the affected agent analysis
% described above, and argues that the function of object omission with verbs of eating
% and drinking is to highlight the effect of action on the agent, by removing the other
% affected argument, the object. For ‘drink’ this leads to an ‘alcohol’ reading because the
% consumption of alcohol is a culturally salient activity, and, more importantly, because
% drinking alcohol is associated with a very specific and often directly observable effect on
% the agent, that of intoxication. For ‘eat’, the intended effect of eating is said to be that of
% becoming full, and the amount of food required to achieve this is conventionally referred
% to as ‘a meal’ – hence the ‘eat a meal’ reading of objectless ‘eat’ (Nжss 2007:141–4).

% *Naess (2011: 421)*  
% VERBS OF CHANGE-OF-STATE IN MOLTI ALTRI PAPER!  
% > However, the semantic properties that set these
% two groups of verbs apart from other two-participant verbs are different: Verbs of eating
% and drinking have an effect on their agent, meaning that the object can be left
% out if the effect one wishes to focus on is that on the agent. Verbs of creation, on
% the other hand, have non-referential objects – the objects do not exist until the verbal
% action has been brought to completion – and it is this non-referentiality which
% most plausibly accounts for the omissibility of effected objects in languages like English
% (Hopper 1985; Næss 2007:127). One might propose a shared property of low
% distinctness of the object; with affected-agent verbs, the object is low in distinctness
% because it shares its defining semantic property, affectedness, with the agent, and with
% effected-object verbs it is low in distinctness because it is non-referential (Næss
% 2007:128).

% *Naess (2007: 136)*
% > The IOD construction with a purpose clause – John murdered for the money – does not necessarily have an iterative reading, 
% as pointed out above. Rather, such clauses are construable as a kind of affected-agent construction where the affected- agent
%  reading is not imposed by the semantics of the verb, but rather by the pur- pose clause. A statement of the agent’s motivation 
%  or purpose in performing an act is essentially a statement of the benefits that the agent hopes to achieve in acting; in other words
%   the intended effect of the act on the agent. Affectedness of the agent, then, is not necessarily inherent to the semantics of a specific
%   verb, but may be introduced by other elements of a clause.  

% *Naess (2007: 52)*  
% LA MAX DIST ARG HYP LA DICO IN "DIRECT CONSEQUENCES" SOPRA! citando naess 2007
% > There is considerable support in the literature for the idea that a prototypical
% agent is unaffected, although systematic empirical evidence for the relevance of
% this property has rarely been presented. For example, Langacker (1991 :238) states
% that in a canonical transitive clause, the agent “remains basically unaffected”; while
% Kittilä (2002a :237) gives as one of the defining features of a transitive event that
% “the agent is not affected”. The Maximally Distinguished Arguments Hypothesis
% places this observation in a larger theoretical context by stating that the require-
% ment that the agent remain unaffected is a function of the more basic requirement
% that the two participants of a transitive event should be maximally distinct, con-
% ceptually as well as physically.

% *Naess (2007: 52)*  
% MOLTO IMPORTANTE! DIMOSTRA CHE EAT/DRINK NON SONO VERBI TRANSITIVI PROTOTIPICI  
% > The discussion in this chapter will center round the morphosyntactic behaviour of
% verbs meaning ‘eat’ and ‘drink’ crosslinguistically. Such verbs have sometimes been
% referred to in linguistic literature as “ingestive verbs”. This term appears to originate
% in Masica (1976), where it refers to “a small set of verbs... having in common a se-
% mantic feature of taking something into the body or mind (literally or figurative-
% ly)” (Masica 1976 :46). As much of this chapter will be dedicated to demonstrating,
% these verbs often show intransitive characteristics. Masica considers them to be
% “occupying a halfway station between intransitives and transitives, since the object
% in question can frequently be dispensed with in favor of concentration on the activ-
% ity as such” (p. 48). He takes this observation to explain certain facts about the se-
% mantic behaviour of these verbs under causativisation in Hindi, cf. 4.3.4 below.

% *Naess (2007: 53)*  
% > A number of linguists have noted that acts of eating and drinking are charac-
% terised by affecting their agent. For example, Starosta (1978) characterises intran-
% sitive uses of ‘eat’ (John is eating) as having a “Patient” subject, by which he pre-
% sumably means that the subject argument is affected by the act of eating. Wierzbicka
% (1982) accounts for constructions of the type have a drink essentially by arguing
% that they describe actions undertaken by an agent for the sole purpose of achiev-
% ing an effect on himself. Nedjalkov and Jaxontov (1988) include “verbs meaning
% ‘to eat’, ‘to drink’” in the class of verbs forming “possessive resultative construc-
% tions”, which apply in cases where “the result of the action affects the underlying
% subject rather than the immediate patient of the action” (Nedjalkov and Jaxontov
% 1988 :9). Haspelmath (1994) similarly notes that verbs whose agent argument is
% “saliently affected” may form “active resultative participles”, meaning that the re-
% sultative participles describe an effect achieved on the agent rather than the patient
% argument. Among such affected-agent verbs Haspelmath counts ‘eat’ and ‘drink’ as
% well as ‘learn’, ‘see’, ‘put on’ and ‘wear’ (Haspelmath 1994 :161).

% *Naess (2007: 54)*  
% > The best-known and most frequently-discussed case of “intransitive behaviour” in
% ‘eat’ verbs is that of so-called indefinite object deletion (IOD), found in English
% and a large number of other languages. In these languages, ‘eat’ verbs exhibit a
% structurally fairly simple alternation: they can be used either with or without an
% overt direct object, without any further morphosyntactic differences in the verb or
% the clause.

\paragraph{DOBJ IS THERE IN SYNTAX}
questa teoria vede l'oggetto sintatticamente rappresentato! (recuperare altri pezzetti)\\
trovare il modo di separare questo paragrafo e quello sull'incorporation in una sezione sintassi vs semantica? o anche rappresentato non rappresentato? oppure syntactic transitivity, semantic transitivity?

% *Cummins & Roberge (2005: 47)*  
% > Based on work by Roberge
% (2002, to appear), we hold that all null objects are syntactically represented.2
% Roberge proposes a Transitivity Requirement (TR), parallel to the Extended
% Projection Principle (EPP) for subjects, whereby the direct-object position,
% complement of the verb, is given by Universal Grammar.

% *Cummins & Roberge (2004: 2)*
% > These possibilities cannot be attributed solely to lexical properties of the verb; if this
% were the case, certain verbs would always be able to appear without their objects
% regardless of the construction or discourse context, and others would never be able to
% appear without an object. As we will show, this is not the case. Rather, following Roberge
% (2003), we propose that null or implicit objects can be attributed to a Transitivity
% Requirement (TR) just as null subjects are ultimately due to the EPP. Recoverability for
% the EPP is morphologically based, as is evident in null subject languages, while
% recoverability involving the TR may also be semantically and pragmatically based; as we
% will show below, such recovery may be based on information derived from the verb's
% lexical semantics and Generalised Conversational Implicatures (formalised as in Levinson
% 2000) involved in the interpretation of reduced nominal forms. The factors that contribute
% to licensing superficial intransitivity—the absence of an overt object—may include lexical
% semantics, functional elements, discourse factors, and trans-clausal structural elements.
% This view is supported by a comparative study of null object possibilities in French and
% English. [...]  
% The concept of transitivity has been interpreted as a continuum in certain works, and a
% distinction has been proposed between syntactic transitivity and semantic transitivity; see,
% among many others, Blinkenberg (1960), Desclés (1998), Hopper et Thompson (1980),
% Lazard (1994). Surprisingly little is ever said about the object position itself. The
% hypothesis in Roberge (2003) is that there exists a Transitivity Requirement (TR),
% whereby an object position is always included in VP, independently of the lexical choice
% of V. The empirical motivation of this hypothesis is the well documented evidence (see in
% particular Blinkenberg (1960), Larjavaara (2000)) that any “transitive” verb has the
% potential to appear without a direct object and any “unergative” verb has the potential to
% appear with a direct object. To account for these facts, there must be a mechanism to
% generate the direct-object position, either optionally or obligatorily. The TR represents the
% second, more restrictive, possibility and conveys the concept of transitivity as a property
% of the predicate (the VP), rather than as a property of the lexical content of V. The TR is
% the internal-argument counterpart to the EPP.

% *Cummins & Roberge (2004: 3)*
% > For the purpose of our
% discussion, we define unexpressed objects interpretatively: there is an x such that x is (1)
% phonologically null, (2) involved in the event denoted by the VP, and (3) not an external
% argument. [...]  
% Two recent studies—Larjavaara (2000) on French and García Velasco & Portero
% Muñoz (2002) on English. [...]  
% The two agree that indefinite or generic null objects do not have a
% contextually available referent. GP point out that generic null objects can give rise to an
% activity rather than an accomplishment reading of the verb; L notes that null objects can
% focus attention on the activity. Both point out that the lexical characteristics of the verb
% can help to identify the referent of the null object. [...]  
% In a third study, Goldberg (2001) investigates unexpressed objects of causative verbs
% (those that entail a change of state in the patient argument) in English. She concludes that
% the option of leaving these arguments unexpressed depends largely on factors relating to
% information structure: the unexpressed object is typically neither topical nor focal, and the
% verb is emphasized somehow, by being iterative or generic, by being contrasted with
% another verb, or by having a narrow focus.

% *Cummins & Roberge (2004: 4)*
% > All of these authors implicitly or explicitly adopt the position that the missing
% argument is not syntactically represented: syntactically the verb is intransitive. In a
% generative framework, this position finds a counterpart in Rizzi (1986:509-510), who
% proposes that both the arbitrary third-person human interpretation, meaning “people in
% general” or “some people”, and the prototypical-object interpretation, where the verb's
% lexical semantics identify the object, are available lexically to saturate the argument's
% theta role and block projection. Thus, the verbs are intransitive in syntax. The absence of a
% syntactic object explains why, in Rizzi’s account, the type of sentence exemplified in (13)
% is impossible in English: there is no object that can bind the anaphor or be modified by the
% adjective. However, such sentences are grammatical in Romance; hence several accounts
% (Rizzi 1986; Authier 1989; Roberge 1991) posit a syntactically present null object.  
% (13)a.b.Ce gouvernement rend __ malheureux.  
% *“This government makes __ unhappy.”  
% Une bonne bière reconcilie __ avec soi-même.  
% *“A good beer reconciles __ with oneself.”  
% Under the TR, the object position is projected and the verb remains transitive in syntax
% in both English and French. Although we do not find sentences like those in (13) in
% English (as shown by the ungrammaticality of the glosses), there is nonetheless evidence
% that a null object has an effect on syntax in both English and French.3 

% *Dvorak (2017: 1)*  
% > Non-overt direct objects with an existential interpretation, as in (1), have been the topic of
% linguistic discussions since Chomsky introduced the transformational rule of ‘Unspecified
% object deletion’ in 1962. In the English-oriented literature, they are usually assumed to be
% represented lexically, either in the form of de-transitivizing rules that operate on individual
% predicates, see (2a), or as two separate predicates, transitive and intransitive, linked by
% a predicate-specific meaning postulate, see (2b). Alternatively, Cote (1996)  enriches the
% lexical conceptual structure (LCS) of a given verb by marking the relevant argument as
% unsubscripted zero [...]  
% More recently, this view has been challenged by Alexiadou et al. (2014), who argue for a
% syntactic re-interpretation of Bresnan’s rule of Existential Closure (EC) in (2a), such that it
% applies to any predicate with an unsaturated internal argument: JECK = λ fhe,stiλ es ∃x[f(x)(e)]
% (see Babko-Malaya 1999 and Martı́ 2011 for related proposals).

% *Lorenzetti (2008: 62)*  
% > Early accounts of the phenomenon proposed the elimination of the object from the syntax
% via “object deletion transformation” [Katz and Postal 1964, Browne 1971, Allerton 1975],
% while later works introduced the view that implicit object constructions have aspectual
% constraints [Mittwoch 1982] and that omitted objects are frequently typical, inferable or
% partially specified by the semantics of the verb [Lehrer 1970, Rice 1988].


\section{A working definition of "indefinite object drop"} \labsec{theory_workingdef}

citare Medina e tutta la teoria possibile che la conforti
è una definizione pre-teoretica che prescinde dalle proiezioni sintattiche?
QUI METTERE IL MIO DIAGRAMMA DI VENN CONCENTRICO

% \labsec{theory_continuous}
% nell'ultimo par della sezione, in fondissimo, anticipo questa sezione! devo dire di nuovo che io resto agnostica e (per la prima volta nel testo) che nei miei esperimenti faccio in modo che il contesto non giochi alcun ruolo (mettere riferimento alla sezione sperimentale, dove devo ancora dire questa cosa)

% hopper thompson 1980: 251 (abstract)
% TORNA BENISSIMO COL MIO DIAGRAMMA DI VENN
% The grammatical and
% semantic prominence of Transitivityis shown to derive from its characteristicdis-
% coursefunction:high Transitivityis correlatedwithforegrounding,and low Transitivity
% with backgrounding

% *Lorenzetti (2008: 63)*  
% QUESTA PAGINA DICE IN MODO SINTETICO COSE MOLTO IMPORTANTI!!!  
% L'ESEMPIO (5) È UN CHIARO ESEMPIO DI COME UN VERBO POSSA CREARE INDEF DROP IN MODI DIVERSI?
% > 3.1.1. Indefinite Null Objects
% This category, also known as Indefinite Null Complements (henceforth INCs) [Fillmore
% 1986] or Unspecified Object Alternation [Browne 1971], is typical of a variety of activity
% verbs of the eat type, such as drink, sing, bake, cook and paint among the others, which have
% a pronounced manner component in their meaning and fairly circumscribed selectional
% restrictions. Hence, the content of the null object is more or less predictable: it will
% correspond to the literal rather than to the metaphorical meaning of the verb , and is
% sometimes argued to be restricted in usage, i.e. an expression such as I’m cleaning IS most
% likely to refer to the interior of a house, rather than to one’s teeth. [...]  
% However, postulating a reading of these verbs in terms of stereotypic entities associated to
% them does not always seem appropriate, since while in example (3) the phantom object is a
% meal, i.e. the apparently stereotypic entity associated to the verb eat, in an example like the
% most likely context would not lead to the interpretation that the person is accustomed to
% having a meal or as many meals as she can during the whole day.
% (5) I started working out, but I would eat all day after that.
% On the contrary, the most typical interpretation in this case is likely to be achieved through
% the underspecified word “food”, a representative of the entire class of edible things. However,
% the fact that the object is unexpressed in this case suggests that what the person eats is
% irrelevant for the current purpose of the interaction.
% We can suppose, in this respect, that a better explanation of the restrictions on these null
% objects is that they have to be consistent with the underlying context, the intentional structure
% of discourse and the shared relevance [Sperber and Wilson 1986] at the time of utterance.

% *Eu (2018: 525)*
% QUESTO TORNA BENISSIMO COL MIO DIAGRAMMA DI VENN CONCENTRICO!
% CITATO IN "OTHER ACCOUNTS"
% > Indefiniteness as unknownness, however, is challenged by cases where INC-objects
% fairly clearly refer to a specific individual in the context. Groefsema (1995: 142, 44)
% introduces (5a, b) against Fillmore’s comment on (3a) and says that here eat and drink
% do take the contextual referents (sandwiches, glass of beer) as their objects, although
% the amount eaten or drunk is ‘unspecified’:
% (5) (a) John brought the sandwiches and Ann ate.
% (b) John picked up the glass of beer and drank.
% (Groefsema 1995: 142, 144)

\begin{figure}[htb]
\caption{CAPTION.}
% lexical and aspectual factors
\labfig{objectdrop_venndiagram}
    \begin{tikzpicture}
\coordinate (O) at (0,0);
\foreach \j in {1,...,3} \draw (O) circle (3.5-\j);
\foreach \k/\text in {0/extra-linguistic context,1/intra-linguistic context,2/lexicon} \draw[decoration={text along path,reverse path,text align={align=center},text={\text}},decorate] (2.6-\k,0) arc (0:180:2.6-\k);
    \end{tikzpicture}
\end{figure}
\setchapterpreamble[u]{\margintoc}
\chapter{Object-dropping verbs}
\labch{objectdrop}

As mentioned in \refch{intro}, this thesis is about indefinite implicit objects. What is "indefinite" about them? In what sense can they be considered "implicit"? And ultimately, what is objecthood itself? This Chapter will answer these questions in reverse order, from the most general to the most specific one. In \refsec{theory_transitivity} I will make reference to the transitivity continuum, as defined by \textcite{HopperThompson1980} and further explored by later literature. In \refsec{theory_def_vs_indef} a crucial distinction will be made between definite and indefinite object drop, following \textcite{Fillmore1986} and subsequent works. The nature of \textit{indefinite} object drop will be finally described in \refsec{theory_defindefinite}, and a working definition (for the purposes of this thesis) will be provided in \refsec{theory_workingdef}.

\section{Transitivity as a prototype} \labsec{theory_transitivity}

\subsection{Transitivity in Hopper \& Thompson (1980)} \labsec{theory_ht1980}
School kids everywhere are used to call "transitive" the verbs which take an overt direct object, with reference to the syntax. In a traditional semantic definition, a clause is deemed "transitive" if it describes an event where the action performed by an Agent "passes over"\sidenote{Hence the name of the concept of transitivity, from Latin \textit{transire} 'to go over'} to a Patient, which usually undergoes some kind of transformation.\\
Going beyond these naive definitions, but still capturing their spirit, \textcite{HopperThompson1980} first proposed an account where transitivity is interpreted as a scalar concept whose strength depends on several parameters, or, to use more modern terminology, as a prototype category \parencite{Naess2007}. In particular, they identified ten parameters \parencite[252]{HopperThompson1980}, reported almost \textit{verbatim} from \reftab{ht1980_parameters}.

\begin{table}[htb] % the "htb" makes table env unfloaty
\caption{\textcite[252]{HopperThompson1980} defined transitivity as a prototype concept determined by ten parameters.}
\labtab{ht1980_parameters}
\begin{tabular}{rl|ll}
 & & \textbf{high transitivity} & \textbf{low transitivity} \\
 \hline
A. & \textbf{Participants} & 2+ (Agent and Object)  & 1 participant  \\
B. & \textbf{Kinesis} & action  & non-action  \\
C. & \textbf{Aspect} & telic  & atelic  \\
D. & \textbf{Punctuality} & punctual  & non-punctual  \\
E. & \textbf{Volitionality} & volitional  & non-volitional  \\
F. & \textbf{Affirmation} & affirmative  & negative  \\
G. & \textbf{Mode} & realis  & irrealis  \\
H. & \textbf{Agency} & A high in potency  & A low in potency   \\
I. & \textbf{Affectedness of O} & O totally affected  & O not affected  \\
J. & \textbf{Individuation of O} & O highly individuated  & O non-individuated  
\end{tabular}
\end{table}

% PUNCTUALITY È LA PERFECTIVITY???
SPIEGARE QUELLI NON OVVI

A terse summary by \textcite[15]{Naess2007} clearly shows the relation between the naive definitions of transitivity and the ten-parameter account by \textcite{HopperThompson1980}. A prototypical cause is understood to describe an event such that:
\begin{itemize}
    \item a volitional Agent (E, H)
    \item performs an action (B)
    \item with a tangible, lasting effect on a Patient (A, I, J),
    \item and it is presented as real and concluded (C, D, F, G).
\end{itemize}

These parameters are potentially active in all languages, but, in a true prototypical fashion, languages may differ from one another with respect to the actual subset of parameters they select as necessary criteria for transitivity.



% *Naess (2007: 18)*
% > Interestingly, an influential line of research in current functional typology ap-
% pears to explicitly contradict the statement that prototypical transitive clauses have
% highly individuated O arguments. The observation sometimes referred to as
% “Comrie’s generalisation” (de Swart 2003, Næss 2007) states that “the most natural
% kind of transitive construction is one where the A is high in animacy and definite-
% ness, and the P [= O] is lower in animacy and definiteness; and any deviation from
% this pattern leads to a more marked construction” (Comrie 1989 :128, my empha-
% sis).

% *Lorenzetti (2008: 78)*  
% >  the phenomenon of the null instantiation of
% objects is best characterised within a revised notion of transitivity to be conceived as a scalar
% property of utterances, and emerging from the interplay of a cluster of parameters [Hopper
% and Thompson 1980]. The latter, however, do not display the same degree of salience, and
% only some of them, such as agentivity, affectedness and individuation of the object are
% assumed to have a primary role in this respect.

% *Newman & Rice (2006: 6)*  
% > An influential and far-reaching re-conceptualization of the notion of
% transitivity is that found in Hopper and Thompson (1980). They identify 10
% parameters which distinguish high and low transitivity of clauses, as
% summarized in Table 2.

% *Olsen & Resnik (1997: 2)*
% > According to H&T, clauses that
% show telic aspect and individuated objects are more transitive than those with
% atelic aspect and nonindividuated objects. Furthermore, if aspect and
% individuation covary, telicity and high individuation should occur together in
% transitive clauses, and atelicity and low individuation together in intransitive
% clauses. 

% *Kardos (2010: 1)*
% > This paper is concerned with the distinct behavior of change-of-state (COS) verbs and
% pseudo-transitive verbs. The members of the former class, such as break, crack, explode, dry,
% harden, melt and open, are considered to exhibit prototypical transitive behavior (Hopper and
% Thompson 1980, Blume 1998, Testelec 1998, inter alia), whereas those of the latter class,
% such as eat, drink, sweep, write, knit, translate, which are also referred to as ambitransitive or
% labile (Næss 2009), are known to feature both transitive and intransitive properties. [...]  
%  First, it is widely assumed that what makes a verb
% prototypically transitive is the semantically maximally distinct behavior of its arguments (e.g.
% Kemmer 1993, Næss 2009). [...] with these verbs the obligatory
% presence of two arguments follows from the fact that they denote complex events1 (Levin and
% Rappaport Hovav 2005, van Hout 1996).

% *Kardos (2010: 3)*  
% > The term 'pseudo-transitive' chararacterizes different types of verbs, such as verbs of
% creation (e.g. cook, write, knit), verbs of ingestion or consumption (e.g. eat, drink), and verbs
% of surface contact (e.g. sweep). They occur in a variety of grammatical environments, unlike
% change-of-state verbs. For instance, they appear in the transitivity alternation, in which the
% alternating verb may appear with an explicit object or with an indefinite null complement
% (INC)3 (Fillmore 1986).

% *Kardos (2010: 4)*  
% V. CENNAMO PER INCREMENTAL THEME!
% > Finally, an interesting feature of verbs of creation, verbs of ingestion and verbs of surface
% contact is that they occur with a special type of direct object, called incremental theme (for a
% detailed discussion of incremental themes, see Dowty 1991, Levin and Rappaport Hovav
% 2005, Wechsler 2005, Ramchand 2008).

% *Kastner & Zu (2014: 19-20)* 
% > Looking at implicit objects in Hebrew,
% episodic contexts allow for any arbitrary object (140) while generic contexts require a [+human] or
% [+animate] object (141). Examples are from Landau (2010:384), and see Cattaneo (2008) for discussion
% of a similar requirement in Italian implicit objects:  
% (140) ‘I hope you managed (=were quick enough) to photograph. (pointing to a bird)’  
% (141) ‘The fact that in Texas, they hang people/*paintings in rooms without windows, drives me crazy.’  
% Landau concludes that a full typology of implicit objects might have to make reference to [+human] or
% [+generic] specifications

\subsection{Later refinements} \labsec{theory_postthompson}
Naess si ricollega perfettamente all'agent affectedness che dico sotto!

% *Naess (2007: 28)*  
% SI RICOLLEGA BENISSIMO ALLA PROTOTYPICAL TRANSITIVITY!  
% > In what Kemmer calls a prototypical two-participant event, corresponding to
% what will here be called a (semantically) transitive event, it is a basic requirement
% that the Initiator and the Endpoint participants be two physically distinct entities,
% and that the event involve some kind of transmission of force from the Initiator to
% the Endpoint participant. Kemmer further specifies that the initiating entity – an
% agent, in this particular case – should be human and acting volitionally, while the
% Endpoint – a patient – should be inanimate, definite, and completely affected by
% the event instigated by the agent.

% *Naess (2007: 29)*  
% > A prototypical transitive event, then, involves a prototypical agent and a proto-
% typical patient, neither of which shares any of the defining properties of the other.
% The fact that prototypicality tends to be recursive is noted by Taylor (1995 :61):
% “the very attributes on whose basis membership in a category is determined are
% more often than not themselves prototype categories.” We remember from the pre-
% vious chapter that a fundamental property of prototype categories is that they are
% defined in maximal distinction to each other.

% *Naess (2007: 30)*  
% > (3.1) The Maximally Distinct Arguments Hypothesis  
% 		 A prototypical transitive clause is one where the two participants are maxi-
% mally semantically distinct in terms of their roles in the event described by
% the clause.


\section{Definite \textit{vs} indefinite object drop} \labsec{theory_def_vs_indef}

\subsection{Either definite or indefinite: discrete accounts} \labsec{theory_discrete}

Definite or indefinite: discrete accounts -- Fillmore (1986)’s seminal work

% *Olsen & Resnik (1997: 7)*  
% IMPORTANTE! DEFINITE VS INDEFINITE  
% > Resnik (1993) also tested whether there was a difference between definite
% and indefinite implicit object alternations, in terms of a correlation between
% selectional preference and implicit object instances. He hypothesizes (Resnik
% 1993:88-89) that "if a verb requires that an antecedent be available in the
% discourse context, the verb itself might not contribute as much information about
% the omitted object." Similarly, the transitivity hypothesis ranks clauses with
% definite implicit objects—those that have discourse antecedents and telic aspect—
% as higher in transitivity than clauses with indefinite implicit objects. [...]  
%  Moreover, H&T's individuation property
% (elaborated here to include selectional constraints) covaries with aspect as
% predicted by the transitivity hypothesis, unifying the "typicality" facts and the
% telicity properties where previously they represented two apparently disparate
% requirements on the implicit object constructions.

% *Olsen & Resnik (1997: 9)*  
% > An intriguing but as yet unexplored question concerns the potential
% relationship between Resnik's (1993,1996) information-theoretic account of
% selectional constraints and H&T's proposal that the transitivity spectrum is closely
% bound up with discourse grounding. Backgrounded material may be distinguished
% from foregrounded material by the amount of information added to the discourse,
% as measured against prior context; this is also the intuitive characterization of
% weakly versus strongly selected arguments. We therefore conjecture that the
% relationship between transitivity and discourse new information may also be
% amenable to an information-theoretic treatment.

% *Liu (2008: 296)*  
% > As Rutherford (1998:191) correctly points out, “Verbs
% exhibiting surface characteristics of intransitivity, however, are a mixed bag and can
% be distinguished syntactically and/or semantically.”

%  *Liu (2008: 289)*
%  tutto il paper riguarda questa distinzione in gruppi per transitività  
%  > verbs used without an object into four categories: 1) pure intransitive verbs, such as
% arrive, rise, and sleep; 2) ergative intransitive verbs, such as break, increase, and open; 3)
% transitive-converted intransitive verbs of activity, such as eat, hunt, and read; 4) object deleting
% verbs, warranted by discourse or situational context, such as know, notice, and
% promise.

% *Medina (2007: 13)*
% > The indefinite implicit object construction is to be distinguished here, as much as possible, from 
% definite implicit objects which are defined here to be objects in which the speaker does need to have 
% the specific identity of the object in mind.This distinction serves to distinguish implicit objects whose 
% particular meaning can be recovered from the preceding discourse or disambiguating physical context 
% (definite implicit objects) from implicit objects whose meaning is recoverable only from the verb in the
%  sentence (indefinite implicit objects)

% *Petho & Kardos (2006: 30)*  
% > so-called indefinite null complements (INC) and definite null complements (DNC). This
% distinction was introduced by Fillmore (1986), and aims to capture a semantic difference between
% two types of verbs. Indefinite null complements (i.e. implicit objects) of verbs receive an
% “existentially quantified” interpretation, e.g. I am eating approximately means ‘I am eating
% something’, but not ‘I am eating it.’. On the other hand, a definite null complement is interpreted
% anaphorically and must therefore have an appropriate antecedent in context to make sense

% *Garcia-Velasco & Munoz (2002: 7-8)*  
% IMPORTANTE APPUNTO SULLA RECOVERABILITY! e sulla world knowledge!  
% > The fact the phenomenon is influenced by two types of factors (lexical and
% discoursive) has led some scholars to suggest the existence of two corresponding types of
% argument omission: contextual and lexical omission. This distinction is first introduced in
% Allerton (1975), to be taken over and developed by Fillmore (1986) and Groefsema (1995).
% Fillmore establishes an interesting distinction between definite null complements (DNC),
% which basically correspond to Allerton’s contextual omission and indefinite null complements
% (INC). Fillmore (1986: 96) employs the following test to distinguish the two categories:  
% << One test for the INC/DNC distinction has to do with determining whether it would sound odd for a speaker to
% admit ignorance of the identity of the referent of the missing phrase. It’s not odd to say things like, “He was
% eating; I wonder what he was eating”; but it is odd to say things like “They found out; I wonder what they found
% out.” The missing object of the surface-intransitive verb EAT is indefinite; the missing object of the surface-
% intransitive verb FIND OUT is definite. The point is that one does not wonder about what one already knows.>>  
% The distinction is then based on the possibility of recovering the missing element. INC verbs
% do not allow recoverability from the context. [...]  
% It seems that those verbs which allow the transition from accomplishment to activity might
% correspond to IO-verbs. That is, the IO type of omission can be considered to be lexical in
% nature, and therefore influenced by both the type and nature of the verbal object and the
% semantic class of the verb itself. [...]  
% The factors of relevance here include the semantic structure of the verb, which
% itself may give prominence to one semantic component (as in the manner-result opposition),
% the speaker’s communicative intentions, which may lead him to focus on the activity itself,
% thus downgrading the referential status of the object, and world knowledge, which allows him
% to construe an action as an autonomous activity

% *Tonelli & Delmonte (2011: 55 sgg.)*  
% IMPORTANTE: TERMINI-OMBRELLO PER DEFINITE+INDEFINITE OBJ DROP  
% > In this work, we focus on null complements, also
% called pragmatically controlled zero anaphora (Fill-
% more, 1986), understood arguments or linguistically
% unrealized arguments [...]  
%  definite null complements or instantia-
% tions (DNI) and are lexically specific in that they ap-
% ply only to some predicates.  
%  indefinite null complements or instantiations (INI) and
% are constructionally licensed in that they apply to
% any predicate in a particular grammatical construc-
% tion.

% *Eu (2018: 523)*
% >  Allerton (1975) and Fillmore (1986) understand omitted objects as being indefinite, by which they mean
% that omitted objects are unknown and insulated from their potential referents available
% in the context. However, this squib presents data that challenge this understanding of
% indefiniteness, and proposes that the indefiniteness of omitted objects may be more
% precisely understood as their indeterminacy over their potential referents. [...]  
% Omitted objects
% are ‘semantically obligatory’ (Somers 1984: 510), and yet object omission is generally
% distinguished from the phenomenon where an object is missing on the surface but
% ‘recoverable’ from the context. The difference between the two types of missing
% objects, however, has not been clearly explained, and this squib investigates this
% difference to gain a more precise view of object omission.
% Allerton (1975) and Fillmore (1986) explain the difference in terms of definiteness:
% omitted objects are indefinite in reference, while recoverable objects are definite.

% *Eu (2018: 524)*
% > For Fillmore and Allerton, however, the unknownness of INC-objects does not
% mean that they are known only as the abstract semantic role assigned to the object.
% Rather, they note that INC-objects are often understood in ‘semantic specialization’,
% for instance, as ‘a meal’ as in (4a), ‘alcoholic beverages’ as in (4b), and ‘breads or
% pastries’ as in (4c):  
% (4) (a) We’ve already eaten.  
% (b) I’ve tried to stop drinking.  
% (c) I spent the afternoon baking.  
% (Fillmore 1986: 96–7)

% *Eu (2018: 525)*
% > The question, then, is how semantic specialization differs from the knownness of
% DNC-objects. Allerton (1975: 218) answers this by saying: ‘the sentence: John’s been
% drinking again may imply a particular KIND of object, but it does not refer to one
% established as DEFINITE contextually’. In other words, semantically specialized INC-
% objects always refer to a category, or ‘a particular KIND of object’, and never a specific
% instance of the category, while DNC-objects refer to specific individuals in the context.

% *Eu (2018: 525)*
% > Indefiniteness as unknownness, however, is challenged by cases where INC-objects
% fairly clearly refer to a specific individual in the context. Groefsema (1995: 142, 44)
% introduces (5a, b) against Fillmore’s comment on (3a) and says that here eat and drink
% do take the contextual referents (sandwiches, glass of beer) as their objects, although
% the amount eaten or drunk is ‘unspecified’:
% (5) (a) John brought the sandwiches and Ann ate.
% (b) John picked up the glass of beer and drank.
% (Groefsema 1995: 142, 144)

% *Eu (2018: 527)*
% >  The point here is that INC-objects themselves do not fix the actual referent
% as the obvious one; hence, whenever the context is flexible enough as in (5a–d), it is
% possible to dissociate the actual referent from the obvious one without sounding odd.
% In contrast, DNC-objects require a clear and fixed referent. So in all contexts if the
% actual referent is different from what the hearer expects on the basis of the context, the
% object cannot be deleted; therefore, if an object is deleted as a DNC-object, the actual
% referent cannot differ from the obvious referent no matter how flexible the context may
% be. The referents of DNC-objects are determinate.

% *Ruppenhofer & Michaelis (2010: 159)*  
% > Fillmore (1986) distinguishes two major types of null comple-
% ments, definite and indefinite, based on the potential for a discourse antecedent,
% and Goldberg (2006: Chapter 9) uses the discourse prominence of participants to
% explain why constructions like the English experiential perfect license argument
% omissions that are not allowable in episodic contexts (e.g., She has never failed to
% impress Ø ).

% *Stark & Meier (2018: 11)*  
% > Cummins/Roberge (2004). In their qualitative analysis of object drop in French,
% they distinguish two types of null objects (NOs): “internally-licensed Null Objects” and
% “referential” Null Objects.

% *Stark & Meier (2018: 13)*  
% > This kind of null topic [referential null objects] is controlled pragmatically according to Raposo (1986, 375), i. e.
% it has either to be given linguistically, in the preceding context, or extralinguistically, in
% the situational context.

% *Ahringberg (2015: 9)*  
% > Fillmore (1986, p. 103) further recognizes several complement constructions that may be
% omitted as definite null complements, including lexical noun phrase direct objects, indicative
% that-clause direct objects, subjunctive that-clause direct objects, and prepositional phrase
% complements of transitive verbs, to name a few.

% *Dvorak (2017: 117)*  
% >  The terms were coined by Fillmore (1969, 1986), but the distinction itself goes
% back even further, to Katz and Postal (1964) who distinguished between the deletion of
% it versus something at the level of D-structure. Fillmore (1969) proposed that it has to
% be specified for each predicate whether it can have a null complement with an indfefinite
% interpretation, or with a definite interpretation. Fraser and Ross (1970) assume that the null
% object of Max read / Max is reading undergoes ‘unspecified object deletion’, which makes
% it different from the null object of the verbs like I approved / I began / I insisted, which
% has an anaphoric interpretation. 

% *Dvorak (2017: 118)*  
% > Fillmore (1986) reinforces the lexicon-based view of ‘indefinite null complements’ by
% saying that they are “limited to particular lexically defined environments”, such as the
% object slot of verbs like eat, read, sing, cook, sew, bake (Fillmore 1986:95). He suggests that
% these verbs, when used intransitively, have an understood object that could be paraphrased
% as stuff. The referential identity of such an object is unknown, or a matter of indifference,
% as shown by the follow-up clause in (200-a). On the other hand, ‘definite null complements’
% correspond to something that is already known from the context, so they do not allow the
% same continuation (see (200-b)).  
% (200) a. He was eating; I wonder what he was eating.  
% b. #They found out; I wonder what they found out.  
%  Fillmore 1986:96  
% He is also aware that in some highly restricted mini-genres, the possibility of object omission
% is much higher: Store in a cool place, Shake before using, Keep out of reach of children. For
% a recent take on these special registers, see Ruda 2014.

% *Melchin (2019: 53-54)*  
% > The understood objects of verbs that have undergone UOA are interpreted as indefi-
% nite masses or plurals. For their indefinite status, I give evidence from AnderBois (2012).
% AnderBois notes that the understood objects may not be coreferential with any previous
% discourse referents. [...] The unavailability of coreferential interpretations is a property of indefinite DPs; definite
% DPs (including personal pronouns) typically corefer with other referents in the discourse.

%  *Cote (1996: 110)*
%  > an examination of English null objects leads one into the murky waters of the distinction between arguments of verbs and information available from world
% knowledge about events.

%  *Cote (1996: 113)*
% > null objects found in (3) and (4) which occur with a different set of verbs and which may not co-refer with a discourse antecedent, but which do seem to create a discoure
% entity available for subsequent reference. Following previous work, I refer to the verbs that take this type of null object as Indefinite Object Alternation (IOA) verbs.


% *Liu (2008: 290)*  
% in risposta a quelli che dicono che non c'è differenza tra INDEFINITE object deletion e DEFINITE obj del  
% (dovrei avere una cosa del genere negli appunti, ma dove??)
% > For example, without any contextual
% information preceding or following the sentence Mary read for a while before she
% went to bed, it is impossible to decide whether a specific object, like a book or a magazine,
% has been deleted after the verb read. Making this distinction is important
% because whether a specific object has been deleted determines whether the verb is
% used as an intransitive verb focusing on activity or as an instance of object deletion,
% a distinction that, as I will argue later in this paper, should and can be made.

% *Liu (2008: 293)*  
% corrobora il punto precedente  
% > As some linguists (e.g., Garcia-Velasco & Munoz 2002; Quirk et al. 1985) have
% correctly pointed out, in the case of the second type, the shift of the verbs’ function
% from transitive to intransitive often involves a change in the focus of meaning, i.e.,
% the focus turns from the object in the transitive use to the activity (the verb) itself in
% the intransitive use. There is no such semantic shift in the object-deleting verbs’ shift
% from transitive to intransitive.

% *Liu (2008: 300)*  
% > two groups. For instance, Quirk et al. (1985:1565) list both the ergative and the
% transitive-converted intransitive verbs as instances of “transitive → intransitive.”
% There are three significant differences between the two groups of verbs. First, the
% subject or the sole argument of an ergative verb plays the theta role of theme, but the
% sole argument of a transitive-converted intransitive verb plays the theta role of agent
% (e.g., The window broke vs. Mary ate). Second, while there is a subject change in the
% use of an ergative intransitive verb compared to its transitive counterpart (e.g., from
% They opened the door to The door opened), no such a change is involved in the use
% of a transitive-turned intransitive verb [...]  
% The third difference between the two groups is that there is no object deletion
% involved at all in ergative intransitive verbs when they shift from transitive to intransitive
% because the shift entails only a movement of the object into the subject position,
% i.e., the object is not deleted, just moved to a different position in the sentence. [...]  
% Of course, some scholars argue that there are a few instances
% in which a transitive-converted intransitive verb of activity may assume “a more specific
% meaning, so a particular kind of object is ‘understood’” (Quirk et al. 1985:1169).
% An example given by Quirk et al. is John drinks heavily. They suggest that the verb
% drink in the utterance means ‘drink alcohol,’ i.e., alcohol is the omitted object. While
% it is true that drink here means ‘drink alcohol,’ the focus of the utterance, in the final
% analysis, is not on the object because it does not really refer to any specific kind or
% amount of alcoholic drink, something that is usually mentioned if the focus is on the
% object. In short, transitive-turned-intransitive verbs of activity focus on the activity,
% not the object.4

% *Liu (2008: 302)*  
% > The third compelling reason to consider these verbs intransitive verbs of activity
% is that they can function meaningfully without a clear discourse or situational context
% as shown in the following BNC examples:  
% (26) She likes to read.  
% (27) The python has already eaten.  
% (28) She might have been drinking.  
% In contrast, true object-deleting verbs, such as know, promise, and understand,
% cannot be used this way

% *AnderBois (2012: 44)*
% >  two properties which distinguish overt indefinites
% and definites in English: the possibility of apparently anaphoric readings (§2.1) and the ability to serve
% as the inner antecedent for Sluicing (§2.2) [...]  
% Sluicing is the name given by Ross (1969) to the phenomenon in (7) in which the crossed out
% material can be elided in the presence of a suitable antecedent clause in prior discourse:  
% (7) [ Someone left]A , but I don’t know [who left]E .

\subsection{Genre-based implicit objects: a special case of definite object drop} \labsec{theory_recipes}

Recipes and special registers

% Syntactic *Cote (1996: 120-122)* (130 del pdf)
% > By syntactic explanation I mean any analysis which argues that the null object is represented in the syntactic structure of the sentence/utterance. [...] Culy (1987) argues this case for null objects in recipe constructions in English (a separate issue
% which will be discussed in Chapter 4) and then adds that he rejects the idea of a difference between the
% grammar of recipe contexts and the grammar of ‘Standard’ English. In particular, he argues that null
% objects in general are zero anaphors.

% *Ahringberg (2015: 9)*  
% > 2.3.3 Constructional licensing
% As previously mentioned, definite null instantiation is licensed not only by the predicate’s
% lexical properties but also by the grammatical construction in which it is found (Fillmore, 1986,
% p. 97; Lambrecht & Lemoine, 2005, p. 20). This kind of licensing is, for example, evident in
% imperative constructions (Prytz, 2009, pp. 11-12) which are common in recipes or manuals.

% *Garcia-Velasco & Munoz (2002: 9)*  
% RECIPES != INDEFINITE OBJECT  
% > Hypothesis 1  
% Indefinite Objects do not present available referents in the surrounding linguistic or
% extralinguistic context. [...]  
% for example, bake tends to appear in recipe contexts, exemplifying the so called “instructional
% imperative”, and consequently a type of structural omission. Bake, in this example does not
% take an activity reading, and, therefore, it may have a referent in the surrounding linguistic
% context:


% *Weir (2017: 157)*  
% PARALLELISMO VP/DP! È ANCHE IN QUALCHE APPUNTO QUA SOTTO
% > This paper discusses object drop in English ‘reduced written register’ (RWR),
% such as recipes (Haegeman 1987a, b, Massam & Roberge 1989, Massam 1992)
% and diaries. [...] I propose that object drop in RWR is dependent on article drop. [...]  
% This phenomenon is restricted to written English. Haegeman (1990) points out that
% object drop of this type is impossible in speech.

% *Weir (2017: 158)*  
% > Previous work on the construction (Haegeman 1987a, b; Massam & Roberge 1989;
% Massam 1992) concentrates primarily on the absence of objects in recipes, as in
% (1a, b). It is also possible to see this phenomenon in instructional or directive reg-
% isters more generally, as (1c) shows. Furthermore, it can appear in more informal
% written registers, such as in diaries, SMS (text) messages, internet communication,
% etc., as (1d, e) exemplify. It is, then, one of the hallmarks of the register which is
% called ‘block language’ by Straumann (1935), ‘abbreviated English’ by Stowell (1991,
% 1999), and ‘reduced written register’ by Weir (2013), the term I shall adopt here,
% abbreviated as ‘RWR’. [...]  
% Something about ‘reduced written register’ licenses the possibility of null objects,
% a factor which is not present in spoken English.  
% In this paper I propose an analysis which explains this variation. Concretely, I
% capitalize on an apparently independent feature of RWR: it also allows article drop,
% as shown below.  
% (3) Bought Ø new phone today. (= I bought a new phone today.)

% *Weir (2017: 160)*  
% > Another interesting fact about null objects in RWR is that there are restrictions
% on their co-occurrence with overt subject pronouns, as Massam & Roberge (1989)
% and Massam (1992) discuss. Null objects are licit in imperatives, as in (12a), or in
% a declarative sentence in which the subject has been dropped, as in (13a), but if
% subjects are included in these structures, they become ungrammatical. 

% *Weir (2017: 167)*  
% > One of the phenomena which an account of object drop will have to explain is the
% subject/object asymmetry: object pronouns cannot be dropped to the exclusion of
% subject pronouns. Interestingly, this asymmetry has a very similar correlate in an-
% other phenomenon which only takes place in ‘reduced written registers’ of English,
% namely article drop. This phenomenon is most usually discussed with respect to
% the grammar of newspaper headlines (Mårdh 1980, Stowell 1991, 1999, 2013, Reich
% this issue, Weir 2013), but it is possible in other subregisters of RWR as well,

% *Weir (2017: 168)*  
% > A very interesting parallel with the object drop cases appears. Just as object
% pronouns cannot be dropped if subject pronouns are present, articles in object DPs
% cannot be dropped if articles in subject DPs are present. This subject-object asym-
% metry was noted by Mårdh (1980) in a corpus study of English headlines. [...]  
% Stowell (1991, 1999) also notes the asymmetry, and gives it a characterization in
% terms of c-command: an article-ful DP may not c-command an article-less DP.

% *Ruppenhofer & Michaelis (2010: 158)*
% >  We consider
% five genre-based omission types: instructional imperatives (Culy 1996, Bender
% 1999), labelese, diary style (Haegeman 1990), match reports (Ruppenhofer 2004)
% and quotative clauses. We show that these omission types share important traits;
% all, for example, have anaphoric rather than indefinite construals. We also show,
% however, that the omission types differ from each other in idiosyncratic ways.

% *Ruppenhofer & Michaelis (2010: 159)*
% > But genre and argument omission
% are closely connected conventions — so much so that speakers and writers can
% often evoke a rich genre by simply omitting the appropriate argument in a predi-
% cation. For example, the title of Cynthia P. Lawrence’s 2002 novel, Chill Ø before
% Serving Ø: A Mystery Novel for Food Lovers,1 evokes the Instructional Imperative

% *Ruppenhofer & Michaelis (2010: 162)*
% > For an omission to be genre-based, we require that the construction at issue allow the omission only
% in some of the genres that it occurs in, or that the construction itself be limited to
% particular genres, if it always allows omission. Thus, for example, the omissibility
% of the subject of an imperative is not genre-based, as imperatives have this affor-
% dance regardless of genre. By contrast, in ordinary, non-instructional imperatives
% like (14), the object must be overt.  
% (14) Take *(the money) and run.

% *Ruppenhofer & Michaelis (2010: 163)*  
% CFR. QUELLO CHE DICO SOTTO SUL LINGUAGGIO CALCISTICO! LA SITUAZIONE È PIÙ COMPLESSA DI COSÌ  
% > genres enable omissions that the predicator or
% construction would not otherwise permit, rather than blocking omissions that are
% possible outside of the genre

% *Ruppenhofer & Michaelis (2010: 165)*  
% Thus, constructionally licensed null complements are not intrinsically anaphoric.2
% Why then do genre-based omissions entail anaphoric construal? Perhaps because
% recoverability of the missing argument requires recourse to cultural stereotypes, in
% the same way that certain (nonanaphoric) lexically licensed omissions do:
% (24) I love to read (magazines and books).
% (25) She drinks (alcohol).
% (26) He smokes (tobacco).
% Just as reading, drinking and smoking conventionally involve certain types of
% participants, so recipes, product labels and match reports invoke a specific set of
% conventional participants (an edible substance, a product, a ball, etc.). This expla-
% nation does not go through, however, because the missing arguments in (24)–(26)
% are not in fact anaphoric: none are replaceable, for example, by the definite pro-
% noun it. Genre-sensitive argument omissions are anaphoric because the relevant
% genres presuppose the salience of certain entities. 

% *Ruppenhofer & Michaelis (2010: 166)*  
% > genre-based omissions are never obligatory [...] Both halves of
% the Gricean quantity maxim would seem to encourage speakers to exploit the null
% complementation opportunities offered by genres: while effort conservation fa-
% vors argument omission, informativeness favors using linguistic conventions (like
% argument omission) that signal what text type is in play. 

% *Ruppenhofer & Michaelis (2010: 175)*  
% > Accordingly, the grammar is viewed as consisting of a lexicon — a finite set of
% lexical descriptions (descriptions of feature structures whose type is either lexeme
% or word) — and a set of constructions. We propose (pace Kay, 2004) that argument
% omission can but need not be licensed by a lexeme. In the case of activity verbs
% like eat, drink and read, which allow existentially interpreted null-complements
% in episodic contexts (e.g., Sue ate Ø at noon/drank Ø at the party/reads ø during
% breakfast), we concur that the zero argument’s potential for null expression is en-
% coded in the lexical entry of the particular verb, as per Kay. 

% *Haegeman (1987: 237)*  
% > Verbs like cut, place, pour, Combine, chill and sertpe are transitive in
% English, but they appear in (14) to be objectless. At first glance one
% might opt for a register-determined lexical rule which &dquo;intransitivises&dquo;
% these verbs, thus no longer requiring that the object be projected syn-
% tactically. However, there are clear arguments against this solution (cf.
% also Haegeman 1987).
% First of all we note that the set of verbs allowing indefinite object
% drop as discussed above is confined to a relatively small set which is
% lexically determined. But in the register of instructional language any
% verb may appear without its object, as in (14), suggesting that a
% lexically governed rule would miss a generalization.

% ## Recipes

% *Megitt (2019)*  
% > recipe minilect  
% chiarisce meglio a p. 8:  
% > Nordman (1994) argues that the language of cookbooks belongs to the minilect genre (Swedish:
% minilekt). The minilect is a subgroup of technolects, which in turn belongs to the wider concept
% of non-fiction texts. According to Nordman (1994:53), the recipe is a minilect in that it is limited
% in terms of vocabulary, that the syntax is strictly formalised and that the general structure and
% layout of cookbooks often follow a pre-defined pattern. Furthermore, the stylistics of cookbook
% language is, according to Nordman (1994:49), similar beyond language barriers. [...]  
% Nordman (1994:67) observes that the recipe has an imperative function, in that it
% is intended to make the reader perform a number of tasks in a pre-determined order.

% *Megitt (2019: 20)*  
% > McShane (2005:33) claims that in the English language, object ellipsis is in fact only possible
% in narrowly defined registers, such as recipes.

% *Ruda (2014: 341)*  
% SI RICOLLEGA AL DISCORSO DELLA RECOVERABILITY  
% > The identification of the referents of the objects in this case is fa-
% cilitated by the restrictions on the set of possible referents of the objects imposed
% by the nature of the registers and by the clearly defined purpose of communication. [...]  
% The topical status of missing objects in recipes makes their content recoverable. An-
% tecedents are restricted to a very limited set, as within the interpretive frame there
% are only the ingredients, products of stages of cooking, and tools. Thus, the possible
% reference of the objects is contextually determined and the interpretation is guided
% by the nature of the recipe register. 

% *Ruda (2014: 350-351)*  
% >  It seems that this feature of objects in recipes adds to
% the specific characteristics of this register. The omission of the object increases the
% prominence of the remaining elements in the sentence, thereby making it possible for
% the hearer's attention to be fully devoted to the phonetically realised elements (see
% also Mittwoch 2005). Moreover, object drop is facilitated by the backgrounding of
% information by linguistic or extra-linguistic context. Taking into account that the reg-
% ister under discussion has a clearly defined structure dictated by the communicative
% purpose of the texts, and that the set of possible referents of the omitted objects can
% be predicted from the type of discourse, it seems that the nature of the given register
% encourages object drop even further (see also Cummins and Roberge 2005 for some
% discussion of the interpretive properties of null objects).

% *Ruda (2014: 352)*  
% ATTENZIONE! DA QUALCHE PARTE QUI, UN APPUNTO DICE L'ESATTO CONTRARIO!  
% > Summarizing, it seems that the analyses of object drop in recipes employing
% argument ellipsis are unworkable,

% *Paesani (2006: 154)*
% > the absence of objects, this characteristic is common in some special registers, in
% particular cooking recipes.20 The following examples illustrate object omission in En-
% glish (34), French (35), German (36), Serbian (37), and Spanish (38) cooking recipes.

% *Paesani (2006: 155)*
% > In each of these examples, the antecedent is a nonovert topic; context (linguistic or
% nonlinguistic) determines the referent of the null object (cf. Haegeman 1987; Zwicky
% 1990). This context is usually the list of ingredients at the start of the recipe, and the
% first mention of the particular ingredient in the instructions. No repetition of the in-
% gredient (topic) is required after it is first introduced into the discourse context (cf.
% Sadock 1974).

% *Paesani (2006: 158)*
% > Syntactically, the diary register appears to resort to a sentential grammar more eas-
% ily than the headline and recipe registers, for instance. Although diaries exhibit some
% of the properties that characterize a nonsentential grammar such as null subjects and
% the absence of tense inflection, rarely do these properties co-occur. Indeed, the diary
% register is typified by utterances having a null subject and a tensed verb. Similarly, the
% note-taking register, although characterized by more nonsentential properties than the
% diary register, appears to be less distinctive than headlines and recipes. The note-taking
% register exhibits most of the nonsentential properties outlined earlier, yet, like the diary
% register, does not productively exhibit the co-occurrence of tenselessness and caseless-
% ness typical of nonsententials in casual adult speech. The headline and recipe registers,
% on the other hand, appear to be more distinctive because they resort to a nonsenten-
% tial grammar more easily than do the diary and note-taking registers. Headlines and
% recipes are typified by the types of base-generated phrases and root small clauses that
% form the basis of the nonsentential analysis proposed by Progovac (Chap. 2, this vol-
% ume). In particular, both registers provide strong support for the correlation between
% tenselessness and caselessness characteristic of adult nonsententials.
% One constant across the syntactic characteristics presented in Section 2 is the
% clear link between discourse/pragmatic context and missing elements in the syntax.

% *Paesani (2006: 161)*
% > As argued in Progovac, TP and DP are interrelated in syntactic theory. A DP is visible
% as a subject argument when it is assigned abstract (nominative) Case; TP then checks
% the abstract Case of the subject DP. As a result of this interrelationship, utterances
% that lack Tense cannot check structural Case. 

% *Paesani (2006: 164)*
% > As was noted in Section 2, missing objects are a common characteristic of the recipe
% register. The absence of objects in this context may be linked to the co-occurrence of
% tenselessness and caselessness characteristic of the nonsentential grammar. 

% *Paesani (2006: 165)*  
% OBJECT DROP AS TOPIC DROP IN RECIPES
% > In English, objects tend to be focal in nature; however, in recipes, objects appear to
% be topical. [...] Haegeman (1987) also argues that object drop in recipes is
% topic linked; the antecedent of a missing object is a sentence-peripheral topic position. [...]  
% Given the preponderance of imperative and infinitive constructions in the recipe
% register, and the fact that the pragmatic frame for recipes is the “here and now,” the lack
% of a Tense node may create the syntactic context for missing objects, as it does for the
% absence of structural Case assignment on subjects. Massam and Roberge (1989) note
% a link between the absence of subjects and objects in cooking recipes, stating that “it is
% difficult to tell whether the imperative aspect or the no-subject aspect of the sentences
% is relevant to the [missing object]” (135)

% *Paesani (2006: 166)*  
% TP === DP!!! SUBJECT/DETERMINER IN SPEC-TP/DP  
% > There is thus a one-way correlation between the absence of objects and the absence of
% subjects: if the object is omitted, then the subject must also be omitted, but not vice
% versa. [...]  
% This one-way relationship is similar to the generalization made by Stowell (1991)
% regarding the omission of determiners in subject and object position in headlines (see
% n. 11). The pattern of determiner omission in (72) is similar to that for subject and
% object omission in (71).  
% (72)a. An old man finds a rare gold coin  
% b. Old man finds a rare gold coin  
% c. Old man finds rare gold coin  
% d. *An old man finds rare gold coin  

% *Paesani (2006: 167)*  
% LAW OF LEAST EFFORT  
% > In addition, the co-occurrences typical of
% cooking recipes may be related to the fact that both the subject (i.e., the person read-
% ing the recipe) and the objects (i.e., the list of ingredients) are known. As a result, what
% minimally needs to be expressed is the action performed by the subject on the objects.


% *Garcia-Velasco & Munoz (2002: 6)*
% > An interesting case is that of recipes, where we can find examples such as the following:
% (13) Cook _____ gently for four minutes in plenty of boiling, salted water to obtain an
% “al dente” texture. Drain _____ and serve _____
% Massam & Roberge (1989) study the properties of understood objects in recipe contexts. The
% authors observe inter alia that omitted objects tend to receive a specific (non-arbitrary)
% interpretation and do not need to be present in the linguistic context.

% *Massam & Roberge (1989: 135)*  
% (EO = Empty Object)
% > First, we note that the EO receives a specific interpretation, rather than, for example, an arbitrary reference. The
% empty category need not have a particular linguistic antecedent;
% instead, its reference appears to be contextually defined [...]  
% 1 Evidence from French suggests that it is the no-subject aspect
% of these constructions that is relevant for licensing RCEOs, as pointed
% out by an LI reviewer. French RCEOs, although they may appear with
% an imperative, are more usually found with infinitives [+ ESEMPI IN INGLESE NELLA STESSA PAGINA!]

% *Cote (1996: 122)*
% > Culy (1987) argues this case for null objects in recipe constructions in English (a separate issue
% which will be discussed in Chapter 4) and then adds that he rejects the idea of a difference between the
% grammar of recipe contexts and the grammar of ‘Standard’ English. In particular, he argues that null
% objects in general are zero anaphors.

% *Cote (1996: 167)*
% > Objects are generally omitted much more frequently from recipes and instructions than they are
% from ‘Standard’ English, and with fewer restrictions. 

% *Cote (1996: 168)*
% > In practice, it seems that these ‘recipe’ null objects are quite different from those in ‘Standard’
% English. In particular, they may occur with strictly transitive verbs and do not in fact seem to be lexically-
% constrained at all.  

% Le ricette non sono comandi come gli altri  
% > Note that it is not sufficient to say that these null objects must occur inside of commands with
% salient objects. 

% *Bender (1999: 2)*  
% cita Culy (1996)  
% > There are several important things to note in Culy's results. The rst is
% that null objects occur frequently in recipes, and that this frequency varies
% across cookbooks.

% *Bender (1999: 4)*  
% > As far as the syntax is concerned, Culy (1996) identi es three types of analysis:
% deletion accounts, which involve a rule that deletes an object noun phrase;
% semantic accounts, where a special entry for the selecting verb causes the object
% position to be present in the semantics but not in the syntax; and empty category
% accounts.

% *Culy (1996: 91)*  
% Culy critica Mohanan! infatti, in inglese la situazione non è affatto quella citata  
% > Mohanan (1983: 643): "It is well known that [unrealized grammatical functions] in English 
% are restricted to the subject position of nonfinite clauses"

% *Culy (1996: 92 sgg.)*  
% Culy nel paper fa una regressione multipla su set diacronici di ricette. La dipendente è la scelta di zero anaphors as opposed to pronouns and
% noun phrases, 5 fattori come predittori --> stile (libro di ricette specifico) e semantica/discourse factors sono i fattori più importanti, sintassi no  
% 5 factors = source, verb form, type of object, lookback, grammatical function of the antecedent (p. 93)  
% stepwise multiple regression analysis with VARBRUL2 program (p. 95)

% *Culy (1996: 92)*  
% i detransitivized verbs come gli usi intransitivi di eat NON SONO zero anaphors!!!  
% (quindi le ricette sono diverse dalle cose che interessano a me!)  

% *Culy (1996: 101)*  
% Sadock (1974) sostiene che "zero objects occur only when there is no overt subject of the clause" (cfr. Paesani 2006: 166)

% *Liu (2008: 302)*  
% IMPORTANTISSIMO! I NULL OBJECT NELLE RICETTE SONO DEFINITE, NON IMPLICIT!!!  
% POSSO CONSIDERARLI NELL'OTTICA DI UN CONTINUUM TRA DEFINITE E IMPLICIT?  
% DEVO RIFLETTERE SU QUESTA COSA E COMUNQUE PROBLEMATIZZARLA NEL CAPITOLO TEORICO
% > Situational context-warranted object deletion is found mostly in instructional language
% on product labels and manuals as well as on warning signs. The deleted
% objects are concrete nouns, as seen in the following collected examples

% ## Telegraphic register

% *Cote (1996: 170)*
% > As with recipe contexts, the telegraphic register or ‘sublanguage’ (eg., the language used in
% telegraphs, memos, signs, etc...) makes more extensive use of null arguments than does ‘Standard’
% English. This register, also known as “telegraphese”, is exemplified by (9) and (10) below.

% ## Communities

% *Glass (2013: 1)*
% > First, I argue that recoverability is a matter of degree. At a minimum, one simply knows that
% an object exists; at a maximum, one knows exactly what it is; and there is plenty of middle ground
% in between. For an object to be omissible, it must be sufficiently recoverable along this continuum
% for speakers to pursue their communicative goals in the context. Second, I argue that a given ob-
% ject may easier to recover, and thus easier to omit, against the common ground of a particular
% community of practice

% ## Football language  

% *Bergh & Ohlander (2016)*  
% "Iniesta passed and Messi finished clinically": CFR. GLASS 2020 per la routine!  
% In football you can't pass the salt, so it's ok to omit "the ball" in "passed" because it can only mean that  
% In the context of a family dinner, you cannot say your big brother "passed *(the salt) and finished *(his steak)", let alone clinically

% *Bergh & Ohlander (2016: 21)*  
% > Object omission, however, is not the only special verb usage to be
% found in football language. There are other types of special or
% “unconventional” verb behaviour related to transitivity, often involving
% violation of “normal” selectional constraints, as illustrated in expressions
% such as kill the match and rest a player. Such verb usage will also be
% dealt with in due course.

% *Bergh & Ohlander (2016: 22)*  
% >  As pointed out by Huddleston & Pullum
% (2002:216), the notion of transitivity is often more usefully applied to
% verb use than to verbs as such, since not all verbs are either transitive or
% intransitive: “although faint is always intransitive many verbs can occur
% either with or without an object. For example, read is intransitive in She
% read and transitive in She read the letter” (cf. Jespersen 1924:158, Quirk
% et al. 1985:1169). In a similar vein, Biber et al. (1999:147) note: “It is
% striking that a lot of English verbs have both transitive and intransitive
% uses”, pass being a prime example. 

% *Bergh & Ohlander (2016: 22)*  
% > From an overall communicative perspective, omission of objects – or
% indeed any linguistic element – can be seen as a special exponent of a
% general principle of expressive economy, underlying various types of
% ellipsis, along the lines of: “Do not say more than you have to for the
% message to get across!”8 [...]  
% 8 Cf. “Zipf’s Law”, embodying the “principle of least effort” (Zipf 1949). From a
% pragmatic perspective, object omission may be seen as a syntactic reflection of Grice’s
% well-known ”maxim of quantity”, implying (second sub-maxim) that one should avoid
% being more informative than necessary (Levinson 1983:101). Cf. also Jespersen’s
% (1924:309) discussion of ”suppression”, akin to Grice’s quantity maxim: ”we suppress a
% great many things which it would be pedantic to say expressly”; ”Only bores want to
% express everything, but even bores find it impossible to express everything.”

% *Bergh & Ohlander (2016: 24)*  
% LA RECOVERABILITY È COLLEGATA ALLA PRAGMATICA (world knowledge, routine...)  
% > From a communicative perspective, the basic point of the above
% discussion is the recoverability of the omitted object, which, in turn, is
% dependent on a variety of linguistic, contextual and situational factors, as
% well as background knowledge. This can be illustrated by the most basic
% of all ball-sport verbs, i.e. play. A sentence like They played beautifully
% is, in isolation, multiply ambiguous, its full interpretation heavily
% dependent on context.

% *Bergh & Ohlander (2016: 26)*  
% CONTINUA IL DISCORSO SUBITO SOPRA  
% >  Knowledge of football’s conceptual framework is thus a
% necessary requirement for the proper identification of missing objects in
% football language; the same applies to the special languages of other
% subject fields. Also, object omission within a special subject field may be
% at least partially genre- or register-dependent (cf. Ferguson 1983, Müller
% 2008). According to Ruppenhofer & Michaelis (2010:163f.), stressing
% “the connection between genres and argument omission”, “certain genres
% license object omissions that are otherwise permitted only in generic-
% habitual contexts”, match reports being mentioned as one such genre.
% Further, live match reporting may, due to time pressure, give rise to a
% higher frequency of omitted objects than, say, a post-match analysis of a
% game.

% *Bergh & Ohlander (2016: 28)*  
% QUINDI NON SI PUÒ CONCLUDERE SEMPLICEMENTE CHE I LINGUAGGI SPECIALISTICI AMMETTANO PIÙ OBJECT DROP
% RISPETTO ALLA LINGUA D'USO COMUNE  
% > It should also be stressed that not all transitive football verbs may
% occur, with equal ease, without an object. For example, the football verb
% par excellence, kick, whose self-evident object is normally the ball
% (occasionally, less normally, a player in the opposition). Nonetheless,
% kick does not readily permit object omission, as evidenced by the rarity
% (as confirmed by web searches) of, e.g., ?She kicked into the net instead
% of She kicked the ball into the net, in contrast to the perfectly normal She
% shot (fired, sidefooted, struck, backheeled, etc.) into the net.; the same
% goes for ?She hit into the net for She hit the ball into the net.
% Interestingly, the verb deliver – which often occurs without an object in
% general language (cf. section 2) – apparently does not gladly part from its
% object (ball) in football language:

% *Ruppenhofer & Michaelis (2010: 167)*  
% >  we take an entity to
% be topical if it can readily be reconstructed as the filler of a given argument posi-
% tion. By contrast, we take an entity to be important in a given text only if it has
% ongoing relevance to readers, and readers can infer qualities of the entity from the
% text. In match reports, for example, the players are important while the ball is not:
% readers do not typically make inferences about the ball based on the events of the
% match, so one ball is as good as another for the purpose of match reporting. There
% is, however, only one object of play, while there are multiple players. Thus, the ball
% is highly topical: it is the only entity that could occupy the argument position it
% does in a given match-report predication, e.g., object of the verb headed in (18).
% One could therefore refer to the object of play by means of a pronoun without cre-
% ating ambiguity. Accordingly, we might view null expression as the match-report
% analog to anaphoric reference.

% *Ruppenhofer & Michaelis (2010: 169)*  
% > A further difference among genres is that some genre-sensitive argument-omission
% constructions can combine with any predicate, whereas others require predicates
% to denote actions or properties that are canonical in the genre. For instance, in di-
% ary style, the subject of any predicate is subject to omission. Similarly, in the case of
% object omission in the quotative construction, no particular judgment-expressing
% reporting verbs can be considered canonical. By contrast, labelese predications are
% limited to those that describe the provenance, constitution, qualities or efficacy
% of the product, e.g., ø contains no hydrogenated oils, ø creates visibly fuller, thicker
% hair. Labelese predications like that in (36), an altered version of a statement found
% on a box of quinoa, are not attested:
% (36) *Ø has flourished in cultivation for over 5,000 years.
% Match reports may include non-canonical events involving the ball. For instance,
% players sometimes step on the ball, kiss or rub it for good luck, hide it under their
% shirts, etc. However, we find omissions only with predicates denoting canonical
% aspects of the game such as taking various kinds of kicks or headers. Omissions
% like that in (37) are not attested:
% (37) Before he took that free kick he kissed *(the ball) for luck.

% ## Whatsapp messages

% *Stark & Meier (2018)*


\subsection{Neither definite nor indefinite: continuous accounts} \labsec{theory_continuous}

Neither definite nor indefinite: continuous accounts

% *Olsen & Resnik (1997: 1)*
% > Early accounts eliminate the object from the syntax via an "object
% deletion" transformation (Katz and Postal 1964, Browne, 1971). More recent
% work has observed that implicit object constructions have aspectual constraints
% (Mittwoch 1982, Brisson 1994) and that omitted objects must be "typical,"
% inferable, or partially specified by the semantics of the verb (Brisson 1994,
% Fellbaum and Kegl 1989, Lehrer 1970, Mittwoch 1982, Rice 1988). Recently
% Resnik (1996) has substantiated and formalized the inferability claim using an
% information-theoretic account of selectional constraints. In this paper we show
% that the aspectual and selectional criteria for implicit objects are accounted for
% within the framework described by Hopper and Thompson's (1980; H&T)
% transitivity hypothesis. We locate English implicit object constructions on a
% continuum of transitivity, with indefinite implicit object constructions (1a) closer
% to intransitives, and definite implicit object constructions (1b) closer to transitives.

% *Glass (2013: 1)*  
% UTILI CONSIDERAZIONI TEORICHE E BIBLIOGRAFICHE  
% > I'll sketch the scope of these data before I turn to the analysis. Although the literature (e.g.,
% Fillmore1986) distinguishes between “indefinite” and “definite” IOs, I have some qualms about
% this distinction, which I elaborate later on (see also Anderbois 2012, Scott 2006). Therefore, I
% consider data from both sides of this distinction. However, I limit myself to IOs that seem to
% stand in for DP’s (rather than CP’s). I also don't take a stand on whether IOs are represented in the
% semantics, as in e.g., Anderbois (2012), or whether the verbs are simply intransitive and a patient
% argument (the thing eaten) is pragmatically inferred (as in Recanati 2007); the term “IO” is de-
% scriptive only.


% *AnderBois (2012: 43)*
% > Dating back to Fillmore (1969)’s seminal work, the literature on implicit arguments (IAs) has
% consistently distinguished two types of IAs : those which can be paraphrased with an overt indefinite, as
% in (1a)1 , and those which are better paraphrased with a pronoun or definite description, such as (1b).  
% (1) a. John ate [THEME].  ⇐ Indefinite Implicit Argument  
% b. Maribel noticed [S TATE O F A FFAIRS ]. ⇐ Definite Implicit Argument [...]  
% a third category, to which we give the descriptive moniker flexible IAs. As we will show below, flexible
% IAs appear to pattern with definite IAs on some occasions and indefinite IAs on others.  
% (2) The Giants won [CONTEST]. ⇐ Flexible Implicit Argument

% *Cummins & Roberge (2005: 46)*  
% CFR. ANDERBOIS 2012 SUBITO SOPRA IN QUESTO NOTEBOOK!  
% > Larjavaara’s generic/latent distinction is based on
% the ability of the hearer to identify a possible referent. Lambrecht and
% Lemoine’s categories of indefinite (those that cannot refer to an entity in the
% discourse) and definite (those that must be interpreted as referring to an entity
% in the discourse) capture a similar distinction, and they add a third
% category—ÔÔlibreÕÕ (ÔfreeÕ)—to handle the indeterminate cases.

% *Naess (2007: 130-134)*
% CONTINUOUS ACCOUNT SU BASE PRAGMATICA
% CRITICA DI QUESTA COSA --> TORNA BENE CON IL MIO "CONTINUUM GRADATUM"
% In a departure from the traditional approach based on verb semantics, Goldberg
% (2001) analyses IOD as essentially a discourse-pragmatic phenomenon. Her start-
% ing-point is the data presented in (6.1) above, which demonstrate that in the right
% context, just about any transitive verb in English can be used without an object.
% In order to account for these data, Goldberg introduces the notion of discourse
% prominence, which subsumes both topic and focus. An argument that is either
% topical or focal is discourse-prominent, and English generally requires discourse-
% prominent arguments to be expressed. In the typical situation the patient argument
% of a causative verb is a highly prominent one (cf. 3.2.2); “one typically does not as- --  
% sert that a participant changes state unless one wishes to discuss or draw attention
% to that participant” (Goldberg 2001 :510). Nevertheless, a patient of a causative
% verb which is neither topical nor focal, and therefore has low discourse promi-
% nence, may be omitted under the further condition that the action is emphasised.
% Emphasis on the action can take a number of forms; iteration, discussed earlier, is
% one possibility, but there are others. Goldberg among others mentions that the ac-
% tion can be the discourse topic, as in the example He was always opposed to the idea
% of murder, but in the middle of the battlefield, he had no trouble killing; or an action
% can be emphasised by the speaker’s strong affective stance toward the action, as in
% Why would they give this creep a light prison term? He murdered!
% Goldberg concludes that patient arguments can be omitted when they are
% deemphasised vis à vis the action. This leaves her with the cases of apparently
% lexically conditioned omission – the familiar “core set” of IOD verbs like drink,
% smoke, sing, bake, read, eat, which frequently omit their objects even when none of
% the above-mentioned trappings of “action emphasis” are present. She states that
% intuitively there does appear to be a stronger emphasis on the action in Pat read in
% the car than in Pat read a book in the car, but does not attempt to explain or sub-
% stantiate this intuition. Instead, she notes that this set of verbs frequently appears
% in generic contexts with a habitual interpretation: Pat drinks; Pat smokes; Chris
% sings; Sam bakes. As this usage is licensed by her discourse-prominence analysis,
% she assumes that the frequent usage of these verbs in such contexts has led to a
% grammaticalisation of object omission as a lexical option for these verbs.
% There are two main difficulties with this analysis. Firstly, it is unclear exactly
% what the condition of “emphasis on the action” should be taken to include. There
% are instances where object omission is possible but where it is not clear that there
% is any intended emphasis on the action as such. In a sentence like John murdered
% for the money, no iteration is necessarily implied; John might perfectly well have
% committed just one murder for the purposes of financial gain. But neither does the
% action as such seem to be emphasised in any other way; if anything, the emphasis
% here seems to be on John’s motivation for performing the act in question rather
% than on the action itself, and in fact the NP specifying this motivation may take
% contrastive stress: John murdered for MONEY, not for love. This goes against Gold-
% berg’s claim that emphasis on the action as such is required, rather than just the
% presence of some focal element other than the patient (p. 513).
% Secondly, the attempt to locate the properties licensing IOD exclusively at the
% level of discourse leads to a rather vague and unsatisfactory account of the in-
% stances where IOD does seem to be related to verbal semantics. There is no expla-
% nation for why just the sort of verbs cited by Goldberg (note that all of these are
% either affected-agent or effected-object verbs) should be so frequent in generic/ha-
% bitual contexts as to grammaticalise the possibility of IOD. Furthermore, it is sim- --  
% ply not true that all of these verbs are frequent in “generic contexts with a habitual
% interpretation”. The most frequently-cited IOD verb of them all, ‘eat’, is highly un-
% natural in a generic/habitual sentence: ??John eats. Similarly, the generic/habitual
% construction with drink only applies when drink is understood to mean ‘drink al-
% cohol’; if taken to mean something like ‘John habitually imbibes liquids’, ??John
% drinks is just as peculiar as ??John eats. It is not obvious, therefore, how the pro-
% posed grammaticalisation of the possibility of IOD would have come about with
% these verbs; and the question also arises of exactly where the ‘drink alcohol’ mean-
% ing of objectless ‘drink’ comes from (see 6.3.3 below).

% *Naess (2007: 134)*
% > Structurally, the most obvious characteristic of the indefinite object deletion construction is
% that it is a formally intransitive clause, as opposed to the transitive construction
% which appears when the verb in question is used with ano overt object NP. The
% logical explanation to such an alternation between higher vs. lower formal transi-
% tivity would be that it reflects a corresponding difference in semantic transitivity.
% From this perspective, IOD is most felicitously analysed not as a lexical quirk
% of certain specific verbs or classes of verbs, but as a syntactic detransitivisation
% mechanism, a means of expressing in a formally intransitive clause events which
% are construed as deviating from the transitive prototype.

% *Naess (2007: 135)*  
% IMPORTANTISSIMO!!! SI RICOLLEGA AL DISCORSO DI DRINK=ALCOHOL ETC  
% > Analysing IOD as a construction which applies only to verbs or clauses which
% are not fully transitive semantically explains a striking observation made by Fill-
% more (1986): certain verbs undergo what he calls a semantic “specialisation” when
% used in an IOD context. He cites the verb bake as an example: in the sentence I
% spent the afternoon baking, “the missing object is taken to include breads or pas-
% tries, but not potatoes or hams” (Fillmore 1986 :96).
% If we take into consideration the difference in semantic transitivity between
% affected-object and effected-object verbs, an explanation for this “semantic spe-
% cialisation” is readily forthcoming. The verb bake in English is ambiguous between
% an affected-object (bake potatoes, where the potatoes already existed prior to the
% action and are only affected, not effected by the act of baking) and an effected-ob-
% ject reading (bake pastries, where the act of baking brings the pastries into exist-
% ence). Affected-object bake is high in transitivity and so does not easily undergo
% IOD. Effected-object bake, on the other hand, unproblematically omits its object,
% because this verb has a nonreferential, nonaffected object and therefore is not ful-
% ly transitive semantically. Consequently, under IOD, only the effected-object read-
% ing is possible – the intransitive construction can only be read as containing a verb
% relatively low in transitivity.
% Similar behaviour can be found e.g. with the ambiguous verb paint in generic/
% habitual statements. If we say of someone that He paints, we mean that he paints
% pictures – either for a living or as a hobby – not that he is a housepainter.

% *Naess (2007: 136)*  
% > The IOD construction with a purpose clause – John murdered for the money –
% does not necessarily have an iterative reading, as pointed out above. Rather, such
% clauses are construable as a kind of affected-agent construction where the affected-
% agent reading is not imposed by the semantics of the verb, but rather by the pur-
% pose clause. A statement of the agent’s motivation or purpose in performing an act
% is essentially a statement of the benefits that the agent hopes to achieve in acting;
% in other words the intended effect of the act on the agent. Affectedness of the
% agent, then, is not necessarily inherent to the semantics of a specific verb, but may
% be introduced by other elements of a clause.


\section{Defining the indefinite} \labsec{theory_defindefinite}


\subsection{Two meanings, two verbs: the naive account} \labsec{theory_twoentries}

Two entries in the lexicon: the naive account

% *Petho & Kardos (2006: 29)*  
% DECISAMENTE NO! MOLTO NAIVE, DIRE CHE IO NON PENSO QUESTO
% >   Some of these can be characterised as properties of
% the verbal predicate itself, i.e. the verb has both a transitive and an intransitive version in the
% lexicon, like eat, drink, cook, read.

% Pragmatic *Cote (1996: 120)* 
% > The optional argument (pragmatic) explanation supposes that the verb may optionally have no
% linguistic representation for an object, even in the lexicon. In other words, this explanation asserts that
% there is a true transitive/intransitive alternation at work. 

% *Bourmayan & Recanati (2013: 122)*  
% con notazioni logiche lambda! utili per la tesi   
% > on the ‘intransitive’ uses of the verb the object
% remains covert—it is not articulated in surface syntax, though it shows up at
% LF. The logical form of ‘John eats’ is therefore something like ‘John eats [some
% thing]’, where the materials within square brackets correspond to a covert syntactic
% element with indefinite value. We will refer to this view as the covert indefinite view
% (CIV). The other view, which we call the genuine-intransitive view (GIV), assigns
% two distinct (though related) lexical entries to transitive and intransitive ‘eat’.



% *Lorenzetti (2008: 60)*  
% ## CONTRO LA DISTINZIONE IN DUE DIVERSE ENTRATE LESSICALI
% > we argue that positing different lexical entries in the case of null-object verbs
% is often counterintuitive and inappropriate,

% *Naess (2007: 130-134)*  
% ## CONTRO LA DISTINZIONE IN DUE DIVERSE ENTRATE LESSICALI
% Rather than focus on the relations between individual verbs and the types of
% objects they typically take, a number of approaches to IOD appeal to event-struc-
% ture analyses along the lines of Vendlerian semantics. Mittwoch (1982) character-
% ises intransitive ‘eat’ as an activity and transitive ‘eat’ as an accomplishment, thus
% supporting on event-semantic grounds the view that the two uses of ‘eat’ must on
% some level be analysed as involving two separate lexical items. Van Valin and La-
% Polla (1997) similarly assume a distinction between an “activity” and an “active
% accomplishment” reading of verbs such as ‘eat’, where the activity version has a
% nonreferential second argument which is not instantiated as a semantic macrorole
% in the verb’s logical structure, whereas the active accomplishment version has two
% macrorole arguments (Van Valin and LaPolla 1997 :148–150). Van Valin and La-
% Polla’s analysis of ‘eat’ was discussed in 4.5.
% In Brisson (1994), a case is made for two distinct classes of verbs permitting
% IOD (in Brisson’s terms, verbs allowing unspecified objects): “write verbs”, exam-
% ples of which are write, knit, bake, draw, paint, sew, drink, type, dig, and eat; and
% “sweep verbs” such as sweep, plow, pack, dust, vacuum, clean, mow and rake – study
% and read are given in parentheses as atypical examples of this class because they
% behave like these verbs under some, but not all, conditions.
% In brief and somewhat simplified terms, Brisson’s argument is that the sweep
% verbs are not true accomplishment verbs (among other things, she argues that they
% do not necessarily entail a result state, and they can take adverbs of duration with-
% out acquiring an iterative reading) and that this accounts for their ability to occur
% without an overt object, a claim essentially parallel to that made for ‘eat’ by Van
% Valin and LaPolla to the effect that the second argument of activity predicates is
% qualitatively different from that of accomplishment predicates, and therefore omis-
% sible. The write verbs, on the other hand, are accomplishment verbs proper and
% therefore require a different account.
% However, Brisson is not able to provide such an account, beyond proposing
% that write verbs do not really “delete” their objects but rather have two distinct
% variants represented in the lexicon, one intransitive activity verb and one transitive
% accomplishment verb. With respect to the intransitive activity verbs, she discusses ---  
% whether there is not in fact an “implicit argument” present; does not John wrote
% seem to entail that he wrote something? However, she claims that this may not be
% relevant and that “the assumption that ‘something’ was written might be related to
% real world knowledge” (Brisson 1994 :99).
% The appeal to real-world knowledge appears rather ad hoc in this case, as it is
% not clear why this would apply only to the write verbs. If the assumption that
% something is written is attributable to real-world knowledge, why would the as-
% sumption that something is swept require a different analysis? If the interpretation
% of objectless clauses depends on real-world knowledge in some cases, could this
% not be generalised to all cases, making all further analysis superfluous?
% A further problem with this analysis is that it produces a complicated and
% rather vague account of the verb eat, which has traditionally been considered one
% of the most central instances of the IOD phenomenon. Eat in this account is clas-
% sified with the write verbs, but behaves rather differently from the rest of this class
% when used intransitively: “monadic eat seems to strongly suggest that a meal has
% been eaten in the unmarked case: John ate means that he ate dinner, or lunch, or
% whatever may constitute a meal in John’s diet. This completive sense is in strong
% contrast to the monadic write verbs: John wrote does not have any such completive
% sense. However, it is still possible to obtain a “pure” activity reading for monadic
% eat, as in John ate for ten minutes. It is likely that monadic eat would have to be
% lexically marked to account for these properties” (Brisson 1994 :100).
% An analysis in terms of event structure, then, cannot account for the fact that
% a number of presumed accomplishment verbs are frequent in the IOD construc-
% tion, beyond claiming that such verbs occur in two distinct lexical variants. It also
% has difficulties in accounting for the behaviour of ‘eat’ under IOD, which will be
% further discussed in 6.3.3.

\subsection{One verb, two meanings: the state-of-the-art account} \labsec{theory_incorporation}

qui parlare dell'"oggetto logico" e del focus sull'activity!!!\\
è questo il paragrafo in cui distinguere object drop e verbi intransitivi\\
cambiare il titolo in "focus sull'activity" e fare sottoparagrafi, uno per noun incorporation\\
manner/result complementarity?

\paragraph{Inherent objects}
testo

% *Naess (2007: 130-134)* 
% Rice appeals to the notion of prototypical complement of a verb: there is, she
% claims, a basic-level NP which is evoked as the understood object when a verb is
% used without an overt object NP: “Clearly, an omitted object should not be read as
% zero. Rather, on a neutral reading, an omission activates a prototype or a particular
% semantic frame in which the action is prototypical” (Rice 1988 :204). This assump-
% tion is meant to account for the fact that the presence of an object is implied with
% IOD constructions, even though no specific reference is assigned to it.
% The notion of a prototypical complement, however, begs the question of how
% one determines, for a given verb, which possible complement is “prototypical”.
% One of Rice’s examples of such a verb is drink, which is assumed to have the pro-
% totypical complement ‘alcohol’. On this analysis, the fact that the English verb
% drink may be used without an object is a consequence of this verb having a proto-
% typical object associated with it, and this prototypical object is recoverable from
% the objectless sentence, giving the reading of John drinks as John drinks alcohol.
% The question is, apart from the fact that objectless drink in English (and many ---  
% other languages) has this reading, what is the evidence for alcohol being a proto-
% typical object of drink? In the daily life of most people, a typical act of drinking
% does not involve alcohol, and it is not clear why such a specialised notion as that of
% ingesting intoxicating fluids should be “prototypical” with respect to the everyday
% concept of drinking things in general. Without any independent justification, such
% an assumption is circular: Object NPs may be omitted if they represent “proto-
% typical complements” of the verb in question, and since in English objectless drink
% is generally interpreted as meaning ‘drink alcohol’, ‘alcohol’ must be the prototypi-
% cal object of ‘drink’.

% *Garcia-Velasco & Munoz (2002: 4)*  
% >  it is not only the presence/absence of a verbal object that allows
% the transition from an activity to an accomplishment reading with some verbs. When the
% verbal object is non-specific, indefinite or generic, it is possible to obtain the same effect:  
% (3) a. He ate a plate of spaghetti in ten minutes (accomplishment)  
% b. He ate spaghetti for ten minutes (activity)  
% Van Valin & LaPolla (1997) note that this situation is frequent with verbs of creation or
% consumption. According to the authors (1997: 122), the second argument in activity
% predicates does not show referential properties, which usually imparts a generic or habitual
% interpretation to the predication. [...]  
% On the basis of these observations, the authors make the following claim (1997: 122-123):  
% << Thus, the second argument with an activity verb like eat will be called an INHERENT ARGUMENT, an
% argument which expresses an intrinsic facet of the meaning of the verb and does not refer specifically to any
% participants in an event denoted by the verb; it serves to characterize the nature of the action rather than to refer
% to any of its participants. >>

% *Cote (1996: 120)* 
% Lexical (the winner, page 130) -- la spiegazione del perché gli altri due fattori (in particolare il sintattico) non sono vincenti è GENIALE!!!  
% e credo che mi consenta anche di spiegare per quale motivo lo stesso modello funzioni (speriamo) sia per i dObj, che sono argomenti prototipici, sia per gli Strumenti, molti dei quali sono aggiunti prototipici
% > The lexical analysis approach requires that an argument be present in the lexicon but not projected to syntactic structure. 

% *Cote (1996: 156)*  
% CRUCIALE per distinguere gli indefinite dObj verbs dai veri intransitivi (così PM è tranquillo)
% > Also, unlike strict intransitives, when IOA verbs like eat are used, there is a new discourse entity
% (an “eatee”) which is made salient enough to be referred to with a pronoun in a subsequent utterance.

% *Dvorak (2017: 2)*
% > The literature on null arguments
% generally advocates two major theoretical approaches: (A) a null argument is syntactically
% represented and it corresponds to a null pronoun/DP; (B) a null argument is not syntacti-
% cally represented at all and it is a part of a lexical entry for a given predicate.

% *Dvorak (2017: 112)*
% >  Examples like John ate were discussed within the transformati-
% onal generative grammar already in Chomsky 1964 where they were analyzed as ‘deleted
% unspecified object’ (see also Chomsky 1962, Katz and Postal 1964). [...]  
% Bresnan (1978) proposed an alternative, lexical solution to the same issue. She argues that
% the verb eat has a logical object even if it lacks a grammatical object, which is what makes
% it different from the verbs like sleep which have no object at all. 

% *Yasutake (1987: 48)*  
% > they are different from pure intransitives in that the action will not be complete without some lexically implied (but unspecified) object

% *Melchin (2019: 52)*
% > some properties of the omitted objects. I show that they are
% interpreted as indefinite masses, corresponding to the “typical” object of the verb; however,
% there is no evidence that they are present in the syntax. This suggests that it is really an
% instance of omission of an argument,4 rather than the presence of a null pronoun or variable
% present in the syntax

% *Naess (2007: 130-134)*  
% > “In English... condi-
% tions for the omission of non-subject complements are limited to particular lexi-
% cally defined environments. The most commonly discussed of these is the object
% slot for such verbs as EAT, READ, SING, COOK, SEW and BAKE... which are un-
% derstood as having, when used intransitively, an understood object roughly repre-
% sented as the word STUFF” (Fillmore 1986 :95). Fillmore assumes that the ability to
% occur with what he calls Indefinite Null Complements (that is, “deleted” objects) is
% specified in the lexicon for each verb: “From the reality that omissibility phenom-
% ena of the sort discussed in this paper are tighly connected with specific senses of
% specific words, it seems unavoidable that (at least in these cases) closely related
% word senses must be listed separately in lexical entries” (Fillmore 1986 :106).
% Rice (1988) suggests that IOD is not a possibility specified in the lexical entry
% of a particular lexeme, but rather a function of the different ways in which the
% events denoted by certain verbs can be construed: “[C]ertain construals of transi-
% tive events are such that they focus on the active participant and leave the acted-
% upon participant unspecified and, most importantly, to be filled in by a default
% value” (Rice 1988 :203). She argues that objects can be omitted when there is a
% default interpretation for a missing object, exemplifying this with ‘John ate a meal’
% as the default interpretation for John ate. The verbs allowing such omission must
% be “semantically neutral”, not conflating action and manner (contrast John ate with
% *John nibbled), and the omitted object shows a “low degree of semantic independ-
% ence from the verb” (Rice 1988 :204).

\paragraph{Prototypicality}

Prototypicality and other aspects such as noun incorporation

% *Mittwoch (2005: 20)*  
% DUE CONSIDERAZIONI (IN NOTA) FONDAMENTALI!!!
% > It is sometimes claimed that in such restrictions the omitted object is
% prototypical. As an attempt at explanation, this risks circularity. What makes us
% regard the interior of a house as a ‘prototypical’ object of the verb clean is
% precisely the fact that it is typically ‘understood’ in the absence of an overt
% object. [...]  
% In Mittwoch (1971) and (1982) I pointed out that the deleted element could not
% be the indefinite pronoun something, since this would be incompatible with the
% atelic nature of the resulting sentence.

% *Yasutake (1987: 48-50)*  
% 3 tipi di implicit object:
% * read/telephone = "activities where there is some standard or typical kind of object"
% * drink/shave (+ expect/propose/drive) = "the objectless use of this subtype has, conventionally or socially, acquired 
% a slightly different sense from the habitual one" (CFR. GLASS 2020! ROUTINE, WORLD KNOWLEDGE con riferimenti in Yasutake stesso)
% * steal/see/annihilate = "a highly specialized kind of activity, and in many cases imposes practically no limitation on possible objects"

% *Dvorak (2017: 118)*  
% TUTTE COSE FONDAMENTALI DA SPOSTARE NEI PARAGRAFI GIUSTI!  
% >  For Rice (1988), INO represent variation
% which is “not strictly a function of the verb’s inherent meaning”, therefore, it “does not
% warrant additional lexical entries”. In her view, this forces researchers like Bresnan (1982)
% or Hale and Keyser (1986) to make unwarranted assumptions about the lexicon’s power.
% Rice sees object omission as a result of a collection of paradigmatic rather than idiosyncratic
% semantic factors, such as the verb type (verbs that conflate action and manner tend to resist
% object omission: *Celia nibbled/chewed/bit versus Celia ate), or the object type (objects
% denoting wholes are more likely to be left out than objects denoting parts: Travis let Billy
% drive (the car) versus Travis let Billy gun *(the motor)). In general, Rice claims, the se-
% mantically ‘neutral’ verbs with objects that are neither too specific nor too general are the
% most prone to object omission. The omitted object then represents the verb’s ‘prototypical
% complement’, giving rise to the default interpretation, cf. When he goes to Boston, John
% drives (a car / *a Toyota / *a motorcycle / *a vehicle).

% *Ahringberg (2015: 7)*  
% > 2.3.1 Lexical licensing
% According to Fillmore (1986, p. 98) the lexical properties of the predicate are essential in the
% licensing of null instantiation, which implies, in in broad terms, that some transitive predicates
% can allow their objects to be left out whereas other cannot. This is generally conceived as the
% fundamental factor also in other studies on the subject (e.g. Goldberg, 1995; Lambrecht &
% Lemoine, 2005; Prytz, 2009; Bäckström, 2013). Fillmore (1986, p. 98) further claims that this
% kind of lexical description is more consistent than a semantic one, as some verbs allow null
% instantiation whereas predicates with synonymous meanings may require their complements to
% be expressed. For example, the verbs protest and find out can be used without objects, whereas
% oppose and discover cannot, even though the verbs are “semantically related” (Fillmore, 1986,
% p. 98).

% *Ahringberg (2015: 8)*  
% > A single predicate may denote several senses and
% Fillmore (1986, p. 99) emphasises that with polysemous transitive verbs, in other words verbs
% with several different senses, it is rather certain types of the senses and not the predicates per
% se that permit leaving out the object. Likewise, it appears that only particular kinds of
% complements allow being omitted in some cases. For example, a left out object of the verb lose
% (Fillmore, 1986, p. 100) can only refer to a certain kind of competition or election, but not to
% an item which one has forgotten or mislaid. It therefore appears that null instantiation cannot
% be fully accounted for from lexical factors only. 

% *Newman & Rice (2006: 4)*  
% > Huddleston and Pullum (2002: 303-305) refine Huddleston [1988]’s notion of
% intransitivity by offering a sub-categorization of types of ‘unexpressed
% objects’ of intransitive verbs. EAT and DRINK participate in two such
% patterns of omissibility: ‘specific category indefinites’ and ‘normal
% category indefinites’. The former refers to the possibility of understanding
% the intransitive uses of EAT and DRINK specifically as ‘eat a meal’ and
% ‘drink alcoholic drink’ respectively; the latter refers to the use of
% intransitive EAT and DRINK when the unexpressed object is interpreted as
% the ‘indefinite, typical, unexceptional’ exemplar (‘food’ in the case of EAT,
% ‘water’ or ‘beverage’ presumably, in the case of DRINK).  
% The traditional view of an intransitive vs. (mono)transitive distinction,
% as enunciated in Huddleston (1988) and Huddleston and Pullum (2002), is
% by no means compelling. One could just as well argue that the intransitive
% use in (3a) really involves one participant (the agent phrase) and describes
% an activity of that participant, similar to the way in which the intransitive
% verb run in English describes an activity of a runner. Other associated
% entities can be a necessary part of a larger semantic frame of intransitive
% verbs (legs in the case of run, food in the case of eat), but this does not
% require us to say that they are second participants which are simply
% unexpressed. [...]  
% A more provocative view of transitivity can be found in Van Valin and
% LaPolla (1997: 115). They speak of the English predicate as having either
% one or two arguments in its logical structure, similar to Huddleston’s
% distinction between intransitive and monotransitive uses of EAT. Their
% representation of the logical form of EAT expresses the alternatives through
% the parenthesized (y) embedded in the argument structure.  
% (4) do' (x, [eat' (x, (y))]  
% x=CONSUMER, y=CONSUMED

% *Dvorak (2017: 116)*  
% > It is somewhat disturbing that the most indepth discussions of INO in 1970s and 1980s
% were revolving around a single verb to eat. The notable breakthrough in this tradition is
% Levin (1993:33), who listed over forty verbs in English as examples of the ‘unspecified object
% alternation’. They include the verbs bake, carve, clean, cook, drink, eat, hunt, paint, play,
% sing, study, wash, write, etc.22 Levin also notes that the intransitive variants of these verbs
% are ‘understood to have as object something that qualifies as a typical object of the verb’.  
% (199) Mike ate. (→ Mike ate a meal or something one typically eats.)

% *Lorenzetti (2008: 59)*  
% QUESTA COSA DELLO SHIFT SEMANTICO LA DICO ALTROVE! FORSE IN SPECIAL REGISTERS?  
% > This paper argues that the phenomenon of the null instantiation of objects, i.e. the property of
% some transitive verbs to omit their direct complements, can be viewed as a polysemy-trigger.
% Our study, adopting a lexical complexity perspective, suggests that in the majority of cases
% verbs retain traits of their prototypical meaning, which becomes the starting point for possible
% inferences, contributing to the overall interpretative process, and leading to the dynamic
% emergence of different semantic interpretations and nuances through complex mechanisms of
% figure and ground.

\paragraph{INDEFINITE = "SOMETHING"}

testo testo

% *Melchin (2019: 55)*  
% > Thus, the understood objects of UOA verbs pattern with bare masses and plurals, which are
% weak indefinites.
% Further evidence for a weak indefinite reading of understood objects comes from the
% analysis of Steedman (2015). Steedman starts with the observation, attributed to Fodor and
% Fodor (1980), that these understood objects, unlike indefinite pronouns like something, always
% have low scope with respect to other quantifiers in the sentence.

% *Lorenzetti (2008: 63)*  
% QUESTA PAGINA DICE IN MODO SINTETICO COSE MOLTO IMPORTANTI!!!  
% > 3.1.1. Indefinite Null Objects
% This category, also known as Indefinite Null Complements (henceforth INCs) [Fillmore
% 1986] or Unspecified Object Alternation [Browne 1971], is typical of a variety of activity
% verbs of the eat type, such as drink, sing, bake, cook and paint among the others, which have
% a pronounced manner component in their meaning and fairly circumscribed selectional
% restrictions. Hence, the content of the null object is more or less predictable: it will
% correspond to the literal rather than to the metaphorical meaning of the verb , and is
% sometimes argued to be restricted in usage, i.e. an expression such as I’m cleaning IS most
% likely to refer to the interior of a house, rather than to one’s teeth. [...]  
% However, postulating a reading of these verbs in terms of stereotypic entities associated to
% them does not always seem appropriate, since while in example (3) the phantom object is a
% meal, i.e. the apparently stereotypic entity associated to the verb eat, in an example like the
% most likely context would not lead to the interpretation that the person is accustomed to
% having a meal or as many meals as she can during the whole day.
% (5) I started working out, but I would eat all day after that.
% On the contrary, the most typical interpretation in this case is likely to be achieved through
% the underspecified word “food”, a representative of the entire class of edible things. However,
% the fact that the object is unexpressed in this case suggests that what the person eats is
% irrelevant for the current purpose of the interaction.
% We can suppose, in this respect, that a better explanation of the restrictions on these null
% objects is that they have to be consistent with the underlying context, the intentional structure
% of discourse and the shared relevance [Sperber and Wilson 1986] at the time of utterance.


\paragraph{Focus on the activity}
testo

% *Mittwoch (2005: 2)*  
% PARAGRAFO IMPORTANTISSIMO!!!  tra activity e prototypicality
% > There is a well-known transitivity alternation involving a class of common
% process verbs which can be both transitive and intransitive with the same
% subject argument, so that the intransitive variety is unergative:  
% (1) John is reading / drinking.  
% John is reading a letter / drinking juice.  
% A representative list of English verbs participating in the alternation as it will be
% understood here is given in (2).  
% (2) a. read, study, revise (what has been learnt) rehearse, practise  
% b. sing, dance, play (music), act  
% c. write, compose (music) paint (a picture), draw, etch, sew, knit, crochet,
% weave, spin, cook, bake2  
% d. type, print, photocopy, dictate, record, film  
% e. eat, drink, chew, smoke  
% f. sow, plough, harvest, weed, hunt  
% g. wash, iron, mend, darn, clean, sweep, dust, hoover, paint (apply paint
% to), embroider, tidy up  
% The verbs all have a pronounced manner component in their meaning, and fairly
% circumscribed selection restrictions. Hence the content of the phantom object is
% more or less predictable. It will correspond to the literal rather than
% metaphorical meaning of the verb (e.g. read written or printed material rather
% than, say, the stars or coffee grounds) and may be further restricted in usage
% (e.g. the understood object of intransitive clean in he is cleaning is the interior of
% a house, rather than a car, shoes or teeth; that of mend is clothes rather than
% electric gadgets or roads).3
% Aspectually, intransitive predicates with these verbs pattern together, as atelics,
% with VPs consisting of transitive verbs + bare NP objects [−DELIMITED
% QUANTITY], whereas transitive verbs + quantized object yield telic VPs.

% *Ahringberg (2015: 6)*  
% > In cases where the omitted complement is indefinite, the focus is placed on the action
% which the predicate denotes (Fillmore, 1986, p. 96), and some other verbs that allow this type
% of null instantiation include clean, drink, embroider, hunt, iron, read, sing, study, teach and
% write (Levin, 1993, p. 33). It should be stressed that compared to typical intransitive verbs, such
% as those mentioned in the introduction section, there is always an implied object involved,
% which could be interpreted as representing either the word “stuff” or “something” as exemplified
% in (9) (Fillmore, 1986, p. 95), or a more specific concept which is generally conceived and
% linked with the verb, such as dinner in (10).
% (9) He’s too stressed out to be able to eat {stuff}.

% *Yasutake (1987: 50)*
% cita Munro (1982) --> nell'uso intransitivo, "the general action is of more interest than the specific unspecified object"

% *Yasutake (1987: 52)*  
% > verbs with decategorized objects thus share a common property with objectless transitive,
% viz. the communicative intent of the speaker in both is to provide information concerning the
% subject by way of emphasizing the action-type. Removal of an object noun phrase is hence
% regarded as an extreme form of de-categorization

% *Melchin (2019: 52-53)*  
% (questa cosa la diceva anche PM: preparare obiezioni!) + dire anche in altro par, ma quale?
% >  Chomsky (1986) cites an observation by Howard Lasnik
% that the meaning of the intransitive use of eat is somewhat different from the transitive use,
% and is specifically something more like dine; [...]  Similarly, Fillmore
% (1986) notes that while one can, and often does, bake a number of foods besides typical
% “baked goods” such as breads and pastries, including potatoes and hams, John is baking
% generally cannot be taken to mean John was baking, for example, potatoes.

% *Lorenzetti (2008: 66)*
% QUI È USATO PER GIUSTIFICARE ANCHE **DEFINITE** DOBJ DROP!
% > As to the factors more directly connected to the domain of discourse, it is worth
% mentioning the topic/focus distinction. The omitted arguments in (11) and the following are
% all highly predictable, and therefore they are not good candidate for focal status, since "the
% focus is that portion of a proposition which cannot be taken for granted at the time of speech,
% the unpredictable and pragmatically non-recoverable element in an utterance" [Lambrecht
% 1994: 207].  
% A sentence topic, by contrast, is usually defined as “a matter of already established current
% interest which a statement is about and with respect to which a given proposition is to be
% interpreted as relevant” [Lambrecht 1994: 119].  
% (11) a. I thought you said your dog doesn’t bite!  
% b. Religion integrates and unifies.  
% Every sentence requires at least one focus, namely an assertion containing new information
% (Chafe 1994). It would be tempting to claim that when objects are omitted, the focus is on the
% activity itself.

% *Kardos (2010: 5-6)*  
% IMPORTANTE CLASSIFICAZIONE DELLE TEORIE E SPIEGAZIONE!!!  
% > When it comes to matters of argument realization, researchers are split into several camps
% regarding the role of the lexicon and that of aspectual notions in this process. As I will discuss
% in section 4 in greater detail, some bolster the so-called Free Argument Projection Hypothesis
% which says that ′′arguments of verbs are projected freely onto syntax, with verbs being
% unspecified for those components of meaning that determine argument expression5′′
% (Rappaport Hovav and Levin 2005: 275). Others attribute argument realization patterns purely
% to aspectual properties such as incremental theme and measure. Unlike the former two
% ' camps' , Levin and Rappaport Hovav strongly reject the Free Argument Projection Hypothesis
% and also question the sole role of aspect in argument expression. Their theory is
% fundamentally lexically-based as is seen below in the exposition of the system. [...]  
% << THE ARGUMENT-PER-SUBEVENT CONDITION: There must be at least one argument XP in the
% syntax per subevent in the event structure.  
% (Rappaport Hovav and Levin 2001: 779) >>  
% The popularity of this condition is apparent from the fact that it has been accommodated in
% the work of numerous researchers, e.g Goldberg (2005)6, Grimshaw and Vikner (1993), van
% Hout (1996), inter alia. An important consequence of this is that verbs denoting simple events
% must project a single argument, whereas verbs expressing complex eventualities alway occur
% with at least two obligatory elements in the syntax. The former class is represented by activity
% type verbs. For instance, run lexicalizes a single event in its lexical representation, which in
% turn yields the obligatory appearance of a single argument (i.e. an agent) in subject position.
% A prototypical verb exemplifying the latter class is the causative verb, break, which is
% associated with two subevents, one being the causing event and the other one, the event
% caused by the causing event. The causing event is a simple activity, while in the second
% subevent an externally caused result state is brought about.

% *Sugayama (2007: 1)*  
% PRAGMATIC AND DISCOURSE FACTORS?
% > researchers have proposed that causative verbs obligatorily express the argument that
% undergoes the change of state in all contexts (Browne 1971; Brisson 1994; van Hout 2000; Ritter and
% Rosen 1996, 1998; Rappaport Hovav & Levin 1998). This generalisation is too strong to accommodate
% real fact [...]  
% counterexamples may be accounted for by the following Principle of Omission under Low Discourse
% Prominence in Goldberg’s (2005) Construction Grammar. [...]  
% (2) Principle of Omission under Low Discourse Prominence: Omission of the patient argument
% may be possible when the patient argument is construed to be de-emphasized/unprofiled in the
% discourse vis-а-vis the action [...]  
% That is, omission is possible when the patient argument is (or focal) in the discourse, and
% NOT TOPICAL the action is particularly emphasised [...]  
% the attention can be shifted away from the (definite) argument in favour of the action, if the action is sufficiently emphasised due to the
% TOPICAL patient argument being present and salient in the IMMEDIATE NON-LINGUISTIC CONTEXT

% *Mittwoch (2005: 3)* 
% TRA ACTIVITY E NOUN INCORPORATION
% > In the early days of generative grammar the intransitive version of these verbs
% was derived by a transformation deleting the object. Today it is generally
% thought that the objects of the verbs concerned in this alternation, though
% appearing in the lexicon, need not be projected in the syntax.4
% Thus an influential paper by Grimshaw (1993) draws a distinction between
% structural and content components of meaning. The objects of change-of-state
% verbs are structural, and must be projected; the objects of activity verbs are
% content arguments, and are in principle optional (subject to certain ill-
% understood restrictions). The reason is that change-of-state verbs have a
% complex event structure involving something like x cause y to change state,
% where y represents the object, whereas activity verbs have a simple structure: x
% act. Additional components of meaning that distinguish between different verbs
% in each structure are content components in this theory.

% *Wierzbicka (1982: 758)*  
% RIFLETTERE SOPRATTUTTO SUL SECONDO CAPOVERSO  
% > the action reported in a have a V frame cannot have an external
% goal: it must be either aimless, or aimed at some experience of the agent. [...]  
% Finally, the action (or process) must be seen as repeatable. Having a swim
% (or a read, or a try) is something that can be done again and again. There is
% something arbitrary about the length of a walk, a lie-down, or a read. Since
% these activities (when reported in a have a V frame) are aimless, devoid of any
% external goal, they can not only be extended or terminated at will, but can also
% be resumed at will. Thus actions which cannot be repeated cannot be described
% in a have a V frame. For example, the contrast of have a bite or a lick or a
% taste vs. ?have an eat may result at least partly from the contrast in repeatability
% of the actions in question. One could bite John's sandwich, or lick his ice
% cream, or taste his soup-not once but twice, or more-but one could eat his
% sandwich only on[ce]

% *Wierzbicka (1982: 759)*  
% > A translation into a more conventional metalanguage could read: 'The have
% a V construction is agentive, experiencer-oriented, antidurative, atelic, and
% reiterative.' 

% *Wierzbicka (1982: 771)*  
% > Examples are have a bite, a lick, a suck, a chew, a nibble.
% The verbs included in this subtype are transitive or semi-transit
% meaning of the verb ensures that the undergoer of the action is on
% affected by it. A lick, bite, or nibble is not enough to make muc
% to the object involved; and a chew or a suck would also not affect
% greatly, because of the slow nature of the process. This means tha
% verb is transitive, and the action requires an undergoer as well as
% undergoer can be ignored; the action can be viewed as really invo
% one participant (the agent). However, if someone eats an apple or
% the object in question is TOTALLY affected, and thus impossible to ignore

% *Wierzbicka (1982: 776)*  
% QUESTO VA INSERITO NEL DISCORSO SULLA TELICITY  
% > This means that the 'unlimited substance' type, like the 'objectless action' type
% (?4), demands strict atelicity. Drinking or smoking (like walking or swimming)
% is something that can be prolonged for as long as the agent wishes; but drinking
% a glass of water or smoking a cigarette cannot be indefinitely extended (just
% as walking to the post office or swimming to the shore cannot). Thus

% *Wierzbicka (1982: 788)*  
% > . This does not mean that the frame takes only intransitive verbs. It
% allows transitive verbs, too, but only if the sentence with a transitive verb
% permits an interpretation compatible with the requirement of one core partic-
% ipant: the agent/experiencer. Only two types require one-argument verbs: the
% type which refers to aimless and objectless individual action conducive to
% feeling good, and that which refers to potentially therapeutic, semi-voluntary,
% 'corrective' individual act

% *Naess (2011: 419)*  
% UTILE ANCHE PER I VERBI DI MOTO!!! fare una nuotata/caduta/corsa/scalata != *fare un'andata  
% (V. CONSIDERAZIONI IN FONDO ALLA STESSA PAGINA SUL VERBO "RUN")  
% > In other words, affectedness of the agent
% is associated not only with specific case-marking patterns, but also with the possibility of
% intransitive behaviour. [...]  
% Wierzbicka (1982) discusses constructions of the type have a drink, have a swim, and
% accounts for their distribution to a large extent in terms of affectedness of the agent participant. [...]  
% Wierzbicka links this affected-agent semantics explicitly to reduced formal transitivity:
% ‘have works as a detransitiviser: the object is de-emphasised, the predication concerning
% the object is backgrounded, and at the same time the emphasis on the agent increases’
% (Wierzbicka 1982:291). [...]  
% Nжss claims that it is a crucial property of transitive constructions that
% their arguments are maximally distinct – not just as physical entities, but in terms of the role
% they play in an event. Thus a prototypical transitive construction has one single controlling
% participant and one single affected participant. Verbs of eating and drinking deviate from this
% pattern because both the agent and the patient are affected

\paragraph{Object drop as noun incorporation}
queste teorie sostengono che l'oggetto non sia rappresentato sintatticamente! (vero?)

% *Dvorak (2017: 119)*  
% > An unorthodox approach to INO is presented by Martı́ (2011), who is primarily moti-
% vated by defeating the view that English INO are purely pragmatic in nature (Groefsema
% 1995, a.o.). Martı́ argues that the INO of verbs like eat, bake, smoke, drink, read, write,
% hunt, cook, sing, carve, knit, weed, file, write, etc. are grammatically represented, number-
% neutral nouns, not too different from nouns incorporating into verbs in noun-incorporating
% languages. Her argumentation is based on the fact that English verbs with implicit indefi-
% nite objects are generally atelic (except for John ate for/in and hour ), and they describe
% conventional, name-worthy, institutionalized, habitual activities – just like verbs that have
% undergone noun-incorporation (cf. Mithun 1984, Dayal 2011b:164). I get back to the ties
% between INO derivation and noun-incorporation in 5.2.3.

% *Martì (2015: 453)*  
% NOUN INCORPORATION! TORNA SPESSO, MERITA UN PARAGRAFO A PARTE?  
% > That there is a notional, even if unuttered, object (indicated in parentheses) in
% the interpretations of the object-less versions of these sentences is uncontrover-
% sial. The controversial question is whether there is an object at any point in the
% grammatical derivation of the object-less sentences, despite its lack of phonological
% realization.
% Carston (2004), Groefsema (1995), Hall (2009), Iten et al. (2004), Recanati
% (2002) and Wilson and Sperber (2000) have suggested that, in at least some of
% the uses of these seemingly object-less sentences, there is indeed no object at any
% level of linguistic representation. Instead, that object is provided for pragmatically.
% In these cases, given appropriate pragmatic pressures, language users enrich gram-
% matical interpretations in such a way as to provide the missing material. Here I
% argue that an object is, on the contrary, part of the linguistic representation of
% sentences such as (37)–(41) in all of their uses. This object, however, is not a
% run-of-the-mill syntactic object, but an object whose properties are those of nouns
% in noun incorporation constructions found in languages such as West Greenlandic,
% Frisian, and many others. The argument rests on the following logic: if these
% notional objects behave like linguistic category X (i.e. incorporated nouns), then
% the null hypothesis is that the notional objects are themselves instantiations of that
% linguistic category X. The only difference between the objects of interest here and
% those that appear in noun incorporation is that the former, but not the latter, are
% phonologically null.14

% *Martì (2015: 454)*  
% > Noun incorporation is a word formation process whereby a verb and a noun (usually,
% its object) are put together to form a verb. [...]  
% When the noun and the verb are put together, a number of associated effects follow.
% Importantly, the object noun loses its status as a regular syntactic object. This can
% be seen in the fact that the nouns in these constructions must always appear bare

% *Martì (2015: 455)*  
% ALTRE IMPORTANTI CONSIDERAZIONI SULLA NOUN INCORPORATION  
% > In West Greenlandic, an erga-
% tive language, objects of transitive verbs and subjects of intransitive (unergative) verbs
% are marked with ABSOLUTIVE Case, and subjects of transitive verbs are marked with
% ERGATIVE Case. In (42)b, where the noun has incorporated, the subject is marked
% with ABSOLUTIVE, not ERGATIVE, Case.
% Word order is also affected, so that incorporated nouns always precede the
% verb, independently of the canonical position of regular syntactic objects in the
% language in question. [...]  
% There are a number of semantic characteristics that are associated with
% noun incorporation. The incorporated noun is interpreted indefinitely and
% non-specifically: compare the a and b examples in (42)–(44). The indefinite,
% non-specific semantics of the nouns that participate in noun incorporation is a
% widely noted fact in the literature (see Carlson, 2006; van Geenhoven, 1998; Gerdts
% and Marlett, 2008; Mardirussian, 1975, p. 386; Mithun, 1984; de Reuse, 1994;
% Spencer, 1991; Sullivan, 1984, among others). A second cross-linguistically stable
% semantic property of incorporated nouns is that they typically take narrow scope
% with respect to other operators in the sentence 

% *Martì (2015: 456)*  
% > Verbs that are incorporated into typically designate name-worthy, typical activi-
% ties. For example, Axelrod (1990, p. 193) says that ‘ ... incorporation provides the
% lexicalized expression of a typical activity’. And Mithun (1984, p. 848) says that
% ‘some entity, quality or activity is recognized sufficiently often to be considered
% name-worthy in its own right’.

% *Martì (2015: 457)*  
% > It is well known that implicit indefinite objects are interpreted as non-specific
% indefinites with plain existential import (Bresnan, 1978; Dowty, 1981; Fillmore,
% 1969, 1986; Fodor and Fodor, 1980; Gillon, 2012; Mittwoch, 1980; Shopen, 1973;
% Thomas, 1979). In (49), the speaker does not convey the idea that something in
% particular has been eaten today, the same way that that would not have been conveyed
% had s/he said ‘John has already eaten food/a meal today’:
% (49) John has already eaten today.

% *Martì (2015: 458)*  
% > Another property that implicit indefinite objects share with incorporated nouns
% is that they take obligatory narrow scope with respect to other operators in the
% sentence, such as negation or intensional verbs (Fillmore, 1986; Fodor and Fodor,
% 1980; Mittwoch, 1982; Wilson and Sperber, 2000): [...]  
% Again like incorporated nouns, implicit indefinite objects are semantically
% number-neutral. As far as I know, this fact has not been noted before. Thus, if (64)
% is true, then it is immaterial whether John is smoking half, one or many cigarettes:

% *Martì (2015: 459)*  
% SIGNIFICATO CONVENZIONALIZZATO DEGLI USI INTRANSITIVI  
% > When verbs such as eat, drink, write, etc. take on implicit indefinite objects, they
% typically give rise to ‘conventionalized’ meanings, in a similar fashion to incorpo-
% rated nouns. Thus, if John is eating, then he can only be eating edible things, things
% that are conventionally eaten. Whereas it is perfectly possible to say that John is eat-
% ing his bed, strange as that may be, when one says that John is eating, one means
% that he is eating things that are normally eaten. 

% *Martì (2015: 461)*  
% QUESTO CONSENTE DI DISTINGUERE INDEFINITE DROP DA DEFINITE DROP!!!
% > Frisian allows us to test the predictions of the proposed analysis in an interest-
% ing way. That’s because this language has both implicit indefinite objects and noun
% incorporation, as we saw in Section 3.1. The fact that Frisian is such a close relative
% of English allows for a controlled comparison between the two languages.
% In fact, this is precisely the argument made by Dyk (1997) regarding the nature
% of what he calls ‘detransitivized’ verbs (in Dowty’s 1989 terminology), which are
% the verbs that take implicit indefinite indefinite objects, such as English eat. [...]  
% This means that unergatives, detransitivized and incorporated-into verbs form a
% natural class. Then, there a number of restrictions on the type of verb that can be
% incorporated into in Frisian and these very same restrictions are strikingly observed
% for detransitivized verbs. For example, only verbs that select for a Patient object
% allow incorporation. The verbs corresponding to English notice, hate and know don’t
% take Patients as objects and do not allow noun incorporation or detransitivized uses
% (Dyk, 1997, pp. 95, 108):

% *Martì (2015: 463)*  
% QUESTO SIGNIFICA CHE NON DIPENDE DALL'OBJ=PATIENT, MA DALL'AGENT AFFECTEDNESS?  
% (conferma: esiste bird-watching, soggetto Agent, Obj != Patient)  
% > A verb like know never has a volitional subject, and, accordingly, in Frisian this verb
% never allows incorporation or detransitivization.

% *Martì (2015: 466)*  
% > A common denominator in various proposals for the analysis of noun incorpo-
% ration is the semantic function that is assigned to the incorporating noun: that of
% restricting the internal argument of the verb. I build upon Chung and Ladusaw’s
% (2004) implementation of this idea.
% Chung and Ladusaw (2004) propose that there is a mode of composition called
% Restrict that applies in cases of formal incorporation. This rule applies in cases where
% a predicate is being combined with a property. The property argument is interpreted
% as a restrictive modifier of the predicate. The operation does not reduce the predi-
% cate’s degree of unsaturation; i.e. semantically, the predicate is still missing an internal
% argument. Restrict is illustrated in (87):

% *Martì (2015: 467)*  
% > Any pragmatic approach must consider the cluster of properties we discussed in
% the previous sections as accidental. Nothing in this type of approach predicts that
% implicit indefinite objects always take narrow scope, are number-neutral, or are
% interpreted indefinitely and non-specifically.

% *Martì (2010: 7)*  
% IN RISPOSTA A CHI DICE CHE NON C'È DIFFERENZA TRA DEFINITE E INDEFINITE?  
% > Notice the difference in meaning between incorporated and non-incorporated
% objects in these languages. For example, whereas in (15), the grammatical and
% notional object is interpreted specifically, in (12) the notional object is interpreted
% indefinitely. The indefinite, non-specific semantics of incorporated nouns is a widely
% noted fact in the incorporation literature (see Mithun 1984, Sullivan 1984, de Reuse
% 1994, Spencer 1995, etc.).

% *Martì (2010: 7)*  
% >  the semantic properties of implicit indefinite objects
% in languages like English are in fact the same as the “cross-linguistically stable
% properties of the semantics of incorporation” (in Farkas and de Swart’s 2003 and
% Carslon’s 2006 terminology). After presenting the case for semantic incorporation, I
% show that, formally, implicit indefinite nouns undergo compound noun incorporation,
% not classificatory noun incorporation.

% *Yasutake (1987: 51)*  
% OBJECT DE-CATEGORIZATION AND NON-INDIVIDUALITY  
% parla di object incorporation (John did some deer-hunting) and "the use of an explicit but non-individuated object" (John hunted deer)

% *Melchin (2019: 56)*  
% > 3.2.2  Omitted objects are not present in the syntax

% *Bourmayan & Recanati (2013: 125)*  
% drink alcohol --> alcohol-drink --> drink(2)  
% write with a pen --> pen-write --> write(2)  [questa è una mia ipotesi, devo ragionarci su]  
% > Through free enrichment, a lexical item (or, for that matter, a complex phrase) can be
% understood in a more specific sense than the sense it literally has. For example,
% intransitive ‘drink’ is often understood in the specific sense DRINK ALCOHOL [...]
% can be achieved by morphosyntactic means. In languages such as
% West Greenlandic, some verbs can undergo a process of ‘incorporation’ of their direct
% objects which yields noun-verb combinations behaving like single, verbal, morphological items [...]
% According to van Geenhoven (1998), the incorporated object of the verb does not
% correspond to a genuine argument: it does not denote an individual of type e, but
% rather a property of type <e,t>.

% *Naess (2007: 129)*
% ALTRI QUI NEL PARAGRAFO AVEVANO DETTO CHE EAT E DRINK SONO VERBI PROTOTIPICI
% > Two main proposals have been made as to
% how to integrate such verbs into a broader theory of valency in general: either the
% “intransitive” and the “transitive” variants of a verb are counted as two different
% lexical entries with different argument structures, or the lexemes in question have
% as part of their lexical entry that an indefinite direct object need not be overtly ex-
% pressed. The problem, then, becomes one of identifying the set of verbal lexemes
% which have the property of being able to participate in this “alternation”, and, if pos-
% sible, providing a semantic characterisation of the verbs belonging to this set.
% As Marantz (1984 :192) puts it, “it is an interesting and important problem to
% characterize the transitive verbs that permit indefinite object deletion”. Marantz
% assumes the ingestive verbs, of which he gives ‘eat’, ‘drink’, and ‘learn’ as examples,
% to be the core candidates for IOD, recurring in language after language. In other
% instances, he claims, “the alternation is created not by a productive lexical rule but
% by generalization by analogy with certain core verbs exhibiting the alternation”
% (Marantz 1984 :193).
% (NOTA MIA: eat, drink e learn sono affected agent verbs! v. naess 2011: 413 + haspelmath negli appunti al prossimo paragrafo)

\paragraph{Agent affectedness} 
Agent affectedness: why eating and drinking are poor choices for typical transitivity examples

% *Naess (2007: 144)*  
% > IOD is considerably more common with some verb types than with others (affected-agent
% verbs, where the agent can be construed as the “endpoint” of the event, and ef-
% fected-object verbs, whose objects are inherently nonreferential and therefore eas-
% ily omissible),

% *Naess (2007: 150)*  
% AMBITRANSITIVE/LABILE VERBS  
% > In other words, what characterises the verb cook semantically is the same
% property that was demonstrated for bake in 6.3.1 above: the transitive cook is am-
% biguous between an affected-object and an effected-object reading. We saw above
% that for such verbs, indefinite object deletion is only permissible with the effected-
% object reading. That is, an intransitive clause where the S is interpreted as an agent
% can only be read as referring to the agent’s bringing something into existence.
% Dad is baking can only mean that he is producing things from loose ingredients
% through a process of mixing and heating them (baking bread or cookies); it cannot
% mean that he is putting already existent objects through a process which produces
% an alteration in them (baking potatoes). On the other hand, the patient-subject
% variety, for instance the potatoes are baking, clearly has the affected-object reading,
% as predicted by the restriction of inchoative-causative verb pairs to verbs referring
% to a change of state. 

% *Naess (2007: 141-144)*  
% ANCORA SU EAT/DRINK --> CI DEVO FARE UN PARAGRAFO A PARTE!  
% > A recurring question in discussions of IOD with ingestive verbs is not simply why
% such verbs so frequently undergo IOD, but also why the use of these verbs without
% an object tends to give rise to certain very specific readings. In the words of Fill-
% more, “EAT is used to mean something like ‘eat a meal’ – not merely ‘eat some-
% thing’, and DRINK is used to mean ‘drink alcoholic beverages’” (Fillmore 1986 :96).
% It is this use that motivates Rice’s analysis of ‘alcohol’ as being somehow a proto-
% typical object for drink, and it leads Fillmore to assume that IOD in fact comes in
% two distinct varieties, “one involving a semantic object of considerable generality,
% the other requiring the specification of various degrees of semantic specialization”
% (Fillmore 1986 :96). There are obvious weaknesses in this analysis, since it does
% not explain how a language user determines whether an absent object should be
% understood as being of “considerable generality”, or, on the contrary, representing
% a “semantic specialization”. [...]  
% The semantic property which gives rise to these
% specialised meanings is in fact the same property that makes these verbs eligible for
% indefinite object deletion in the first place: affectedness of the agent. I suggested in
% 4.3.1 that omitting the object with these verbs amounts to highlighting the effects
% that the action has on the agentive participant, while backgrounding those on the
% patient, by omitting the patient argument altogether. The so-called specialised
% readings arise precisely as a result of this emphasis on the effect on the agent.
% To understand how objectless eat and drink acquire their conventionally un-
% derstood meanings of ‘eat a meal’ and ‘drink alcohol’, respectively, we must con-
% sider the contexts in which these verbs are generally used without an object. As it
% turns out, these two verbs seem to be in almost complementary distribution with
% respect to which grammatical contexts favour their use without an overt object NP
% (see also Newman and Rice 2006).
% The English verb eat is mostly used without an object in the past tense or per-
% fect: John ate, I have eaten, or in the progressive, I am eating. The generic/habitual
% ??John eats, on the other hand, is highly unnatural, if not downright ungrammati-
% cal.  
% By contrast, objectless uses of the verb drink are most frequently found pre-
% cisely in a generic/habitual construction, as in John drinks, or Fillmore’s example
% I’ve tried to stop drinking. The past-tense John drank seems to be mainly interpret-
% able as a habitual statement in the past tense, while the perfect ??John has drunk is
% extremely peculiar without an object. The progressive John is drinking is possible,
% though perhaps a little strange out of context; it does not appear to favour the
% reading ‘drinking alcohol’ to quite the extent that the generic statement does, al-
% though this reading is facilitated by the inclusion of a context evoking the act of
% drinking alcohol: John is in the pub drinking.  
% In fact, if we attempt to interpret drink as referring simply to the ingestion of
% any kind of drinkable fluid, then the same thing happens with this verb as with eat:
% it becomes very unnatural in a generic/habitual clause without an object – ??John
% drinks. Most analyses seem to assume that this is because the conventional use of
% drink to mean ‘drink alcohol’ somehow blocks the reading ‘drink things in general’,
% or that the verb drink somehow selects ‘alcohol’ as a kind of privileged argument
% (Rice 1988). However, this does not explain how this “privileged” reading arose in
% the first place. Rather than explain the absence of a generic use of drink to mean
% ‘drink anything’ by invoking an “overriding” reading of such clauses as ‘drink alco-
% hol’, I suggest that the latter should in fact be explained in terms of the former. It is
% not the case that the reading ‘drink anything’ is unavailable for objectless clauses
% because the reading ‘drink alcohol’ overrides it; such a statement has little or no
% explanatory value, saying in effect that objectless drink means ‘drink alcohol’ be- --  
% cause it means ‘drink alcohol’. Rather, the reading ‘drink alcohol’ is available for
% generic statements precisely because the ‘drink anything’ reading is pragmatically
% unnatural and therefore unlikely to be expressed.  
% The naturalness of generic statements is constrained by properties of the world.
% When verbs like eat and drink do not easily occur in generic objectless statements
% of the type John smokes, it is for a very simple reason. Eating and drinking are the
% two most fundamental actions of human existence; their habitual performance is
% the most basic prerequisite for sustaining life. ??John eats, in the sense that eating
% is a habit of John’s, is highly peculiar for the obvious reason that humans and ani-
% mates who do not have this habit simply do not exist – they would be dead (cf. also
% Fillmore 1977 :135). Referring to inanimate beings, on the other hand, such sen-
% tences are much less peculiar; one can imagine, for example, an advertisement
% listing the attractions of a lifelike doll: She eats, she cries, she sleeps!
% While the sentence John drinks, taken to mean ‘John habitually imbibes liq-
% uids’ is unnatural because is is close to tautological – if John did not have this
% habit, he would not exist – there is nothing in the English grammar which pre-
% cludes the construction of such a sentence. Drink is an affected-agent verb highly
% suited to appear in low-transitivity contexts such as objectless generic construc-
% tions; it is the real-world interpretation of this construction which makes it odd.
% The sentence is syntactically well-formed but semantically strange – which leaves
% open the possibility of assigning it an alternative interpretation.  
% There are a couple of reasons why a suitable candidate for such an interpreta-
% tion should be ‘drink alcohol’. The consumption of alcoholic beverages has an im-
% portant social and cultural function in the Western cultural sphere, and increas-
% ingly in the rest of the world. Newman and Rice (2006) take the ‘alcohol’ reading
% of intransitively-used ‘drink’ to reflect “the prominence of alcohol consumption as
% a topic of discourse”.  
% In addition to these pragmatic considerations, however, the semantic proper-
% ties of objectless ‘drink’ – the emphasis on the affectedness of the agent – are in
% themselves conducive to this particular reading. The reading ‘drink alcohol’ very
% much preserves, if not augments, the affected-agent semantics of ‘drink’ and of the
% objectless construction. The act of drinking alcohol involves a highly specific in-
% tended effect on the agent – intoxication – which is not only clearly noticeable to
% the agent himself but also frequently observable to others. It is this intention of
% achieving a particular effect, and subsequently the habit of being in a particular
% state, namely intoxication, that we attribute to people when we use drink without
% an overt object: a statement like John drinks clearly implies that he drinks for the
% purpose of getting drunk, and that he does so frequently – not that he occasion-
% ally sips a glass of wine with his dinner because he enjoys the taste.  
% Given that the default reading ‘drink liquids’ is unavailable, or rather unneces-
% sary, for pragmatic reasons, then, the ‘drink alcohol’ reading is ideally suited to
% take over as standard interpretation for generic drink: affectedness of the agent is
% crucially relevant to its semantics, and it has a sociocultural significance which
% makes it highly eligible for being cast in some simplified, conventionalised expres-
% sion. In short, to say of someone that ‘he drinks things’ does not convey any inter-
% esting information about the person in question, while saying that ‘he drinks alco-
% hol (frequently/habitually/too much)’ does; so the construction which would
% normally be used to express the former is more usefully employed as a conven-
% tionalised expression of the latter.  
% Objectless eat, on the other hand, shows a different distribution. There does
% not appear to be any such socially significant variety of the act of eating which
% could usefully take over the unnecessary generic construction He eats in English;
% which means that eat is not normally found in this kind of construction – cer-
% tainly not with reference to humans or animates. On the other hand, objectless eat
% is found with reference to specific acts of eating that are either ongoing or con-
% cluded; and in this use it appears to have the sense of ‘eat a meal’.
% Recall again that the semantic effect of using an affected-agent verbs in an ob-
% jectless construction is to emphasise the affectedness of the agent, to the extent that
% the patient, being immaterial to the meaning conveyed, is suppressed. An object-
% less ‘eat’ sentence, then, is a statement about the effect on himself that the agent
% achieves through the act of eating. Typically, the effect one seeks to achieve by eat-
% ing is that of eliminating hunger; a typical act of eating consists of eating until one
% is full. Since the amount of food required to achieve this is what we usually refer to
% as ‘a meal’, this explains the interpretation of objectless ‘eat’ as ‘eat a meal’: if nothing
% else is specified, we interpret ‘eat’ as meaning ‘eat until full’ – i.e. eat a meal.
% Indefinite object deletion, then, functions much like antipassivisation in that
% it is essentially a syntactic mechanism used in contexts of low semantic transitivity.

% *Naess (2007: 55)*  
% > A phenomenon which in a sense is similar to that of indefinite object deletion
% is the occurrence of suppletive verb pairs for transitive and intransitive uses of ‘eat’.

% *Naess (2007: 55-56)*  
% SPOSTARE IN CAPITOLO SU ASPECT (TELICITY?)  
% > An affected argument is described by Tenny (1994) as “one that makes the
% event described by the verb delimited, by undergoing a change of state that marks
% the temporal end of the event”. As a result of this, affected arguments are “measur-
% ing arguments”, an argument type which “measures out and imposes delimited-
% ness on the event” (Tenny 1994 :158). In other words, affectedness is a property
% which defines the perceived endpoint of a verbal event; [...]  
% This view of affectedness has interesting implications for the notion of Af-
% fected Agent: If affected arguments “measure out” events, then the agent of a verb
% like ‘eat’, by virtue of being affected, might function as an argument serving to de-
% limit the verbal action. In other words, an Affected Agent is, or can be construed
% as, the endpoint of the act of eating. [...]  
%  It is essential to note that it is this effect, rather than that on the
% patient, which is the main goal of the agent’s act: we eat in order to achieve an ef-
% fect on ourselves, not primarily on the food.  
% In Tenny’s model, an event cannot show more than one instance of “measur-
% ing out” by an argument, and only internal arguments can measure out events.
% Facts such as those just discussed, however, suggest that it is in fact possible for an
% event to have two potentially measuring arguments, although only one of them
% may function as an actual measuring argument in any given clause. Furthermore,
% they suggest that the restriction on which arguments may be measuring arguments
% is not a formal restriction in terms of internal vs. external arguments, but rather a
% semantic restriction in terms of affectedness: any affected argument has the po-
% tential of serving to measure out an event.

% *Naess (2007: 57)*  
% > The choice of which argument should be taken to measure out an event such as
% that of eating is in reality a question of perspective. Either one may choose to focus
% on the objectively most affected participant in the event, the Patient, whose af-
% fectedness is immediately observable to any objective bystander. Or else one may
% focus on the most salient effect from the agent’s point of view, namely the effect
% that the agent registers directly on himself and which constitutes his motivation
% for engaging in the act of eating.

% *Naess (2007: 61-63)*  
% QUINDI EAT NON È TRANSITIVO PROTOTIPICO MANCO PE GNENTE  
% > A similar, though more complex, situation is found in the Mayan language
% Yukatek of Mexico. To begin with, it should be noted that Yukatek has no under-
% ived transitive verb meaning ‘eat’; han ‘eat’ is intransitive, and can only be used
% with a direct object if transitivised with an overt transitivising suffix (Jürgen
% Bohnemeyer p.c.). [...]  
% In general, the class of inherently imperfective verbs, which take the transitive
% subject cross-referencing marker in their unmarked aspect, consists of “activity
% verbs” – verbs such as ‘work’, ‘run’, ‘bathe’ and ‘dance’ belong to this class. The class
% of inherently perfective verbs, which in their unmarked aspect show the marker
% otherwise used to cross-reference transitive objects, consists mainly of “change-of-
% state” verbs such as ‘arrive’, ‘awaken’, ‘die’ and ‘fall’.
% Han ‘eat’, however, despite being considered an activity verb by Krämer and
% Wunderlich (1999 :447) belongs to the inherently perfective class. In other words,
% when it is unmarked for aspect, it takes the subject cross-referencing marker which
% with transitive verbs is used for objects. [...]  
% The classification of ‘eat’ as an “inherently perfective” verb in Yukatek can sim-
% ilarly be understood as a consequence of the verb’s having an Affected Agent argu-
% ment. 

% *Naess (2007: 63)*  
% > Yukatek is by no means the only language where the process of causativisation
% functions differently with respect to ‘eat’ than for other verbs. Amberber (2002)
% notes that in a number of languages, causative morphemes or constructions which
% in general are restricted to occurring on intransitive verbs exceptionally occur on
% just one small class of transitives: the ingestive verbs.

% *Naess (2007: 75)*  
% > 4.4.3 ‘Eat’ grammaticalised as marker of agent affectedness  
% The semantics of affectedness inherent to ‘eat’ is reflected in a number of languag-
% es through the use of this verb as an auxiliary, as a light verb in noun-verb con-
% structions, and other grammaticalised uses. In most of these languages such uses
% of ‘eat’ express a sense of undergoing, affectedness or adversativity:

% *Naess (2007: 77)*  
% SPOSTARE IN CAPITOLO SU TELICITY! (CONTINUA DOPO, CREDO)  
% > Of the properties included in Hopper and Thompson’s list of Transitivity pa-
% rameters, the most obvious candidate for an explanation for the intransitive be-
% haviour of ‘eat’ verbs would appear to be parameter C, “aspect” – telic vs. atelic. As
% Hopper and Thompson’s Transitivity notion is explicitly defined as a property of
% clauses, not all their parameters are directly applicable to individual verbs; but for
% those that are, the verb ‘eat’ must be ranked as “high” at least for parameters A, B,
% E, H, and I: it involves two participants, denotes an “action”, is necessarily voli-
% tional, and takes an A which is “high in potency” (human or animate) and a high-
% ly affected O. Telicity, on the other hand, has sometimes been appealed to as an
% explanation for the aberrant behaviour of ‘eat’ verbs. Thus Van Valin and LaPolla
% characterise ‘eat’ as “not inherently telic” (1997 :112), while Tenny (1994) main-
% tains that the telicity of ‘eat’ depends on the “delimitedness” or “non-delimited-
% ness” of its measuring argument, that is, its object: “If Chuck eats an apple, he fin-
% ishes eating when the apple is gone, but if he eats ice cream, he continues eating for
% an indefinite period of time, because there is an indefinite quantity of ice cream.
% (He may even continue eating ice cream forever, if he is in a world that never runs
% out of ice cream)” (Tenny 1994 :24).
% Quite apart from the fact that the real world does not actually work this way

% *Naess (2007: 78-79)*  
% COME SI CONCILIA QUESTO CON MEDINA? PER LEI EAT È TELICO O ATELICO?  
% > It is well known that telicity depends not only on the predicate of a clause but
% also on the nature of its object, if it has an object, so that bare-plural or mass noun
% objects lead to atelic readings (Dowty 1991): John built a chair in an hour/*for an
% hour vs. John built chairs for an hour/*in an hour. This variation is not specific to
% ‘eat’ but is apparently characteristic of verbs with incremental themes (Dowty
% 1991, Tenny 1994). The interesting question in the case of ‘eat’ is how this verb
% behaves when it is used intransitively, i.e. when the presence of different kinds of
% objects cannot influence the reading.
% Intransitive ‘eat’ is in fact perfectly compatible with an adverbial of comple-
% tion: I ate in five minutes, then rushed off to work. This clearly shows that intransi-
% tive ‘eat’ does in fact have a delimited – that is, a telic – reading; and since no object
% is present, this telic reading cannot derive from the “delimitedness” of the object.
% Rather, the telic reading arises from the affectedness of the agent. [...]  
% On the other hand, we also find ‘eat’ with adverbials of duration: We ate all
% evening. This is a rather striking alternation which to my knowledge is relatively
% rare with intransitive verbs, though it does occur with at least two other types of
% English intransitives. One is certain “reflexive” verbs of body care such as shower
% and bathe, which can also be construed either with adverbials of completion or of
% duration: I showered in five minutes or I showered for half an hour. The completive
% reading here implies a particular result state of the agent; I showered in five minutes
% means that it took me five minutes to attain the desired degree of cleanness, where-
% as I showered for half an hour only means that I spent half an hour standing under
% the shower. In the latter case, no result state is entailed; the implication of clean-
% ness can be cancelled by a sentence such as I showered for half an hour but I still
% couldn’t get all the dirt off my skin.
% Secondly, the alternation is shown by at least some verbs whose subject under-
% goes a process which is conceived of as typically leading to a result state, but which
% is in principle independent of this result state; that is, it may be halted before the
% state is achieved or extended past the point where the state is achieved, while still
% being essentially the same process. This is the case for the verbs cook and bake; [...]  
% What these different verbs have in common is that they all refer, when used with
% an adverb of completion, to a result state of the subject argument; that is, their sub-
% jects are affected by the action in question. Once again, then, eat here patterns with
% verbs which are characterised by affecting their subject argument; reflexive verbs
% such as shower and bathe and patient-subject verbs like (intransitive) cook and bake. [...]  
% Furthermore, the possibility of a telic reading for intransitive ‘eat’ is unusual for
% intransitive uses of verbs that may delete their objects in English; we may say Dad
% cooked dinner in half an hour or Dad cooked for half an hour but not *Dad cooked
% in half an hour; Mary sewed a dress in an hour or Mary sewed for an hour but not
% *Mary sewed in an hour.

% *Naess (2007: 79)*  
% TELICITY BOCCIATA COME FATTORE PER EAT VERBS  
% > What is relevant is that the verb has a
% telic reading even when it is used intransitively. In other words, the intransitive use
% of ‘eat’ verbs cannot be explained by appealing to the atelicity of such verbs: if ‘eat’
% can be used intransitively because it has a low value for the parameter of telicity,
% then we would expect an intransitive clause with ‘eat’ to necessarily have an atelic
% reading. But if a verb may be telic and still occur in intransitive contexts, then it can-
% not be the telic-atelic parameter which is the source of its reduced transitivity.

% *Naess (2007: 80)*  
% >  (Van Valin and LaPolla 1997 :149). It is this lack of a second macrorole which is taken to
% account for the frequently observed intransitive behaviour of ‘eat’ verbs.

% *Naess (2007: 81)*  
% >  The claim made by the analysis presented here is that it is precisely the
% affectedness of the agent which allows the object of ‘eat’ and similar verbs to be
% construed as “irrelevant”; since the effect of the act on the agent himself is the
% primary goal of the agent’s acting, the patient is of secondary interest and rela-
% tively low in distinctness, and may therefore be demoted or omitted altogether.

% *Naess (2007: 126)*
% > For most languages for which I have been able to find data, it is the case that if
% they allow indefinite object deletion with any verb, they allow it with ‘eat’ (and
% usually also ‘drink’). [...]  
% many of the languages which may omit the objects of ‘eat’ and ‘drink’ also
% include in their set of IOD verbs other verbs denoting events which affect their
% agents or their A participant. Another example given for Malayalam in Asher and
% Kumari (1997) is vaayikkuka ‘read’, another act having an effect on its agent. 

% *Naess (2007: 127-128)*  
% EFFECTED OBJECTS  
% > However, there is another type of verb which are crosslinguistically common
% in IOD constructions, which do not fall under this explanation as their agent par-
% ticipant is not typically affected, namely verbs with effected objects, discussed in
% 5.3.5. As noted there, an effected object is one that comes about as a result of the
% verbal action; it did not exist before the action began, nor does it come into exist-
% ence if the action is interrupted before it is completed. If someone is, for instance,
% writing a letter but is interrupted before they finish, we cannot say that a letter has
% been produced. [...]  
% The unique property of effected-object verbs is that
% nonreferentiality of the object is inherent to the semantics of the verb itself; that is,
% the use of the verb itself indicates that the object is nonreferential and therefore
% less prominent in the discourse context, less distinct from the general background,
% and not affected in the usual sense. Note that IOD with such verbs is most felici-
% tous in grammatical contexts presenting the act as ongoing, and consequently, the
% object as not yet completed: I am/was writing. By contrast, in perfective contexts,
% where the object must be understood as having been completed, and therefore
% referential, omission is all but impossible:??I have written,??I have baked. (Com-
% pare these to the very common perfect use of objectless eat: I have eaten; cf 6.3.3). [...]  
% The property shared by the omissible objects with affected-agent and effected-
% object verbs is a relatively low degree of distinctness. We saw in chapter 4 that Pa-
% tient participants of affected-agent verbs are not maximally distinct from the Af-
% fected Agent, and that they are omissible because the Affected Agent may itself be
% construed as the endpoint of the event, by virtue of being affected. Effected objects
% are nonaffected and nonreferential, and therefore show a low degree both of dis-
% tinguishability from the Agent and distinctness from the general background; this
% low degree of distinctness makes them highly susceptible to omission.

% *Naess (2007: 138)*  
% > In fact, a similar effect can be observed for English. Consider the verb kill, by
% any account a very highly transitive verb, with an active and, in the default inter-
% pretation at least, controlling agent, and a radical effect on its patient. A generic
% objectless clause with a human subject is not particularly good for kill: *John kills,
% meaning that he habitually kills people or animals, is unacceptable in English, al-
% though for the reasons outlined above it may be improved by the addition of a
% purpose clause: John kills for the sheer thrill of it. However, with an inanimate and
% therefore necessarily nonagentive subject a generic objectless clause is unproblem-
% atic: Smoking kills. Note that both the generic reading and the inanimate subject
% are required for the object to be omissible; *Smoking killed is unacceptable.

qui iniziava il vecchio paragrafo "eating and drinking"

% *Naess (2007: 13)*  
% KILL OR CUT, NOT EAT! discutere sul fatto che tutti i paper in genere parlano di "eat", che non è molto prototipico...  
% > Thus, for example,
% verbs like ‘kill’ or ‘cut’ are expected to be treated as transitive across languages – in-
% deed, verbs like these are frequently the basis for determining what should count
% as the basic transitive clause of a language. 

% *Naess (2011: 413)*
% > Verbs referring to acts of eating and drinking show a crosslinguistic tendency to behave in ways
% which distinguish them from other verbs in a language. Specifically, they tend to pattern like
% intransitive verbs in certain respects, even though they appear to conform to the definition of
% ‘prototypical transitive verbs’. The explanations which have been suggested for this behaviour fall
% into two main categories: those referring to telicity or Aktionsart, and those referring to the fact
% that such verbs describe acts which have ‘affected agents’, i.e. they have an effect on their agent as
% well as on their patient participant. The latter observation has further led to reexaminations of the
% notion of transitivity in general. [...]
% the notion of transitivity as a prototype concept. This idea was first articulated by Hopper and Thompson (1980)

% *Petho & Kardos (2006: 30)*  
% INVECE È CHIARO! LOCK NON È UN AFFECTED-AGENT VERB
% > It does not become clear
% why e.g. eat can be used intransitively, as opposed to lock, which requires its object to appear on
% the syntactic surface, even though its relevant selection restrictions do not seem to be any less
% specific (eat requires food as its object, whereas lock requires an object that has a lock, e.g. a car
% or a door).

% *Newman & Rice (2006: 5-6)*  
% > Van Valin and LaPolla (1997:112) explicitly remark that
% “...eat is not inherently telic, unlike kill and break; hence it must be
% analyzed as an activity verb, with an active accomplishment use”. For
% them, the ‘activity verb’ use (He ate, He ate spaghetti for ten minutes) is
% the ‘basic’ meaning of EAT.

% *Newman & Rice (2006: 14)*  
% > There is proportionately more intransitive usage with DRINK than there is
% with EAT . The difference is arguably influenced by the existence of
% specialized meanings associated with the intransitive (the ‘specific category
% indefinite’ kind of interpretation à la Huddleston and Pullum 2002: 303-
% 305 or Rice 1988). In the case of EAT the specific interpretation is ‘meal’,
% whereas with DRINK it is ‘alcoholic beverage’ (especially when consumed
% in an habitual and/or excessive manner). This use of intransitive DRINK is a
% very familiar one in casual conversation (some examples from sBNC are
% All they do in that house is drink and smoke; Because her daddy drinks in
% there in the pub...; He bought a bottle of brandy at the first liquor store he
% found and he began to drink), reflecting the prominence of alchol
% consumption as a topic of discourse. Comparing EAT and DRINK in this way
% is instructive for demonstrating the kind of variation that can exist between
% lexical items, even those which define and exhaust a class (cf. Levin 1993:
% 213-214). 

% *Naess (2011: 414)*  
% IMPORTANTISSIMO ELENCO DI CARATTERISTICHE + ESEMPI ESOTICI  
% > Types of ‘intransitive behaviour’ exhibited by eating and drinking verbs include (but
% are not restricted to) the following

% *Kardos (2010: 1)*  
% IMPORTANTISSIMO!!! SI RICOLLEGA ALL'AFFECTED-AGENT ACCOUNT (riflettere, ma penso di sì!)  
% >  it is often argued that pseudo-transitives, some core examples of which are eat
% and drink, exhibit both transitive and intransitive properties as the roles that their arguments
% play in the denoted event show overlapping properties. 


% *Naess (2011: 420)*  
% PER SPIEGARE "not clear why alcoholic drinks should be seen as ‘prototypical’; most acts of drinking by
% most people involve objects other than alcohol"  
% > An alternative explanation which has been proposed starts from the affected agent analysis
% described above, and argues that the function of object omission with verbs of eating
% and drinking is to highlight the effect of action on the agent, by removing the other
% affected argument, the object. For ‘drink’ this leads to an ‘alcohol’ reading because the
% consumption of alcohol is a culturally salient activity, and, more importantly, because
% drinking alcohol is associated with a very specific and often directly observable effect on
% the agent, that of intoxication. For ‘eat’, the intended effect of eating is said to be that of
% becoming full, and the amount of food required to achieve this is conventionally referred
% to as ‘a meal’ – hence the ‘eat a meal’ reading of objectless ‘eat’ (Nжss 2007:141–4).

% *Naess (2011: 421)*  
% VERBS OF CHANGE-OF-STATE IN MOLTI ALTRI PAPER!  
% > However, the semantic properties that set these
% two groups of verbs apart from other two-participant verbs are different: Verbs of eating
% and drinking have an effect on their agent, meaning that the object can be left
% out if the effect one wishes to focus on is that on the agent. Verbs of creation, on
% the other hand, have non-referential objects – the objects do not exist until the verbal
% action has been brought to completion – and it is this non-referentiality which
% most plausibly accounts for the omissibility of effected objects in languages like English
% (Hopper 1985; Næss 2007:127). One might propose a shared property of low
% distinctness of the object; with affected-agent verbs, the object is low in distinctness
% because it shares its defining semantic property, affectedness, with the agent, and with
% effected-object verbs it is low in distinctness because it is non-referential (Næss
% 2007:128).

% *Naess (2007: 136)*
% > The IOD construction with a purpose clause – John murdered for the money – does not necessarily have an iterative reading, 
% as pointed out above. Rather, such clauses are construable as a kind of affected-agent construction where the affected- agent
%  reading is not imposed by the semantics of the verb, but rather by the pur- pose clause. A statement of the agent’s motivation 
%  or purpose in performing an act is essentially a statement of the benefits that the agent hopes to achieve in acting; in other words
%   the intended effect of the act on the agent. Affectedness of the agent, then, is not necessarily inherent to the semantics of a specific
%   verb, but may be introduced by other elements of a clause.  

% *Naess (2007: 52)*  
% LA MAX DIST ARG HYP LA DICO IN "DIRECT CONSEQUENCES" SOPRA! citando naess 2007
% > There is considerable support in the literature for the idea that a prototypical
% agent is unaffected, although systematic empirical evidence for the relevance of
% this property has rarely been presented. For example, Langacker (1991 :238) states
% that in a canonical transitive clause, the agent “remains basically unaffected”; while
% Kittilä (2002a :237) gives as one of the defining features of a transitive event that
% “the agent is not affected”. The Maximally Distinguished Arguments Hypothesis
% places this observation in a larger theoretical context by stating that the require-
% ment that the agent remain unaffected is a function of the more basic requirement
% that the two participants of a transitive event should be maximally distinct, con-
% ceptually as well as physically.

% *Naess (2007: 52)*  
% MOLTO IMPORTANTE! DIMOSTRA CHE EAT/DRINK NON SONO VERBI TRANSITIVI PROTOTIPICI  
% > The discussion in this chapter will center round the morphosyntactic behaviour of
% verbs meaning ‘eat’ and ‘drink’ crosslinguistically. Such verbs have sometimes been
% referred to in linguistic literature as “ingestive verbs”. This term appears to originate
% in Masica (1976), where it refers to “a small set of verbs... having in common a se-
% mantic feature of taking something into the body or mind (literally or figurative-
% ly)” (Masica 1976 :46). As much of this chapter will be dedicated to demonstrating,
% these verbs often show intransitive characteristics. Masica considers them to be
% “occupying a halfway station between intransitives and transitives, since the object
% in question can frequently be dispensed with in favor of concentration on the activ-
% ity as such” (p. 48). He takes this observation to explain certain facts about the se-
% mantic behaviour of these verbs under causativisation in Hindi, cf. 4.3.4 below.

% *Naess (2007: 53)*  
% > A number of linguists have noted that acts of eating and drinking are charac-
% terised by affecting their agent. For example, Starosta (1978) characterises intran-
% sitive uses of ‘eat’ (John is eating) as having a “Patient” subject, by which he pre-
% sumably means that the subject argument is affected by the act of eating. Wierzbicka
% (1982) accounts for constructions of the type have a drink essentially by arguing
% that they describe actions undertaken by an agent for the sole purpose of achiev-
% ing an effect on himself. Nedjalkov and Jaxontov (1988) include “verbs meaning
% ‘to eat’, ‘to drink’” in the class of verbs forming “possessive resultative construc-
% tions”, which apply in cases where “the result of the action affects the underlying
% subject rather than the immediate patient of the action” (Nedjalkov and Jaxontov
% 1988 :9). Haspelmath (1994) similarly notes that verbs whose agent argument is
% “saliently affected” may form “active resultative participles”, meaning that the re-
% sultative participles describe an effect achieved on the agent rather than the patient
% argument. Among such affected-agent verbs Haspelmath counts ‘eat’ and ‘drink’ as
% well as ‘learn’, ‘see’, ‘put on’ and ‘wear’ (Haspelmath 1994 :161).

% *Naess (2007: 54)*  
% > The best-known and most frequently-discussed case of “intransitive behaviour” in
% ‘eat’ verbs is that of so-called indefinite object deletion (IOD), found in English
% and a large number of other languages. In these languages, ‘eat’ verbs exhibit a
% structurally fairly simple alternation: they can be used either with or without an
% overt direct object, without any further morphosyntactic differences in the verb or
% the clause.

\paragraph{DOBJ IS THERE IN SYNTAX}
questa teoria vede l'oggetto sintatticamente rappresentato! (recuperare altri pezzetti)\\
trovare il modo di separare questo paragrafo e quello sull'incorporation in una sezione sintassi vs semantica? o anche rappresentato non rappresentato? oppure syntactic transitivity, semantic transitivity?

% *Cummins & Roberge (2005: 47)*  
% > Based on work by Roberge
% (2002, to appear), we hold that all null objects are syntactically represented.2
% Roberge proposes a Transitivity Requirement (TR), parallel to the Extended
% Projection Principle (EPP) for subjects, whereby the direct-object position,
% complement of the verb, is given by Universal Grammar.

% *Cummins & Roberge (2004: 2)*
% > These possibilities cannot be attributed solely to lexical properties of the verb; if this
% were the case, certain verbs would always be able to appear without their objects
% regardless of the construction or discourse context, and others would never be able to
% appear without an object. As we will show, this is not the case. Rather, following Roberge
% (2003), we propose that null or implicit objects can be attributed to a Transitivity
% Requirement (TR) just as null subjects are ultimately due to the EPP. Recoverability for
% the EPP is morphologically based, as is evident in null subject languages, while
% recoverability involving the TR may also be semantically and pragmatically based; as we
% will show below, such recovery may be based on information derived from the verb's
% lexical semantics and Generalised Conversational Implicatures (formalised as in Levinson
% 2000) involved in the interpretation of reduced nominal forms. The factors that contribute
% to licensing superficial intransitivity—the absence of an overt object—may include lexical
% semantics, functional elements, discourse factors, and trans-clausal structural elements.
% This view is supported by a comparative study of null object possibilities in French and
% English. [...]  
% The concept of transitivity has been interpreted as a continuum in certain works, and a
% distinction has been proposed between syntactic transitivity and semantic transitivity; see,
% among many others, Blinkenberg (1960), Desclés (1998), Hopper et Thompson (1980),
% Lazard (1994). Surprisingly little is ever said about the object position itself. The
% hypothesis in Roberge (2003) is that there exists a Transitivity Requirement (TR),
% whereby an object position is always included in VP, independently of the lexical choice
% of V. The empirical motivation of this hypothesis is the well documented evidence (see in
% particular Blinkenberg (1960), Larjavaara (2000)) that any “transitive” verb has the
% potential to appear without a direct object and any “unergative” verb has the potential to
% appear with a direct object. To account for these facts, there must be a mechanism to
% generate the direct-object position, either optionally or obligatorily. The TR represents the
% second, more restrictive, possibility and conveys the concept of transitivity as a property
% of the predicate (the VP), rather than as a property of the lexical content of V. The TR is
% the internal-argument counterpart to the EPP.

% *Cummins & Roberge (2004: 3)*
% > For the purpose of our
% discussion, we define unexpressed objects interpretatively: there is an x such that x is (1)
% phonologically null, (2) involved in the event denoted by the VP, and (3) not an external
% argument. [...]  
% Two recent studies—Larjavaara (2000) on French and García Velasco & Portero
% Muñoz (2002) on English. [...]  
% The two agree that indefinite or generic null objects do not have a
% contextually available referent. GP point out that generic null objects can give rise to an
% activity rather than an accomplishment reading of the verb; L notes that null objects can
% focus attention on the activity. Both point out that the lexical characteristics of the verb
% can help to identify the referent of the null object. [...]  
% In a third study, Goldberg (2001) investigates unexpressed objects of causative verbs
% (those that entail a change of state in the patient argument) in English. She concludes that
% the option of leaving these arguments unexpressed depends largely on factors relating to
% information structure: the unexpressed object is typically neither topical nor focal, and the
% verb is emphasized somehow, by being iterative or generic, by being contrasted with
% another verb, or by having a narrow focus.

% *Cummins & Roberge (2004: 4)*
% > All of these authors implicitly or explicitly adopt the position that the missing
% argument is not syntactically represented: syntactically the verb is intransitive. In a
% generative framework, this position finds a counterpart in Rizzi (1986:509-510), who
% proposes that both the arbitrary third-person human interpretation, meaning “people in
% general” or “some people”, and the prototypical-object interpretation, where the verb's
% lexical semantics identify the object, are available lexically to saturate the argument's
% theta role and block projection. Thus, the verbs are intransitive in syntax. The absence of a
% syntactic object explains why, in Rizzi’s account, the type of sentence exemplified in (13)
% is impossible in English: there is no object that can bind the anaphor or be modified by the
% adjective. However, such sentences are grammatical in Romance; hence several accounts
% (Rizzi 1986; Authier 1989; Roberge 1991) posit a syntactically present null object.  
% (13)a.b.Ce gouvernement rend __ malheureux.  
% *“This government makes __ unhappy.”  
% Une bonne bière reconcilie __ avec soi-même.  
% *“A good beer reconciles __ with oneself.”  
% Under the TR, the object position is projected and the verb remains transitive in syntax
% in both English and French. Although we do not find sentences like those in (13) in
% English (as shown by the ungrammaticality of the glosses), there is nonetheless evidence
% that a null object has an effect on syntax in both English and French.3 

% *Dvorak (2017: 1)*  
% > Non-overt direct objects with an existential interpretation, as in (1), have been the topic of
% linguistic discussions since Chomsky introduced the transformational rule of ‘Unspecified
% object deletion’ in 1962. In the English-oriented literature, they are usually assumed to be
% represented lexically, either in the form of de-transitivizing rules that operate on individual
% predicates, see (2a), or as two separate predicates, transitive and intransitive, linked by
% a predicate-specific meaning postulate, see (2b). Alternatively, Cote (1996)  enriches the
% lexical conceptual structure (LCS) of a given verb by marking the relevant argument as
% unsubscripted zero [...]  
% More recently, this view has been challenged by Alexiadou et al. (2014), who argue for a
% syntactic re-interpretation of Bresnan’s rule of Existential Closure (EC) in (2a), such that it
% applies to any predicate with an unsaturated internal argument: JECK = λ fhe,stiλ es ∃x[f(x)(e)]
% (see Babko-Malaya 1999 and Martı́ 2011 for related proposals).

% *Lorenzetti (2008: 62)*  
% > Early accounts of the phenomenon proposed the elimination of the object from the syntax
% via “object deletion transformation” [Katz and Postal 1964, Browne 1971, Allerton 1975],
% while later works introduced the view that implicit object constructions have aspectual
% constraints [Mittwoch 1982] and that omitted objects are frequently typical, inferable or
% partially specified by the semantics of the verb [Lehrer 1970, Rice 1988].


\section{A working definition of "indefinite object drop"} \labsec{theory_workingdef}

citare Medina e tutta la teoria possibile che la conforti
è una definizione pre-teoretica che prescinde dalle proiezioni sintattiche?
QUI METTERE IL MIO DIAGRAMMA DI VENN CONCENTRICO

% hopper thompson 1980: 251 (abstract)
% TORNA BENISSIMO COL MIO DIAGRAMMA DI VENN
% The grammatical and
% semantic prominence of Transitivityis shown to derive from its characteristicdis-
% coursefunction:high Transitivityis correlatedwithforegrounding,and low Transitivity
% with backgrounding

\begin{figure}[htb]
\caption{CAPTION.}
% lexical and aspectual factors
\labfig{objectdrop_venndiagram}
    \begin{tikzpicture}
\coordinate (O) at (0,0);
\foreach \j in {1,...,3} \draw (O) circle (3.5-\j);
\foreach \k/\text in {0/extra-linguistic context,1/intra-linguistic context,2/lexicon} \draw[decoration={text along path,reverse path,text align={align=center},text={\text}},decorate] (2.6-\k,0) arc (0:180:2.6-\k);
    \end{tikzpicture}
\end{figure}

% *Naess (2007: 124-125)*
% > In this chapter, the term “indefinite object deletion” (IOD) will be used to describe
% the situation where a verb which allows an argument which syntactically must be
% characterised as a direct object is employed without any overt direct-object argu-
% ment. We will restrict the discussion to cases of so-called context-independent
% object deletion.
% Many languages allow omission of objects whose reference is expected to be
% retrievable from context. Such context-dependent object deletion may take place
% when the object in question has already been mentioned in the previous discourse,
% or where the general context provides sufficient clues as to the identity [...]  
% By contrast, context-independent object deletion does not depend on the ref-
% erent of the missing object being retrievable or identifiable by the hearer. In order
% to correctly understand a sentence such as He is eating, it is not necessary to be
% able to infer what is being eaten; in fact, there is no expectation whatsoever that
% the referent of the object should be identifiable. This is not to say that the use of the
% verb eat without an object does not imply that an object is present; the act of eating
% unquestionably presupposes that something is eaten. However, it is only this pres-
% ence of an object in general that must be inferred by the hearer in such cases: the
% hearer knows that something is being eaten, but is not expected to be able to iden-
% tify which specific edible object is being referred to. [...]  
% Some approaches assume a further division between contexts where objects
% are omitted because they are obvious and contexts where they are omitted because
% they are indefinite. For example, Heath (1976a) claims that in the sentence He
% drinks, “the T[ransitive] O[bject] is deleted because it is obvious in the context,
% rather than because of indefiniteness. On the other hand, in the type Speed kills the
% deleted TO is indefinite rather than obvious”.
% It is not clear, however, how the object of He drinks can be “obvious in the
% context” when the sentence is in fact presented without any context whatsoever.
% The only way to interpret this would seem to be that the object is “obvious in the
% context of deletion”, that is, it is obvious because it is deleted; the very fact that no
% object is present allows us to assign it the “obvious” interpretation. But this is a
% circular argument: the object is deleted because it is obvious, and it is obvious be-
% cause it is deleted. No such difference between “obvious” and “indefinite” deleted
% objects will be assumed here. 

% *Haegeman (1987: 236)*  
% > The (b)-examples illustrate what has come to be known as indefinite
% object drop: a transitive verb lacks an object, but an indefinite object
% is understood.
%  Rizzi ( 1986) argues convincingly that these cases of
% indefinite object drop in English are lexically governed. While eat, for
% instance, allows object drop, the semantically closely related devour
% does not:

% *Yasutake (1987: 46)*  
% calls them "detransitive verbs"

% *Naess (2007: 55)*  
% >  However, for English and many other languages,
% indefinite object deletion only applies to a small subset of verbs, which as a result
% are frequently labelled “pseudo-intransitive”, “labile”, or “(S/A) ambitransitive” (cf.
% chapter 6).

% *Glass (2013: 1)*  
% > “Implicit object” (IO) is a name for what happens when a verb we normally consider transitive
% appears without an object

% *Petho & Kardos (2006: 29)*  
% >  implicit object arguments, also referred
% to in the literature as null complements or understood arguments

% *Glass (2020: 1)*
% > Other words for this phenomenon include object drop,  unspecified object,  null object,  null argument,  null complement, null instantiation, diathesis alternation, implicit object  

% *Gillon (2012: 314)*
% > Fodor and Fodor (1980) seem to have been the first authors to have brought to the
% attention of linguists that implicit indefinite objects pose a problem for grammatical
% theory. Though Fodor and Fodor did not put it this way, we can succinctly formulate
% the problem as a dilemma. Either there is only one verb to read or there are two.

% *Glass (2020: 1)*
% > It is a longstanding question in lexical semantics which normally-transitive verbs in Englishcan omit their objects to describe an event with an unexpressed theme, which cannot, and why1.Even near-synonyms differ in how natural they sound with their objects omitted (1) (Fillmore1986, Rice 1988, Mittwoch 2005, Gillon 2012), making it difficult to explain omission in termsof meaning, and leading some researchers to characterize this phenomenon as partly or fullyarbitrary (Fillmore 1986, Jackendoff 2002: 134, Ruppenhofer 2004, Gillon 2012, Ruppenhofer& Michaelis 2014).
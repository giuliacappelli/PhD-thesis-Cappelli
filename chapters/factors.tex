% \setchapterimage[6.5cm]{seaside}
\setchapterpreamble[u]{\margintoc}
\chapter{Factors allowing indefinite object drop}
\labch{factors}

In \nrefch{objectdrop} I presented indefinite object drop as a marked construction deviating from the transitive prototype. Indefinite object drop has been argued to be binarily distinct from definite object drop, to be possible with some or all transitive verbs, to imply an unsaid \textit{something} or a prototypical object different for each verb, to make the verbs participating in this construction have an additional, intransitive entry in the lexicon or just the transitive one. Several answers were proposed to these conundrums in the literature, and I provided my own perspective.\\
In this Chapter, I am going to focus on the main intra-linguistic (semantic, aspectual, and pragmatic) factors allowing a transitive verb to participate in the indefinite object drop construction, based on literature on this topic. I will return to the subject of recoverability, manner specification, telicity, perfectivity, and iterativity in \nrefch{predictors}, where I will define them as predictors of indefinite object drop in my Stochastic Optimality Theoretic model.


\section{Semantic factors} \labsec{semanticfactors}

\subsection{Recoverability} \labsec{recoverability}

\paragraph{An intuitive notion of recoverability}

As observed several times in \refch{objectdrop}, recoverability is the \textit{sine qua non} of object omission. \textcite{Cote1996} even went as far as to identify it as "the only absolute constraint on null arguments". Intuitively, this notion can be used to tell apart definite implicit objects (whose meaning is recoverable from context, be it extra- or intra-linguistic) and implicit indefinite objects (whose meaning is recoverable from the semantics of the verb itself). Even authors who cautiously \parencite{Resnik1993, Resnik1996, OlsenResnik1997} or openly \parencite{Glass2013, Glass2020, glass2022english} reject a distinction between definite and indefinite object drop still maintain a certain notion of recoverability as a fundamental requirement for object drop. Let us discuss this in some more detail.\\
Decades of literature on the matter teem with pre-theoretical definitions of object recoverability. The oldest is in \textcite[321]{Jespersen1927}, where the author argues that "the omission of an obvious object probably produces more intransitive uses of transitive verbs than anything else". Later on, \textcite[105]{Ohlander1943} observed that object-less utterances "may appear complete enough" by virtue of the fact that "the element to be understood or supplied is so self-evident that the gap is mentally filled in by the audience more or less unreflectingly". Like many others, Ohlander makes reference to the notion of recoverability without using this exact wording. Similarly, \textcite{HickmanEtAl2016} resort to the idea of "conceptual defaultness", i.e., the property of unmentioned direct objects whose omission does not depend on an informational failure on the part of the speaker/writer \textemdash on the contrary, defaults are omitted to comply with the Gricean maxim of quantity, since their very mention "would be unnecessary and, perhaps, awkward" \parencite[516]{HickmanEtAl2016}.\\
Interestingly, recoverability appears to be such an intuitive notion as to become a major determinant of argument omission (not limited to indefinite object drop) in the early stage of grammar acquisition crosslinguistically, even in context and languages where adult grammar would normally prohibit it \parencite{allen2000discourse, RatitamkulEtAl2004, Medina2007, sopata2016null, Rasetti2003, PerezLerouxEtAl2011, PerezLerouxEtAl2013, Perez-LerouxEtAl2018, OGradyEtAl2008, Ingham1993}. Young children are shown to omit objects (and other arguments) of verbs they are exposed to, both in transitive and intransitive utterances, if extra-linguistic context makes them sufficiently recoverable.


\paragraph{Between lexical and contextual recoverability}

In \posscite{Kardos2010} words, omitted objects are recoverable "either through lexical stereotypes or based on the context" \parencite[7]{Kardos2010}. In a sense, it is not even necessary to postulate a binary distinction between the two facilitators of recoverability, since "lexical stereotypes" (i.e., the selectional preferences of a verb) descend from world knowledge, situational context is where meaningful conversations happen, and narrower context enables more object omissions without contradicting world knowledge (more on this in \refch{objectdrop}, and in \textcite{Glass2013, Glass2020, glass2022english}, where recoverability is intended as a matter of degree). A prime example of this is \ref{berghplay}, taken by \textcite[24]{BerghOhlander2016}, whose full interpretation depends on additional context. Indeed, this sentence in isolation has no single interpretation. \textit{What} did they play beautifully? Was it an instrument or a game? And what kind, exactly? However, while context would make it possible to know the referent of this implicit indefinite object, the semantics of the verb (in particular, its selectional restrictions) still provide us relevant information without the need for additional context. Indeed, we know that they played either a game, or a musical instrument, or a role in a theater piece. This would not be possible, for instance, with a selectionally un-restricted (or, better, very loosely restricted) verb such as \textit{to make}.

\ex. \label{berghplay} They played beautifully.

This also holds true for verbs with much stricter selectional preferences than \textit{to play}, e.g., the verb \textit{to eat}, as discussed in \refch{objectdrop}. As noted by \textcite[149]{Cote1996} among others, intransitive \textit{to eat} tends to refer to a meal (which is the prototypical item humans eat), but it does not have to. In context-rich utterances, the actual referent may be different, depending on "the underlying context and intentional structure of the discourse structure at the time of utterance". Crucially, context may make it obvious that the omitted Patient is some specific kind of edible item (e.g., pasta, hamburgers, or even something as extravagant as Hawaiian pizza), but it cannot make it deviate from the basic selectional preferences of the verb\sidenote{Unless, of course, the verb takes a specialized meaning in a specific sub-genre, e.g., \textit{to eat} in a game of chess would mean \textit{to capture enemy pawns}.} \textemdash the omitted Patient has to be something edible.\\
Semantically-licensed recoverability also interacts with world knowledge, specifically with societal norms, in \textcite{Goldberg2005}. In this case, the author argues that politeness is a driving factor in the omission of direct objects occurring with verbs of bodily emission, since they are very imageable (hence, recoverable), but also very taboo in usual social contexts. On the other hand, it would be easy to imagine that in contexts where such verbs are not taboo (e.g., in medical or adult-only literature, as a future corpus study could ascertain) the main driver of object omission would be contextual inferability, rather than misguided concerns for politeness. Either way, context and world knowledge (about the verb itself, but also about proper customs) comply with verbal semantics.\\
Consistently with all the previous observations on recoverability, \textcite{Mittwoch2005} and \textcite{Glass2013} also observe that the referent of implicit indefinite objects corresponds to the literal meaning of the verb, rather than to metaphorical or idiomatic meanings. For instance, intransitive \textit{to read} refers to "written or printed material rather than, say, the stars or coffee grounds" \parencite[2]{Mittwoch2005}. Likewise, when a verb has strict selectional preferences (e.g., \textit{to eat} selects for edible items) and one of its near-synonyms has broader preferences\sidenote{An in-depth discussion of such verb pairs will be tackled in \refsec{mannerspec}.} (e.g., \textit{to devour} selects for edibles, but also for metaphorical items such as books), direct objects are much more likely to be dropped with the former than with the latter \parencite[5]{Glass2013}. Clearly, this mechanism is in place to maximize recoverability, since the literal selectional preferences of a verb are known and predictable (based on lexical semantics and world knowledge), while its metaphorical or idiomatic behavior is largely arbitrary and unpredictable. More in general, the more an object is semantically dependent from a verb, the more likely it is to be omitted \parencite[203-204]{Rice1988}.


\paragraph{Semantic selectivity as a proxy to recoverability}

Let us now look in more detail into semantic selectivity as the main verb-internal, context-independent factor allowing for object recoverability. So far in this Section, I made the case that knowing the specific type of objects a verb favors in its selectional preferences is the first step towards object recoverability and, consequently, felicitous object omission. World knowledge and context shape the way we process object-less transitive verbs when accessing their selectional preferences. However, as hinted before, there is a close correlation between indefinite object drop and the breadth of a given verb's selectional preferences, regardless of the actual items or family of items it tends to occur with \parencite{Garcia-VelascoMunoz2002, Liu2008, Glass2020, Medina2007, MaoueneEtAl2011, OlsenResnik1997, Resnik1993, Resnik1996}. The intuition behind the use of a verb's selectional preferences as a means to gauge the recoverability of its objects stems from the observation that implicit indefinite objects "are clearly understood because they are inferred from a very narrow, if not exclusive, range of possibilities" \parencite[4]{Garcia-VelascoMunoz2002}. This is the reason why native speakers of English are more likely to find indefinite object drop grammatical with \textit{to read} than with \textit{to know}, since there are way fewer readable than knowable things in our lives \parencite[302]{Liu2008}.\\
\textcite{Resnik1993, Resnik1996} was the first to provide a more data-grounded definition of recoverability, by means of a computational model of a verb's selectional preferences. In particular, his Selectional Preference Strength taxonomy-based measure is shown to be inversely proportional to the semantic narrowness of a verb's selectional preferences, so that a verb will receive a higher score if its direct objects are semantically similar (e.g., \textit{to eat, to read}), and a lower score if they are semantically different (e.g., \textit{to make, to know}). I will present the mathematical details of Resnik's measure, discuss the implications of such an approach, and propose my own distributional semantics-based alternatives in \refsec{predictor_sps}, where I expand upon both my Preference In Selection of Arguments (PISA) computational measure presented in \textcite{CappelliLenciPISA} and on my behavioral variant of Computational PISA, inspired by \posscite{Medina2007} Object Similarity measure. An important implementation-related aspect to note here, common to all these measures of semantic selectivity used as proxies to object recoverability (Resnik's SPS, Medina's OS, my own PISAs), is that recoverability is modeled as being gradient, and these models capture the semantic narrowness/breadth of the semantic categories\sidenote{This is less true for semantic similarity measures based on behavioral judgments, such as Object Similarity and Behavioral PISA, but still not completely off the mark. Please refer to \refsec{predictor_sps} for a full account of such measures.} the potential direct objects of a verb belong to, rather than focusing on the specific objects themselves.\\
Most importantly, with his computational experiment \textcite[88]{Resnik1993} could conclude that recoverability, as quantified via gradient semantic selectivity, is a necessary (albeit insufficient) condition for object omission. This conclusion, consistent with pre-theoretical intuitions about recoverability and previous theory-informed, non-experimental statements about the role of selectional preferences in determining object recoverability, stems from the observation that object-less transitive verbs never receive low semantic selectivity scores\sidenote{However, some verbs high in semantic selectivity fail to license indefinite object drop. Resnik explains this apparent failure of his measure with reference to the aspectual properties of such verbs. I will come back to this in \refsec{aspectualfactors} and \refch{predictors}.} in Resnik's experiment.


\subsection{Agent affectedness} \labsec{agentaffect}

Back on \refpage{affectedagent}, I argued with plenty of references to relevant literature (first and foremost, \textcite{Naess2007}) that verbs whose Agent is in some ways affected by the action (e.g., \textit{to eat, to learn}) described by the verb tend to be more likely to participate in the implicit indefinite object construction than unaffected-Agent transitive verbs (e.g., \textit{to kill, to break}). This is a direct consequence of the need for the arguments of prototypical transitive verbs to be maximally distinct from a semantic point of view, captured in \posscite{HopperThompson1980} parameters H (agency) and I (affectedness of the object)\sidenote{Refer back to \reftab{ht1980_parameters}.}, and later on in \posscite{Naess2007} Maximally Distinguished Arguments Hypothesis. \textcite[335]{malchukov2006transitivity} captures the same intuition in his Relevance Principle, stating that \posscite{HopperThompson1980} transitivity parameters have to be marked on the relevant constituent (e.g., volitionality on the Agent, affectedness on the Patient) in prototypical transitive clauses.\\
I will not go again over the affected-Agent account of object drop, since I already discussed it in \refch{objectdrop}. For the purposes of the review of the main factors facilitating indefinite object drop provided in this Chapter, I will just mention that Agent affectedness can manifest in two ways. One is inherent to the semantics of the verb, as it happens with ingestion verbs such as \textit{to eat}, \textit{to drink}, and, in a sense, \textit{to learn}. The other is instead context-dependent, as in \ref{agentaffect1}, where the verb \textit{to murder} gets an affected-Agent interpretation and thus participates in felicitous object drop due to the purpose clause \textit{for the money}.

\ex. \label{agentaffect1} John murdered for the money. \hfill \parencite[136]{Naess2007}

It is important to note, as seen before in this Chapter and \refch{objectdrop}, that the context enables verb behavior that is already possible, virtually, thanks to the semantics of the verb itself. After all, does a murder not affect the murderer even without mention of the cause? Stating it explicitly puts the focus on the motive, putting in the background both the Patient (which was already backgrounded, due to being unmentioned) and the murdering activity itself (which would be the focus of cause-less \textit{John murdered}). Crucially, I would like to point out that it is not possible to use purpose clauses to induce an affected-Agent reading on object-less transitive verbs lacking this possibility in their semantics, e.g., the verb \textit{to build} in \ref{agentaffect2}.

\ex. \label{agentaffect2} *John built for the money.

Indeed, \ref{agentaffect2} would only be considered grammatically acceptable provided it is inserted in a larger context (just like plain \textit{*John built}). However, one could object to this account of the verb \textit{to build} making reference to the effected-Patient account I discussed in \refch{objectdrop} together with the affected-Agent account \textemdash the object of \textit{to build} comes into existence via the act itself of building it, unlike the Patient of \textit{to murder} (which exists before the murder, and ceases to do so due to it). Why does the verb \textit{to build} not allow for indefinite object drop, even though it is a handbook effected-Patient verb just like \textit{to bake}? Based on \textcite[512]{Goldberg2001} and \textcite[139]{Naess2007}, \textit{to bake} and \textit{to build} are actually more different than it would seem at first glance, since verbs like the latter (e.g., \textit{to break}) refer to events whose interpretation strictly depends on the Patient itself. In other words, while it is possible to imagine a baking event without having a precise baked good\sidenote{A baked good which, we will remember from \refch{objectdrop}, is much more likely to be bread or some pastry rather than rotisserie chicken.} in mind, it is impossible to picture a breaking or building event without having a precise broken or built object in mind. Without going into idiom territory, where one could "break a bank note" to get change or "build one's hope", it is clear that the act of breaking a glass is quite different from the act of breaking a leg, just as building a sand castle is quite different from building an airplane. Thus, these examples go to show that recoverability (introduced in \refsec{recoverability}) is the preminent factor in determining object drop, and neither Agent affectedness nor Patient effectedness cannot overpower it.


\subsection{Manner specification} \labsec{mannerspec}

\paragraph{Introduction}
Manner specification is a tricky semantic predictor of indefinite object drop to define, due to the different interpretations the concept of "manner" received\sidenote{For an extensive discussion on the concept of "manner", touching several aspects that are beyond the scope of this dissertation, refer to \textcite{Stosic2019, stosic2020defining}.} from different authors. This word is used, fundamentally, in two different ways:
\begin{itemize}
    \item to refer to "semantically marked" counterparts of "semantically neutral" verbs, e.g., \textit{to devour, to nibble} with respect to \textit{to eat} \parencite{Naess2007, FellbaumKegl1989taxonomic, Rice1988};
    \item in contrast with "result", to separate "manner" activity verbs such as \textit{to sweep} from "result" causative verbs such as \textit{to break}\sidenote{The concept of "manner", here discussed in relation to transitive verbs, is also central in studies about motion verbs \parencite{Iwata2002, CennamoLenci2019, BeaversEtAl2010}.} \parencite{BeaversKoontzGarboden2012, BeaversKoontzGarboden2017, Beavers2013, RappaportLevin1998building, RappaportHovavLevin2005, LevinRappaportHovav2008, RappaportHovavLevin2010, Melchin2019}.
\end{itemize}

Keeping these two senses apart is important to avoid drawing forced conclusions about the nature of verbs exhibiting "manner" components in their meanings. \textcite[7]{Garcia-VelascoMunoz2002} run exactly into this problem when they argue that "manner-of-ingesting verbs may be the exception to the rule", namely, the fact that manner-of-action verbs allow for object drop in \textcite{RappaportLevin1998building} while result verbs do not. Indeed, they acknowledge that "both \textcite{Rice1988} and \textcite{FellbaumKegl1989taxonomic} suggest that the presence of a manner component in manner-of-eating verbs accounts for the impossibility of omitting the object", but they fail to recognize that these authors are using "manner" in a very different sense from Rappaport Hovav and Levin. In a way, these two senses are so different to become almost opposites, given that manner-specified verbs being semantically marked counterparts of other verbs would be the least manner-y of all in the account interpreting "manner" as opposed to "result", since they also encode a result component (as I will show in this Section).\\ % manner-specified-as-semantically-marked verbs would be the least manner-y of all in the manner-as-opposed-to-result account, since they also encode a result component (as I will show in this Section).\\
In this thesis, as I will also argue in \refsec{predictor_mannspec}, I am only interested in the first sense of the word "manner". However, since the two senses overlap in significant ways, despite their fundamental difference, I will also comment now the second sense in some detail.

\paragraph{"Manner" as opposed to "result"} \labpage{mannerresultrappaport}
With respect to indefinite object drop, \textcite{RappaportLevin1998building, RappaportHovavLevin2005, LevinRappaportHovav2008, RappaportHovavLevin2010} argue that verbs expressing manner in their meaning, such as \textit{to eat}, are much more likely to allow for object drop than verbs expressing result, such as \textit{to devour}. In particular, \textit{to devour} is considered a result verb because it entails complete consumption of the Patient\sidenote{This analysis would have manner specification be collinear, or at least highly correlated, with telicity as \textcite{Olsen1997} (whose account is used in the experimental setting by \textcite{Medina2007}, which is also mine) intends it. In \refsec{predictor_mannspec} I will demonstrate that this is not the case at all.} by the Agent, unlike \textit{to eat}, at least in an unmarked, uninterrupted scenario \parencite{Melchin2019, smollett2005quantized, pinon2008aspectual}.\\
This idea, which \textcite{Goldberg2001} and \textcite{Onozuka2007} oppose on the basis of the aforementioned Principle of Omission under Low Discourse Prominence, is famously exemplified by \textcite{RappaportLevin1998building} with the examples in \ref{resultmanner}. The rationale behind this account is that result verbs specify scalar change (see \ref{resultmanner2}), while manner verbs specify non-scalar change (see \ref{resultmanner1}). Crucially, the entity changing along the scale specified by result verbs (i.e., the Patient object) is argued to be ungrammatical to omit, giving rise to a test for result-lexicalization used by \textcite{BeaversKoontzGarboden2012} and \textcite{Rissman2016}.

\ex. \label{resultmanner} \a. \label{resultmanner1} Phil swept.
\b. \label{resultmanner2} *Tracy broke.

In such an account, manner and result are to be considered complementary, in that a verb can only lexicalize one of them. However, \textcite{LevinRappaportHovav2008} also observe that there is some understood manner component in many result verbs (e.g., the result of \textit{to clean}\sidenote{\textcite[52]{Melchin2019} actually argues that "dynamic verbs" such as \textit{to clean} are specified for neither manner nor result. However, delving into this debate would not bring my own argumentation further.} is achieved by acting in a specific manner), and likewise, some understood result component in many manner verbs (e.g., \textit{to scrub} is a manner of cleaning, that will likely generate cleanliness of a surface as a result). This particular observation serves to bridge the gap between this account, where lexicalized manner leads to felicitous object drop, to the other account, where an overt, specified manner component blocks object drop. \textcite[5]{BeaversKoontzGarboden2012} even make the case that so-called "poison verbs", a sub-class of manner-of-killing verbs identified by \textcite[230-233]{Levin1993} in opposition to "murder verbs", actually entail both manner and result, \textit{contra} \textcite{LevinRappaportHovav2008, RappaportLevin1998building}. Crucially, \textcite[71, 89]{Melchin2019} adds to this by arguing that also \textit{to devour} entails both manner and result, because the Agent acting in a very specific manner brings forth the "manner" interpretation, while the "scalar change affecting another participant" brings forth the "result" interpretation (which was also in Levin and Rappaport Hovav's original proposal).

\paragraph{Introduction to "manner" as "semantic narrowness"}
The other interpretation of the concept of "manner", which is the one I employ in my experiments (\refch{predictors} and \refch{judgments}) and probabilistic model of object drop (\refch{results} and \refch{model}), is offered by \textcite{Rice1988}, \textcite{FellbaumKegl1989taxonomic}, and \textcite{Naess2007}, among others. These authors argue that the impossibility of omitting the object with verbs like \textit{to devour} is explained by the presence of a manner component in their meaning (see also \textcite{Garcia-VelascoMunoz2002} for further considerations). In other words, while \textit{to eat} (which \textcite{Rice1988} calls a "semantically neutral" verb) is a base verb referring to a general activity, \textit{to devour} (which \textcite{Rice1988} calls an "action-plus-manner" verb) has an additional manner specification in that it refers to a particular \textit{manner} of eating. Most importantly, \textcite[49-50]{Melchin2019} shows that this distinction holds crosslinguistically with examples in French, Dutch, and Arabic. The same also goes, for instance, for \textit{to guzzle, to chug} with respect to basic \textit{to drink}, and moving from transitive to motion verbs, for \textit{to saunter, to stride} with respect to basic \textit{to walk}.

\paragraph{Links between manner specification and Agent affectedness}
% \paragraph{Effects of manner specification as a by-product of Agent affectedness}
\textcite[139]{Naess2007} provides an intriguing link between manner specification and the affected-Agent account discussed in \refch{objectdrop} and \refsec{agentaffect}. In particular, she observes that indefinite object drop is infelicitous with manner-specified verbs because they typically refer to the way in which the Patient (crucially, not the Agent) is affected, in true prototypical transitive behavior. However, manner specification in the verb root stops being an obstacle to object drop if proper context is provided to imply Agent affectedness, as in her example in \ref{affectedmanner}.

\ex. \label{affectedmanner} The dinner was delicious, but Jane had no appetite and only nibbled.

Bringing Agent affectedness into the equation can also solve a decades-old conundrum by \textcite{FellbaumKegl1989taxonomic}. Why do \textit{to mush, to nosh, to graze} allow for implicit indefinite objects while \textit{to gobble, to gulp, to devour} do not, despite them all being manner-specified troponyms\sidenote{I am using terminology from \textcite{FellbaumKegl1989taxonomic}. Troponymy is a relation among verbs akin to what hyponymy is for nouns, "although the resulting hierarchies are much shallower" \parencite{Miller1995}.} of \textit{to eat}? Based on everything I observed about manner specification so far, both the first and the second group of verbs should block indefinite object drop on the basis of their manner component. In their taxonomic account, \textcite{FellbaumKegl1989taxonomic} explain this issue by positing two lexical entries for \textit{to eat}\sidenote{Refer back to \refsec{theory_entries} for a discussion of whether or not to have two separate lexical entries for transitive verbs used transitively and intransitively.}, one meaning roughly "to eat a meal" and another meaning "to ingest food". The first entry would be intransitive, and its manner-specified troponyms (\textit{to mush, to nosh, to graze}) are too. The second entry would instead be strictly transitive, and its manner-specified troponyms (\textit{to gobble, to gulp, to devour}) are too. As I argued in \refsec{theory_entries}, positing two separate lexical entries does little more than restate the problem, without actually providing substantial explanatory power to the discussion. I would instead explain the difference in transitivity between these two groups of manner-specified troponyms of \textit{to eat} with reference to the affected-Agent analysis. In particular, verbs like \textit{to mush, to nosh, to graze} are activity verbs with a clear focus on the way the action affects the Agent (just like plain \textit{to eat}), and indeed they do allow for their object to be dropped. On the contrary, verbs like \textit{to gobble, to gulp, to devour} tend to highlight the affectedness of the Patient, making it necessary to express it overtly with a direct object in the syntax.\\
A link between manner specification and Agent affectedness also emerges from \posscite{lemmens2006} corpus analysis of verbs of killing in English, such as \textit{to kill, to murder, to execute, to assassinate, to massacre}. He finds that, while \textit{to kill} is not unlikely to be used intransitively in several corpora, the same does not hold for the other verbs of killing, which never occur with null objects. He attempts an explanation by observing that manner-specified verbs of killing may have "a stronger Patient-orientation, as they incorporate a more salient reference to Patients that are considered important in some socio-economical context (\textit{to assassinate}) or to a high number of Patients (\textit{to massacre})". Flipping this perspective, \textit{to kill} could be argued to be more likely to license object drop because it projects Agent affectedness more strongly than Patient affectedness, if compared to the other verbs of killing. Even disregarding this possible explanation of \posscite{lemmens2006} findings, they still confirm the relevance of manner specification in the implicit indefinite object construction.

\paragraph{Recoverability explains failures of a manner-based account}
Interestingly, \textcite[207]{Rice1988} notes in passing that "verbs that are very neutral, but that furthermore sustain a wide variety of complements, tend always to require objects", considering the ungrammaticality of intransitive \textit{to love} as an example. Once again, object recoverability (as an effect of a transitive verb's semantic selectivity) is shown to be prominent with respect to other drivers of indefinite object drop. Recoverability also accounts for some examples \textcite[134]{Jackendoff2003} cites as "immediate counterexamples" to the idea that manner specification is relevant for the issue of argument drop, such as \textit{serve/give the food to Sally} as opposed to \textit{serve/*give the food}, and \textit{insert/put the letter in the slot} as opposed to \textit{insert/*put the letter}. These examples feature the omission of a Recipient/Goal instead of the omission of a Patient (which I am focusing on), but the principle holds in this case too. Indeed, \textit{to serve} is a manner-specified troponym of \textit{to give} ("a more specific form of giving", in Jackendoff's words) and \textit{to insert} is in the same relation with respect to \textit{to put}, but their hypernyms select for a much wider range of arguments, making it much more difficult for a speaker to recover them if they are unexpressed. Jackendoff also brings two other examples to his argumentations, this time relative to direct objects, i.e., \textit{juggle (six balls), flirt (with Kim)}. The two verbs are shown to be grammatical both in their transitive and in their intransitive use. Since the author calls them "highly specific verbs", it seems that he is conflating both semantic selectivity and manner specification into the same label of "semantic specificity". I argue that these two properties have to be kept separated instead, and that whenever manner specified verbs allow for their object (or other internal argument) to be dropped, they do so by virtue of the high recoverability of the intended object/argument (stemming from the high semantic selectivity of the verb).


\section{Aspectual factors} \labsec{aspectualfactors}

\subsection{Telicity} \labsec{telicity}

\paragraph{Vendler's aspectual classes}
Telicity, as a component of "lexical aspect", is a well-known predictor of indefinite object drop. As shown in \refsec{theory_transitivity}, \textcite{HopperThompson1980} included it in their ten-parameter account of transitivity as a prototype concept, associating telic aspect with high transitivity and atelic aspect with low transitivity. Telicity is one of the facets of so-called "lexical aspect", first organized by \textcite{Vendler1957} into four \textit{Aktionsarten} ("types of action") as in \reftab{vendler_classes}, to which \textcite{smith1991parameter} then added a fifth class of "semelfactives"\sidenote{The term was actually coined by \textcite[42]{comrie1976aspect} to refer to "a situation that takes place once and once only", such as "one single cough" abiding by Latin etymology (\textit{semel}, 'once'). He observes in a footnote that in Slavic linguistics the equivalent of this term was used to refer both to proper semelfactives and to "clearly iterative" utterances such as \textit{He coughed five times}. The use of the label "semelfactive" applied to such iterative sentences became then widespread in linguistics due to its un-etymological use by \textcite{smith1991parameter}.}. 

\begin{table}[htb] % the "htb" makes table env unfloaty
\caption{\textit{Aktionsarten} as defined by \textcite{Vendler1957}, plus semelfactives.}
\labtab{vendler_classes}
\begin{tabular}{l|cc}
 & \textbf{punctual} & \textbf{durative} \\
 \hline
\textbf{telic} & \begin{tabular}[c]{@{}c@{}}achievement\\(e.g., \textit{to find})\end{tabular} & \begin{tabular}[c]{@{}c@{}}accomplishment\\(e.g., \textit{to build})\end{tabular} \\
\textbf{atelic} & \begin{tabular}[c]{@{}c@{}}semelfactive\\(e.g., \textit{to knock})\end{tabular} & \begin{tabular}[c]{@{}c@{}}activity\\(e.g., \textit{to run})\end{tabular} \\
\hline
\textbf{stative} & - & \begin{tabular}[c]{@{}c@{}}state\\(e.g., \textit{to know})\end{tabular}
\end{tabular}
\end{table}

To simplify a very complex issue, one might observe the following. The first distinction to be made is between states, which cannot be used in progressive aspects (e.g., \textit{*John is knowing}), and the other categories. The two dimensions along which non-stative verbs vary are durativity and telicity. Durativity, which is quite self-explanatory, was also included among the ten transitivity parameters by \textcite{HopperThompson1980}, with punctual verbs being high in transitivity and durative verbs being instead low in transitivity. The other dimension, i.e., telicity, is defined as the property of having an endpoint of some kind. Crucially, literature on telicity tends to envision it as a property of \textit{predicates}, not just of verbal heads \parencite[270]{HopperThompson1980}. Thus, a durative activity such as \textit{John is running} can be made into an accomplishment by specifiying a terminal point for the event, as in \textit{John is running home}\sidenote{Please refer to \textcite{CennamoLenci2019} and \textcite{cappelli2019argumenthood} for considerations on the argumenthood of the added locative phrase to motion-verb activities in Italian.}, and \textit{vice versa}. Under this account, then, intransitive uses of transitive verbs would just be transforming accomplishments (durative and telic) into activities (durative and atelic), as in \textit{John is smoking (a cigarette)}. Such an account was explored by \textcite{Mittwoch1982} with respect to the intransitive uses of transitive verbs.\\ % Considering this interpretation of the role of telicity, the intransitivization-as-focus-on-the-activity account I discussed on \refpage{activityfocus}, and the fact that activities are the only \textit{Aktionsart} to bear two low-transitivity features based on \posscite{HopperThompson1980} account (i.e., atelicity and durativity), one could be tempted to say that only activities are involved in the implicit indefinite object construction. 
One could be tempted to say that only activities are involved in the implicit indefinite object construction, considering that, based on three pieces of evidence I discussed so far,
\begin{itemize}
    \item object drop turns accomplishments into activities (as argued just now in this Section);
    \item intransitivization is a mechanism employed to focus on the activity (refer to \refpage{activityfocus});
    \item activities are the only \textit{Aktionsart} to bear two low-transitivity features in \posscite{HopperThompson1980} account (i.e., atelicity and durativity).
\end{itemize}
However, as \textcite[151]{Vendler1957} himself observes, this is not the case. He makes an interesting point relative to what he calls a "habit-forming" semantic behavior of some verbs, which also takes us back to the observations about noun incorporation and the focus on the activity in \refsec{theory_incorporation}. In particular, while it is true that some activities are "habit-forming"\sidenote{All examples here are from \textcite[151]{Vendler1957}.} (such as \textit{to smoke} in \textit{Do you smoke?}), this behavior is also shown by accomplishments (e.g., a writer is someone who writes books for a living, just like a cabdriver is someone who drives a cab to earn money) and achievements (e.g., dogcatchers catch dogs for a living).

\paragraph{(Non-) "Inherently telic" verbs}
Moving closer to the account of telicity I am going to employ in my experimental setting (following \textcite{Medina2007}), \textcite[112]{vanvalinlapolla1997syntax} propose a distinction between "inherently telic" verbs such as \textit{to kill} and \textit{to break} on one hand, and activity verbs made into accomplishments, such as \textit{to eat}, on the other hand. This account still understands telicity as a feature of predicates, but it also acknowledges that it can be somewhat embedded in the meaning of a verb. I will return to this issue later in this Section. Crucially, as observed by \textcite[5-6]{NewmanRice2006}, the activity use of \textit{to eat} is considered the "basic" meaning of the verb in \textcite{vanvalinlapolla1997syntax}. The presence of a direct object in sentences featuring such verbs begets telicity depending on the nature of the object itself \parencite{tenny1994aspectual, dowty1991thematic, filip2004telicity}, so that a verb is telic if its "measuring argument" is delimited (i.e., a quantized\sidenote{The standard view that maximally affected quantized objects determine telicity \parencite{verkuyl1972compositional, tenny1994aspectual} is challenged by \textcite{smollett2005quantized} and \textcite{pinon2008aspectual}. For an extensive discussion of this issue, which is not within the scope of these pages, the reader is referred to \textcite{Melchin2019}.} object), atelic otherwise (i.e., bare plurals and mass nouns, as noted by \textcite{verkuyl1972compositional, verkuyl1989aspectual}). Adapting an example from \textcite[24]{tenny1994aspectual}, \textit{to eat} is telic in \ref{tennytelic1} because the apple is a delimited measuring argument \textemdash the eating event ends when the apple is gone. On the contrary, in \ref{tennytelic2} \textit{to eat} is atelic, because there is no fixed quantity of ice-cream for Chuck to consume. If he happened to live in a universe blessed with neverending, incredibly cheap ice-cream, his eating act could go on forever.

\ex. \label{tennytelic} \a. \label{tennytelic1} Chuck eats an apple.
\b. \label{tennytelic2} Chuck eats ice-cream.

This behavior, far from being expressed by \textit{to eat} alone, is shown by all incremental-theme verbs (already mentioned in \refsec{whichverbs}). It is also consistent with the classic \textit{in/for} telicity test\sidenote{I will delve into more detail about telicity tests in \refsec{telicitytestsexp}.}, as shown in \ref{testtelic}.

\ex. \label{testtelic} \a. \label{testtelic1} Chuck ate an apple in an hour / *for an hour.
\b. \label{testtelic2} Chuck ate ice-cream *in an hour / for an hour.

\textcite{Naess2007, Ruda2017, willim2006event, Naess2011} note that intransitive \textit{to eat} is compatible both with a telic reading, as in \ref{naesstelic1}, and with an atelic reading, as in \ref{naesstelic2}. With reference to her affected-Agent account of object drop, which I discussed extensively in \refsec{theory_incorporation}, \textcite[78-79]{Naess2007} argues that the telic reading is granted by Agent affectedness, as if the Agent itself worked as a measuring argument in this case. The atelic reading, instead, is argued to be typical of an event "leading to a result state, but which is in principle independent of this result state" (much like \posscite{Vendler1957} "habit-forming" verbs), e.g., \textit{to cook}. However, it has to be noted that the Agent is indeed affected in some measure in \ref{naesstelic2}, making Agent affectedness a non-decisive factor. \labpage{agentaffectfail}

\ex. \label{naesstelic} \a. \label{naesstelic1} I ate in five minutes, then rushed off to work.
\b. \label{naesstelic2} We ate all evening.

Crucially, and consistently with the particular perspective adopted by \textcite{Medina2007} (and, following her, by me) following \textcite{Olsen1997}, the atelic interpretation of intransitive \textit{to eat} seems to be more easily attained than the telic interpretation, since it does require less processing effort. In other words, both interpretations imply a focus on the activity (rather than on the Patient object, which is missing altogether), but the telic interpretation also requires that one understands the sentence as if putting additional focus on the way the Agent is affected by the activity. Quoting \textcite[4]{OlsenResnik1997}, implicit indefinite objects need to appear "in the appropriate context" in order to get a telic interpretation. \textcite[79]{Naess2007} actually leverages the bivalent behavior of intransitive \textit{to eat} with respect to telicity to weaken the use \textcite{HopperThompson1980} make of this parameter to determine transitivity, at least in the specific case of this verb. However, since a rule specific to a single verb would not make for a strong grammar, I argue that telicity as a factor determining (or blocking) object drop is there to stay.

\paragraph{A note on telicity and (in)definite object drop} "Inherent telicity" has interesting consequences on the theory of object drop, and in particular on the distinction between definite and indefinite object drop discussed in \refsec{theory_def_vs_indef}, as observed by \textcite[3-4]{OlsenResnik1997} with reference to \textcite{Allerton1975}, \textcite{Mittwoch1982}, and \textcite{Olsen1997}. What they note is that implicit objects tend to receive indefinite interpretations with atelic verbs (unless they appear in particularly favorable contexts as to license a definite interpretation, as seen in \ref{naesstelic1}) and definite interpretations with telic verbs. An example of this argumentation is provided in \ref{olsenresniktelic}, adapted from \textcite[3]{OlsenResnik1997}. In \ref{olsenresniktelic1}, the inherently telic verb \textit{to win} is shown to require a definite interpretation for the missing object, while it can be interpreted as indefinite when occurring with an inherently atelic verb such as \textit{to eat} in \ref{olsenresniktelic2}.

\ex. \label{olsenresniktelic} \a. \label{olsenresniktelic1} Benjamin won, \#but I don't know what. 
\b. \label{olsenresniktelic2} Benjamin ate, but I don't know what. 

However, such a telicity-as-definiteness account is not bulletproof. In this Paragraph I already made reference to the possibility of inducing a definite interpretation for implicit objects of inherently atelic verbs, provided sufficient context. I argue that it is also possible to induce an indefinite interpretation for implicit objects of inherently telic verbs, e.g., by presenting the action as iterative or habitual \parencite[507-509]{Goldberg2001}, as in \ref{goldbergtelic}. Example \ref{goldbergtelic1} would be ungrammatical if the missing object was given a definite interpretation, but it can actually be considered at least partially acceptable (even though \textit{to kill} is inherently telic) under a habitual reading. Indeed, the Joker is a notorious fictional villain from the Batman universe known for his ruthlessness, so it would be quite easy to imagine killing to be a frequent habit of his. This effect is much more evident in \ref{goldbergtelic2}, where the iterative reading is made explicit by the addition of the adverb \textit{again}, without need for extra-linguistic context.

\ex. \label{goldbergtelic} \a. \label{goldbergtelic1} \# The Joker killed.
\b. \label{goldbergtelic2} The Joker killed again.

I will expand more on this in \refsec{iterativity} and \refsec{predictor_iterativity}. For now, suffice it to say that implicit objects occurring with both inherently telic and inherently atelic verbs can be made to yield either a definite or an indefinite interpretation based on context. In particular, with relevant implications for my own model of object drop, both telic and atelic verbs can occur felicitously with implicit indefinite objects.

\paragraph{Olsen (1997)'s account of telicity}
Let us now comment on the specific account of telicity I am going to employ in my experimental setting (see also \refsec{predictor_telicity} for details on the implementation). Since I intend my probabilistic model of the implicit indefinite object construction as an expansion upon the original model by \textcite{Medina2007}, I am going to base my interpretation of telicity on the same source she chose for her study, i.e., \textcite{Olsen1997}.\\
In her "semantic and pragmatic model" of aspect, \textcite{Olsen1997} interprets telicity as a privative feature. This means that a verb can either have or not have the [+telic] feature. This feature, assigned to achievements and accomplishments, denotes in her words "the existence of an end or result to which a situation naturally will lead, not necessarily the actual attainment of such an end". The interpretation of the attainment of this end depends, in her view, not only on telicity, but also on perfectivity and tense (as I will discuss in more detail in \refsec{telperftense}). Crucially, the [+telic] feature is "semantic", in Olsen's words, and cannot be canceled by additional constituents. Consider, for instance, her examples in \ref{olsentelic}. She notes that "although durative adverbials are supposed to turn accomplishments into activities, \ref{olsentelic1} and \ref{olsentelic2} represent iterative accomplishments".

\ex. \label{olsentelic} \a. \label{olsentelic1} Eli won for years.
\b. \label{olsentelic2} Eli ran a mile for years.

On the contrary, she considers the [-telic] feature\sidenote{Actually, Olsen indicates the absence of telic denotation with [0 Telic], rather than with [-telic]. However, I will use the latter notation here for clarity, and also for consistency with my later use of this feature throughout my dissertation.} to be "a cancelable conversational implicature", as exemplified in \ref{olsenatelic}. This means that atelic verbs can receive a telic interpretation by adding a measuring argument, such as a bounded object in \ref{olsenatelic1} or a Goal in \ref{olsenatelic2}. Indeed, "progressive forms of atelic verbs are said to entail the corresponding perfect form" (e.g., \textit{Eli is running} entails \textit{Eli has run}), but neither sentence in \ref{olsenatelic} obeys this requirement (e.g., \textit{Eli is running a mile} does not entail \textit{Eli has run a mile}). This particular state of affairs is known as the "imperfective paradox" \parencite{dowty2012word1979, white1993imperfective}, stating that progressive aspect overrides the result entailment (refer to \textcite{CopleyHarley2015force}, \textcite[115]{Dvorak2017thesis}, and \textcite{Melchin2019} for more on this issue).

\ex. \label{olsenatelic} \a. \label{olsenatelic1} Eli is running a mile.
\b. \label{olsenatelic2} Eli is running to the store.

This is consistent with other views of telicity discussed before in this Section, where implicit objects were argued to be more easily accepted with atelic verbs than with telic verbs, since an overt object is usually required by telic verbs (for which it works as an explicit endpoint\sidenote{This traditional account \parencite{tenny1994aspectual} gets re-interpreted by \textcite{smollett2005quantized}, who argues that quantized objects do not strictly delimit the event, but they just make the delimiting endpoint contextually available. Either way, Olsen's account still holds.} or, in other words, a "measuring argument"), while it is \textit{admitted}, but not required, by atelic verbs.\\
Following Olsen, \textcite{Medina2007} uses telicity as a binary predictor of object drop. Crucially, she assigns the [+telic] or [-telic] feature to the target verbs themselves, rather than to the predicates they head. Thus, a verb in her experimental setting can either be inherently telic or inherently atelic, on the basis of rigorous tests she performed beforehand (refer to \refsec{predictor_telicity} for more details on the tests).


\subsection{Perfectivity} \labsec{perfectivity}

\paragraph{Introduction}
Perfectivity, as a component of "grammatical aspect" or "viewpoint aspect", is a property assigned to a verb based on the perspective the speaker has on the temporal constituency of the event the verb describes \parencite{comrie1976aspect}. Thus, an event seen as complete\sidenote{Not necessarily \textit{completed}, as \textcite[18]{comrie1976aspect} observes with examples from several languages.}, having an initial and a final point, will be encoded by a verb in the perfective aspect, while an event seen as ongoing, having neither an initial nor a final point, will be encoded by a verb in the imperfective aspect. For instance, the perfective/imperfective opposition can be seen in \ref{myperf}, relative to English \textit{to write}, which is perfective in \ref{myperf1} and imperfective in \ref{myperf2}.

\ex. \label{myperf} \a. \label{myperf1} John had written a thesis.
\b. \label{myperf2} John was writing a thesis.

The grammatical relevance of the property of perfectivity varies crosslinguistically. In Slavic languages, such as Russian, perfectivity is embedded in the lexicon itself, so that sentences like the ones in \ref{myperf} would actually require two different lexical entries to refer to the same writing event (i.e., \textit{napisat'} for the perfective form and \textit{pisat'} for the imperfective form). In other languages, such as English and Italian, (im)perfectivity is encoded with morphological means on a single verb lexical entry, and both perfective and imperfective aspects are compatible with any tense\sidenote{This is not true of every language marking grammatical aspect morphologically. For instance, in Latin it is only possible to encode the perfective/imperfective distinction in the past tense.} (more on the relation between aspect and tense in \refsec{telperftense}).\\
Crucially, while telicity is a property of verbs themselves\sidenote{Which gets fully realized only when interpreted within the full linguistic context.} (as argued in \refsec{telicity} on the basis of \textcite{Olsen1997} and \textcite{Medina2007}) or of predicates, perfectivity is a property of events, which gets encoded on verbal heads in different ways (morphologically, in the two languages I am interested in). As discussed in \refch{predictors} and shown in \refch{judgments}, this will have important consequences on my experimental setting.

\paragraph{Perfectivity and object drop}
Grammatical aspect has been argued to play a role in licensing or blocking indefinite object drop, even though it appears to be a path less trodden than lexical aspect. \textcite[30]{Medina2007} explains the lesser attention devoted to perfectivity as a predictor of object drop with reference to the fact that for many years scholars interpreted object omission as a verb-specific phenomenon (refer back to \refch{objectdrop} for a full commentary on this). Nevertheless, literature on the matter \parencite{Lorenzetti2008, Cote1996, Naess2007, TsimpliPapadopoulou2006, Dvorak2017thesis} agrees that imperfective aspect (encoding duration) is more likely than perfective aspect (encoding completion) to license implicit objects. Compare, for instance, \ref{myperfex1}, where \textit{to write} appears in perfective aspect and blocks object drop, and \ref{myperfex2}, where it appears in imperfective aspect and allows for indefinite object drop.

\ex. \label{myperfex} \a. \label{myperfex1} *John had written.
\b. \label{myperfex2} John was writing.

\textcite[1609]{TsimpliPapadopoulou2006} explain this by observing that "perfectivity is understood as involving an endpoint", which is made explicit by the use of an overt object.\\
This different behavior expressed by perfective and imperfective clauses holds crosslinguistically. For instance, \textcite[1597]{TsimpliPapadopoulou2006} observe that while null objects are acceptable both with perfective and with imperfective verbs in Greek, they tend to be favored more by imperfective aspect. They also note \parencite[1601]{TsimpliPapadopoulou2006} that the strict ungrammaticality of indefinite null objects occurring with perfective verbs in Russian (and Polish, as found in \textcite[89]{sopata2016null}) is not found in Greek. As for the languages I will base my model on, \textcite{Medina2007} provides experimental evidence in support of imperfective clauses being more prone to favor object drop than perfective clauses in English, while \textcite{Cennamo2017} comments on Italian sentences reaching the same conclusion. Moving to typologically different languages, \textcite[118]{Naess2007} observes that in Kalkatungu\sidenote{Kalkatungu is a language belonging to the Pama–Nyungan family, spoken in Australia.} variation in grammatical aspect is accompanied by changes in case-marking, i.e., ergative-absolutive in perfective clauses and absolutive-dative in imperfective clauses. This becomes relevant for the role of perfectivity in the implicit indefinite object construction when compared to previous observations about ergative languages made in \refsec{theory_incorporation}, where the case was made that subjects of intransitive verbs and transitive verbs used intransitively are in the absolutive case (as they are in imperfective clauses in Kalkatungu), while subjects of transitive verbs used transitively are in the ergative case (as they are in perfective clauses in Kalkatungu). To put it more simply, since object drop is favored by imperfective aspect, it stands to reason that in such languages case gets assigned accordingly. The limitation of ergative constructions\sidenote{It has to be noted that this account is only valid for languages exhibiting a kind of split ergativity that is conditioned by the grammatical aspect.} to perfective environments is also noted in \textcite[271]{HopperThompson1980}, specifically with evidence from Hindi and Georgian, and references to literature about other languages.

\subsection{Interactions among telicity, perfectivity, and tense} \labsec{telperftense}

\paragraph{Telicity and perfectivity}
There are several comparisons to be drawn between telicity and perfectivity, and also between tense and these two facets of aspect \parencite[394-397]{YousefiMardian2019}. I will come back to this in \refsec{anoteontense}, focusing on observations bearing direct consequences for my experimental design. Here, I will focus instead on broader concerns.\\
\textcite[162]{Lazard2002} notes "an affinity between the incompleteness of the process and the low individuation of the object", which is one of the ten parameters of prototypical transitive clauses in the account by \textcite{HopperThompson1980} (see \refsec{theory_transitivity}, and \reftab{ht1980_parameters} in particular). According to Lazard, an event can be construed as incomplete when the verb is in "an incompletive aspect" (progressive, habitual, imperfective...) or when "the object is only partly affected" (which is typical of atelic verbs and indefinite objects, which are interpreted as non-measuring plurals or mass nouns, as seen in \refsec{whichobjects}). Relatedly, \textcite[118]{Naess2007} puts a particular focus on the association between the concept of delimitedness and the concept of affectedness. Thus, Lazard shows that atelicity and imperfectivity share the property of favoring indefinite object drop by construing the event as incomplete. Moreover, similarly to what \textcite{Olsen1997} concludes about telicity (telic aspect being uncancelable, unlike atelic aspect), \textcite{Dvorak2017thesis} notes that perfective aspect is the marked form and imperfective aspect is the unmarked form.\\
The close relation between telicity and perfectivity is also found in \textcite{HopperThompson1980} themselves, who explicitly state they use these two terms "interchangeably" \parencite[270]{HopperThompson1980}. They justify this choice, which nowadays would be untenable \parencite{bertinetto2001frequent, bertinetto-delfitto2000aspect, CivardiBertinetto2015semantics}, by acknowledging the poverty of the literature on the matter up to their time of writing \textemdash indeed, they note that it would be "risky to infer a distinction between the two types of aspect when none is explicitly discussed". However, consistently with previous observations about telicity and perfectivity provided in this Chapter, Hopper and Thompson recognize that telicity "can be determined generally by a simple inspection of the predicate" while "perfectivity is a property that emerges only in discourse".

\paragraph{Preminence of telicity over perfectivity}
\textcite[1598]{TsimpliPapadopoulou2006} observe that both imperfective and perfective activity verbs in Greek receive an atelic reading when combined with bare plurals (e.g., \textit{Helen painted / was painting portraits}), hinting towards a more preminent role of (a)telicity than (im)perfectivity in determining indefinite object drop, even though the two are related in multiple ways. Indeed, if imperfectivity played a bigger role than atelicity, data would show that only imperfective verbs, regardless of their telicity feature, could occur with bare plurals (which, I insist, are the only possible interpretation for indefinite null objects together with mass nouns). The hypothesis of the preminence of telicity on perfectivity with respect to their role in the implicit indefinite object construction is also consistent with experimental evidence from English and Italian I will provide in \refch{results}. Even more strongly, \textcite{Stoica2017} finds that native speakers of Romanian are equally avoidant of indefinite null objects both in perfective and in imperfective contexts.\\
The preminence of telicity on perfectivity is found not only in adult grammar, but also in the early stages of grammar acquisition by children. In particular, three-year-olds and younger children have been shown to assign preferably imperfective aspect to atelic verbs and perfective aspect to telic verbs both in production and in comprehension \parencite{Medina2007, olsenEtAl1998acquiring, wagner2001aspectual}. However, this account of L1 acquisition, which is known as "aspect-first hypothesis" in the field \parencite{antinucci1976children}, needs to be taken with a pinch of salt. An alternative view exists in the literature claiming that children, instead of first acquiring aspect as entangled with Aktionsart and tense (i.e., atelic=imperfective=present \textit{vs} telic=perfective=past), actually extract the relevant information out of the morphological structure of their L1 instead of relying on a pre-built strategy \parencite{bertinetto2021acquisition, BertinettoLenciEtAl2015}.

\paragraph{Tense and aspect}
As noted by \textcite[68]{Medina2007}, tense and aspect are independent but interrelated properties, "to the extent that one encourages certain interpretations of the other". In adult grammar \parencite{comrie1976aspect} and especially in child grammar \parencite{wagner2001aspectual}, for instance, past tense may induce a perfective interpretation of the event. I will return on the relation between past tense and perfective aspect in \refsec{anoteontense}.\\
With specific regard to the implicit indefinite object construction, \textcite{Dixon1992, Goldberg2005a, Glass2020} note that verbs in the past tense tend to block object drop. On the contrary, \textcite[9]{Garcia-VelascoMunoz2002} note that present tense, interpreted as an expression of imperfectivity, favors object drop.


\section{Pragmatic factors} \labsec{pragmaticfactors}

In this Section, I use "pragmatic factors" as an umbrella term for several factors (neither verb-specific nor aspect-related) involved in the implicit indefinite object construction. The term covers not only purely pragmatic factors (such as a routine interpretation), but also phenomena related to intra-linguistic context (such as iterativity and habituality) and discourse factors (such as emphasis and contrastive focus). These factors, which \textcite{Medina2007} did not include in her novel model of indefinite object drop, are nevertheless crucial in a comprehensive analysis of this construction. Indeed, as \textcite[54]{delancey1987transitivity} observes in his cognition-oriented account of transitivity parameters, the interpretation of utterances in actual language use is based on real-world context or, failing such, on discourse context. This also echoes the conclusion, reached by \textcite[176]{Prytz2016}, that the "structural sides of linguistic meaning" go hand in hand with the "contextual, pragmatic, and encyclopedic sides of meaning".

\subsection{Purely pragmatic factors: routine} \labsec{pragmaroutine}
Routine is described by \textcite[2]{Glass2020} as "a series of recognized, conventional actions within a community", whose association with a given verb is shown by experimental evidence to vary gradiently across different communities of speakers. The author observes, with no dearth of experimental evidence from Reddit communities, that transitive verbs describing routines facilitate object drop and, likewise, object drop induces hearers to imagine scenarios where the described event is routine for its performers. She also makes an explicit connection between this account and the object-drop-as-noun-incorporation account I discussed in \refsec{theory_incorporation}.\\
Crucially, routine cannot be construed as yet another semantic factor, because it is strongly rooted in extra-linguistic context (unlike semantic factors such as the ones I discussed in \refsec{semanticfactors}, which are strictly related to the very meaning of the verb itself). In particular, speakers encode a lot of world knowledge in their utterances when they use a verb to convey a routine interpretation, and hearers (or readers) of such utterances have to be aware of the specific context they stem from in order to make sense of the intended meaning. For instance, example \ref{doyoulift} from \textcite[9]{Glass2020} would only be correctly interpreted as "you are not in the habit of lifting weights to exercise" if one knew that it was uttered in the context of a conversation about fitness.

\ex. \label{doyoulift} You don't lift.

The role of world knowledge in facilitating object drop is also noted by \textcite[528]{Eu2018}, where reference is made to "contextually established semantic specialization". Here, "context" is taken to refer both to "immediate context" such as the time and place of utterance, and to "general context" such as world knowledge and the life habits of the speaker. Similarly, the routine-licensed account of object drop contributes to explaining the "specialized readings" of verbs such as \textit{to eat} (i.e., "to eat a meal") and \textit{to drink} (i.e., "to drink alcohol")\sidenote{Which I discussed extensively in \refch{objectdrop}.} noted, among others, by \textcite[420]{Naess2011}.


\subsection{Linguistic-context factors: iterativity and habituality} \labsec{iterativity}

\textcite[237]{Mittwoch2005} observes that "the omissibility of unspecified objects is for many verbs subject to contextual factors". For instance, she notes that habitual contexts, "where the lexicon interacts with more general properties of the sentence", are more likely to license object drop than episodic contexts. Mittwoch uses the term "habitual" in a way that closely resembles the account provided by \textcite{Glass2020} (refer back to \refsec{pragmaroutine}). Indeed, she understands the "imbibe alcoholic beverages" interpretation of intransitive \textit{to drink} to be a habitual reading of the verb, since it refers to a habit the Agent is shown to have. She also applies the term, admittedly "rather freely", to sentences such as \ref{mittwochhabitual1}, where the habitual reading is not inferred by means of verb semantics and world knowledge (as in the case of \textit{to drink}), but instead via extra-linguistic context \textemdash the Agent is obviously not reading the Iliad in his sleep, but he is in the habit of doing so while awake. Dispositional properties and professions appear to be another class of broadly-defined habits giving rise to felicitous object drop, as Mittwoch shows in \ref{mittwochhabitual2} and \ref{mittwochhabitual3}, respectively. Similar considerations are also found in \textcite{FellbaumKegl1989taxonomic}, \textcite[39]{Levin1993}, \textcite[518]{Goldberg2001}, and \textcite[29]{PethoKardos2006}.

\ex. \label{mittwochhabitualA} \a. \label{mittwochhabitual1} He is reading the Iliad at the moment. \\ (said about somebody who is asleep)
\b. \label{mittwochhabitual2} Fido bites.
\c. \label{mittwochhabitual3} She directs (films), produces (films), conducts (music), dyes (textiles), programmes (computers).

\textcite[518]{Goldberg2001} argues that these "characteristic property" examples, where some typical transitive verbs can be used intransitively to elicit the interpretation that the action is somewhat characteristic of the agent, get easily explained by her principle of Omission under Low Discourse Prominence (which I introduced on \refpage{activityfocus}). However, she also observes that the characteristic-property interpretation is not strictly required \textemdash I would say, the characteristic property can be interpreted as either permanent or temporary, as long as it is implied in some way. She provides an example of this in \ref{goldz}.

\ex. \label{goldz} That dog has been known to occasionally bite, but he is generally very loving.  
% \b. \label{goldz2} The frightened toddler scratched and bit until his mother arrived.

In addition to habitual contexts, \textcite[248]{Mittwoch2005} also discusses other examples of pluractionality \parencite{lasersohn1995plurality2013}, i.e., contexts fostering event plurality, as in \ref{mittwochhabitualB}\sidenote{It should be noted that \textcite{Lavidas2013}, a lone voice, argues that in English indefinite null objects of the kind illustrated by \ref{mittwochhabitualB} "are not accepted by all native speakers". He even argues that only definite implicit objects are possible in modern English, but I would note that decades of literature on the matter, as well as my own contribution in these pages, demonstrate that indefinite object drop is indeed a possibility in modern English.}. With respect to pluractionality, \textcite{BertinettoLenci2012habituality} observe that habituality, where "the resulting habit is regarded as a characterizing property of a given referent", is closely related to iterativity (both being expressions of pluractionality) but distinct from it (since habituality, but not iterativity, belongs to the class of "gnomic imperfectives", i.e., constructions having "a characterizing function" expressed in the imperfective aspect in languages with explicit aspectual marking).

\ex. \label{mittwochhabitualB} \a. \label{mittwochhabitual4} They murdered, raped, and plundered.
\b. \label{mittwochhabitual5} [International tribunals] are valuable, she argues, because when they punish criminals, they also affirm, condemn, purge, and purify.

Most importantly, it should be noted that \ref{mittwochhabitual4} is not a good example of habituality belonging to "gnomic imperfectives", since Simple Past is aspectually neutral in English. Let us consider the two possible translations of Mittwoch's example in Italian, provided in \ref{mittwoch4ita}. Since both \ref{mittwoch4ita1} (with verbs in the \textit{passato remoto} tense, aspectually perfective) and \ref{mittwoch4ita2} (with verbs in the \textit{imperfetto} tense, aspectually imperfective) are grammatical, then, it is pluractionality and not grammatical aspect making \ref{mittwochhabitual4} (as well as the Italian equivalents in \ref{mittwoch4ita}) acceptable. As a consequence, imperfectivity has to be interpreted as a facilitating factor for indefinite object drop only inasmuch it encodes a progressive, rather than habitual, meaning.

\ex. \label{mittwoch4ita} \a. \label{mittwoch4ita1} Uccisero, stuprarono, saccheggiarono.
\b. \label{mittwoch4ita2} Uccidevano, stupravano, saccheggiavano.

% Vero, ma l'es. (18a) non dice nulla rispetto al contrasto aspettuale, perché il Simple Past è aspettualmente neutro. Al di là di ciò, per il tuo scopo è sufficiente la plurazionalità. I due ess. seguenti vanno altrettanto bene:
% - Essi uccisero, stuprarono, saccheggiarono
% - Essi uccidevano, stupravano, saccheggiavano
% Quindi, cita pure le nostre cose sull'abitualità, ma per dire che ai tuoi fini conta la plurazionalità, comunque si manifesti aspettualmente.
% Ciò è importante, perché quando tu sopra hai parlato dell'imperfettivo, era evidente che ti riferivi all'accezione progressiva. E' bene precisarlo, altrimenti si può pensare che sia l'imperfettività in quanto tale a favorire l'omissione.

\textcite[250]{Mittwoch2005} further observes that habitual contexts also favor object drop with verbs that would usually block it, such as the verbs of destruction used in \ref{mittwochhabitualC}.

\ex. \label{mittwochhabitualC} \a. \label{mittwochhabitual6} They usually demolish rather than restore.
\b. \label{mittwochhabitual7} They fell indiscriminately.

I briefly mentioned iterative contexts in \ref{goldbergtelic}, here reported again in \ref{itermine}, where the verb \textit{to kill} was shown to admit an implicit indefinite object more easily when used iteratively, as in \ref{itermine1}, than in an episodic context, as in \ref{itermine2}.

\ex. \label{itermine} \a. \label{itermine1} The Joker killed again.
\b. \label{itermine2} \# The Joker killed.

\textcite[5]{Glass2013} explains this difference between iterative and episodic sentences by claiming that more information is lost when object drop happens in the latter than in the former. In other words, "in these sentences describing iteration, it becomes less likely that interlocutors' communicative purposes would be thwarted" by object drop. In an interesting account of the transitivity of iterative constructions in Warrungu\sidenote{Warrungu is a language of the Pama–Nyungan family, spoken in Australia.}, \textcite[4-5]{Tsunoda1999} notes that in this language the iterative suffix, typically having imperfective readings (e.g., iterative and habitual), tends to combine with verbs that are very low on the transitivity scale, consistently with everything I observed about iterativity so far in this Section.\\
The facilitating role of iterativity with respect to the implicit indefinite object construction, amply discussed by \textcite{Goldberg2001}, will be further explored in \refsec{predictor_iterativity} relatively to my experimental setting.
% "is used most commonly with intransitive verb roots" and makes transitive verbs into antipassives (verbs with reduced valence, as I mentioned on \refpage{antipassives})


\subsection{Discourse factors: emphasis, coordination and contrastive focus} \labsec{pragmadiscourse}

According to \textcite[251-252]{Mittwoch2005}, "the most permissive contexts for object drop involve pairs of verbs that stand in some sort of semantic contrast", even when these verbs are "some of the poorest candidates for object drop" (e.g., \textit{to break} and other prototypical change-of-state verbs). Contrastive focus is also discussed in these terms by \textcite{Dixon1992}. Examples \ref{mittwochhabitual9} and \ref{mittwochhabitual10} are from Mittwoch, while examples \ref{cotehabitual1} and \ref{cotehabitual2} (featuring coordination instead of constrastive focus) are from \textcite[112, 143]{Cote1996}.

\ex. \label{mittwochhabitualD} \a. \label{mittwochhabitual9} This one creates, that one destroys.
\b. \label{mittwochhabitual10} A few people bought, most just looked.
\c. \label{cotehabitual1} You wash and I'll dry.
\d. \label{cotehabitual2} Bert pushed and Ernie pulled.

% \a. \label{mittwochhabitual8} Those who give bribes are just as guilty as those who take.

In these cases, where the verbs "prop each other up", object drop "is clearly a rhetorical device" used to put all the focus on the verb itself (refer to \refpage{activityfocus} for more on this).\\
In addition to constrastive focus, \textcite[196-197]{goldberg2006constructions} mentions several other pragmatic factors licensing\sidenote{She actually claims that "the underlying motivation for the expression of arguments is at root pragmatic", in a much more stronger account of the role of pragmatics than the one I am arguing for here. \textcite{Ahringberg2015} observes that Goldberg's claim is in direct constrast with the account provided by \textcite{Fillmore1986}. A mild critique of all-pragmatic accounts of object drop is also found in \textcite{NemethEniko2014}.} object drop, or, as she calls it, the "Deprofiled Object construction". These additional factors are "repeated action" (which \textcite{Rissman2016} terms "x-and-x construction"), as in \ref{goldadd1}, and "strong affective stance", as in \ref{goldadd2}. In particular, \textcite{Rissman2016} interprets repeated-action contexts to target agentive meaning, i.e., to highlight "an atelic event in which the agent repeatedly performs an action". This, consistently with the affected-Agent account I discussed on \refpage{affectedagent}, would appear to favor the implicit indefinite object construction.

\ex. \label{goldadd} \a. \label{goldadd1} Pat gave and gave but Chris just took and took.
\b. \label{goldadd2} He murdered!

\textcite[506-514]{Goldberg2001} also identifies additional discourse factors facilitating object drop, i.e., generic statements, as in \ref{goldadd3}, and infinitives, as in \ref{goldadd4}. Once again, object drop is shown to be favored by pluractionality. She notes \parencite[507]{Goldberg2001} that in these cases "atelicity could supply the appropriate constraint", given that "repeated actions are often construed as atelic or temporally unbounded events", and atelicity (as discussed in \refsec{telicity}) strongly facilitates object drop.

\ex. \label{goldaddB} \a. \label{goldadd3} Tigers only kill at night.
\b. \label{goldadd4} The singer always aimed to please/impress.

To all these discourse factors, \textcite[3]{Glass2020} also adds modal statements, as in \ref{glassmodal}.

\ex. \label{glassmodal} Dresses I would murder for.

\textcite[66]{Lorenzetti2008} uses "structural omission" as a cover term for the discourse factors discussed here. Finally, \textcite[46]{CumminsRoberge2005} provide an interesting review of not-so-recent pieces of literature in French language from the late '80s to the year 2000 about the role of pragmatics (i.e., contextual, discursive, constructional, and intention-related factors) in favoring object drop. Together with \textcite{Goldberg2001, Goldberg2005, Goldberg2005a, goldberg2006constructions}, \textcite{groefsema1995understood} is another strong advocate for the idea that object drop is driven, first and foremost, by pragmatic factors (\textit{contra} \textcite[29]{PethoKardos2006}). Lamenting the scarcity of literature on the interaction between lexicon and pragmatics, \textcite[7]{Garcia-VelascoMunoz2002} praise Groefsema, and also \textcite{Allerton1975, Fillmore1986, FellbaumKegl1989taxonomic} as "notable exceptions".


\section{A note on frequency} \labsec{frequencyfail}

Frequency, intended as the number of times a verb (or a verb-object pair) occurs in a corpus, has often been observed to be somewhat correlated with other factors playing a role in felicitous object drop. For instance, \textcite[149-151]{Resnik1996} hypothesizes that semantic selectivity (discussed in \refsec{recoverability} and \refsec{resnik_sps}) should be positively correlated with the corpus frequency of transitive verbs used without a direct object, based on the idea that verbs selecting for highly recoverable objects should occur more easily without them than verbs selecting for scarcely recoverable objects. His results show that while high omission frequency correlates with high semantic selectivity, some verbs deviate from this trend by failing to participate in the implicit indefinite object construction despite being high-recoverability verbs. Interestingly, the opposite (i.e., low-recoverability verbs omitting their object very frequently) never happens, which Resnik takes to mean that "this pattern reflects an underlying hard requirement, namely that strong selection is a necessary condition for object omission". However, these findings are not replicated by \textcite[441]{ruppenhofer2004interaction}, who finds no association between verb frequency and their tendency to allow implicit objects in a study using a 34-verb set only partially overlapping with Resnik's 30-verb set.\\
\textcite[165]{Medina2007}, using the same verb set and semantic selectivity measure as \textcite{Resnik1996}, further observes that frequency fails to show a precise correlation with gradient grammaticality judgments provided by human subjects about implicit indefinite objects, since some verbs received intermediate judgments despite never being used intransitively in the Brown corpus. Medina provides two interpretations for this phenomenon, one where this is simply an artifact depending on the small size of the corpus, and another ascribing this mismatch between mid-way judgments and null corpus frequency to the existence of a threshold grammaticality value in the minds of speakers blocking them to utter sentences less grammatical than that ideal value (refer to \textcite{KempenHarbusch2005} for a similar account of the existence of what they call a "production threshold"). The existence of this threshold, and its actual numerical value, is an open question for future studies comparing native judgments and language production.\\
% one considering it a corpus artifact, and another ascribing it to "a criterion such that any output form with less than a certain level of grammaticality is simply never produced".\\
\textcite{Goldberg2005a} offers another relevant insight as to the role frequency plays in object drop. She suggests that frequent use of some verbs (e.g., \textit{to smoke, to drink, to sing, to write}) in contexts favoring a habitual interpretation (discussed in \refsec{pragmaticfactors}) may give rise to "the grammaticalization of a lexical option, whereby they can appear intransitively in less constrained contexts" such as generic contexts\sidenote{Such as the one provided by the sentence \textit{John ate early today}, where generic \textit{to eat} is understood to refer to the act of eating a meal.} not implying habituality. Under this view, further discussed by \textcite[65]{Lorenzetti2008}, frequency is not a direct cause of object drop, but it appears to be a rather strong facilitator in a diachronic perspective. Another example of frequency facilitating object drop, both in a synchronic and in a diachronic account, is that of verb pairs where the less frequent near-synonym does not allow implicit indefinite objects (e.g., \textit{to devour}, compared to \textit{to eat}). I devoted some space to the conundrum presented by such verb pairs, discussed by \textcite{Glass2020, Goldberg2005a, Lorenzetti2008} among others, in \refch{objectdrop} and in \refsec{mannerspec}.\\
To conclude, indefinite object drop is never a direct consequence of frequency itself, crucially, and frequency has been shown time and again to be a poor correlate of recoverability or object-droppability. Given this, in my experimental setting (presented in \refch{judgments}) I will employ strategies to avoid having frequency be a confounding factor in my experiments on human acceptability judgments (refer to \refsec{frequencycheck} and \refsec{likert_preprocessing}).
% endeavor to minimize the possible effect of frequency on human acceptability judgments.

\section{Final considerations} \labsec{factorsofchoice}
In this Chapter, I presented a series of factors which the literature on indefinite object drop identifies as facilitators of the intransitive use of transitive verbs. In particular, they are:
\begin{itemize}
    \item semantic factors: object recoverability (strongly correlated with the semantic selectivity of a verb), Agent affectedness, and manner specification (intended as a feature of "semantically marked" counterparts of neutral verbs, e.g., \textit{to devour} with respect to \textit{to eat});
    \item aspectual factors: telicity (lexical aspect) and perfectivity (grammatical aspect);
    \item pragmatic, contextual, and discourse factors: iterativity, habituality, routine, emphasis, coordination, and constrastive focus.
\end{itemize}

I also noted that corpus frequency, both of verbs themselves and of indefinite null objects, is a rather unreliable correlate of the likelihood of a transitive verb participating in the implicit indefinite object construction. Moreover, it is never a direct cause of indefinite object drop \textemdash rather, it is a consequence of other factors at play.\\
Crucially, no factor among the ones I discussed here is able to predict indefinite object drop with absolute certainty. Indeed, while they all contribute to the phenomenon, none alone is responsible for it. Moreover, there is considerable difference in the effect of each factor on the omissibility of indefinite objects. For instance, while the literature consistently acknowledges that recoverability is the main driver of indefinite object drop, there are way fewer accounts relative to Agent affectedness, and this has also been noted to be a controversial factor on \refpage{agentaffectfail}. Among semantic factors, manner specification (in the sense I use in this thesis) is considered to be a rather strong factor when comparing pairs of hypernym-troponym verbs where one is the manner specified counterpart of the other. Aspectual factors are reliable predictors of object drop, by and large, with telicity playing a somewhat larger role than perfectivity. Among pragmatic factors, which appear to be less powerful and more constrained than semantic and aspectual factors in facilitating object drop, some are expressed by linguistic means (iterativity, coordination, contrastive focus), while others are (also, or only) dependent on extra-linguistic context (habituality, routine, emphasis).\\
I picked my factors of choice based on several considerations:
\begin{itemize}
    \item I want to expand upon the original model of indefinite object drop by \textcite{Medina2007}, who used semantic selectivity, telicity, and perfectivity as predictors;
    \item since the same semantic and aspectual factors are at play in context-poor and context-rich utterances (as I argued in \refsec{theory_workingdef}), I want to avoid using context-dependent factors in my experiments in order to avoid context-related confounding effects;
    \item based on previous observations in this Section and throughout this Chapter, Agent affectedness does not appear to be strong enough of a factor to be included in my models;
    \item corpus frequency is not going to be featured in the model, for reasons stated in \refsec{frequencyfail} and before in this Section.
\end{itemize}

Thus, the predictors I will use in my Stochastic Optimality Theoretic models of indefinite object drop are: semantic selectivity (as a proxy to object recoverability), telicity, perfectivity, manner specification, and iterativity (because, among the context-free pragmatic factors, it is the only one requiring just the one verb in the stimulus sentence). I will discuss the experimental implementation of each in \refch{predictors}.
% \setchapterimage[6.5cm]{seaside}
\setchapterpreamble[u]{\margintoc}
\chapter{Factors allowing indefinite object drop}
\labch{factors}

In \nrefch{objectdrop} I presented indefinite object drop as a marked construction deviating from the transitive prototype. Indefinite object drop has been argued to be binarily distinct from definite object drop, to be possible with some or all transitive verbs, to imply an unsaid \textit{something} or a prototypical object different for each verb, to make the verbs participating in this construction have an additional, intransitive entry in the lexicon or just the transitive one. Several answers were proposed to these conundrums in the literature, and I provided my own perspective.\\
In this Chapter, I am going to focus on the main intra-linguistic (semantic, aspectual, and pragmatic) factors allowing a transitive verb to participate in the indefinite object drop construction, based on literature on this topic. I will return to the subject of recoverability, manner specification, telicity, perfectivity, and iterativity in \nrefch{predictors}, where I will define them as predictors of indefinite object drop in my Stochastic Optimality Theoretic model.


\section{Semantic factors} \labsec{semanticfactors}

\subsection{Recoverability} \labsec{recoverability}

\paragraph{An intuitive notion of recoverability}

As observed several times in \refch{objectdrop}, recoverability is the \textit{sine qua non} of object omission. \textcite{Cote1996} even went as far as to identify it as "the only absolute constraint on null arguments". Intuitively, this notion can be used to tell apart definite implicit objects (whose meaning is recoverable from context, be it extra- or intra-linguistic) and indefinite implicit objects (whose meaning is recoverable from the semantics of the verb itself). Even authors who cautiously \parencite{Resnik1993, Resnik1996, OlsenResnik1997} or openly \parencite{Glass2013, Glass2020, glass2022english} reject a distinction between definite and indefinite object drop still maintain a certain notion of recoverability as a fundamental requirement for object drop. Let us discuss this in some more detail.\\
Decades of literature on the matter teem with pre-theoretical definitions of object recoverability. The oldest is in \textcite[105]{Ohlander1943}, where the author argues that object-less utterances "may appear complete enough" by virtue of the fact that "the element to be understood or supplied is so self-evident that the gap is mentally filled in by the audience more or less unreflectingly". Like many others, Ohlander makes reference to the notion of recoverability without using this exact wording. Similarly, \textcite{HickmanEtAl2016} resort to the idea of "conceptual defaultness", i.e. the property of unmentioned direct objects whose omission does not depend on an informational failure on the part of the speaker/writer \textemdash on the contrary, defaults are omitted to comply with the Gricean maxim of quantity, since their very mention "would be unnecessary and, perhaps, awkward" \parencite[516]{HickmanEtAl2016}.\\
Interestingly, recoverability appears to be such an intuitive notion as to become a major determinant of argument omission (not limited to indefinite object drop) in the early stage of grammar acquisition crosslinguistically, even in context and languages where adult grammar would normally prohibit it \parencite{allen2000discourse, RatitamkulEtAl2004, Medina2007, sopata2016null, Rasetti2003, PerezLerouxEtAl2011, PerezLerouxEtAl2013, Perez-LerouxEtAl2018, OGradyEtAl2008, Ingham1993}. Young children are shown to omit objects (and other arguments) of verbs they are exposed to, both in transitive and intransitive utterances, if extra-linguistic context makes them sufficiently recoverable.


\paragraph{Between lexical and contextual recoverability}

In \textcite[7]{Kardos2010}'s words, omitted objects are recoverable "either through lexical stereotypes or based on the context". In a sense, it is not even necessary to postulate a binary distinction between the two facilitators of recoverability, since "lexical stereotypes" (i.e. the selectional preferences of a verb) descend from world knowledge, situational context is where meaningful conversations happen, and narrower context enables more object omissions without contradicting world knowledge (more on this in \refch{objectdrop}, and in \textcite{Glass2013, Glass2020, glass2022english}, where recoverability is intended as a matter of degree). A prime example of this is \ref{berghplay}, taken by \textcite[24]{BerghOhlander2016}, whose full interpretation depends on additional context. Indeed, this sentence in isolation has no single interpretation. \textit{What} did they play beautifully? Was it an instrument or a game? And what kind, exactly? However, while context would make it possible to know the referent of this indefinite implicit object, the semantics of the verb (in particular, its selectional restrictions) still provide us relevant information without the need for additional context. Indeed, we know that they played either a game or a musical instrument. This would not be possible, for instance, with a selectionally un-restricted (or, better, very loosely restricted) verb such as \textit{to make}.

\ex. \label{berghplay} They played beautifully.

This also holds true for verbs with much stricter selectional preferences than \textit{to play}, e.g. the verb \textit{to eat}, to which I devoted some pages in \refch{objectdrop}. As noted by \textcite[149]{Cote1996} among others, intransitive \textit{to eat} tends to refer to a meal (which is the prototypical item humans eat), but it does not have to. In context-rich utterances, the actual referent may be different, depending on "the underlying context and intentional structure of the discourse structure at the time of utterance". Crucially, context may make it obvious that the omitted Patient is some specific kind of edible item (e.g. pasta, hamburgers, or even something as extravagant as Hawaiian pizza), but it cannot make it deviate from the basic selectional preferences of the verb\sidenote{Unless, of course, the verb takes a specialized meaning in a specific sub-genre, e.g. \textit{to eat} in a game of chess would mean \textit{to capture enemy pawns}.} \textemdash the omitted Patient has to be something edible.\\
Semantically-licensed recoverability also interacts with world knowledge, specifically with societal norms, in \textcite{Goldberg2005}. In this case, the author argues that politeness is a driving factor in the omission of direct objects occurring with verbs of bodily emission, since they are very imageable (hence, recoverable), but also very taboo in usual social contexts. On the other hand, it would be easy to imagine that in contexts where such verbs are not taboo (i.e. in medical literature, or among \textit{aficionados} of certain sexual kinks) the main driver of object omission would be contextual inferability, rather than misguided concerns for politeness. Either way, context and world knowledge (about the verb itself, but also about proper customs) comply with verbal semantics.\\
Consistently with all the previous observations on recoverability, \textcite{Mittwoch2005} and \textcite{Glass2013} also observe that the referent of indefinite implicit objects corresponds to the literal meaning of the verb, rather than to metaphorical or idiomatic meanings. For instance, intransitive \textit{to read} refers to "written or printed material rather than, say, the stars or coffee grounds" \parencite[2]{Mittwoch2005}. Likewise, when a verb has strict selectional preferences (e.g. \textit{to eat} selects for edible items) and one of its near-synonyms has broader preferences\sidenote{An in-depth discussion of such verb pairs will be tackled in \refsec{mannerspec}.} (e.g. \textit{to devour} selects for edibles, but also for metaphorical items such as books), direct objects are much more likely to be dropped with the former than with the latter \parencite[5]{Glass2013}. Clearly, this mechanism is in place to maximize recoverability, since the literal selectional preferences of a verb are known and predictable (based on lexical semantics and world knowledge), while its metaphorical or idiomatic behavior is largely arbitrary and unpredictable. More in general, the more an object is semantically dependent from a verb, the more likely it is to be omitted \parencite[203-204]{Rice1988}.


\paragraph{Semantic selectivity as a proxy to recoverability}

Let us now look in more detail into semantic selectivity as the main verb-internal, context-independent factor allowing for object recoverability. So far in this Section, I made the case that knowing the specific type of objects a verb favors in its selectional preferences is the first step towards object recoverability and, consequently, felicitous object omission. World knowledge and context shape the way we process object-less transitive verbs when accessing their selectional preferences. However, as hinted before, there is a close correlation between indefinite object drop and the breadth of a given verb's selectional preferences, regardless of the actual items or family of items it tends to occur with \parencite{Garcia-VelascoMunoz2002, Liu2008, Glass2020, Medina2007, MaoueneEtAl2011, OlsenResnik1997, Resnik1993, Resnik1996}. The intuition behind the use of a verb's selectional preferences as a means to gauge the recoverability of its objects stems from the observation that indefinite implicit objects "are clearly understood because they are inferred from a very narrow, if not exclusive, range of possibilities" \parencite[4]{Garcia-VelascoMunoz2002}. This is the reason why native speakers of English are more likely to find indefinite object drop grammatical with \textit{to read} than with \textit{to know}, since there are way fewer readable than knowable things in our lives \parencite[302]{Liu2008}.\\
\textcite{Resnik1993, Resnik1996} was the first to provide a more data-grounded definition of recoverability, by means of a computational model of a verb's selectional preferences. In particular, his Selectional Preference Strength taxonomy-based measure is shown to be inversely proportional to the semantic narrowness of a verb's selectional preferences, so that a verb will receive a higher score if its direct objects are semantically similar (e.g. \textit{to eat, to read}), and a lower score if they are semantically different (e.g. \textit{to make, to know}). I will present the mathematical details of Resnik's measure, discuss the implications of such an approach, and propose my own distributional semantics-based alternatives in \refsec{predictor_sps}, where I expand upon both my Preference In Selection of Arguments (PISA) computational measure presented in \textcite{CappelliLenciPISA} and on my behavioral variant of Computational PISA, inspired by \textcite{Medina2007}'s Object Similarity measure. An important implementation-related aspect to note here, common to all these measures of semantic selectivity used as proxies to object recoverability (Resnik's SPS, Medina's OS, my own PISAs), is that recoverability is modeled as being gradient, and these models capture the semantic narrowness/breadth of the semantic categories\sidenote{This is less true for semantic similarity measures based on behavioral judgments, such as Object Similarity and Behavioral PISA, but still not completely off the mark. Please refer to \refsec{predictor_sps} for a full account of such measures.} the potential direct objects of a verb belong to, rather than focusing on the specific objects themselves.\\
Most importantly, with his computational experiment \textcite[88]{Resnik1993} could conclude that recoverability, as quantified via gradient semantic selectivity, is a necessary (albeit insufficient) condition for object omission. This conclusion, consistent with pre-theoretical intuitions about recoverability and previous theory-informed, non-experimental statements about the role of selectional preferences in determining object recoverability, stems from the observation that object-less transitive verbs never receive low semantic selectivity scores\sidenote{However, some verbs high in semantic selectivity fail to license indefinite object drop. Resnik explains this apparent failure of his measure with reference to the aspectual properties of such verbs. I will come back to this in \refsec{aspectualfactors} and \refch{predictors}.} in Resnik's experiment.


\subsection{Agent affectedness} \labsec{agentaffect}

Back on \refpage{affectedagent}, I argued with plenty of references to relevant literature (first and foremost, \textcite{Naess2007}) that verbs whose Agent is in some ways affected by the action (e.g. \textit{to eat, to learn}) described by the verb tend to be more likely to participate in the indefinite implicit object construction than unaffected-Agent transitive verbs (e.g. \textit{to kill, to break}). This is a direct consequence of the need for the arguments of prototypical transitive verbs to be maximally distinct from a semantic point of view, captured in \textcite{HopperThompson1980}'s parameters H (agency) and I (affectedness of the object)\sidenote{Refer back to \reftab{ht1980_parameters}.}, and later on in \textcite{Naess2007}'s Maximally Distinguished Arguments Hypothesis.\\
I will not go again over the finer details of the affected-Agent account of object drop, since I already devoted sufficient space to it in \refch{objectdrop}. For the purposes of the review of the main factors facilitating indefinite object drop provided in this Chapter, I will just mention that Agent affectedness can manifest in two ways. One is inherent to the semantics of the verb, as it happens with ingestion verbs such as \textit{to eat}, \textit{to drink}, and, in a sense, \textit{to learn}. The other is instead context-dependent, as in \ref{agentaffect1}, where the verb \textit{to murder} gets an affected-Agent interpretation and thus participates in felicitous object drop due to the purpose clause \textit{for the money}.

\ex. \label{agentaffect1} John murdered for the money. \hfill \parencite[136]{Naess2007}

It is important to note, as seen before in this Chapter and \refch{objectdrop}, that context enables verb behavior that is already possible, virtually, thanks to the semantics of the verb itself. After all, does a murder not affect the murderer even without mention of the cause? Stating it explicitly puts the focus on the motive, putting in the background both the Patient (which was already backgrounded, due to being unmentioned) and the murdering activity itself (which would be the focus of cause-less \textit{John murdered}). Crucially, I would like to point out that it is not possible to use purpose clauses to induce an affected-Agent reading on object-less transitive verbs lacking this possibility in their semantics, e.g. the verb \textit{to build} in \ref{agentaffect2}.

\ex. \label{agentaffect2} *John built for the money.

Indeed, \ref{agentaffect2} would only be considered grammatically acceptable provided it is inserted in a larger context (just like plain \textit{*John built}). However, one could object to this account of the verb \textit{to build} making reference to the effected-Patient account I discussed in \refch{objectdrop} together with the affected-Agent account \textemdash the object of \textit{to build} comes into existence via the act itself of building it, unlike the Patient of \textit{to murder} (which was very much existing before the murder, and ceased to do so due to it). Why does the verb \textit{to build} not allow for indefinite object drop, even though it is a handbook effected-Patient verb just like \textit{to bake}? Based on \textcite[512]{Goldberg2001} and \textcite[139]{Naess2007}, \textit{to bake} and \textit{to build} are actually more different than it would seem at first glance, since verbs like the latter (e.g. \textit{to break}) refer to events whose interpretation strictly depends on the Patient itself. In other words, while it is possible to imagine a baking event without having a precise baked good\sidenote{A baked good which, we will remember from \refch{objectdrop}, is much more likely to be bread or some pastry rather than rotisserie chicken.} in mind, it is impossible to picture a breaking or building event without having a precise broken or built object in mind. Without going into idiom territory, where one could "break a bank note" to get change or "build one's hope", it is clear that the act of breaking a glass is quite different from the act of breaking a leg, just as building a sand castle is quite different from building an airplane. Thus, these examples go to show that recoverability (introduced in \refsec{recoverability}) is the preminent factor in determining object drop, and neither Agent affectedness nor Patient effectedness cannot overpower it.


\subsection{Manner specification} \labsec{mannerspec}

Manner specification is a tricky semantic predictor of indefinite object drop to define, due to the different interpretations the concept of "manner" received from different authors. This word is used, fundamentally, in two different ways:
\begin{itemize}
    \item to refer to "semantically marked" counterparts of "semantically neutral" verbs, e.g. \textit{to devour, to nibble} with respect to \textit{to eat} \parencite{Naess2007, FellbaumKegl1989taxonomic, Rice1988};
    \item in contrast with "result", to separate "manner" activity verbs such as \textit{to sweep} from "result" causative verbs such as \textit{to break}\sidenote{The concept of "manner", here discussed in relation to transitive verbs, is also central in studies about motion verbs \parencite{Iwata2002, CennamoLenci2019, BeaversEtAl2010}.} \parencite{BeaversKoontzGarboden2012, BeaversKoontzGarboden2017, Beavers2013, RappaportLevin1998building, RappaportHovavLevin2005, LevinRappaportHovav2008, RappaportHovavLevin2010}. % per indicare, sostanzialmente, activity vs accomplishment/achievement (AL diceva che poteva essere collineare con telicity!)
\end{itemize}

In this thesis, as I will also argue in \refsec{predictor_mannspec}, I am only interested in the first sense of the word "manner". However, since the two senses overlap in significant ways, making it necessary to keep them well separated, I will also comment the second sense in some detail.\\
With respect to indefinite object drop, \textcite{RappaportLevin1998building, RappaportHovavLevin2005, LevinRappaportHovav2008, RappaportHovavLevin2010} argue that verbs expressing manner in their meaning, such as \textit{to eat}, are much more likely to allow for object drop than verbs expressing result, such as \textit{to devour}. In particular, \textit{to devour} is considered a result verb because it entails complete consumption of the Patient by the Agent, unlike \textit{to eat}, at least in an unmarked, uninterrupted scenario \parencite{Melchin2019, smollett2005quantized, pinon2008aspectual}. This idea\sidenote{Which \textcite{Goldberg2001}, praisingly quoted by \textcite[539]{Onozuka2007}, opposes on the basis of the aforementioned Principle of Omission under Low Discourse Prominence.} is famously exemplified by \textcite{RappaportLevin1998building} with the examples in \ref{resultmanner}. The rationale behind this account is that result verbs specify scalar change (see \ref{resultmanner2}), while manner verbs specify non-scalar change (see \ref{resultmanner1}). Crucially, the entity changing along the scale specified by result verbs (i.e. the Patient object) is argued to be ungrammatical to omit, giving rise to a test for result-lexicalization used by \textcite{BeaversKoontzGarboden2012} and \textcite{Rissman2016}.

\ex. \label{resultmanner} \a. \label{resultmanner1} Phil swept.
\b. \label{resultmanner2} *Tracy broke.

In such an account, manner and result are to be considered complementary, in that a verb can only lexicalize one of them. However, \textcite{LevinRappaportHovav2008} also observe that there is some understood manner component in many result verbs (e.g. the result of \textit{to clean} is achieved by acting in a specific manner), and likewise, some understood result component in many manner verbs (e.g. \textit{to scrub} is a manner of cleaning, that will likely generate cleanliness of a surface as a result). This particular observation serves to bridge the gap between this account, where lexicalized manner leads to felicitous object drop, to the other account, where an overt, specified manner component blocks object drop. \textcite[5]{BeaversKoontzGarboden2012} even make the case that so-called "poison verbs", a sub-class of manner-of-killing verbs identified by \textcite[230-233]{Levin1993} in opposition to "murder verbs", actually entail both manner and result, \textit{contra} \textcite{LevinRappaportHovav2008, RappaportLevin1998building}.\\
The other interpretation of the concept of "manner", which is the one I employ in my experiments (\refch{predictors} and \refch{judgments}) and probabilistic model of object drop (\refch{results} and \refch{model}), is offered by \textcite{Rice1988}, \textcite{FellbaumKegl1989taxonomic}, and \textcite{Naess2007}, among others. These authors argue that the impossibility of omitting the object with verbs like \textit{to devour} is explained by the presence of a manner component in their meaning (see also \textcite{Garcia-VelascoMunoz2002} for further considerations). In other words, while \textit{to eat} (which \textcite{Rice1988} calls a "semantically neutral" verb) is a base verb referring to a general activity, \textit{to devour} (which \textcite{Rice1988} calls an "action-plus-manner" verb) has an additional manner specification in that it refers to a particular \textit{manner} of eating. Most importantly, \textcite[49-50]{Melchin2019} shows that this distinction holds crosslinguistically with examples in French, Dutch, and Arabic. The same also goes, for instance, for \textit{to guzzle, to chug} with respect to basic \textit{to drink}, and moving from transitive to motion verbs, for \textit{to saunter, to stride} with respect to basic \textit{to walk}.\\
\textcite[139]{Naess2007} provides an intriguing link between manner specification and the affected-Agent account discussed in \refch{objectdrop} and \refsec{agentaffect}. In particular, she observes that indefinite object drop is infelicitous with manner-specified verbs because they typically refer to the way in which the Patient (crucially, not the Agent) is affected, in true prototypical transitive behavior. However, manner specification in the verb root stops being an obstacle to object drop if proper context is provided to imply Agent affectedness, as in her example in \ref{affectedmanner}.

\ex. \label{affectedmanner} The dinner was delicious, but Jane had no appetite and only nibbled.

Bringing Agent affectedness into the equation can also solve a decades-old conundrum by \textcite{FellbaumKegl1989taxonomic}. Why do \textit{to mush, to nosh, to graze} allow for indefinite implicit objects while \textit{to gobble, to gulp, to devour} do not, despite them all being manner-specified troponyms\sidenote{I am using terminology from \textcite{FellbaumKegl1989taxonomic}. Troponymy is a relation among verbs akin to what hyponymy is for nouns, "although the resulting hierarchies are much shallower" \parencite{Miller1995}.} of \textit{to eat}? Based on everything I observed about manner specification so far, both the first and the second group of verbs should block indefinite object drop on the basis of their manner component. In their taxonomic account, \textcite{FellbaumKegl1989taxonomic} explain this issue by positing two lexical entries for \textit{to eat}\sidenote{Refer back to \refsec{theory_entries} for a detailed discussion of whether or not to have two separate lexical entries for transitive verbs used transitively and intransitively.}, one meaning roughly "to eat a meal" and another meaning "to ingest food". The first entry would be intransitive, and its manner-specified troponyms (\textit{to mush, to nosh, to graze}) are too. The second entry would instead be strictly transitive, and its manner-specified troponyms (\textit{to gobble, to gulp, to devour}) are too. As I argued extensively in \refsec{theory_entries}, positing two separate lexical entries does little more than describe a state of affairs, without actually providing substantial explanatory power to the discussion. I would instead explain the difference in transitivity between these two groups of manner-specified troponyms of \textit{to eat} with reference to the affected-Agent analysis. In particular, verbs like \textit{to mush, to nosh, to graze} are activity verbs with a clear focus on the way the action affects the Agent (just like plain \textit{to eat}), and indeed they do allow for their object to be dropped. On the contrary, verbs like \textit{to gobble, to gulp, to devour} tend to highlight the affectedness of the Patient, making it necessary to express it overtly with a direct object in the syntax.\\
Interestingly, \textcite[207]{Rice1988} notes in passing that "verbs that are very neutral, but that furthermore sustain a wide variety of complements, tend always to require objects", considering the ungrammaticality of intransitive \textit{to love} as an example. Once again, object recoverability (as an effect of a transitive verb's semantic selectivity) is shown to be prominent with respect to other drivers of indefinite object drop.



% *Melchin (2019: 47-48)*  
ribattere a ognuna di queste obiezioni di jackendoff!
% CONTROESEMPIO ALLA MANNER SPECIFICITY
% > Jackendoff (2002, p. 134, fn. 15) explicitly argues that the differences between these verb
% classes is, in fact, essentially due to arbitrary lexical specifications of the verbs. He notes that
% it is tempting to argue that whether a verb’s object will be obligatory or optional depends
% on the specificity of the verb’s semantics, so that verbs with a narrow meaning (like devour)
% will require an object, while for verbs with a wider meaning (like eat), the object will be
% optional. Jackendoff counterexemplifies this claim with pairs of verbs like serve and give,
% where the former may drop its indirect object while the latter may not, despite the fact that
% serve seems like a more specific counterpart to give. Jackendoff also raises the examples of
% juggle (six balls) and flirt (with Kim), stating that these verbs have highly specific semantics
% yet may drop arguments


% *Melchin (2019: 51)* 
citare melchin 2019 nella grande parentesi iniziale dove parlo degli studi manner-vs-result
% (ATTENZIONE che per Melchin la "manner specification" è abbastanza diversa dalla versione naive!)  
% >  Part of the meaning of eat is that the event denoted is undertaken by
% an agentive causer. According to the theory that I develop below, this means that it is a
% Manner verb.  [...] I claim that if a verb expresses Manner, then its agentive subject must be overtly present
% in the syntax, while if a verb expresses Result, then the argument of which the result state
% is predicated must be present. 

% *Melchin (2019: 52)*  
% > there are dynamic verbs, like clean, which are specified for neither Manner and Result and
% therefore may express either. 

% *Melchin (2019: 59)*  
% > In this section I propose that argument omission is constrained by whether or not the predicate
% in question expresses Manner or Result, components of lexical meaning defined in terms of
% the relationships of the core arguments to the event denoted by the predicate. I borrow the
% terms “Manner” and “Result” from Rappaport Hovav and Levin (1998, 2005, 2010), though I
% define them in a somewhat different way than they are used in those papers. A verb that
% expresses Manner requires that its external argument be an Agent of the event; a verb that
% expresses Result requires that its internal argument undergo some scalar change resulting
% from the event. When a verb expresses Manner (i.e., it is a “Manner verb”), the agentive
% subject cannot be omitted or replaced in the sentence, and likewise, when a verb expresses
% Result (a “Result verb”), the object undergoing the scalar change must be overtly expressed

% *Melchin (2019: 62)*  
% > Manner and Result, as characterized by Rappaport Hovav and Levin, are not mutually
% exclusive in the event structures of sentences; this is demonstrated in (20), in which the
% matrix verb introduces the Manner, and the resultative PP or AP provides the Result. What
% they claim is that an individual verb root can only express one or the other [...]  Beavers and Koontz-Garboden
% (2012, 2017) have shown evidence of a group of verbs, including many manner-of-killing and
% manner-of-cooking verbs, that express both Manner and Result, contrary to the predictions of
% Rappaport Hovav and Levin; 

% *Melchin (2019: 63)*  
% MANNER = AGENT  
% >  I propose that the Manner component of verb meaning is an
% entailment that the event is brought about by a participant acting in a particular manner
% – that is, an entailment that there is an Agent in the event. This entailment is implicit in
% Rappaport Hovav and Levin’s event-schematic characterization of Manner, shown in (22)
% above, in which the Manner component modifies an ACT predicate; however, they do not
% explicitly link Manner to the Agent role. Furthermore, I propose that the entailed Agent
% must be realized, and may not be omitted. 
% The link between Manner and agentivity is made by Beavers and Koontz-Garboden (2012),
% who show that while non-Manner verbs, such as the Result verbs break and shatter, allow
% instruments and inanimate causers as subjects, Manner verbs like scrub and brush do not,

% *Melchin (2019: 64)*  
% > A similar observation is made by Reinhart (2002, 2010), who links the property of requiring
% an Agent to the causative-inchoative alternation. Specifically, she notes that verbs that
% require an Agent do not undergo this alternation, while verbs that do not require an Agent
% may do so.

% *Melchin (2019: 65)*  
% POSSIBLE TEST FOR MANNER-HOOD
% > Thus, Manner can be characterized as the requirement that there is an Agent, or a
% participant acting in a certain manner, in the event structure.  Furthermore, this Agent must
% be expressed as the external argument of the verb, and may not be omitted, as shown in (27),
% or replaced with a non-agentive external argument, as shown in (25). This allows a more
% precise and testable definition of Manner than that of Rappaport Hovav and Levin (2010) as
% non-scalar change; rather than involving a change that cannot be characterized as scalar, it
% involves entailments about the role one of the participants plays in the event, and is made
% testable by the fact that this participant must be expressed as the external argument.

% *Melchin (2019: 66)*  
% WHAT IS AN AGENT
% > extend the definition of agentive causation to include any case where the force
% causing the event is spontaneously generated by inherent properties of the causing entity itself,
% and sustained throughout the event. This notion is referred to by Folli and Harley (2008) as
% teleological capability. 

% *Melchin (2019: 66)*  
% RESULT = THEME = OVERT DOBJ
% >  a Result verb entails that the Theme argument undergoes a scalar change in its
% entirety. Furthermore, as with the Agent entailed by a Manner verb, this Theme must be
% realized as a syntactic argument of a verb, in this case the internal argument. 
% When I claim that the Theme must undergo a scalar change “in its entirety”, I do not
% mean that the change must reach some specified endpoint or that the event must be telic;
% instead, the change (to whatever extent it occurs) must affect the entire entity denoted by
% the Theme. 

% *Melchin (2019: 71)*  
% DEVOUR = MANNER+RESULT
% > Now that Manner and Result are characterized in terms of agentivity and themehood in
% the event, rather than scalar versus nonscalar change, one can raise the question of what the
% properties of a verb with both of these meaning components would be. The event described
% by such a verb would require an Agent to act in a certain manner to bring it about, and
% would result in a scalar change affecting another participant. Syntactically, such a verb would
% not allow the omission of either the external argument or the internal argument; both would
% need to be realized in a sentence containing the verb. Rappaport Hovav and Levin (2010)
% claim that no such verbs exist, and that Manner-Result Complementarity holds (which they
% explain using the Lexicalizaton Constraint). However, in the next section, I show that devour
% is just such a verb.

% *Melchin (2019: 75)*  
% >  eat can take a resultative XP predicated by a non-selected argument, while
% devour cannot,  
% (44)  John ate the restaurant out of business.  
% (45) #The lion devoured the zoo out of business.

% *Melchin (2019: 77)*  
% CONSEGUENZE GROSSE SULLA TELICITY!
% > Given the claim above that devour is a Result verb and eat is not, this predicts that
% an event of devouring must maximally affect the edible parts of the Theme, while events
% of eating do not necessarily maximally affect the relevant parts. The standard view in the
% literature (e.g., Verkuyl 1972, 1993; Tenny 1994; Jackendoff 1996; Krifka 1998) holds that
% this is not the case, and that incremental theme verbs uniformly require that a quantized or
% bounded object be maximally affected by the event – in present terms, that all incremental
% theme verbs, including both eat and devour, have the Result component. This means that if
% the Theme is quantized or bounded (in the sense discussed in Footnote 6) the event will be
% telic. However, Smollett (2005) challenges this view, claiming that sentences with incremental
% theme verbs such as eat and build may be atelic even if the Theme is quantized, though
% this reading may be difficult to access without some contextual support (see Piñón 2008 for
% further discussion of this claim).

% *Melchin (2019: 84)*  
% > Thus, while clean is not necessarily specified for
% either Manner or Result, it must take on one or the other meaning in a sentence.
% A possible alternative to treating clean as underspecified for Manner and Result is to
% analyze it as lexical ambiguity between a Manner clean and a Result clean. However, there
% is a conceptual advantage to the approach taken here. 

% *Melchin (2019: 86)*  
% FONDAMENTALE!!! TABELLA CON TYPOLOGY 2X2
% > verbs marked with a negative value for one of the features can still express that 
% component of meaning, but verbs marked positive must necessarily express that meaning.  
% Table 3.1: Typology of Manner and Result [...]  
%  the typology in Table 3.1 can be explained using two principles:
% (i) a transitive dynamic predicate must express at least either Manner or Result; (ii) some
% verbal roots, by virtue of their meaning, necessarily express Manner or Result, or both (i.e.,
% they require that their external argument be an Agent, and/or that their internal argument
% undergo scalar change). Principle (i) is likely a consequence of the nature of dynamicity;
% an event involving one or more participants is not dynamic if there is no participant acting
% agentively, or undergoing some change (or both). 19 Principle (ii) is a consequence of the lexical
% semantics of certain verbs; verbs specified for Manner or Result will satisfy (i) vacuously. 
% The [– Manner, – Result] verbs are simply those transitive verbs which do not satisfy (ii),
% but which must take on one or the other feature in order to appear in a dynamic sentence.

% *Melchin (2019: 87)*  
% > for ambiguous verbs ([– Manner, – Result]), in addition to clean, there is cut, as mentioned
% above and extensively discussed in Levin and Rappaport Hovav (2013, 2014). 

% *Melchin (2019: 89)*  
% > We have now seen that devour is not alone as a Manner-Result verb; it is joined by certain
% verbs of manner-of-killing. I have also shown here that the set of verbs that has been classified
% as “manner-of-killing” by Beavers and Koontz-Garboden (2012) and others cited therein is
% not a unified class,

% *Kardos (2010: 7-8)*  
% Based on similar evidence, Levin and Rappaport Hovav revised their theory in their 2004
% paper in the following way. On the one hand, they still bolster the view that events denoted by
% eat and sweep consist of two subevents, namely, the action carried out by the eater (or
% sweeper) on the one hand, and the gradual disappearance of the apple, or the gradual
% “becoming swept” of the room. On the other hand, however, they note a significant difference
% between this event structure and that of truly complex causative verbs. They posit that the two
% subevents of an eating or a sweeping event occur at the same time and cannot be clearly
% separated since, as they put it, they are temporally dependent on each other. Therefore, they
% refrain from calling these true complex events, claiming that this duality can be grasped only
% conceptually, which explains why these events are simple from a linguistic point of view. In
% particular, the fact that the object of these verbs can be dropped, and that non-subcategorized
% objects can combine with them (see section 2), is regarded as evidence for the “simple event”
% character of eat-type verbs.


\section{Aspectual factors} \labsec{aspectualfactors}

% *Lorenzetti (2008: 66)*  
% > Rice [1988] contends that the manner component adds a certain degree of specificity,
% which makes the verb lose its basic status. The impossibility of omission may be then
% considered to be related to specific semantic components shared by verbal sets, which might
% foster or forbid omission. In addition to the manner component, a feature like completion can
% also make a verb incompatible with object omission. By contrast, the duration component is
% frequently associated to object omission.

% *Dvorak (2017: 115)*  
% SPOSTARE IN TELICITY/PERFECTIVITY!  
% > The issue
% here is that once the verb is in the -ing form, describing an ongoing event, the postulated
% distinction between activities and accomplishments is overridden. This is confirmed by the
% fact that all progressivized process verbs behave like activities for the purpose of Dowty’s
% (1979) classical tests distinguishing accomplishments from activities.

% Medina 224, in nota
% Finally, and perhaps most importantly, children under three years have been shown to distribute
% imperfective / perfective aspect in accordance with telicity in both production and comprehension (Olsen et al., 1998; Wagner, 2001; Weist, 2002; Weist et al., 1984). Thus, rather than assessing the relationship
% between imperfective / perfective aspect and the use of implicit objects, what may actually be being
% % indirectly assessed is the relationship between telic / atelic aspect and implicit objects.

\subsection{Telicity} \labsec{telicity}

Contents

% *Newman & Rice (2006: 5-6)*  
% > Van Valin and LaPolla (1997:112) explicitly remark that
% “...eat is not inherently telic, unlike kill and break; hence it must be
% analyzed as an activity verb, with an active accomplishment use”. For
% them, the ‘activity verb’ use (He ate, He ate spaghetti for ten minutes) is
% the ‘basic’ meaning of EAT.

% *Naess (2007: 77)*  
% SPOSTARE IN CAPITOLO SU TELICITY! (CONTINUA DOPO, CREDO)  
% > Of the properties included in Hopper and Thompson’s list of Transitivity pa-
% rameters, the most obvious candidate for an explanation for the intransitive be-
% haviour of ‘eat’ verbs would appear to be parameter C, “aspect” – telic vs. atelic. As
% Hopper and Thompson’s Transitivity notion is explicitly defined as a property of
% clauses, not all their parameters are directly applicable to individual verbs; but for
% those that are, the verb ‘eat’ must be ranked as “high” at least for parameters A, B,
% E, H, and I: it involves two participants, denotes an “action”, is necessarily voli-
% tional, and takes an A which is “high in potency” (human or animate) and a high-
% ly affected O. Telicity, on the other hand, has sometimes been appealed to as an
% explanation for the aberrant behaviour of ‘eat’ verbs. Thus Van Valin and LaPolla
% characterise ‘eat’ as “not inherently telic” (1997 :112), while Tenny (1994) main-
% tains that the telicity of ‘eat’ depends on the “delimitedness” or “non-delimited-
% ness” of its measuring argument, that is, its object: “If Chuck eats an apple, he fin-
% ishes eating when the apple is gone, but if he eats ice cream, he continues eating for
% an indefinite period of time, because there is an indefinite quantity of ice cream.
% (He may even continue eating ice cream forever, if he is in a world that never runs
% out of ice cream)” (Tenny 1994 :24).
% IF HE LIVED IN A UNIVERSE BLESSED WITH NEVERENDING ICE-CREAM
% Quite apart from the fact that the real world does not actually work this way

% *Naess (2007: 78-79)*  
% COME SI CONCILIA QUESTO CON MEDINA? PER LEI EAT È TELICO O ATELICO?  
% > It is well known that telicity depends not only on the predicate of a clause but
% also on the nature of its object, if it has an object, so that bare-plural or mass noun
% objects lead to atelic readings (Dowty 1991): John built a chair in an hour/*for an
% hour vs. John built chairs for an hour/*in an hour. This variation is not specific to
% ‘eat’ but is apparently characteristic of verbs with incremental themes (Dowty
% 1991, Tenny 1994). The interesting question in the case of ‘eat’ is how this verb
% behaves when it is used intransitively, i.e. when the presence of different kinds of
% objects cannot influence the reading.
% Intransitive ‘eat’ is in fact perfectly compatible with an adverbial of comple-
% tion: I ate in five minutes, then rushed off to work. This clearly shows that intransi-
% tive ‘eat’ does in fact have a delimited – that is, a telic – reading; and since no object
% is present, this telic reading cannot derive from the “delimitedness” of the object.
% Rather, the telic reading arises from the affectedness of the agent. [...]  
% On the other hand, we also find ‘eat’ with adverbials of duration: We ate all
% evening. This is a rather striking alternation which to my knowledge is relatively
% rare with intransitive verbs, though it does occur with at least two other types of
% English intransitives. One is certain “reflexive” verbs of body care such as shower
% and bathe, which can also be construed either with adverbials of completion or of
% duration: I showered in five minutes or I showered for half an hour. The completive
% reading here implies a particular result state of the agent; I showered in five minutes
% means that it took me five minutes to attain the desired degree of cleanness, where-
% as I showered for half an hour only means that I spent half an hour standing under
% the shower. In the latter case, no result state is entailed; the implication of clean-
% ness can be cancelled by a sentence such as I showered for half an hour but I still
% couldn’t get all the dirt off my skin.
% Secondly, the alternation is shown by at least some verbs whose subject under-
% goes a process which is conceived of as typically leading to a result state, but which
% is in principle independent of this result state; that is, it may be halted before the
% state is achieved or extended past the point where the state is achieved, while still
% being essentially the same process. This is the case for the verbs cook and bake; [...]  
% What these different verbs have in common is that they all refer, when used with
% an adverb of completion, to a result state of the subject argument; that is, their sub-
% jects are affected by the action in question. Once again, then, eat here patterns with
% verbs which are characterised by affecting their subject argument; reflexive verbs
% such as shower and bathe and patient-subject verbs like (intransitive) cook and bake. [...]  
% Furthermore, the possibility of a telic reading for intransitive ‘eat’ is unusual for
% intransitive uses of verbs that may delete their objects in English; we may say Dad
% cooked dinner in half an hour or Dad cooked for half an hour but not *Dad cooked
% in half an hour; Mary sewed a dress in an hour or Mary sewed for an hour but not
% *Mary sewed in an hour.

% *Naess (2007: 79)*  
% TELICITY BOCCIATA COME FATTORE PER EAT VERBS  
% > What is relevant is that the verb has a
% telic reading even when it is used intransitively. In other words, the intransitive use
% of ‘eat’ verbs cannot be explained by appealing to the atelicity of such verbs: if ‘eat’
% can be used intransitively because it has a low value for the parameter of telicity,
% then we would expect an intransitive clause with ‘eat’ to necessarily have an atelic
% reading. But if a verb may be telic and still occur in intransitive contexts, then it can-
% not be the telic-atelic parameter which is the source of its reduced transitivity.

% # Telicity (lexical aspect)

% *Medina (2007: 27)*
% > What does telicity have to do with implicit objects?  The idea pursued in this dissertation is that, since a direct object often specifies what constitutes the inherent or natural endpoint of a telic event, telic verbs resist omitting an expression of that endpoint.

% *Medina (2007: 69)*
% > The approach taken here is based on Olsen’s (1997)analysis of telicity as a privative feature. 
% That is, telicity is semantic and no additional constituents can cancel the denotation of an event with an inherent end or goal, 
% while atelicity is a cancelable conversational implicature that allows either the interpretation of having or not having an inherent end.  
% Olsen notates this with a positive feature to indicate telicity [+ Telic], but instead of a negative feature indicating atelicity [-Telic], she indicates
%  the absence of telic denotation as [0 Telic].  What thissuggests is that if the predicate specified in the input is telic, then the only interpretation 
%  possible will be that of an inherent endpoint.  If the input is atelic, both interpretations may be possible depending on the particular overt object 
%  that is used (if any).Crucially, while an atelic predicate may take on a telic interpretation given the addition of a bounded object, a telic predicate 
%  can not be made to have an Atelic interpretation.
 
%  *Olsen & Resnik (1997: 2)*  
% PARLANDO DI HOPPER & THOMPSON (1980)  
% > H&T's A SPECT feature covers two types of aspect: telicity (the inherent
% boundedness of events denoted by verbs, e.g. win, in (4a)) and perfectivity (the
% completion of an event, as in (4b)).  
% (4)a. Benjamin will win (the race). (cf. run)  
% b. Benjamin had won/run the race. (cf. was winning/running)  
% They adopt the perfective/imperfective terminology in their discussion of the
% transitivity hypothesis, but primarily use telicity in their discussion of the
% discourse function of transitivity (H&T:1980:271). 

%  > *Olsen & Resnik (1997: 3)*  
% > The relationship between telicity and the presence of an overt object has
% been much discussed (cf. Dowty 1979, van Hout 1996, Tenny 1994). According
% to Tenny (1994:11), "delimitedness" governs most of the mapping between
% lexical and syntactic structure, with arguments that measure delimitedness
% appearing as direct objects. Van Hout (1996:123) makes a stronger claim, that
% telic verbs "require projection of an argument in direct object position." Data cited
% in support of these hypotheses include the spray/load constructions, in which the
% direct object argument measures the event: in (6a) the truck is full (whether or not
% the hay is gone), and in (6b) the hay is all on the truck (whether or not the truck is
% full). In contrast, the event in (6c) is not bounded at all. 2  
% (6)a. Benjamin loaded the truck with the hay.  
% b. Benjamin loaded the hay on the truck.  
% c. Benjamin ran. [...]  
% Absence of an overt object
% makes implicit objects low on the PARTICIPANTS feature and hence lower in
% transitivity. They are also less likely to have telic ASPECT . Mittwoch (1982) [...]  
% Furthermore, indefinite and definite implicit objects differ on the
% relevance of the telicity feature. Mittwoch argues that inherently telic verbs are
% prohibited from occurring with indefinite implicit objects, since the indefinite
% object constructions must have activity interpretations. Thus telic verbs require
% either an overt object or, if the object is implicit, one that has a definite
% interpretation. As (8) shows, the telic verb win only has a definite interpretation. It
% is therefore infelicitous to claim ignorance of the implicit object.  
% (8)a. Benjamin won the race.  
% b. Benjamin won, *but I don't know what.  
% Allerton (1975) supports this characterization, observing that indefinite implicit
% objects occur with "verbs whose activity may be viewed as self-sufficient without
% an object," whereas definite implicit objects occur when "the meaning of the verb
% is somehow incomplete without mention of a PARTICULAR OBJECT"

% *Olsen & Resnik (1997: 4)*  
% >  In contrast, atelic verbs with implicit objects have indefinite
% interpretations, as Mittwoch points out. We note that they may also have definite
% interpretations: the object in (7b) may be interpreted as a salient meal, as in (9)
% (cf. Olsen (in press) for more examples of telic interpretations of atelic verbs).
% Mittwoch also observes in a footnote that indefinite implicit objects could have a
% telic interpretation in the appropriate context, as in I won't have dinner with you: I
% have already eaten.

% *Ruda (2020: 143)*
% > As is well-known, Vendler’s classification does not apply to verbs, but to verbal predications which include the verb’s complements and adjuncts. As build two houses is a telic accomplishment predicate, it combines with a time-span adverbial and does not license an entailment from the progressive to simple past. By contrast, build houses is atelic and has reverse properties

% *Ruda (2020: 144)*
% > Thus, incremental theme predications with quantised incremental themes are telic.

% *Ruda (2020: 144)*
% >  the telicity of predications with a verb of consumption like eat, a verb of creation like draw, and a verb of directed motion with an incremental path theme like run in fact is variable in English, as observed by Hay, Kennedy & Levin (1999: 139):

% **Ruda (2020) ha un sacco di riflessioni sulle classi vendleriane (cfr. Cennamo! fare tabella che tenga tutto insieme)**

% *Ruda (2020: 145)*
% > Other examples of verbs with variable telicity are given in (46) and (47), suggest-ing a need for a finer-grained approach to telicity that takes into account both the verb’s entailments of incrementality or lack thereof, the referential properties of the incremental theme, as well as linguistic and extra-linguistic context (Willim 2006:  153, 113). 

% *Cote (1996: 159)* su implicit object alternation (parlare di manner/result alternation?)  
% la parte "yet to be shown" si dimostra facendo vedere che la corrispondenza atelico:implicitdobj non è perfetta  
% >  it seems that all the verbs which display IOA are generally classified as process-oriented/non-completive (but it has yet to be shown that all verbs
% that meet this requirement can have null objects.)

% *Melchin (2019: viii)*
% >  devour denotes an event where
% the complement necessarily undergoes a complete scalar change (i.e., it must be fully devoured
% by the end of the event), which means that the complement must be syntactically realized
% (Rappaport Hovav and Levin 2001; Rappaport Hovav 2008). Eat, on the other hand, does
% not entail a complete change of state in its complement, and so the complement is optional.
% I show that the correlation between scalar change entailments and obligatory argument
% realization holds for a wider group of verbs as well. Thus, the c-selectional properties of eat,
% devour, and similar verbs need not be stipulated in their lexical entries

% *Melchin (2019: 54-55)*
% > Evidence from telicity suggests that the understood object of a verb that has undergone
% UOA is interpreted as a bare plural or mass. It is well known (see Tenny 1994, Jackendoff
% 1996, Krifka 1998, Borer 2005b, Ramchand 2008, and others) that the telicity of sentences
% containing UOA verbs depends on the object. If the object is bounded, or quantized,6 then
% the sentence is telic [...] This is shown below, using the in/for X time test for telicity, where predicates which
% may be modified with in X time are telic, and those which can take for X time are atelic:

% *Melchin (2019: 67)*
% > That the Theme of a scalar change must be syntactically present, and cannot be omitted,
% has been observed by Rappaport Hovav and Levin (2001), Rappaport Hovav (2008), and
% Beavers and Koontz-Garboden (2012). 

% *Melchin (2019: 77)*  
% FONDAMENTALE!!! Allora come test di telicità potrei dire che un verbo è telico se consente solo "in 5 minuti",
% atelico se consente solo "per 5 minuti", non-specificato se consente entrambe le forme (com'era più in Olsen e la privativity feature??)  
% > Smollett argues that, contrary to the claims of Tenny (1994) and Krifka (1998) (among
% others), a quantized object does not provide an endpoint to the event (i.e., it does not delimit
% the event). However, it does make an endpoint contextually available. 

% *Petho & Kardos (2006: 32)*
% > As is well-known, these changes in argument structure correlate with a shift in the aspectual
% properties of the verbal predicate, i.e. whereas the relevant simple verbs are of the event type
% process (activity), the resultatives and verbs complemented by similar secondary predicates are
% accomplishments/achievements (cf. Tenny 1994).

% *Naess (2011: 417-418)*
% > Mittwoch (1982) describes the difference between eating and eating something as that of an
% ‘activity’ versus an ‘accomplishment’, [...]  
% A similar analysis is developed by Tenny (1994), who characterises objects as ‘measuring
% arguments’: the effect of an act on its object delimits or ‘measures out’ the event, in
% the sense that the act is brought to an end once the effect on the object is fully achieved  
% Van Valin and LaPolla (1997) characterise the verb ‘eat’ as ‘not inherently telic’ (112).
% They provide an analysis within the framework of Role and Reference Grammar
% (RRG), which assumes the four basic classes of state, activity, accomplishment and
% achievement verbs. ‘Eat’ and similar verbs are analysed within this framework as activity
% verbs with derived accomplishment uses. [...] Valin and LaPolla label them inherent arguments. A further characteristic of inherent
% arguments is that, since they are non-referential, they have no semantic macrorole [...]  
% The second criterion, by contrast, is not delimited to second arguments of activity
% verbs, but is rather characteristic of the phenomenon known as indefinite object deletion:
% wherever an object can be omitted with the implication that its referent is indefinite
% and non-recoverable, the same pattern occurs, regardless of the semantics of the
% verb itself. For example the English verb murder would seem clearly to involve an
% undergoer macrorole and to represent an achievement rather than an activity. [...]

% *Naess (2011: 418)*
% > A more general problem which holds for all accounts based on the notion of telicity/
% or the activity / accomplishment distinction is that even in its intransitive use, eat may have
% a telic reading. If the reason why ‘eat’ has an intransitive use is that it is, or may be construed
% as, atelic, we would expect the intransitive use to necessarily have an atelic reading.
% However, this is not the case; while intransitive ‘eat’ clauses may combine with adverbs
% of duration and so be understood as atelic (We ate all evening), they can also take adverbs
% of completion, giving a telic reading: I ate in five minutes (before rushing off to work). It
% appears, then, that atelicity does not provide a satisfactory explanation for the intransitive
% behaviour of ‘eat’ verbs.

% *Naess (2011: 419)*
% > Nжss further argues that the affectedness of the agent accounts for the possibility of
% using intransitive eat with a telic reading, discussed above; the effect on the agent can
% function to ‘measure out’ the event just as the effect on the patient can, and it is the
% effect on the agent – going from hungry to full – that delimits the event in cases such as
% I ate in five minutes.

% *Tsimpli & Papadopoulou (2006: 1596)*  
% > For instance, a transitive, telic event
% involves the ‘‘subject’’, referred to as the [Originator], in the higher functional position EP (Event
% Phrase), whereas the ‘‘object’’, referred to as the subject-of-quantifiable-change, occupies the
% lower functional position ASPQ (cf. Borer, 1994, 1998, 2004). Differences between telic and
% atelic transitive predicates are captured by the absence of ASPQ in the atelic event. Note that the
% theory aims to provide a syntax-based combination of aspectual verb meanings and argument
% realisation.

% *Tsimpli & Papadopoulou (2006: 1598)*  
% > Activity verbs, which are the focus of study in this paper, are inherently atelic but acquire a telic
% interpretation, and therefore, an accomplishment reading, when followed by a specific direct
% object (cf. Mozer, 1994; Chila-Markopoulou and Mozer, 2001; Sioupi, 2002): [...]  
% > Furthermore, Chila-Markopoulou and Mozer (2001)
% point out that predicates consisting of imperfective activity verbs followed by bare nouns, which are
% non-specific, are atelic and might even denote a permanent property of the subject:  
% (3)the-NOM Eleni-NOM painted-IMP .3S portraits  
% ‘‘Eleni was painting portraits.’’  
% Perfective activity verbs when combined with bare nouns also assign an atelic reading to the
% predicate:  
% (4)the-NOM Eleni painted- PERF .3S portraits  
% ‘‘Helen painted portraits.’’

% *Tsimpli & Papadopoulou (2006: 1599)*  
% > In Greek, object arguments with specific interpretation are necessarily overt in sentences with
% transitive verbs, either as an object clitic or as a full DP.2 The similarity between the clitic and
% D is that both manifest the verb’s transitivity by lexicalising Case (Roussou and Tsimpli,
% 2006).

% *Tsimpli & Papadopoulou (2006: 1600)*  
% > Object omission is not possible in (9a) because verbs like build denote accomplishments and,
% therefore, the predicate must be telic. The unavailability of object omission is thus viewed as a
% consequence of telicity.

% *Tsimpli & Papadopoulou (2006: 1601)*  
% > The perfective/imperfective distinction has been shown to affect object drop in other languages
% too. Babko-Malaya (1999) shows that, in Russian, null objects are allowed in sentences with
% imperfective un-prefixed verbs but not with perfective ones:

% *Tsimpli & Papadopoulou (2006: 1601)*  
% ATTENZIONE! SOPRATTUTTO L'ESEMPIO 12 È DEL TUTTO INEFFICACE E LA CONCLUSIONE È FALLACE (<-- 2-VERB COORD)  
% > Turning to the Greek data, we observe that object drop is more productive than in English and
% possible even with verbs expressing accomplishments:
% (11)the-NOM Petros-NOM built-PERF-3S in-the Halkidhiki
% ‘‘??Petros built in Halkidiki.’’
% (12) the-NOM enemies-NOM destroyed-3P and burnt-3P
% ‘‘*The enemies destroyed and burnt.’’ [...]  
% The grammaticality of the Greek examples in (11) and (12) shows that the Russian asymmetry
% illustrated in (15) and (16) is not attested in Greek. In other words, there is no grammaticality
% effect depending on the perfective/imperfective distinction. [<--- ma subito prima aveva detto che i soggetti preferiscono le imperfettive!]

% *Tsimpli & Papadopoulou (2006: 1603)*  
% > More specifically, we adopt the Transitivity Requirement (TR), which reads as in (20)
% (cf. Basilico, 1998; Bowers, 2002; Erteschik-Shir and Rapoport, 2004; Hale and Keyser, 1993;
% Pirvulescu and Roberge, 1999; Roberge, 2002):
% (20) Transitivity Requirement (TR): An Obj position is always included in VP,
% independently of lexical choice of V. [...]  
% Roberge (2002) argues that TR is the direct object counterpart to the EPP. Crucially, TR is
% assumed to be the subcategorisation requirement of the verb, to be kept distinct from u-marking,
% which involves both the (object) position and the category (Chomsky, 1981). TR dictates the
% representation of a TransP, that is, a phrase headed by the functional head Trans (Basilico, 1998;
% Bowers, 2002). SpecTrans is the EPP position for direct objects, where Case is checked (Bowers,
% 2002).

% *Tsimpli & Papadopoulou (2006: 1605)*  
% > On the basis of the structures in (21) and (22), our prediction is that although both aspectual
% forms allow object omission, and, hence, no grammaticality issue arises, the differences in the
% syntactic representation and the interface interpretation of perfective and imperfective structures
% with null/overt objects, will result in a preference for null objects with imperfectives.

% *Tsimpli & Papadopoulou (2006: 1608-1609)*  
% ANCHE IN QUESTO PAPER C'È LA COSA DEI BAMBINI CHE OMETTONO I DOBJ PIÙ DEGLI ADULTI!  
% TROVARE ALTRO PAPER CHE DICEVA QUESTA COSA

% *Tsimpli & Papadopoulou (2006: 1609)*  
% > Syntactically, the Merge/Merge + Move distinction in the derivation of overt objects
% with perfectives and imperfectives, respectively, was argued to give rise to an economy effect
% favouring the intransitive uses of imperfectives. This preference is further supported by the
% associated effects on predicate interpretation (telic versus atelic reading) at the interface, which
% characterize perfective and not imperfective predicates. In addition, the fact that perfective verbs
% were more frequently used as transitive than imperfective verbs might also be due to the fact that
% perfectivity is understood as involving an endpoint and the use of an overt object makes this
% endpoint visible and the sentence more natural (cf. Horrocks and Stavrou, 2003).

% *Mittwoch (1982: 114)*
% > The differences all hinge on the fact that eat and eat something enter different "time
% schemata" in the sense of Vendler (1957). Eat is an "activity" predicate, whereas eat
% something is, as I shall demonstrate, an "accomplishment".' (I shall henceforth omit
% the word predicate for the sake of conciseness.) Activities and accomplishments contain
% the same verbs, namely "process" verbs, which I have defined elsewhere (Mittwoch
% (1980)) as verbs that can be modified, literally and nonelliptically, by the adverbs quickly
% and slowly. The distinction between them depends on the presence or absence of an
% object NP (or directional phrase, in the case of verbs of motion) and its features if
% present.2 A process verb without an object or with an object NP that lacks a quantifier,
% i.e. that consists of a "bare" plural or mass noun, enters into an activity; a process verb
% with a quantified object NP enters into an accomplishment. I shall distinguish quantified
% and unquantified NPs by the positive and negative values, respectively, of the feature
% [delimited quantity]

% *Mittwoch (1982: 115)*
% > There are other contexts in which atelic durationals can occur with predicates
% consisting of process verbs and quantified object NPs:

% *Mittwoch (1982: 116)*  
% ESPLORARE QUESTA PAGINA! SARÀ IL FULCRO DEL PARAGRAFO SULLA TELICITÀ, PENSO  
% > The restriction of this use of a lot to states and activities may be connected with the fact
% that in its use as part of an NP a lot occurs with bare plurals and mass nouns, witness
% a lot of cakelcakes but *a lot of a cakelsome cakeslthree cakes.4 It has been pointed out
% that there is a certain mereological analogy between Vendler's time schemata and NPs,
% such that states and activities correspond to unquantified NPs whereas accomplishments
% and achievements correspond to quantified ones. 

% *Mittwoch (1982: 117)*  
% PIÙ SUL CONTESTO CHE SULLA TELICITÀ?  
% > 7 With many verbs, an object that is [ +delimited quantity] may be implied. Thus, in the right co
% I've just eaten may mean 'I've just eaten lunch'; and I've just written may mean 'I've just written a lette
% a person presupposed by the co

% *Goldberg (2001: 507-509)*  
% > Many researchers have observed that atelic contexts are more likely to be intran-
% sitive than telic contexts (Mittwoch, 1971; Hopper and Thompson 1980; Dixon,
% 1991, p. 288; Aarts, 1995, p. 87; van Hout, 1996, pp. 166±187; Rappaport Hovav
% et al., 1998). However, atelicity per se is not necessary for object omission. Notice
% example (9) is telic, and yet the example is fully acceptable:  
% 9. Scarface killed again.  
% The use of again in (9) indicates that Scarface has killed before. If the action is
% construed as an isolated occurrence, the sentence is unacceptable:  
% 10.  ?? Pam killed yesterday.6  
% 6  As Nik Gisborne points out, Pam killed for the ®rst time yesterday, is ®ne. I believe this is because the
% possibility of multiple actions is evoked by ®rst. Notice #Pam killed early in the morning yesterday
% requires a special context in which Pam is known to have killed before. [...]  
% It is clear from Table 1 that the repetition of the action is more relevant than the
% atelicity of the event.

% *Lazard (2002: 162)*
% > Several of the constructions surveyed express incomplete processes, either
% because the verb is in an incompletive aspect (progressive, habitual, etc.) or
% because the object is only partly affected. There is an affinity between the
% incompleteness of the process and the low individuation of the object. We have
% seen that different constructions may be used both where the object is weakly
% individuated and where the process is incomplete. This is not surprising, since
% an indefinite object, if plural or a mass noun, can only be partially affected. On
% the whole, incompleteness of the process and low individuation of the object
% can be subsumed under the unifying label of 'low effectiveness'.
% On the other band, the idea of low agentivity Stands somewhat apart.
% However it is sometimes associated with the other factors among the meanings
% which may be conveyed by the antipassive in certain languages. Thus we may
% distinguish two subsets of factors: on the one band, those which constitute dif-
% ferent forms of low effectiveness, and, on the other band, low agentivity,
% although both subsets may happen to be associated in the meaning of some
% constructions.

% *Dvorak (2017: iii)*  
% CFR. STESSA CONSIDERAZIONE DI TSIMPLI PAPADOP 2006 QUI O IN "TELICITY"
% > I argue that the incompatibility of INO
% with perfectives follows from an unvalued EPP-like feature on perfective aspectual heads.

% *Dvorak (2017: 192)*  
% >  Hopper and Thompson (1980) list aspect as one of several components involved in determining the
% degree of transitivity in a clause.

% *Dvorak (2017: 193)*  
% > As for English, one of the early takes on the interaction between null objects and aspect
% is found in Mittwoch 1982. Mittwoch analyzes the difference between John ate and John ate
% something as the difference between an activity predicate and an accomplishment predicate,
% using Vendler’s (1957, 1967) concept of lexical aspectual classes (called ‘time schemata’ at
% that time). She argues that eat is a process verb which becomes an accomplishment if it
% has a quantified object, such as something, but it becomes an activity if it does not have
% any object or if its objects lacks the feature [+delimited quantity], as in the case of bare
% plural and mass nouns. This view was further elaborated by Tenny (1987), who analyzes
% direct internal arguments as “measuring out” the event – giving it the semantic property of
% delimitedness, which she defines as the temporal boundedness of an event. Tenny (1987:155)
% gives the following examples of what she calls “object deletion verbs”:  
% John smoked.  
% John smoked a Cuban cigar.  
% Mary drank.  
% Mary drank a jug of apple wine.

% *Dvorak (2017: 194)*  
% > That predicates can differ in their telicity
% depending on the properties of their direct internal arguments was pointed out already by
% Verkuyl (1972). Verkuyl was probably the first to mention the difference between bare plural
% and mass nouns on one hand and all other nominals on the other when it comes to aspect,
% using the feature [±specified quantity] to distinguish them.

\subsection{Perfectivity} \labsec{perfectivity}

Perfectivity (grammatical aspect / viewpoint aspect), LEGGERE STOICA (2017)

% *Naess (2007: 118)*  
% > The correlation of aspect with transitivity is interesting in that it may function in
% both directions. On the one hand, languages with explicit morphological aspect
% marking may show differences in other properties of the clause co-varying with
% the aspect marking, so that clauses marked for perfective aspect show a highly
% transitive case-marking pattern or other structural features associated with high
% transitivity, while clauses marked for imperfective aspect similarly show structural
% features associated with low transitivity. This is the case for instance in Kalkatun-
% gu, where the difference between perfective and imperfective aspect in (5.30) is
% accompanied by a change in case-marking, from ergative-absolutive in the perfec-
% tive clause to absolutive-dative in the imperfective: [...]  
% The perfective-imperfective contrast is variously described in terms such as
% bounded vs. unbounded, complete vs. noncomplete, or non-reference to the inter-
% nal temporal structure of the situation vs. explicit reference to such structure (see
% discussion in e.g. Comrie 1976 :16ff). The perfective is taken to present the situa-
% tion in question as a delimited whole, whereas the imperfective presents it “from
% the inside”, as an ongoing process.
% The concept of delimitedness is closely associated with that of affectedness, as
% argued in Tenny (1994) and discussed in chapter 4. The limit or endpoint of an
% event is essentially defined in terms of its effect; for example, an act of breaking
% something has reached its endpoint when some object has reached the state of be-
% ing broken. This means that, as far as differences in transitivity are concerned, the
% perfective-imperfective distinction can be described as a difference in the param-
% eter of affectedness: a perfective clause describes a complete situation, including
% its effects, whereas the corresponding imperfective only refers to the act itself,
% viewed as a process, and does not include any effects which this process may even-
% tually achieve. In the terminology of Chung and Timberlake (1985), in an imper-
% fective clause the effect of the situation is “outside the event frame” and is neither
% referred to nor necessarily entailed.


% *Tsimpli & Papadopoulou (2006: 1597)*  
% > According to Smith (1991:6), there are
% three main viewpoint aspectual types: perfective, which views the situation as a whole with
% initial and final points, imperfective, which views part of the situation without initial or final
% points, and neutral, which includes the initial point of the situation and at least one internal
% stage.

% *Medina (2007: 30)*
% > Less attention has been paid to the relationship between grammatical aspect and the presence of an overt object in English, likely because the phenomenon of object omissibility has been construed as being verb-specific. 

% *Melchin (2019: 68)*  
% questo non sembra strettamente legato al ruolo della perfectivity nella dobj omission, ma potrebbe essere interessante pensarci su
% > It should be added that there are grammatical contexts in which the entailment of
% scalar change can be overridden. The most well-known example of this is referred to as the
% “imperfective paradox” (Dowty 1979; see Copley and Harley 2015 for recent discussion and
% analysis): in imperfective and progressive aspects, the usual entailment of a result state in a
% given event can be overridden, as shown in the following contrast (Copley and Harley 2015,
% p. 105):

% *Tsimpli & Papadopoulou (2006: 1597)*  
% >  the morphologically marked aspectual distinction between perfective and
% imperfective verb forms interacts with the null object option in Greek. More specifically, we will
% show that this interaction predicts differences in the rate of occurrence of null objects with
% perfective and imperfective verb forms: with both of these, null objects are acceptable, but they
% are the preferred option with imperfectives.

% *Dvorak (2017: 213)*  
% > Even though aspect is often treated as a syntactic categorial feature with two values, per-
% fective and imperfective (Schoorlemmer 1995, a.o.), this view has been challenged since
% Jakobson 1932. Jakobson describes imperfectivity as a category which is ‘subordinate’ to
% perfectivity: a perfective verb expresses the event in its totality or as bounded; an imper-
% fective verb simply does not say anything about the event’s totality – rather than expressing
% that it is not bounded. In Jakobson’s terminology, applied to morphological categories in
% general, perfectives represent the marked form and imperfectives the unmarked one; see
% also Forsyth 1970. These insights encouraged other researchers to treat imperfective as a
% sort of “leftover category”, or as “aspect non-aspect” because it does not have a uniform
% meaning (see for example Paslawska and von Stechow 2003 for Russian). 

% *Dvorak (2017: 266)*  
% CHIUDERE (L'ATTIVITÀ), RIATTACCARE (IL TELEFONO)...  
% RICOLLEGARE AL DISCORSO DEGLI SPECIAL REGISTERS!!!  E DELLA RECOVERABILITY!!!  
% > 9.2.1  Lexicalized Null Objects (LNO)
% Some transitive verbs have a constant, idiomatized meaning when their object is not ex-
% pressed overtly. These verbs can be perfective or imperfective, but in general, perfective
% verbs with LNO are more common, presumably because the INO strategy is not available
% to them for reasons discussed at length in Chapter 8. [...]  
% It is quite typical that many of the idiomatized meanings of LNO are limited to a particular
% jargon or slang. For example, in the environment of card players, (430-b) means that Charles
% did not use his cards in such a way that it closes the game; in soccer slang, (431-b) means
% that Charles scored a goal. [...]
% The tendency of LNO to appear with transitive verbs that already have an idiomatized
% meaning is matched by the fact that LNO can be often found with predicates that allow
% only one particular entity in the role of an internal argument. For example, in Czech, one
% cannot smeknout ‘to uncap, to tip’ anything else except the hat he’s wearing; one cannot
% zaparkovat ‘to park’ anything else except the vehicle (s)he is driving. Both of these verbs
% appear more often without an overt object than with it in Czech.

% *Dvorak (2017: 268)*  
% > Some lexical-semantic classes of verbs are more prone to having null objects with an idio-
% matized meaning than others. Probably the most numerous is the group of verbs describing
% various chores. All of the following verbs are perfective and their null object could be inter-
% preted as “all the entities in a given household or another given location that the described
% activity normally affects with respect to the agent of the event”. [Gianni ha lavato, stirato, pulito...]

\subsection{A side note on tense}
l'avevo messa anche nel capitolo sperimentale! vedere se spostare direttamente tutto qua

% ## A side note on tense  
% *Medina (2007: 68)*
% > Although Tense and Aspect are independent properties, to the extent that one encourages certain interpretations 
% of the other, Tense may also be shown to affect the grammaticality of an implicit object.  For example, past tense 
% (indicating that the event occurred at a time prior to the speaking act) may encourage aninterpretation of perfectivity 
% (viewing the event from the perspective of its point of completion); in fact, this has been shown for children (Wagner, 2001).

% *Glass (2020: 22)*
% > past-tense uses of verbs are particularly resistant to object omission (Goldberg 2005) 

% *Garcia-Velasco & Munoz (2002: 9)*  
% > Note also that activities tend to be accompanied by grammatical or lexical expressions
% of imperfectivity such as present tense, continuous forms or temporal expressions (all day,
% always, etc.):  
% (16)a. Do you write _____ ?  
% b. He’s been writing _____ all day  
% As Dixon (1991) observes, the acceptability of object omission with past tenses decreases:
% *She knitted. In the same line, punctual verbs such as hit or wrap do not readily take
% understood objects. This can also explain why eat cannot omit its object in example (15)
% above.

% *Garcia-Velasco & Munoz (2002: 12)*  
% > it is expected that, if an element in the expression forces the
% object to become the focus it should be impossible to omit it. This is precisely one of the
% effects caused in expressions by the so-called completive or perfective particles (up and out)
% in phrasal verbs such as drink up, use up, seek out or work out. According to Mittwoch
% (1971), these particles combine with those actions which are capable of completion and,
% consequently, completive particles are incompatible with object omission (see 2.5.). In quite a
% similar line, Quirk et al. (1985: 595) claim that these particles serve to shift the focus of
% attention onto the result of the action, hence, onto the verbal object. That is why we find the
% following contrast:  
% (26) a. He is eating _____  
% b. *He is eating up ______  
% The perfective particle up requires the action to be capable of completion, hence disallowing
% object omission. The hypothesis that we want to explore can thus be formulated as follows:  
% Hypothesis 3  
% Transitive verbs containing a perfective particle cannot omit their objects.


\section{Pragmatic, contextual, and discourse factors} \labsec{pragmaticfactors}

compattare in max 2 sottosezioni! e dire che lo uso come umbrella term to define any predictor of object drop being totally or partially rooted in extra-linguistic terrain\\

mettere solo "pragmatic factors" nel titolo

% # Routine

% *Eu (2018: 528)*
% > Just like the contextual referents discussed above, semantic specialization is
% contextually established, not only through immediate contexts such as the form, time
% and place of utterance, but also through general contexts such as world knowledge and
% speaker’s life pattern, although its contents are always categories rather than specific
% individuals. For instance, when someone says at lunchtime I have already eaten, the
% most likely object is lunch. García Velasco & Portero Muñoz explain that when we
% understand eat as ‘eat a meal’:
% it is our world knowledge, the fact that we eat several times on the day, which leads us to
% the right interpretation of the understood object. (García Velasco & Portero Muñoz 2002:
% 5)
% Furthermore, Levin (1993: 33) notes that Mike ate can mean he ate ‘something one
% typically eats’, and Rice (1988: 204) says that in Each afternoon, John reads, the
% missing object is likely to be books.

% *Glass (2020: 2)*
% > the degree to which a verb describes a series of recognized, conventional actionswithin a community (Mart ́ı 2010, Mart ́ı 2015, Levin & Rapaport Hovav 2014).

% *Glass (2020: 2)*
% >  assumption that a verb’s association with a routine is a gradient notionwhich varies across social communities (Schank & Abelson 1977; Mithun 1984). 

% *Glass (2020: 3)*
% > object-omitting uses verbs are analyzed as intransitive aspectual activitiesdescribing the routine actions of an agent.

% *Glass (2020: 6)*
% > Transitive verbs describing routines have been suggested to facilitate object omission. Mart ́ı2010, Mart ́ı 2015 analogizes object omission to the incorporation of object nouns into verbs,such asberry-pick, which Mithun 1984 argues is cross-linguistically only available for verb-object pairs describing ‘institutionalized,’ ‘name-worthy’ actions within a community.  Mithunnotes that incorporations such asreindeer-huntare only used in communities where reindeerhunting is routine; and that novel incorporations such asladder-climblead hearers to imaginea scenario where that action is routine, for example as part of a fitness test.  Analyzing objectomission as a form of silent noun incorporation, Mart ́ı claims that object-omitting verbs alsodescribe routines.

% *Glass (2020: 6)*
% > Levin & Rapaport Hovav 2014 focus on the verbclean, which they analyze as a ‘result’verb when it conveys that its agent causes a change in its theme, resulting in the theme becom-ingclean(I cleaned the table).  ‘Result’ verbs cannot omit their objects, they say, because thesub-event of the theme changing in state requires its argument to be syntactically realized (Rap-paport Hovav & Levin 1998, Rappaport Hovav & Levin 2005, Rappaport Hovav 2008, Levin& Rappaport Hovav 2011). The puzzle is whycleanactually can omit its object:I cleaned yes-terday.  They propose thatcleanis polysemous between a causative change-of-state requiringan object, and an aspectual activity (dynamic and atelic) describing routine actions of an agenttypically leading to such changes of state (vacuuming, sweeping), in which the object can beomitted. Because changes of state commonly co-occur with the routines that bring them about,cleancomes to polysemously describe both of them, and can omit its object when it describesa routine.

% *Glass (2020: 7)*
% > verbs omit their objects relatively more often in the communities where the actions they describe arerelatively moreroutine.

% *Naess (2011: 415)*  
% QUINDI CI STA CHE QUESTI VERBI SIANO SPESSO USATI COME INTRANSITIVI, STANDO ALL'IPOTESI DI GLASS 2020!
% > Eating and drinking are perhaps the most fundamental of human activities

% *Naess (2011: 420)*  
% > 7. ‘Specialised Readings’
% Many authors have noted that in English, and in many other languages where ‘eat’ and
% ‘drink’ may occur either with or without a direct object, the objectless construction has a
% particular reading which is not inherent to the verb itself. As Fillmore (1986:96) puts it,
% ‘EAT is used to mean something like ‘‘eat a meal’’ – not merely ‘‘eat something’’, and
% DRINK is used to mean ‘‘drink alcoholic beverages’’ ’. Huddleston and Pullum (2002)
% refer to such constructions in English as ‘specific category indefinites’, contrasting with
% ‘normal category indefinites’ where the understood object is taken to belong to the ‘typical,
% unexceptional category for the verb in question’ (Huddleston and Pullum 2002:304).
% A potential objection to such an account is the problem of making a principled distinction
% between the two types – how can you tell whether an absent object is meant to be
% interpreted as being of a ‘specific category’ rather than being ‘typical, unexceptional’? [...]  
% not clear why alcoholic drinks should be seen as ‘prototypical’; most acts of drinking by
% most people involve objects other than alcohol  
% (QUESTO ULTIMO PUNTO SI SPIEGA BENISSIMO TENENDO CONTO DI GLASS 2020!)  
% An alternative explanation which has been proposed starts from the affected agent analysis
% described above, and argues that the function of object omission with verbs of eating
% and drinking is to highlight the effect of action on the agent, by removing the other
% affected argument, the object.

% # Habituality and iterativity

% *Cote (1996: 21)*
% > Fijian is another interesting example because it marks the ‘suppression’ of an obligatory argument with reduplication. 
% > Only truly transitive verbs reduplicate, not intransitives or unaccusatives. Fijian has the additional intriguing property, again described in Jonas and Lathroum (1992), that not all obligatory arguments of a
% verb may be suppressed. 

% *Cote (1996: 113)*
% > GOA verbs are distinct from those which allow Arbitrary Object Alternation (AOA). With
% AOA verbs, the object may be non-overt only if it is arbitrary in reference and the clause has a ‘generic’
% time reference (cf. DeClerck 1991 for one discussion of this term). I discuss whether examples such as
% (15) below and others found in Levin (1989) should be treated separately from this type.
% (15).  This dog bites.
% There is also at least one type of null object which is apparently not lexically-constrained.1
% Habitual Object Alternation (HOA) appears to allow null objects only when there is a repeated or
% habitual action, as in example (8).

% *Cote (1996: 143)*
% > Unlike the other alternations discussed here, HOA does not appear to be restricted to a particular
% class of verbs. Examples of HOA constructions are shown below:
% (75).  Bert pushed and Ernie pulled.

% *Petho & Kardos (2006: 29)*
% > Others seem to be relatively independent of the verb, but have
% to be associated instead with certain (grammatically and semantically characterisable)
% constructions, e.g. habituality (2) and coordination (3):  
% (2) I like to knit.  
% (3) He will steal, rob and murder.

% *Mittwoch (2005: 1)*  
% > The omissibility of unspecified objects is for many verbs subject to contextual
% factors. A relatively small number of verbs allow object drop freely in episodic
% sentences. Many more allow it in habitual sentences. This chapter argues that
% this is, in large part, connected to the fact that such sentences tend to have
% unquantized objects, and that the objects, if present, would be backgrounded. [...]  
% Section 11.1 deals with episodic sentences. It begins with a brief survey of the
% most typical cases, e.g. She is eating, and suggests a very general minimal
% context that is sufficient for unspecified object drop with verbs belonging to this
% category. It then goes on to discuss more marginal cases that require
% considerably more context to license them. Section 11.2 discusses examples of
% habitual uses of verbs, where the lexicon interacts with more general properties
% of the sentence, particularly aspectual ones. Missing objects are much
% commoner in habitual sentences than in episodic ones.

% *Mittwoch (2005: 7)*  
% > I shall use the term ‘habitual’ rather freely to include restricted habituals in the
% progressive (e.g. He is reading the Iliad at the moment, said about somebody (p.
% 244) who is asleep), iterative, and generic uses, as well as additional uses [...]  
% The verbs in (2) can occur freely without further modifiers in habitual sentences,
% though sometimes the meaning is somewhat different. Whereas episodic He is
% writing in the context of (6) denotes making marks with a pen, chalk, etc., the
% same sentence as a restricted habitual or the non-progressive sentence He
% writes could equally be used for working on a typewriter or computer, or even
% dictating into a tape recorder. Intransitive drink has a habitual use in which the
% understood object is restricted to an alcoholic beverage, as well as a use in
% which it is not thus restricted.
% In habitual sentences we find, in addition, a much larger range of verbs that
% permit unspecified direct objects to be dropped, including many that are not
% process verbs. [...]  
% 11.2.1 Dispositional properties and professions

% *Mittwoch (2005: 8)*  
% > 11.2.2 Missing people  
% Rizzi (1986) draws attention to the examples in (23):  
% (23)a. This leads (people) to the following conclusion.  
% b. This sign cautions (people) against avalanches.  
% c. John is always ready to please (people).  
% Rizzi accounts for the objectless uses of the verbs in (23) by postulating a
% lexically governed rule that allows the direct θ-role (of verbs subject to the rule)
% to be saturated in the lexicon rather than being projected in the syntax. The θ-
% role is assigned arb (arbitrary interpretation), i.e. [+HUMAN, +GENERIC,
% PLURAL]. In Italian, by contrast arb can be projected as pro and be syntactically
% active. It can be antecedent for PRO, as in (24), and be modified by secondary
% predicates, as in (25), subject to a restriction to the effect that these predicates
% have to be plural:

% *Mittwoch (2005: 10)*  
% > 11.2.3 Missing things  
% Rizzi did not discuss cases where what is missing is inanimate. Such cases are
% by no means uncommon in English. Apart from the examples listed in 11.2.1 we
% find the following:  
% (26) Online shopping was supposed to revolutionize the way we buy.  
% (27) a. In Mediterranean countries they generally build on the hilltops.
% (p.246)  
% b. Round here they build quickly / inefficiently.  
% c. In the past they built only in stone.

% *Mittwoch (2005: 11)*  
% > (37) As a boy he often stole.  
% (38) We export to three continents.  
% (39) She cleans and polishes all day.  
% Since habitual sentences are (like atelic base sentences) imperfective, the
% quantificational characteristics of the understood objects are the same as those
% discussed in Section 11.1, i.e. [−DELIMITED QUANTITY], and like these they are
% interpreted as nonspecific. [...]  
% natural that the missing object should be understood as a bare
% plural. Note that there are verbs that favour plural event readings; not
% unexpectedly, these permit object drop more easily than verbs similar in
% meaning that do not:  
% (40)a. This factory manufactures for export.  
% b. ?This factory produces for export.  
% (41)a. They pilfer / loot / plunder.  
% b. *They filch / swipe / snitch.

% *Mittwoch (2005: 12)*  
% PLURACTIONALITY  
% > In the following examples some of these verbs occur in a use that is not actually
% habitual or iterative, but shares with these uses the property of event plurality;
% following Lasersohn (1995) I shall adopt ‘pluractionality’ as a cover term for
% event plurality.11 Pluractionality can manifest itself in conjoint sentences, as in
% (45), where object drop in the whole is more natural than in the parts:  
% (45)a. They murdered, raped, and plundered.  
% b. [International tribunals] are valuable, she argues, because when
% they punish criminals, they also affirm, condemn, purge, and
% purify.  
% Pluractionality, according to Lasersohn, also covers intensity and long duration
% of the process involved in an event. 

% *Mittwoch (2005: 13)*  
% CFR. AFFECTED-AGENT ACCOUNT!!! secondo me that's it!  
% > (p.249) Of particular interest in this connection is the productive process of
% outV formation, where the resulting form selects for an object that belongs to
% the same class as the subject (see also Rappaport Hovav and Levin, this volume).
% Not surprisingly, in view of the competitive component in their meaning, outV
% verbs are particular popularly in commercial contexts. The examples in (49)
% were all found on the web:  
% (49)a. I don’t think they can outbuild us.

% *Mittwoch (2005: 14)*  
% > When a habitual sentence can be contextualized so that verb and object
% represent backgrounded information, object drop is facilitated. Thus, although
% verbs of destruction are very unlikely candidates for object drop, in a context
% where the demolition of houses or the felling of trees is the topic of
% conversation, (53a, b) is just about tolerable:
% (53)
% a. They usually demolish rather than restore.
% b. They fell indiscriminately.

% *Mittwoch (2005: 17-18)*  
% > 11.2.5 Contrastive contexts  
% The most permissive contexts for object drop involve pairs of verbs that stand in
% some sort of semantic contrast. Some speakers appear to have an anaphoric
% process whereby in a sequence of clauses in which contrasting verbs share a
% bare NP object, the object is dropped in the second clause; both clauses have
% contrastive focus:  
% (60) Those who give bribes are just as guilty as those who take.
% (HeraldTribune, 23 Sept. 1998)   
% The most typical examples for object drop (and more acceptable than (60)) are
% those in which there is parallelism, with both clauses lacking the objects: [...]  
% It is noteworthy that in such contexts we find even some of the poorest
% candidates for object drop; break is a prototypical change-of-state verb. (Cf.
% Rappaport Hovav and Levin, this volume, for the normal constraint on object
% expression with such verbs.) Even verbs that have no manner content
% whatsoever can form such a pair:  
% (62) This one creates, that one destroys.18  
% Object drop in these cases is clearly a rhetorical device: the absence of the
% object has the effect of adding weight to the verb. The verbs, as it were, prop
% each other up. Just as de-stressing an element increases the phonological
% prominence of what is stressed, omission, a further step, increases the
% prominence of what remains. What is left does not have to share the hearer’s
% attention with what is omitted.

% *Ahringberg (2015: 8)*  
% SPOSTARE/AGGIUNGERE AL PARAGRAFO SUI FATTORI PRAGMATICI!  
% BELLO IL PASSAGGIO FILLMORE VS GOLDBERG, USARLO.  
% > 2.3.2 Pragmatic and semantic licensing
% In contrast to Fillmore (1986), Goldberg (2006, p. 196) claims that pragmatic factors are
% essential for whether or not it is possible to omit an obligatory argument. What she suggests,
% more specifically, is that certain means of accentuation permit leaving out the object in cases
% where the predicate is a verb. These include, for example, “repeated action” (Pat gave and gave
% but Chris just took and took), “strong affective stance” (He murdered!), and “contrastive focus”
% (She could steal but she could not rob.) (Goldberg, 2006, pp. 196-197). The idea is that as the
% focus is put on the action of the verb the omitted objects are of low discourse prominence, and
% thus not necessary to express. 


% *Cummins & Roberge (2005: 46)*  
% INTERESSANTISSIMA LA COSA CHE DICE FONAGY (1985)!!! CFR. ACQUISIZIONE, WORLD KNOWLEDGE, ROUTINE...  
% > This phenomenon has not gone unnoticed and a number of recent
% accounts are on the market. Larjavaara’s (2000) study has a primarily
% interpretive semantic basis: she classifies null objects as either latent
% (having an identifiable referent) or generic (without such a referent), and
% considers that null objects are not represented syntactically, although a
% number of structural factors are correlated with them (such as the dative
% pronoun in (1b), for example.)
% Fonágy’s (1985) study is also based on a contemporary corpus and focuses
% on stylistic and sociolinguistic characteristics. According to his observations,
% the frequency of null objects varies inversely with speaker’s age, and he sees
% some null objects as a recent fashion with an iconic basis, expressing
% nonchalance, haste, or the desire to suppress the referent as well as its
% linguistic representation.
% Noailly (1997) identifies the main function of NOs as assuring cohesion
% with the discourse, in the case of anaphoric NOs, and with the extralinguistic
% context, in the case of deictic null objects.
% Lambrecht and Lemoine’s account (1996), based on the notion of ÔÔdefiniteÕÕ
% and ÔÔindefiniteÕÕ reference, highlights the diverse factors—lexical, construc-
% tional, pragmatic, discursive—that must figure in an understanding of null
% objects.

% # Pragmatic and discourse factors

% *Petho & Kardos (2006: 29)*  
% PASSO AVANTI, dalla recoverability del dobj all'interpretability dell'utterance (questi autori sono contro!)
% > It has been repeatedly suggested, for example by Groefsema (1995) and Németh T. (2001), that
% the omission of complements is primarily driven by pragmatic and discourse factors. More
% exactly, according to this view it is lexical and encyclopedic knowledge or previous contextual
% information that helps the hearer recover the reference of the unrealised element and thus reach
% an interpretation of the utterance that conforms to the pragmatic principle of relevance. This
% approach claims that implicit arguments are always ultimately licensed by the interpretability of
% the utterance.

% *Korkiakangas (2018: 7)*  
% Il presente studio si concentra sul latino notarile delle carte altomedievali toscane (VIII e
% IX secolo d.C.), fino ad ora uno dei pochi materiali latini adeguati allo studio linguistico
% computazionale, in quanto disponibile in formato digitale e sintatticamente annotato  
% IMPORTANTISSIMA SCALA DI TRANSITIVITY HOPPER-THOMPSON-1980!!!  
% > L'omissione dell'oggetto diminuisce il grado HT di ogni
% replica di (2) di 3 punti rispetto al caso con l'oggetto referenziale coinvolto. Ciononostante, i
% predicati complevi e dedi di (2) sono di alta transitivitа per altri aspetti e, pertanto, responsabili di
% quello che si puт definire 'alta transitivitа' nella LLCT2. In altri termini, l'omissione dell'oggetto и
% qualche volta curiosamente collegata ad un'alta transitivitа, se la frase trasmette una forza illocutiva
% dichiarativa.

% *Korkiakangas (2018: 12)*  
% > I verbi di questa sottospecie, tranne subscribo che governa sempre una frase preposizionale, sono
% bivalenti, cioи semanticamente transitivi. A causa della loro funzione performativa, sono di solito
% attivi, telici, volitivi, affermativi, reali e agentivi e spesso anche puntuali, fattori, questi, di alta
% transitivitа sulla scala HT. Il fatto che, per lo piщ, non superano il valore 7, и dato loro essere
% frequentemente privi di oggetto, come in (29) e (31). Sembra che proprio la forza performativa
% inerente di queste espressioni fisse, oltre al contesto formulaico in cui occorrono, tenda a permettere
% l'omissione dell'oggetto.  
% D'altronde, l'omissione dell'oggetto (la cosiddetta object deletion; vedi Naess (2007: 124-128)) è un
% fenomeno tipico dei verba dicendi, che li separa da altri verbi bivalenti e che ha portato alcuni
% studiosi a considerarli fondamentalmente intransitivi (Naess, 2007; Munro, 1982; Thompson, 2002;
% Thompson e Hopper, 2001). Il fenomeno sarebbe almeno in parte causato dalla stessa
% soggettivizzazione del significato che riduce la valenza di verbi mentali del tipo 'pensare' (Traugott,
% 1989; Kärkkäinen, 2003; Thompson, 2002). È ovvio che la mancanza di oggetto abbassa il grado di
% transitività di siffatti verbi sulla scala HT

% ## Salience

% *Cote (1996: 113)*
% > Salient Object Alternation (SOA)  ("the president called")

% *Cote (1996: 125)*
% >  Salient Object Alternation (SOA). The verbs in this lexically-constrained category are roughly the same as those that
% would be included in what Fillmore (1986) and others have called ‘definite’ object alternations. 

% ## Coordination

% *Petho & Kardos (2006: 29)*
% > Others seem to be relatively independent of the verb, but have
% to be associated instead with certain (grammatically and semantically characterisable)
% constructions, e.g. habituality (2) and coordination (3):  
% (2) I like to knit.  
% (3) He will steal, rob and murder.

% *Glass (2013: 3)*
% >  as Goldberg (2001) points out, even though result verbs are generally less accept-
% able with IOs than manner verbs are, result verbs productively allow IOs in certain semantic con-
% texts. For example, many result verbs sound fine with IOs in generic statements, modal state-
% ments (which I add to Goldberg's list), repeated actions, infinitives, emphasis (which Goldberg
% calls “strong affective stance”) and contrastive focus:  
% (13)  Tigers only kill at night.  generic (p. 506)  
% (14)  Dresses I would murder for  modal [W]  
% (15)  Scarface killed again.  repeated action (p. 507)  
% (16)  The singer always aimed to please/impress  infinitive (p. 506)  
% (17)  How can they give this creep a light prison term? He murdered!  emphasis (p. 513)  
% (18)  He burglarized, but she murdered!  contrastive focus (p. 514)
 
% *Glass (2013: 5)*
% > Therefore, when an IO appears in an episodic sentence, more information is lost than when an
% IO appears in a sentence describing some sort of iteration. In other words, in a sentence describing
% iteration, the opportunity cost (measured in terms of information) of using an IO is lower than in
% an episodic sentence. Thus, in these sentences describing iteration, it becomes less likely that in-
% terlocutors' communicative purposes would be thwarted if the IO version is uttered instead of an
% explicit object, perhaps explaining why IOs are more common in these contexts.

% *Garcia-Velasco & Munoz (2002: 2)*
% > Those participants which are given in the context will be more likely to be omitted than those
% which have not been introduced or are introduced for the first time. Obviously, a given object
% can be recovered from the surrounding linguistic context, which is not the case with a new
% participant. Allerton (1975) offers an interesting scale of ‘givenness’, [...]  
%  As pointed out to us by Lachlan Mackenzie (p.c.) presumably the difference between the
% expressions and then we ate and and then we ate dinner must partly be that the eating is in
% Focus in the former and the dinner in the latter. In other words, if the focus of a linguistic
% expression is the activity denoted by the main verb, the participants are more likely to be left
% out.

% *Garcia-Velasco & Munoz (2002: 3)*
% > Some linguistic constructions readily favour argument omission. Among those cited in the
% literature we find the following:
% Contrastive: He theorises about languages but I just describe (Dixon 1991)
% Fixed phrases: Seek and ye shall find; hit or miss (Fellbaum & Kegl 1989)
% Linking or sequential: First she knitted, then she sewed (Dixon 1991); He will steal, rob and
% murder (Kilby 1984)
% Instructional imperatives: Drink up. Push hard. (Levin 1993)
% One property of structural omission is that it seems to override other relevant factors. That is,
% if a verb typically does not allow object omission, in most cases it will be possible to suggest
% a structural context in which it does. What is important to remember in these cases is that the
% omission is motivated by the structure itself and not necessarily by the properties of either the
% verb or the omitted object.

% *Garcia-Velasco & Munoz (2002: 7)*
% > almost nothing at all is said about the interaction of the two areas. [lexicon and discourse/context]
% Notable exceptions are Allerton (1975), Fillmore (1986), Groefsema (1995), Németh (2000)
% and, in a lesser degree, Fellbaum & Kegl (1989).

% *Lorenzetti (2008: 66)*  
% > Moreover, structural omission, i.e. determined by some linguistic constructions, which
% more readily favour object omission, deserves special mention. Among the most frequently
% cited constructions in the literature are contrastive focus (Pussycats eat, but tigers devour),
% fixed phrases (hit or miss), linking or sequential (You wash, I’ll dry), iterated actions (The
% chef fried and baked all afternoon) and instructional imperatives (Take three eggs. Break into
% a bowl).
% Finally, object omission is also enhanced by the extra linguistic context, which frequently
% provides clues as to the identification of the missing information. Frames in this respect prove
% to be of fundamental importance (Restaurant Script: the client entered, he ordered, he ate, he
% paid, he left).

% *Cummins & Roberge (2004: 4)*
% > And both note several structural contexts that favour a non-
% overt object. These contexts are summarized and illustrated in (6) to (12).
% (6) sequences of verbs
% a. He will steal __, rob__, and murder __. (GP)
% b. Elles ont caressé__, pétri__, étreint __, pénétré __... (L:97)
% “They have caressed __, kneaded __, clasped __, penetrated __.”
% (7) imperatives
% a. Push __ hard. (GP:)
% b. Fais voir __. (L:50)
% “Show __.”
% (8) contrastive uses
% a. He theorises about language, but I just describe__. (GP:)
% b. Seulement moi, je n'assassine __ pas, je ressuscite __. (L:91)
% “Only I don't murder __, I resuscitate __.”
% (9) infinitive
% a. This is a lovely guitar, with an uncanny ability to impress __ and delight __.
% (BNC)
% b. Pour compenser __, j'ai décidé d'adopter dorénavant cette graphie. (L:85)
% “To compensate __, I have decided to use that spelling from now on.”
% (10) generic present tense
% a. There are those who annihilate__ with violence—who devour __. (BNC)
% b. Un peintre dérange__ bien moins qu'un écrivain. (L:83)
% “A painter disturbs __ much less than a writer.”

% *Goldberg (2001: 510)*  
% > The account relies on a notion of ``discourse prominence'' that subsumes
% both topic and focus. In English, discourse-prominent arguments, whether promi-
% nent by virtue of being topical or focal, generally need to be expressed. Normally,
% the patient argument of a causative verb is quite prominent in the discourse; one
% typically does not assert that a participant changes state unless one wishes to discuss
% or draw attention to that participant. Therefore patient arguments of causative
% verbs typically need to be expressed. Yet the typical situation does not always hold.
% In certain contexts, it is possible to ®nd patient arguments of causative verbs that
% have very low discourse prominence and therefore need not be expressed. The fac-
% tors outlined above combine to insure that the patient argument receives little pro-
% minence in the discourse: the patient argument is neither focal nor topical.
% Moreover, the action must be emphasized, thereby further shifting discourse pro-
% minence away from the patient argument. 

% *Goldberg (2001: 513)*  
% > Rice (1988) suggests that in many contexts which allow omitted objects generally,
% ``the pragmatic focus is on the activity itself'' (p. 206). It might be tempting to adopt
% this idea, that the action necessarily takes on the role of focus, since by hypothesis,
% the patient argument is not focal, and every sentence requires at least one focus:
% every utterance must contain some assertion or new information (Lambrecht, 1994,
% p. 206). This would give us an explanation for the increased emphasis on the action:
% the action must take on the role of focus because the patient is non-focal.  
% However, there are two reasons to think that the nature of the increased emphasis
% on the action is not captured by the notion of focus.

% *Goldberg (2001: 514)*  
% > We can summarize the constraints on patient omission discussed so far in terms of
% a principle of Omission under Low Discourse Prominence:
% I. Omission of the patient argument is possible when the patient argument is
% construed to be deemphasized in the discourse vis a vis the action. That is,
% omission is possible when the patient argument is not topical (or focal) in the
% discourse, and the action is particularly emphasized (via repetition, strong
% a€ective stance, discourse topicality, contrastive focus, etc.). [...]  
% As noted at the outset of Section 2, languages di€er in their grammatical possibi-
% lities for argument omission. No languages allow focal elements to be omitted,
% because focal elements are by de®nition not predictable from context. In many lan-
% guages, the primary topic, which is the subject, if topical, can be omitted; these are
% the so-called ``pro-drop'' languages (e.g. Spanish). Other languages such as Japanese
% and Korean allow non-subject, topical arguments to be omitted as well. In English,
% with a few lexical exceptions (cf. Fillmore, 1986), all topical arguments including the
% subject must be expressed. However if the action is particularly emphasized (by
% repetition, contrast, etc.), it is possible to omit arguments that are both predictable
% (non-focal) and non-relevant (non-topical).10

% *Goldberg (2001: 517)*  
% > The principle of Omission under Low Discourse Prominence was formulated on
% the basis of patient arguments of causative verbs. However, the same generalization
% can illuminate conditions of inde®nite object omission more generally.

% *Goldberg (2001: 518)*  
% MOLTO IMPORTANTE!!! PROVARE A FARE QUESTO STUDIO DIACRONICO?  
% > Interestingly, the same set of verbs frequently occurs in generic contexts with a
% habitual interpretation: Pat drinks; Pat smokes; Chris sings; Sam bakes. It seems
% likely that the frequent appearance of this usage, which is licensed by the Omission
% under Low Discourse Prominence principle, led to the grammaticalization of a lex-
% ical option for these verbs, whereby they could appear intransitively in less con-
% strained contexts. Corpus and historical work, to determine the frequencies of usage
% and the historical evolution, would be required to determine whether this hypothesis
% is correct.

% *Goldberg (2001: 518)*  
% > The factors outlined as relevant to inde®nite argument omission can help motivate
% the ``characteristic property'' examples noted by Fellbaum and Kegl (1989) and
% Levin (1993, p. 39). Levin cites example (38), and observes that certain other verbs
% including bite, itch, scratch, sting, can appear intransitively, with the interpretation
% that the action is characteristic of the agent.  
% 38. That dog bites.  
% Bite, scratch and sting are arguably causative verbs, so these cases provide further
% evidence that patient arguments of causative verbs need not always be expressed.
% Interestingly, example (38) involves a generic context and a general and non-speci®c
% patient argument; therefore this data is licensed directly by the Omission under Low
% Discourse Prominence principle. The generic context naturally leads to an inter-
% pretation in which the action is characteristic of the subject argument, but such an
% interpretation is not required to license these particular examples. Note that the
% following variants of (38) do not involve characteristic actions:  
% 39. a. That dog has been known to occasionally bite, but he is generally
% very loving.  
% b. The frightened toddler scratched and bit until his mother arrived.

% *Naess (2007: 128)*
% > As IOD is most typically found with certain semantically specifiable classes of
% verbs, most attempts at analysing it have focused on the semantic properties of
% individual verbs or verb classes. However, Goldberg (2001) demonstrates that lex-
% ical semantic factors are not the only properties which may license IOD.
% Goldberg shows that “causative verbs”, that is, verbs causing a change of state
% in their objects, are perfectly felicitous without an overt object in certain specific
% contexts, despite claims from a number of researchers (e.g. Grimshaw and Vikner
% 1993, Brisson 1994, Rappaport Hovav and Levin 1988) that the patient argument
% of such verbs must always be expressed. She quotes the following examples of in-
% definite, nonspecific patient arguments of causative verbs being omitted:  
% (6.1)  Object omission with causative verbs, Goldberg (2001 :506):

% *Eniko (2014: 683)*
% >  Purely pragmatic
% approaches lead to analyses according to which every argument can be omitted if it is
% inferable. [...] According to the first type of pragmatic approaches, the missing content can be inferred as a
% conversational implicature through Gricean maxims (Rice 1988). [...] The other type of pragmatic approaches 
% suggests that pragmatic free enrichment is responsible for the interpretation of implicit contents in these utterances (Sperber & Wilson
% 1986/1995: 189; Carston 2002). According to relevance theory, the decoded meanings of
% utterances in (9)(11) are linguistically underspecified and can be enriched into full-fledged
% conceptual representations of literal utterance meanings by means of free enrichment. The
% process of free enrichment contains general-purpose inference rules (Sperber & Wilson
% 1986/1995: 176).

% *Melchin (2019: 50)*  
% MANNER/RESULT & AGENT = MANNER! CFR. INSTRUMENTS
% > representative examples adapted from Glass 2014, p. 123; most of the contexts and
% examples are originally from Goldberg 2001):  
% (6)  a.  Tigers only kill at night.  generic  
% b.  Dresses I would murder for.  modal  
% c.  Scarface killed again.  repeated action  
% d. The singer always aimed to please/impress. infinitive  
% e. Why would they give this creep a light prison term? He murdered!! emphasis  
% f.  He burglarized, but she murdered! contrastive focus  
% [...]  
% they are all contexts in which the resultant state holding of the theme of the event is less relevant than the nature of the action itself.

\subsection{Iterativity} \labsec{iterativity}

Contents

% *Tsunoda (1999)*
% ANCHE ALTROVE C'ERA UN RIFERIMENTO AGLI ANTIPASSIVI!
% In Warrungu, il suffisso iterativo ha valenza aspettuale e, quando è usato con verbi transitivi, li trasforma in antipassivi  
% (gli antipassivi sono verbi a valenza ridotta come "mangiare (qualcosa)", ma nelle lingue ergative)

% *Goldberg (2001: 505)*
% > Causative verbs entail that there is a change of state in their patient argument,
% which is normally expressed by their object. Several researchers have argued or
% assumed that causative verbs obligatorily express the argument that undergoes the
% change of state in all contexts (Browne, 1971; Grimshaw and Vikner, 1993; Brisson,
% 1994; van Hout, 1996, pp. 5±7; Rappaport Hovav et al., 1998). Initial support for
% this generalization might be drawn from the following examples:  
% 3.a. *The tiger killed.  
% b. *Chris broke.  
% Clearly the generalization must be relativized to English, since many languages do
% allow the patient or theme argument to be unexpressed when it represents topical
% information. This is true for example in Chinese, Japanese and Korean (Li and
% Thompson, 1981; Huang, 1984).

% *Goldberg (2001: 506)*
% > Pace claims in the literature to the contrary, causative verbs often do actually
% allow patient arguments to be omitted, particularly when they are inde®nite and
% nonspeci®c. The following examples illustrate this phenomenon:3  
% 6.a. The chef-in-training chopped and diced all afternoon.  
% b. Tigers only kill at night.  
% c. The singer always aimed to dazzle/please/disappoint/impress/charm  
% d. Pat gave and gave, but Chris just took and took.  
% e. These revolutionary new brooms sweep cleaner than ever  
% (Aarts, 1995, p. 85)  
% f. The sewing instructor always cut in straight lines. [...]  
% 3. [...] These cases would be considered a subtype of ``Inde®nite Null Complementation'' according to Fillmore
% (1986), and a subtype of ``Lexically Conditioned Intransitivity'' according to Fellbaum and Kegl (1989).

% *Goldberg (2001: 507)*  
% IMPORTANTE! SI PARLA ANCHE DI TELICITY  
% > By contrast, the acceptable examples in (6a±f) involve a further relevant factor:
% they designate actions that are iterative (6a,d) or actions that are generic (6b,c,e,f)
% (see also Resnik, 1993, p. 78). In the case of iterative actions, the action designated
% by the verb is interpreted as repeated more than once. In the case of the generic
% statements in (6b,c,e,f), the action is also likely (if not by logical necessity) to be
% repeated more than once, as the statement is understood to be true generally. It is
% also possible that the iterative or generic context be embedded in a negative context
% in which no repetition is entailed, but the possibility of repetition is evoked:  
% 60 . a. The chef-in-training didn't chop or dice all afternoon.  
% b. Tigers never kill at night.  
% c. The singer never aimed to dazzle/please/disappoint/impress/charm.  
% It may be suggested that atelicity could supply the appropriate constraint. Repe-
% ated actions are often construed as atelic or temporally unbounded events.

% *Naess (2007: 136)* 
% CONTRA GOLDBERG?
% > The IOD construction with a purpose clause – John murdered for the money –
% does not necessarily have an iterative reading, as pointed out above. Rather, such
% clauses are construable as a kind of affected-agent construction where the affected-
% agent reading is not imposed by the semantics of the verb, but rather by the pur-
% pose clause. A statement of the agent’s motivation or purpose in performing an act
% is essentially a statement of the benefits that the agent hopes to achieve in acting;
% in other words the intended effect of the act on the agent. Affectedness of the
% agent, then, is not necessarily inherent to the semantics of a specific verb, but may
% be introduced by other elements of a clause.

% *Rissman (2016: 428)*  
% > If instrumental verbs are result verbs, and the instrument is inferred pragmatically,
% then instrumental verbs should pattern with prototypical result verbs such as break
% on a wide range of diagnostics. In this paper, I tested this prediction focusing on
% the diagnostic of object deletion shown in (4):  
% (4)a.b.Cinderella scrubbed all night long.  
% *Cinderella broke all night long.  
% Beavers & Koontz-Garboden (2012) argue that this test diagnoses result: if a verb
% allows object deletion, as in (4a), then this verb does not encode a result. Rappaport-
% Hovav (2008) explains this contrast in terms of the scalar semantics of result verbs:
% if a verb specifies a scale, then the entity changing along the scale cannot be
% omitted. (4b) is infelicitous because break encodes a two-point scale where its
% direct object changes from a non-broken to a broken state.
% This linking hypothesis makes the prediction that instrumental verbs should be
% infelicitous in this construction, as they encode a scalar result. [...]  
% In my studies, I focused on a particular variant of object deletion, which I term
% the "x-and-x construction:"  
% (6)a.b.Cinderella scrubbed and scrubbed all night long.  
% *Cinderella broke and broke all night long.

% *Rissman (2016: 436)*  
% QUESTA CONSIDERAZIONE TORNA SPESSO NEI MIEI APPUNTI  
% > As discussed above, I propose that the x-and-x construction targets agentive
% meaning. In particular, this construction highlights an atelic event in which the
% agent repeatedly performs an action. Given this highlighting of the agent's action,
% the direct object may be deleted, if the verb encodes some type of action on the part
% of the agent.

\subsection{Habituality} \labsec{habituality}

Contents

% *Mittwoch (2005: 4-6)*  
% > The minimal context in which we find intransitive occurrences of the verbs in (2)
% is something like this. We point to a man or to a picture of a man and ask the
% question in (a), where (b) could provide appropriate answers:  
% (6)a. What is he doing?  
% b. He is reading / cooking / knitting / drawing / eating.  
% Similarly the question-answer pair in (7)  
% (7)a. What did you do after dinner last night?  
% b. I read / knitted / cooked.  
% In these contexts the verbs denote ‘activities’ not only in the aspectual sense of
% this term but in the literal sense: they can be used to describe what a person is
% engaged or occupied in doing at a particular moment or interval, just like verbs
% such as work and rest that do not take objects at all. [...]  
% I now turn to two verbs that are included in a list of verbs under the heading of
% ‘unspecified object alternation’ in Levin (1993), but which do not meet the
% [ACTIVITY] criterion and have therefore not been included in (2) above. In the
% minimal context (8) would not be an acceptable answer:  
% (8) He is polishing / chopping.  
% The indefinite pronoun something would be required by default. [...]  
% Build is a verb of creation, and moreover one that selects for a specific range of
% object nouns (unlike make, create, produce, manufacture). Yet (11) is odd as an
% answer to (6a).  
% (11) ??He is building.  
% Insofar as (11) is acceptable, it suggests playing with a Lego set, a fairly
% homogeneous activity, rather than digging foundations, laying real bricks, etc.
% (But note that After supper the child built is impossible.) [...]  
% (12) John is building on the empty lot at the bottom of the road.
% (p.243) But in (12) build is not an [ACTIVITY]; it cannot serve as an answer to
% (6a). It involves an action spread over a much wider time-span than the
% examples above, and it may well be a statement about the place involved rather
% than about John. In fact, the person denoted by the subject of build does not
% have to participate in physical labour her/himself. [NOTA MIA: CFR. AFFECTED-AGENT ACCOUNT!] [...]  
% Verbs other than process verbs cannot drop unspecified objects in episodic
% sentences in English, even in colloquial usage. Consider verbs denoting change
% of possession: [...]  
% Indirect objects denoting recipients are not omissible:  
% (16) I gave it / lent it / promised it / bequeathed it *(to somebody).  
% The most notable exception to this rule is sell as in  
% (17) I sold it (to somebody).7

\subsection{Emphasis} \labsec{emphasis}

Contents

\subsection{Constrastive focus} \labsec{constrastfocus}

Contents

\subsection{Characteristic actions} \labsec{charactions}

Actions that are characteristic of the subject


\section{A note on frequency}

qui dire che la frequenza ha a che fare con vari altri fattori, di cui è un sottoprodotto o una causa (riflettere su questo); crucialmente, non è un fattore indipendente, e come tale non lo considero nell'esperimento (anzi, tento di minimizzarne l'effetto in ogni modo, v. capitolo JUDGMENTS)

% SLIDE MEDINA! CERCARE NEL TESTO DELLA SUA TESI
% Relative SPS is correlated with the relative frequency of an implicit object.

% *Glass (2020: 2)*
% > frequency: the per-million-word frequency of a verb in a corpus (Goldberg 2005)

% *Glass (2020: 4)*
% > Goldberg 2005 notes that in minimal pairs such as (1) (eatvs.devour),the member of the pair that allows omission is the more frequent one.

% *Glass (2020: 5)*
% >In contrast, Ruppenhofer 2004 (Chapter 4) studies 34 verbs, binary-classified into those saidto allow or disallow object omission, and finds no association with their frequency in corpora:‘lemmas with very high token frequency allow [object omission] as well as lemmas with verylow token frequency.’ He concludes that object omission is idiosyncratic.

% *Glass (2020: 5)*
% > if a verb appears frequently in generic or habitual contexts, it is likely to omit its objectin those contexts, a pattern which would diachronically generalize to episodic contexts as well

% *Lorenzetti (2008: 65)*  
% > Not only does semantic neutrality seem to play a role in object omission, but frequency of
% occurrence is also important, as suggested by Goldberg [2005]. Some verbs like smoke, drink,
% sing and write appear without an object even in situations which are not within the purview of
% the de-profiled object construction (i.e. when the action is particularly emphasised), since they
% occur in generic contexts and with an habitual interpretation (Pat smokes; Chris sings; John
% drinks). Goldberg argues that the frequent appearance of some verbs in those contexts
% apparently led to the grammaticalization of a lexical option, whereby they can appear
% intransitively in less constrained contexts, which amounts to saying that the frequent omission
% of an argument in a given context, which allows or favours the omission may lead to the
% creation of a new convention through a process of reanalysis. [...]  
% 58
%  This line of thought is consistent with the claim [Bybee 1998; Hopper 1987] that ‘meanings’ are to be
% considered as generalisations from many repetitions of hearing predicates used in association with certain types of
% human events or situations over the course of a person’s lifetime. Our brain is masterfully adapt at categorising
% and sorting new data and what can initially appear as an extension, loses this status after several hearings, thus
% showing how the dividing line between stored argument structures and extension is constantly changing. [...]  
% 59
%  This is in accordance with the claim that the more frequent a predicate is, the less likely is it to have a fixed
% structure. The most frequent verbs in English, get, say, know, go, know, think, see, come, want, mean (Biber et al.
% 1999) are in fact reported not to have a fixed argument structure, but some are found in lexicalised expressions
% and discourse markers and serve as a basis for innovation and variation (Croft 2000).

% *Lorenzetti (2008: 66)*  
% > Moreover, supporting the idea that the high frequency59 of these verbs is responsible for
% their reanalysis is the fact that verbs considered synonyms, but less frequent, do not allow
% object omission:  
% (10) a. Tom read / *perused last night  
% b. Tom wrote/*drafted last night.
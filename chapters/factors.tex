% \setchapterimage[6.5cm]{seaside}
\setchapterpreamble[u]{\margintoc}
\chapter{Factors allowing indefinite object drop}
\labch{factors}

\section{Semantic factors} \labsec{semanticfactors}

\subsection{Recoverability} \labsec{recoverability}

Contents

% *Lorenzetti (2008: 64)*  
% > The presence or absence of an object may
% affect the type of state of affairs denoted by the predication [Vendler 1967], determining a
% shift from an activity to an accomplishment reading, as in (8) and (9). [...]  
% Another relevant parameter pertains to the specificity of the omitted object, namely the
% capacity of a verb to take just one or a very limited range of objects. In this respect, as shown
% by Rice [1988], it appears that the more predictable a participant is, the more likely it is to be
% deleted. Moreover, there seems to be a close relation between object omission and the
% semantic role of the omitted objects, as documented by Fillmore [1986].

% *Kardos (2010: 7-8)*  
% SI RICOLLEGA A DEFAULTNESS E RECOVERABILITY!  
% > Levin and Rappaport Hovav describe sweep, an example of this class, as a manner verb whose
% manner constant <SWEEP> is inserted into an activity event-structure template (see (17)
% above). However, sweep is a special activity in that it does not only take a single argument
% (which is introduced by the template as such), but it is also associated with an additional,
% optional participant, the surface that is being swept. When accounting for the optionality of
% such arguments, the authors claim that, in fact, there are important differences between the
% structural arguments introduced by an event-structure template, and this latter type of
% arguments that depend on the conceptual content of a constant. The structural arguments are
% required, so they may not be omitted under normal circumstances. However, the non-
% structural arguments belonging to constants may be left implicit if they are recoverable, either
% through lexical stereotypes or based on the context. For example, there is a type of things that
% are normally swept (namely, floors), and this stereotype allows the interpretation that some
% floor is being swept even if the verb is used intransitively and the context does not reveal what
% exactly is being swept. Secondly, they claim that structural objects are relevant to the event
% structure of the predicate, whereas non-structural ones are not. [...]  
% Based on similar evidence, Levin and Rappaport Hovav revised their theory in their 2004
% paper in the following way. On the one hand, they still bolster the view that events denoted by
% eat and sweep consist of two subevents, namely, the action carried out by the eater (or
% sweeper) on the one hand, and the gradual disappearance of the apple, or the gradual
% “becoming swept” of the room. On the other hand, however, they note a significant difference
% between this event structure and that of truly complex causative verbs. They posit that the two
% subevents of an eating or a sweeping event occur at the same time and cannot be clearly
% separated since, as they put it, they are temporally dependent on each other. Therefore, they
% refrain from calling these true complex events, claiming that this duality can be grasped only
% conceptually, which explains why these events are simple from a linguistic point of view. In
% particular, the fact that the object of these verbs can be dropped, and that non-subcategorized
% objects can combine with them (see section 2), is regarded as evidence for the “simple event”
% character of eat-type verbs.

\subsection{Agent affectedness} \labsec{agentaffect}

Contents

\subsection{Manner specification} \labsec{mannerspec}

qui devo dire che "manner" è un termine molto ambiguo, perché io la intendo nel senso tecnico di "manner specification" eat/devour, ma altri lo usano in opposizione a "result" per indicare, sostanzialmente, activity vs accomplishment/achievement


\section{Aspectual factors} \labsec{aspectualfactors}

% *Dvorak (2017: 115)*  
% SPOSTARE IN TELICITY/PERFECTIVITY!  
% > The issue
% here is that once the verb is in the -ing form, describing an ongoing event, the postulated
% distinction between activities and accomplishments is overridden. This is confirmed by the
% fact that all progressivized process verbs behave like activities for the purpose of Dowty’s
% (1979) classical tests distinguishing accomplishments from activities.

\subsection{Telicity} \labsec{telicity}

Contents

\subsection{Perfectivity} \labsec{perfectivity}

Contents


\section{Pragmatic factors} \labsec{pragmaticfactors}

% *Ahringberg (2015: 8)*  
% SPOSTARE/AGGIUNGERE AL PARAGRAFO SUI FATTORI PRAGMATICI!  
% BELLO IL PASSAGGIO FILLMORE VS GOLDBERG, USARLO.  
% > 2.3.2 Pragmatic and semantic licensing
% In contrast to Fillmore (1986), Goldberg (2006, p. 196) claims that pragmatic factors are
% essential for whether or not it is possible to omit an obligatory argument. What she suggests,
% more specifically, is that certain means of accentuation permit leaving out the object in cases
% where the predicate is a verb. These include, for example, “repeated action” (Pat gave and gave
% but Chris just took and took), “strong affective stance” (He murdered!), and “contrastive focus”
% (She could steal but she could not rob.) (Goldberg, 2006, pp. 196-197). The idea is that as the
% focus is put on the action of the verb the omitted objects are of low discourse prominence, and
% thus not necessary to express. 


% *Cummins & Roberge (2005: 46)*  
% INTERESSANTISSIMA LA COSA CHE DICE FONAGY (1985)!!! CFR. ACQUISIZIONE, WORLD KNOWLEDGE, ROUTINE...  
% > This phenomenon has not gone unnoticed and a number of recent
% accounts are on the market. Larjavaara’s (2000) study has a primarily
% interpretive semantic basis: she classifies null objects as either latent
% (having an identifiable referent) or generic (without such a referent), and
% considers that null objects are not represented syntactically, although a
% number of structural factors are correlated with them (such as the dative
% pronoun in (1b), for example.)
% Fonágy’s (1985) study is also based on a contemporary corpus and focuses
% on stylistic and sociolinguistic characteristics. According to his observations,
% the frequency of null objects varies inversely with speaker’s age, and he sees
% some null objects as a recent fashion with an iconic basis, expressing
% nonchalance, haste, or the desire to suppress the referent as well as its
% linguistic representation.
% Noailly (1997) identifies the main function of NOs as assuring cohesion
% with the discourse, in the case of anaphoric NOs, and with the extralinguistic
% context, in the case of deictic null objects.
% Lambrecht and Lemoine’s account (1996), based on the notion of ÔÔdefiniteÕÕ
% and ÔÔindefiniteÕÕ reference, highlights the diverse factors—lexical, construc-
% tional, pragmatic, discursive—that must figure in an understanding of null
% objects.

\subsection{Iterativity} \labsec{iterativity}

Contents

% *Goldberg (2001: 505)*
% > Causative verbs entail that there is a change of state in their patient argument,
% which is normally expressed by their object. Several researchers have argued or
% assumed that causative verbs obligatorily express the argument that undergoes the
% change of state in all contexts (Browne, 1971; Grimshaw and Vikner, 1993; Brisson,
% 1994; van Hout, 1996, pp. 5±7; Rappaport Hovav et al., 1998). Initial support for
% this generalization might be drawn from the following examples:  
% 3.a. *The tiger killed.  
% b. *Chris broke.  
% Clearly the generalization must be relativized to English, since many languages do
% allow the patient or theme argument to be unexpressed when it represents topical
% information. This is true for example in Chinese, Japanese and Korean (Li and
% Thompson, 1981; Huang, 1984).

% *Goldberg (2001: 506)*
% > Pace claims in the literature to the contrary, causative verbs often do actually
% allow patient arguments to be omitted, particularly when they are inde®nite and
% nonspeci®c. The following examples illustrate this phenomenon:3  
% 6.a. The chef-in-training chopped and diced all afternoon.  
% b. Tigers only kill at night.  
% c. The singer always aimed to dazzle/please/disappoint/impress/charm  
% d. Pat gave and gave, but Chris just took and took.  
% e. These revolutionary new brooms sweep cleaner than ever  
% (Aarts, 1995, p. 85)  
% f. The sewing instructor always cut in straight lines. [...]  
% 3. [...] These cases would be considered a subtype of ``Inde®nite Null Complementation'' according to Fillmore
% (1986), and a subtype of ``Lexically Conditioned Intransitivity'' according to Fellbaum and Kegl (1989).

% *Goldberg (2001: 507)*  
% IMPORTANTE! SI PARLA ANCHE DI TELICITY  
% > By contrast, the acceptable examples in (6a±f) involve a further relevant factor:
% they designate actions that are iterative (6a,d) or actions that are generic (6b,c,e,f)
% (see also Resnik, 1993, p. 78). In the case of iterative actions, the action designated
% by the verb is interpreted as repeated more than once. In the case of the generic
% statements in (6b,c,e,f), the action is also likely (if not by logical necessity) to be
% repeated more than once, as the statement is understood to be true generally. It is
% also possible that the iterative or generic context be embedded in a negative context
% in which no repetition is entailed, but the possibility of repetition is evoked:  
% 60 . a. The chef-in-training didn't chop or dice all afternoon.  
% b. Tigers never kill at night.  
% c. The singer never aimed to dazzle/please/disappoint/impress/charm.  
% It may be suggested that atelicity could supply the appropriate constraint. Repe-
% ated actions are often construed as atelic or temporally unbounded events.

% *Naess (2007: 136)* 
% CONTRA GOLDBERG?
% > The IOD construction with a purpose clause – John murdered for the money –
% does not necessarily have an iterative reading, as pointed out above. Rather, such
% clauses are construable as a kind of affected-agent construction where the affected-
% agent reading is not imposed by the semantics of the verb, but rather by the pur-
% pose clause. A statement of the agent’s motivation or purpose in performing an act
% is essentially a statement of the benefits that the agent hopes to achieve in acting;
% in other words the intended effect of the act on the agent. Affectedness of the
% agent, then, is not necessarily inherent to the semantics of a specific verb, but may
% be introduced by other elements of a clause.

\subsection{Habituality} \labsec{habituality}

Contents

% *Mittwoch (2005: 4-6)*  
% > The minimal context in which we find intransitive occurrences of the verbs in (2)
% is something like this. We point to a man or to a picture of a man and ask the
% question in (a), where (b) could provide appropriate answers:  
% (6)a. What is he doing?  
% b. He is reading / cooking / knitting / drawing / eating.  
% Similarly the question-answer pair in (7)  
% (7)a. What did you do after dinner last night?  
% b. I read / knitted / cooked.  
% In these contexts the verbs denote ‘activities’ not only in the aspectual sense of
% this term but in the literal sense: they can be used to describe what a person is
% engaged or occupied in doing at a particular moment or interval, just like verbs
% such as work and rest that do not take objects at all. [...]  
% I now turn to two verbs that are included in a list of verbs under the heading of
% ‘unspecified object alternation’ in Levin (1993), but which do not meet the
% [ACTIVITY] criterion and have therefore not been included in (2) above. In the
% minimal context (8) would not be an acceptable answer:  
% (8) He is polishing / chopping.  
% The indefinite pronoun something would be required by default. [...]  
% Build is a verb of creation, and moreover one that selects for a specific range of
% object nouns (unlike make, create, produce, manufacture). Yet (11) is odd as an
% answer to (6a).  
% (11) ??He is building.  
% Insofar as (11) is acceptable, it suggests playing with a Lego set, a fairly
% homogeneous activity, rather than digging foundations, laying real bricks, etc.
% (But note that After supper the child built is impossible.) [...]  
% (12) John is building on the empty lot at the bottom of the road.
% (p.243) But in (12) build is not an [ACTIVITY]; it cannot serve as an answer to
% (6a). It involves an action spread over a much wider time-span than the
% examples above, and it may well be a statement about the place involved rather
% than about John. In fact, the person denoted by the subject of build does not
% have to participate in physical labour her/himself. [NOTA MIA: CFR. AFFECTED-AGENT ACCOUNT!] [...]  
% Verbs other than process verbs cannot drop unspecified objects in episodic
% sentences in English, even in colloquial usage. Consider verbs denoting change
% of possession: [...]  
% Indirect objects denoting recipients are not omissible:  
% (16) I gave it / lent it / promised it / bequeathed it *(to somebody).  
% The most notable exception to this rule is sell as in  
% (17) I sold it (to somebody).7

\subsection{Emphasis} \labsec{emphasis}

Contents

\subsection{Constrastive focus} \labsec{constrastfocus}

Contents

\subsection{Characteristic actions} \labsec{charactions}

Actions that are characteristic of the subject


\section{A note on frequency}

qui dire che la frequenza ha a che fare con vari altri fattori, di cui è un sottoprodotto o una causa (riflettere su questo); crucialmente, non è un fattore indipendente, e come tale non lo considero nell'esperimento (anzi, tento di minimizzarne l'effetto in ogni modo, v. capitolo JUDGMENTS)

% *Glass (2020: 2)*
% > frequency: the per-million-word frequency of a verb in a corpus (Goldberg 2005)

% *Glass (2020: 4)*
% > Goldberg 2005 notes that in minimal pairs such as (1) (eatvs.devour),the member of the pair that allows omission is the more frequent one.

% *Glass (2020: 5)*
% >In contrast, Ruppenhofer 2004 (Chapter 4) studies 34 verbs, binary-classified into those saidto allow or disallow object omission, and finds no association with their frequency in corpora:‘lemmas with very high token frequency allow [object omission] as well as lemmas with verylow token frequency.’ He concludes that object omission is idiosyncratic.

% *Glass (2020: 5)*
% > if a verb appears frequently in generic or habitual contexts, it is likely to omit its objectin those contexts, a pattern which would diachronically generalize to episodic contexts as well

% *Lorenzetti (2008: 65)*  
% > Not only does semantic neutrality seem to play a role in object omission, but frequency of
% occurrence is also important, as suggested by Goldberg [2005]. Some verbs like smoke, drink,
% sing and write appear without an object even in situations which are not within the purview of
% the de-profiled object construction (i.e. when the action is particularly emphasised), since they
% occur in generic contexts and with an habitual interpretation (Pat smokes; Chris sings; John
% drinks). Goldberg argues that the frequent appearance of some verbs in those contexts
% apparently led to the grammaticalization of a lexical option, whereby they can appear
% intransitively in less constrained contexts, which amounts to saying that the frequent omission
% of an argument in a given context, which allows or favours the omission may lead to the
% creation of a new convention through a process of reanalysis. [...]  
% 58
%  This line of thought is consistent with the claim [Bybee 1998; Hopper 1987] that ‘meanings’ are to be
% considered as generalisations from many repetitions of hearing predicates used in association with certain types of
% human events or situations over the course of a person’s lifetime. Our brain is masterfully adapt at categorising
% and sorting new data and what can initially appear as an extension, loses this status after several hearings, thus
% showing how the dividing line between stored argument structures and extension is constantly changing. [...]  
% 59
%  This is in accordance with the claim that the more frequent a predicate is, the less likely is it to have a fixed
% structure. The most frequent verbs in English, get, say, know, go, know, think, see, come, want, mean (Biber et al.
% 1999) are in fact reported not to have a fixed argument structure, but some are found in lexicalised expressions
% and discourse markers and serve as a basis for innovation and variation (Croft 2000).

% *Lorenzetti (2008: 66)*  
% > Moreover, supporting the idea that the high frequency59 of these verbs is responsible for
% their reanalysis is the fact that verbs considered synonyms, but less frequent, do not allow
% object omission:  
% (10) a. Tom read / *perused last night  
% b. Tom wrote/*drafted last night.
\setchapterpreamble[u]{\margintoc}
\chapter{Indefinite implicit objects}
\labch{indefinitedrop}

Summary: focus on the objects

\section{RECIPES} \labsec{recipes}

qui parlare di ricette e di linguaggi settoriali (paper sul calcio?)

% Syntactic *Cote (1996: 120-122)* (130 del pdf)
% > By syntactic explanation I mean any analysis which argues that the null object is represented in the syntactic structure of the sentence/utterance. [...] Culy (1987) argues this case for null objects in recipe constructions in English (a separate issue
% which will be discussed in Chapter 4) and then adds that he rejects the idea of a difference between the
% grammar of recipe contexts and the grammar of ‘Standard’ English. In particular, he argues that null
% objects in general are zero anaphors.

% *Ahringberg (2015: 9)*  
% > 2.3.3 Constructional licensing
% As previously mentioned, definite null instantiation is licensed not only by the predicate’s
% lexical properties but also by the grammatical construction in which it is found (Fillmore, 1986,
% p. 97; Lambrecht & Lemoine, 2005, p. 20). This kind of licensing is, for example, evident in
% imperative constructions (Prytz, 2009, pp. 11-12) which are common in recipes or manuals.

% *Garcia-Velasco & Munoz (2002: 9)*  
% RECIPES != INDEFINITE OBJECT  
% > Hypothesis 1  
% Indefinite Objects do not present available referents in the surrounding linguistic or
% extralinguistic context. [...]  
% for example, bake tends to appear in recipe contexts, exemplifying the so called “instructional
% imperative”, and consequently a type of structural omission. Bake, in this example does not
% take an activity reading, and, therefore, it may have a referent in the surrounding linguistic
% context:
\chapter*{Abstract}
\addcontentsline{toc}{chapter}{Abstract} % Add the preface to the table of contents as a chapter

Optionally transitive verbs, whose Patient participant is semantically obligatory but syntactically optional (e.g., \textit{to eat, to drink, to write}), deviate from the transitive prototype defined by \textcite{HopperThompson1980}. Following \textcite{Fillmore1986}, unexpressed objects may be either indefinite (referring to prototypical Patients of a verb, whose actual entity is unknown or irrelevant) or definite (with a referent available in the immediate intra- or extra-linguistic context). This thesis centered on \textit{indefinite} null objects, which the literature argues to be a gradient, non-categorical phenomenon possible with virtually any transitive verb (in different degrees depending on the verb semantics), favored or hindered by several semantic, aspectual, pragmatic, and discourse factors. In particular, the probabilistic model of the grammaticality of indefinite null objects hereby discussed takes into account a continuous factor (semantic selectivity, as a proxy to object recoverability) and four binary factors (telicity, perfectivity, iterativity, and manner specification).\\
This work was inspired by \textcite{Medina2007}, who modeled the effect of three predictors (semantic selectivity, telicity, and perfectivity) on the grammaticality of indefinite null objects (as gauged via Likert-scale acceptability judgments elicited from native speakers of English) within the framework of Stochastic Optimality Theory. In her variant of the framework, the constraints get floating rankings based on the input verb's semantic selectivity, which she modeled via the Selectional Preference Strength measure by \textcite{Resnik1993, Resnik1996}. I expanded Medina's model by modeling implicit indefinite objects in two languages (English and Italian), by using three different measures of semantic selectivity (Resnik's SPS; Behavioral PISA, inspired by Medina's Object Similarity measure; and Computational PISA, a novel similarity-based measure by \textcite{CappelliLenciPISA} based on distributional semantics), and by adding iterativity and manner specification as new predictors in the model.\\ % medina-like adjRsq 0.422 eng 0.391 ita
Both the English and the Italian five-predictor models based on Behavioral PISA explain almost half of the variance in the data, improving on the Medina-like three-predictor models based on Resnik's SPS. Moreover, they have a comparable range of predicted object-dropping probabilities (30-100\% in English, 30-90\% in Italian), and the predictors perform consistently with theoretical literature on object drop. Indeed, in both models, atelic imperfective iterative manner-specified inputs are the most likely to drop their object (between 80\% and 90\%), while telic perfective non-iterative manner-unspecified inputs are the least likely (between 30\% and 40\%). The constraint re-ranking probabilities are always directly proportional to semantic selectivity, with the exception of \textsc{Telic End} in Italian. Both models show a main effect of telicity, but the second most relevant factor in the model is perfectivity in English and manner specification in Italian.



% Moreover, atelic inputs are more likely to occur with implicit objects than telic inputs, imperfective inputs more than perfective inputs, iterative inputs more than non-iterative inputs, and manner-unspecified inputs more than manner-specified inputs, as expected. 

% binary faithfulness constraints (penalizing null objects) deriving from the binary predictors get re-ranked with respect to the \textsc{*Int Arg} markedness constraint (penalizing overt objects) as a function


% The full five-predictor models explain the variance in the data better than the three- and the four-predictor models regardless of the chosen measure of semantic selectivity (Resnik's SPS, Computational PISA, or Behavioral PISA). However, the four-predictor models (including iterativity in addition to Medina's telicity, perfectivity, and semantic selectivity) do not perform better than the three-predictor models, with basically identical adjusted R\textsuperscript{2} values in English and slightly smaller adjusted R\textsuperscript{2} values in Italian. Taken together, these results mean that iterativity alone is not a sufficient addition to Medina's model (rather, it makes the model needlessly more complicated, since it does not explain more variance in the data), but models including iterativity and manner specification together have a noticeably stronger explanatory power.

% I showed that the addition of manner specification determines a much stronger qualitative leap in the full models of Italian than in English, where the increase in the performance of the models is rather modest.

% In English SPS-based models are the worst-performing, while PISA-based models are noticeably better (with Behavioral PISA being better than Computational PISA). In Italian, instead, Computational PISA-based models are the worst-performing among all, followed by SPS-based models and, lastly, by Behavioral PISA-based models.

%  I argued that the English facts (SPS-based models being worse than PISA-based models, and SPS correlating poorly with PISAs while Behavioral PISA and Computational PISA correlate well with each other) may depend on the nature of these measures, considering that both PISA measures are based on the computation of pairwise similarity scores (distributional cosine similarity for Computational PISA, Likert-scale human judgments of similarity for Behavioral PISA), while SPS suffers from all the problems of taxonomy-based measures, as discussed in \refsec{predictor_sps}. The Italian picture is different, since in this case Computational PISA-based models perform worse than SPS-based models. Interestingly, as shown in \refsec{ita_judgresult_semsel}, Computational PISA in Italian correlates very well with SPS (even better than with Behavioral PISA). I take this to mean that even after the manual cleansing I performed to purge any artifacts from the corpus data (recounted and motivated on \refpage{cleanthecorpus}), the itWaC corpus, upon which I based the computation of SPS and Computational PISA relative to Italian, has a stronger effect on the resulting scores than the ukWaC corpus, which I used to model the computational measures of semantic selectivity in English.
 
% %  However, there may also be some undesirable side effect generated by the choice of ukWaC for English, given that 
%  I was able to reproduce Medina's findings relative to indefinite object drop in English with my three-predictor PISA-based models, but not with SPS, which is the measure she herself used (refer to \refsec{concl_medinacompare}).
 
%  To conclude, I observe that both in English and in Italian the best-performing models are based on Behavioral PISA. This result does not surprise at all, since this measure, due to being based on human similarity judgments, can be considered a benchmark model of semantic selectivity.
